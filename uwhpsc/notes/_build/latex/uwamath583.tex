% Generated by Sphinx.
\def\sphinxdocclass{report}
\documentclass[letterpaper,10pt,english]{sphinxmanual}

\usepackage[utf8]{inputenc}
\ifdefined\DeclareUnicodeCharacter
  \DeclareUnicodeCharacter{00A0}{\nobreakspace}
\else\fi
\usepackage{cmap}
\usepackage[T1]{fontenc}
\usepackage{amsmath,amssymb}
\usepackage{babel}
\usepackage{times}
\usepackage[Bjarne]{fncychap}
\usepackage{longtable}
\usepackage{sphinx}
\usepackage{multirow}
\usepackage{eqparbox}


\addto\captionsenglish{\renewcommand{\figurename}{Fig. }}
\addto\captionsenglish{\renewcommand{\tablename}{Table }}
\SetupFloatingEnvironment{literal-block}{name=Listing }

\addto\extrasenglish{\def\pageautorefname{page}}

\setcounter{tocdepth}{1}


\title{UW AMath 483/583 Class Notes}
\date{May 22, 2016}
\release{1.0}
\author{Randall J. LeVeque}
\newcommand{\sphinxlogo}{}
\renewcommand{\releasename}{Release}
\makeindex

\makeatletter
\def\PYG@reset{\let\PYG@it=\relax \let\PYG@bf=\relax%
    \let\PYG@ul=\relax \let\PYG@tc=\relax%
    \let\PYG@bc=\relax \let\PYG@ff=\relax}
\def\PYG@tok#1{\csname PYG@tok@#1\endcsname}
\def\PYG@toks#1+{\ifx\relax#1\empty\else%
    \PYG@tok{#1}\expandafter\PYG@toks\fi}
\def\PYG@do#1{\PYG@bc{\PYG@tc{\PYG@ul{%
    \PYG@it{\PYG@bf{\PYG@ff{#1}}}}}}}
\def\PYG#1#2{\PYG@reset\PYG@toks#1+\relax+\PYG@do{#2}}

\expandafter\def\csname PYG@tok@gd\endcsname{\def\PYG@tc##1{\textcolor[rgb]{0.63,0.00,0.00}{##1}}}
\expandafter\def\csname PYG@tok@gu\endcsname{\let\PYG@bf=\textbf\def\PYG@tc##1{\textcolor[rgb]{0.50,0.00,0.50}{##1}}}
\expandafter\def\csname PYG@tok@gt\endcsname{\def\PYG@tc##1{\textcolor[rgb]{0.00,0.27,0.87}{##1}}}
\expandafter\def\csname PYG@tok@gs\endcsname{\let\PYG@bf=\textbf}
\expandafter\def\csname PYG@tok@gr\endcsname{\def\PYG@tc##1{\textcolor[rgb]{1.00,0.00,0.00}{##1}}}
\expandafter\def\csname PYG@tok@cm\endcsname{\let\PYG@it=\textit\def\PYG@tc##1{\textcolor[rgb]{0.25,0.50,0.56}{##1}}}
\expandafter\def\csname PYG@tok@vg\endcsname{\def\PYG@tc##1{\textcolor[rgb]{0.73,0.38,0.84}{##1}}}
\expandafter\def\csname PYG@tok@m\endcsname{\def\PYG@tc##1{\textcolor[rgb]{0.13,0.50,0.31}{##1}}}
\expandafter\def\csname PYG@tok@mh\endcsname{\def\PYG@tc##1{\textcolor[rgb]{0.13,0.50,0.31}{##1}}}
\expandafter\def\csname PYG@tok@cs\endcsname{\def\PYG@tc##1{\textcolor[rgb]{0.25,0.50,0.56}{##1}}\def\PYG@bc##1{\setlength{\fboxsep}{0pt}\colorbox[rgb]{1.00,0.94,0.94}{\strut ##1}}}
\expandafter\def\csname PYG@tok@ge\endcsname{\let\PYG@it=\textit}
\expandafter\def\csname PYG@tok@vc\endcsname{\def\PYG@tc##1{\textcolor[rgb]{0.73,0.38,0.84}{##1}}}
\expandafter\def\csname PYG@tok@il\endcsname{\def\PYG@tc##1{\textcolor[rgb]{0.13,0.50,0.31}{##1}}}
\expandafter\def\csname PYG@tok@go\endcsname{\def\PYG@tc##1{\textcolor[rgb]{0.20,0.20,0.20}{##1}}}
\expandafter\def\csname PYG@tok@cp\endcsname{\def\PYG@tc##1{\textcolor[rgb]{0.00,0.44,0.13}{##1}}}
\expandafter\def\csname PYG@tok@gi\endcsname{\def\PYG@tc##1{\textcolor[rgb]{0.00,0.63,0.00}{##1}}}
\expandafter\def\csname PYG@tok@gh\endcsname{\let\PYG@bf=\textbf\def\PYG@tc##1{\textcolor[rgb]{0.00,0.00,0.50}{##1}}}
\expandafter\def\csname PYG@tok@ni\endcsname{\let\PYG@bf=\textbf\def\PYG@tc##1{\textcolor[rgb]{0.84,0.33,0.22}{##1}}}
\expandafter\def\csname PYG@tok@nl\endcsname{\let\PYG@bf=\textbf\def\PYG@tc##1{\textcolor[rgb]{0.00,0.13,0.44}{##1}}}
\expandafter\def\csname PYG@tok@nn\endcsname{\let\PYG@bf=\textbf\def\PYG@tc##1{\textcolor[rgb]{0.05,0.52,0.71}{##1}}}
\expandafter\def\csname PYG@tok@no\endcsname{\def\PYG@tc##1{\textcolor[rgb]{0.38,0.68,0.84}{##1}}}
\expandafter\def\csname PYG@tok@na\endcsname{\def\PYG@tc##1{\textcolor[rgb]{0.25,0.44,0.63}{##1}}}
\expandafter\def\csname PYG@tok@nb\endcsname{\def\PYG@tc##1{\textcolor[rgb]{0.00,0.44,0.13}{##1}}}
\expandafter\def\csname PYG@tok@nc\endcsname{\let\PYG@bf=\textbf\def\PYG@tc##1{\textcolor[rgb]{0.05,0.52,0.71}{##1}}}
\expandafter\def\csname PYG@tok@nd\endcsname{\let\PYG@bf=\textbf\def\PYG@tc##1{\textcolor[rgb]{0.33,0.33,0.33}{##1}}}
\expandafter\def\csname PYG@tok@ne\endcsname{\def\PYG@tc##1{\textcolor[rgb]{0.00,0.44,0.13}{##1}}}
\expandafter\def\csname PYG@tok@nf\endcsname{\def\PYG@tc##1{\textcolor[rgb]{0.02,0.16,0.49}{##1}}}
\expandafter\def\csname PYG@tok@si\endcsname{\let\PYG@it=\textit\def\PYG@tc##1{\textcolor[rgb]{0.44,0.63,0.82}{##1}}}
\expandafter\def\csname PYG@tok@s2\endcsname{\def\PYG@tc##1{\textcolor[rgb]{0.25,0.44,0.63}{##1}}}
\expandafter\def\csname PYG@tok@vi\endcsname{\def\PYG@tc##1{\textcolor[rgb]{0.73,0.38,0.84}{##1}}}
\expandafter\def\csname PYG@tok@nt\endcsname{\let\PYG@bf=\textbf\def\PYG@tc##1{\textcolor[rgb]{0.02,0.16,0.45}{##1}}}
\expandafter\def\csname PYG@tok@nv\endcsname{\def\PYG@tc##1{\textcolor[rgb]{0.73,0.38,0.84}{##1}}}
\expandafter\def\csname PYG@tok@s1\endcsname{\def\PYG@tc##1{\textcolor[rgb]{0.25,0.44,0.63}{##1}}}
\expandafter\def\csname PYG@tok@gp\endcsname{\let\PYG@bf=\textbf\def\PYG@tc##1{\textcolor[rgb]{0.78,0.36,0.04}{##1}}}
\expandafter\def\csname PYG@tok@sh\endcsname{\def\PYG@tc##1{\textcolor[rgb]{0.25,0.44,0.63}{##1}}}
\expandafter\def\csname PYG@tok@ow\endcsname{\let\PYG@bf=\textbf\def\PYG@tc##1{\textcolor[rgb]{0.00,0.44,0.13}{##1}}}
\expandafter\def\csname PYG@tok@sx\endcsname{\def\PYG@tc##1{\textcolor[rgb]{0.78,0.36,0.04}{##1}}}
\expandafter\def\csname PYG@tok@bp\endcsname{\def\PYG@tc##1{\textcolor[rgb]{0.00,0.44,0.13}{##1}}}
\expandafter\def\csname PYG@tok@c1\endcsname{\let\PYG@it=\textit\def\PYG@tc##1{\textcolor[rgb]{0.25,0.50,0.56}{##1}}}
\expandafter\def\csname PYG@tok@kc\endcsname{\let\PYG@bf=\textbf\def\PYG@tc##1{\textcolor[rgb]{0.00,0.44,0.13}{##1}}}
\expandafter\def\csname PYG@tok@c\endcsname{\let\PYG@it=\textit\def\PYG@tc##1{\textcolor[rgb]{0.25,0.50,0.56}{##1}}}
\expandafter\def\csname PYG@tok@mf\endcsname{\def\PYG@tc##1{\textcolor[rgb]{0.13,0.50,0.31}{##1}}}
\expandafter\def\csname PYG@tok@err\endcsname{\def\PYG@bc##1{\setlength{\fboxsep}{0pt}\fcolorbox[rgb]{1.00,0.00,0.00}{1,1,1}{\strut ##1}}}
\expandafter\def\csname PYG@tok@mb\endcsname{\def\PYG@tc##1{\textcolor[rgb]{0.13,0.50,0.31}{##1}}}
\expandafter\def\csname PYG@tok@ss\endcsname{\def\PYG@tc##1{\textcolor[rgb]{0.32,0.47,0.09}{##1}}}
\expandafter\def\csname PYG@tok@sr\endcsname{\def\PYG@tc##1{\textcolor[rgb]{0.14,0.33,0.53}{##1}}}
\expandafter\def\csname PYG@tok@mo\endcsname{\def\PYG@tc##1{\textcolor[rgb]{0.13,0.50,0.31}{##1}}}
\expandafter\def\csname PYG@tok@kd\endcsname{\let\PYG@bf=\textbf\def\PYG@tc##1{\textcolor[rgb]{0.00,0.44,0.13}{##1}}}
\expandafter\def\csname PYG@tok@mi\endcsname{\def\PYG@tc##1{\textcolor[rgb]{0.13,0.50,0.31}{##1}}}
\expandafter\def\csname PYG@tok@kn\endcsname{\let\PYG@bf=\textbf\def\PYG@tc##1{\textcolor[rgb]{0.00,0.44,0.13}{##1}}}
\expandafter\def\csname PYG@tok@o\endcsname{\def\PYG@tc##1{\textcolor[rgb]{0.40,0.40,0.40}{##1}}}
\expandafter\def\csname PYG@tok@kr\endcsname{\let\PYG@bf=\textbf\def\PYG@tc##1{\textcolor[rgb]{0.00,0.44,0.13}{##1}}}
\expandafter\def\csname PYG@tok@s\endcsname{\def\PYG@tc##1{\textcolor[rgb]{0.25,0.44,0.63}{##1}}}
\expandafter\def\csname PYG@tok@kp\endcsname{\def\PYG@tc##1{\textcolor[rgb]{0.00,0.44,0.13}{##1}}}
\expandafter\def\csname PYG@tok@w\endcsname{\def\PYG@tc##1{\textcolor[rgb]{0.73,0.73,0.73}{##1}}}
\expandafter\def\csname PYG@tok@kt\endcsname{\def\PYG@tc##1{\textcolor[rgb]{0.56,0.13,0.00}{##1}}}
\expandafter\def\csname PYG@tok@sc\endcsname{\def\PYG@tc##1{\textcolor[rgb]{0.25,0.44,0.63}{##1}}}
\expandafter\def\csname PYG@tok@sb\endcsname{\def\PYG@tc##1{\textcolor[rgb]{0.25,0.44,0.63}{##1}}}
\expandafter\def\csname PYG@tok@k\endcsname{\let\PYG@bf=\textbf\def\PYG@tc##1{\textcolor[rgb]{0.00,0.44,0.13}{##1}}}
\expandafter\def\csname PYG@tok@se\endcsname{\let\PYG@bf=\textbf\def\PYG@tc##1{\textcolor[rgb]{0.25,0.44,0.63}{##1}}}
\expandafter\def\csname PYG@tok@sd\endcsname{\let\PYG@it=\textit\def\PYG@tc##1{\textcolor[rgb]{0.25,0.44,0.63}{##1}}}

\def\PYGZbs{\char`\\}
\def\PYGZus{\char`\_}
\def\PYGZob{\char`\{}
\def\PYGZcb{\char`\}}
\def\PYGZca{\char`\^}
\def\PYGZam{\char`\&}
\def\PYGZlt{\char`\<}
\def\PYGZgt{\char`\>}
\def\PYGZsh{\char`\#}
\def\PYGZpc{\char`\%}
\def\PYGZdl{\char`\$}
\def\PYGZhy{\char`\-}
\def\PYGZsq{\char`\'}
\def\PYGZdq{\char`\"}
\def\PYGZti{\char`\~}
% for compatibility with earlier versions
\def\PYGZat{@}
\def\PYGZlb{[}
\def\PYGZrb{]}
\makeatother

\renewcommand\PYGZsq{\textquotesingle}

\begin{document}

\maketitle
\tableofcontents
\phantomsection\label{index::doc}


See the \href{http://faculty.washington.edu/rjl/classes/am583s2014/index.html}{Class Webpage}
for information on instructor, TA, office hours, etc.

Skip to... \DUrole{xref,std,std-ref}{technical\_topics} ... \DUrole{xref,std,std-ref}{applications}

\DUrole{xref,std,std-ref}{toc\_condensed}


\chapter{2013 Versions of some files}
\label{index:versions-of-some-files}\label{index:contents-of-class-notes-for-amath-483-583}
These pages may be referred to in Lecture Videos filmed in 2013,
particularly the homeworks and project.


\section{2013 versions of some files}
\label{2013/index:versions-of-some-files}\label{2013/index::doc}\label{2013/index:id1}
These pages may be referred to in Lecture Videos filmed in 2013,
particularly the homeworks and project.


\subsection{2013 Homework}
\label{2013/homeworks:homeworks}\label{2013/homeworks::doc}\label{2013/homeworks:homework}
\begin{notice}{warning}{Warning:}
These are the 2013 homework assignment.
See {\hyperref[homeworks:homeworks]{\crossref{\DUrole{std,std-ref}{Homework}}}} for this year's assignments.
\end{notice}

There will be 6 homeworks during the quarter with
tentative due dates listed below:
\begin{itemize}
\item {} 
\DUrole{xref,std,std-ref}{2013\_homework1}: Wednesday of Week 2, April 10

\item {} 
\DUrole{xref,std,std-ref}{2013\_homework2}: Wednesday of Week 3, April 17

\item {} 
\DUrole{xref,std,std-ref}{2013\_homework3}: Wednesday of Week 5, May 1

\item {} 
\DUrole{xref,std,std-ref}{2013\_homework4}: Wednesday of Week 6, May 8

\item {} 
\DUrole{xref,std,std-ref}{2013\_homework5}: Wednesday of Week 8, May 22

\item {} 
\DUrole{xref,std,std-ref}{2013\_homework6}: Friday of Week 9, May 29

\item {} 
\DUrole{xref,std,std-ref}{2013\_project}: Wednesday of Week 11, June 12

\end{itemize}


\subsection{Virtual Machine for this class {[}2013 version{]}}
\label{2013/vm::doc}\label{2013/vm:vm}\label{2013/vm:virtual-machine-for-this-class-2013-version}
We are using a wide variety of software in this class, much of which is
probably not found on your computer.  It is all open source software (see
\DUrole{xref,std,std-ref}{licences}) and links/instructions
can be found in the section {\hyperref[software_installation:software\string-installation]{\crossref{\DUrole{std,std-ref}{Downloading and installing software for this class}}}}.

An alternative, which many will find more convenient, is to download and
install the \phantomsection\label{2013/vm:id1}{\hyperref[biblio:virtualbox]{\crossref{{[}VirtualBox{]}}}} software and then download a Virtual Machine (VM)
that has been built specifically for this course.  VirtualBox will run this
machine, which will emulate a specific version of Linux that already has
installed all of the software packages that will be used in this course.

You can find the VM on the \href{http://faculty.washington.edu/rjl/classes/am583s2013/}{class
webpage}.
Note that the file is quite
large (approximately 750 MB compressed), and if possible you should
download it from on-campus to shorten the download time.  The TA's will also
have the VM on memory sticks for transferring.


\subsubsection{System requirements}
\label{2013/vm:system-requirements}
The VM is around 2.1 GB in size, uncompressed, and the virtual disk
image may expand to up to 8 GB, depending on how much data you store
in the VM.  Make sure you have enough free space available before
installing.  You can set how much RAM is available to the VM when
configuring it, but it is recommended that you give it at least 512
MB; since your computer must host your own operating system at the
same time, it is recommended that you have at least 1 GB of total RAM.


\subsubsection{Setting up the VM in VirtualBox}
\label{2013/vm:setting-up-the-vm-in-virtualbox}
Once you have downloaded and uncompressed the virtual machine disk
image from the class web site, you can set it up in VirtualBox, by
doing the following:
\begin{enumerate}
\item {} 
Start VirtualBox

\item {} 
Click the \emph{New} button near the upper-left corner

\item {} 
Click \emph{Next} at the starting page

\item {} 
Enter a name for the VM (put in whatever you like); for \emph{OS Type},
select ``Linux'', and for \emph{Version}, select ``Ubuntu''.  Click \emph{Next}.

\item {} 
Enter the amount of memory to give the VM, in megabytes.
512 MB is the recommended minimum.  Click \emph{Next}.

\item {} 
Click \emph{Use existing hard disk}, then click the folder icon next to
the disk list.  In the Virtual Media Manager that appears, click
\emph{Add}, then select the virtual machine disk image you downloaded
from the class web site.  Ignore the message about the recommended
size of the boot disk, and leave the box labeled ``Boot Hard Disk
(Primary Master)'' checked.  Once you have selected the disk image,
click \emph{Next}.

\item {} 
Review the summary VirtualBox gives you, then click \emph{Finish}.  Your
new virtual machine should appear on the left side of the VirtualBox
window.

\end{enumerate}


\subsubsection{Starting the VM}
\label{2013/vm:starting-the-vm}
Once you have configured the VM in VirtualBox, you can start it by
double-clicking it in the list of VM's on your system.  The virtual
machine will take a little time to start up; as it does, VirtualBox
will display a few messages explaining about mouse pointer and
keyboard capturing, which you should read.

After the VM has finished booting, it will present you with a login
screen; the login and password are both \code{uwhpsc}.  (We would have
liked to set up a VM with no password, but many things in Linux assume
you have one.)

Note that you will also need this password to quit the VM.


\subsubsection{Running programs}
\label{2013/vm:running-programs}
You can access the programs on the virtual machine through the Applications
Menu (the mouse on an \emph{X} symbol in the upper-left corner of the
screen), or by clicking the quick-launch icons next to the menu
button.  By default, you will have quick-launch icons for a command
prompt window (also known as a \emph{terminal window}), a text editor, and
a web browser.  After logging in for the first time, you should start
the web browser to make sure your network connection is working.


\subsubsection{Fixing networking issues}
\label{2013/vm:fixing-networking-issues}
When a Linux VM is moved to a new computer, it sometimes doesn't
realize that the previous computer's network adaptor is no longer
available.

Also, if you move your computer from one wireless network to another while
the VM is running, it may lose connection with the internet.

If this happens, it should be sufficient to shut down the VM (with the 0/1
button on the top right corner) and then restart it.
On shutdown, a script is automatically run that does the following, which in
earlier iterations of the VM had to be done manually...
\begin{quote}

\$ sudo rm /etc/udev/rules.d/70-persistent-net.rules
\end{quote}

This will remove the incorrect settings; Linux should then autodetect
and correctly configure the network interface it boots.


\subsubsection{Shutting down}
\label{2013/vm:shutting-down}
When you are done using the virtual machine, you can shut it down by
clicking the 0/1 button on the top-right corner of the VM.
You will need the password \titleref{uwhpsc}.


\subsubsection{Cutting and pasting}
\label{2013/vm:cutting-and-pasting}
If you want to cut text from one window in the VM and paste it into another,
you should be able to highlight the text and then type ctrl-c (or in a
terminal window, ctrl-shift-C, since ctrl-c is the interrupt signal). To
paste, type ctrl-v (or ctrl-shift-V in a terminal window).

If you want to be able to cut and paste between a window in the VM and a
window on your host machine, click on Machine from the main VitualBox menu
(or \titleref{Settings} in the Oracle VM VirtualBox Manager window), then click on
\titleref{General} and then \titleref{Advanced}.  Select \titleref{Bidirectional} from the \titleref{Shared
Clipboard} menu.


\subsubsection{Shared Folders}
\label{2013/vm:shared-folders}
If you create a file on the VM that you want to move to the file system of
the host machine, or vice versa, you can create a ``shared folder'' that is
seen by both.

First create a folder (i.e. directory) on the host machine, e.g. via:

\begin{Verbatim}[commandchars=\\\{\}]
\PYGZdl{} mkdir \PYGZti{}/uwhpsc\PYGZus{}shared
\end{Verbatim}

This creates a new subdirectory in your home directory on the host machine.

In the VirtualBox menu click on \titleref{Devices}, then click on
\titleref{Shared Folders}.  Click the + button on the right side and then type in the
full path to the folder you want to share under \titleref{Folder Path}, including the
folder name, and then the folder name itself under \titleref{Folder name}.
If you click on \titleref{Auto-mount} then this will be mounted every time you start
the VM.

Then click \titleref{OK} twice.

Then, in the VM (at the linux prompt), type the following commands:

\begin{Verbatim}[commandchars=\\\{\}]
sharename=uwhpsc\PYGZus{}shared   \PYGZsh{} or whatever name the folder has
sudo mkdir /mnt/\PYGZdl{}sharename
sudo chmod 777 /mnt/\PYGZdl{}sharename
sudo mount \PYGZhy{}t vboxsf \PYGZhy{}o uid=1000,gid=1000 \PYGZdl{}sharename /mnt/\PYGZdl{}sharename
\end{Verbatim}

You may need the password \titleref{uwhpsc} for the first \titleref{sudo} command.

The folder should now be found in the VM in \titleref{/mnt/\$sharename}.
(Note \titleref{\$sharename} is a variable set in the first command above.)

If auto-mounting doesn't work properly, you may need to repeat the final
\titleref{sudo mount ...} command  each time you start the VM.


\subsubsection{Enabling more processors}
\label{2013/vm:enabling-more-processors}
If you have a reasonably new computer with a multi-core
processor and want to be able to run parallel programs across multiple
cores, you can tell VirtualBox to allow the VM to use additional
cores.  To do this, open the VirtualBox
\emph{Settings}.  Under \emph{System}, click the \emph{Processor}
tab, then use the slider to set the number of processors the VM will
see.  Note that some older multi-core processors do not support the
necessary extensions for this, and on these machines you will only be
able to run the VM on a single core.


\subsubsection{Problems enabling multiple processors...}
\label{2013/vm:problems-enabling-multiple-processors}
Users may encounter several problems with enabling mutliple processors. Some users may not
be able to change this setting (it will be greyed out). Other users when may find no improved performance after enabling multiple processors. Still others may encounter an error such as:

\begin{Verbatim}[commandchars=\\\{\}]
\PYG{n}{VD}\PYG{p}{:} \PYG{n}{error} \PYG{n}{VERR\PYGZus{}NOT\PYGZus{}SUPPORTED}
\end{Verbatim}

All of these problems indicate that virtualization has not been enabled on your processors.

Fortunately this has an easy fix. You just have to enable virtualization in your BIOS
settings.

1. To  access the BIOS settings you must restart your computer and press a certain
button on startup. This button will depend on the company that manufactures your computer
(for example for Lenovo's it appears to be the f1 key).

2. Next you must locate a setting that will refer to either virtualization, VT, or VT-x.
Again the exact specifications will depend on the computer's manufacturer, however
it should be found in the Security section (or the Performance section if you are using a Dell).

3. Enable this setting,
then save and exit the bios settings.After your computer reboots you should be able to start the VM using multiple processors now.

4. If your BIOS does not have any settings like this it is possible that your BIOS is set up to hide this option from you, and you
may need to follow the advice here: \url{http://mathy.vanvoorden.be/blog/2010/01/enable-vt-x-on-dell-laptop/}

Note: Unfortunately some older hardware does not support virtualization, and so if these solutions don't work for you it may
be that this is the case for your processors. There also may be other possible problems...so don't be afraid to ask the TAs for help!


\subsubsection{Changing guest resolution/VM window size}
\label{2013/vm:changing-guest-resolution-vm-window-size}

\strong{See also:}


The section {\hyperref[vm:vm\string-additions]{\crossref{\DUrole{std,std-ref}{Guest Additions}}}}, which makes this easier.



It's possible that the size of the VM's window may be too large for
your display; resizing it in the normal way will result in not all of
the VM desktop being displayed, which may not be the ideal way to
work.  Alternately, if you are working on a high-resolution display,
you may want to \emph{increase} the size of the VM's desktop to take
advantage of it.  In either case, you can change the VM's display size
by going to the Applications menu in the upper-left corner, pointing to
\emph{Settings}, then clicking \emph{Display}.  Choose a resolution from the
drop-down list, then click \emph{Apply}.


\subsubsection{Setting the host key}
\label{2013/vm:setting-the-host-key}

\strong{See also:}


The section {\hyperref[vm:vm\string-additions]{\crossref{\DUrole{std,std-ref}{Guest Additions}}}}, which makes this easier.



When you click on the VM window, it will capture your mouse and future mouse
actions will apply to the windows in the VM.  To uncapture the mouse you
need to hit some control key, called the \emph{host key}.  It should give you a
message about this.  If it says the host key is Right Control, for example,
that means the Control key on the right side of your keyboard (it does \emph{not}
mean to click the right mouse button).

On some systems, the host key that transfers input focus between the
VM and the host operating system may be a key that you want to use in
the VM for other purposes.  To fix this, you can
change the host key in VirtualBox.  In the main VirtualBox window (not
the VM's window; in fact, the VM doesn't need to be running to do
this), go to the \emph{File} menu, then click \emph{Settings}.  Under \emph{Input},
click the box marked ``Host Key'', then press the key you want to use.


\subsubsection{Guest Additions}
\label{2013/vm:guest-additions}\label{2013/vm:vm-additions}
While we have installed the VirtualBox guest additions on the class
VM, the guest additions sometimes stop working when the VM is moved to
a different computer, so you may need to reinstall them.
Do the following so that the VM will automatically capture and uncapture
your mouse depending on whether you click in the VM window or outside it,
and to make it easier to resize the VM window to fit your display.
\begin{enumerate}
\item {} 
Boot the VM, and log in.

\item {} 
In the VirtualBox menu bar on your host system, select Devices --\textgreater{}
Install Guest Additions...  (Note: click on the window for the class
VM itself to get this menu, not on the main ``Sun VirtualBox'' window.)

\item {} 
A CD drive should appear on the VM's desktop, along with a popup
window.  (If it doesn't, see the additional instructions below.)
Select ``Allow Auto-Run'' in the popup window.  Then enter the
password you use to log in.

\item {} 
The Guest Additions will begin to install, and a window will appear,
displaying the progress of the installation.  When the installation is done,
the window will tell you to press `Enter' to close it.

\item {} 
Right-click the CD drive on the desktop, and select `Eject'.

\item {} 
Restart the VM.

\end{enumerate}

If step 3 doesn't work the first time, you might need to:
\begin{quote}
\begin{description}
\item[{Alternative Step 3:}] \leavevmode\begin{enumerate}
\item {} 
Reboot the VM.

\item {} 
Mount the CD image by right-clicking the CD drive icon, and clicking
`Mount'.

\item {} 
Double click the CD image to open it.

\item {} 
Double click `autorun.sh'.

\item {} 
Enter the VM password to install.

\end{enumerate}

\end{description}
\end{quote}


\subsubsection{How This Virtual Machine was made}
\label{2013/vm:how-this-virtual-machine-was-made}\begin{quote}
\begin{enumerate}
\item {} 
Download Ubuntu 12.04 PC (Intel x86) alternate install ISO from
\url{http://cdimage.ubuntu.com/xubuntu/releases/12.04.2/release/xubuntu-12.04.2-alternate-i386.iso}

\item {} 
Create a new virtual box

\item {} 
Set the system as Ubuntu

\item {} 
Use defualt options

\item {} 
After that double click on your new virtual machine...a dropdown
box should appear where you can select
your ubuntu iso

\item {} 
As you are installing...at the first menu hit F4 and install a
command line system

\item {} 
Let the install proceed following the instructions as given. On most
options the default answer will be appropriate.
When it comes time to format the hard drive, choose the manual option.
Format all the free space and set it as the mount
point. From the next list choose root (you dont need a swap space).

\item {} 
Install the necessary packages

\begin{Verbatim}[commandchars=\\\{\}]
\PYG{c}{\PYGZsh{} \PYGZdl{}UWHPSC/notes/install.sh}
\PYG{c}{\PYGZsh{} }
\PYG{n}{sudo} \PYG{n}{apt}\PYG{o}{\PYGZhy{}}\PYG{n}{get} \PYG{n}{update}
\PYG{n}{sudo} \PYG{n}{apt}\PYG{o}{\PYGZhy{}}\PYG{n}{get} \PYG{n}{upgrade}
\PYG{n}{sudo} \PYG{n}{apt}\PYG{o}{\PYGZhy{}}\PYG{n}{get} \PYG{n}{install} \PYG{n}{xfce4}
\PYG{n}{sudo} \PYG{n}{apt}\PYG{o}{\PYGZhy{}}\PYG{n}{get} \PYG{n}{install} \PYG{n}{jockey}\PYG{o}{\PYGZhy{}}\PYG{n}{gtk}
\PYG{n}{sudo} \PYG{n}{apt}\PYG{o}{\PYGZhy{}}\PYG{n}{get} \PYG{n}{install} \PYG{n}{xdm}
\PYG{n}{sudo} \PYG{n}{apt}\PYG{o}{\PYGZhy{}}\PYG{n}{get} \PYG{n}{install} \PYG{n}{ipython}
\PYG{n}{sudo} \PYG{n}{apt}\PYG{o}{\PYGZhy{}}\PYG{n}{get} \PYG{n}{install} \PYG{n}{python}\PYG{o}{\PYGZhy{}}\PYG{n}{numpy}
\PYG{n}{sudo} \PYG{n}{apt}\PYG{o}{\PYGZhy{}}\PYG{n}{get} \PYG{n}{install} \PYG{n}{python}\PYG{o}{\PYGZhy{}}\PYG{n}{scipy}
\PYG{n}{sudo} \PYG{n}{apt}\PYG{o}{\PYGZhy{}}\PYG{n}{get} \PYG{n}{install} \PYG{n}{python}\PYG{o}{\PYGZhy{}}\PYG{n}{matplotlib}
\PYG{n}{sudo} \PYG{n}{apt}\PYG{o}{\PYGZhy{}}\PYG{n}{get} \PYG{n}{install} \PYG{n}{python}\PYG{o}{\PYGZhy{}}\PYG{n}{dev}
\PYG{n}{sudo} \PYG{n}{apt}\PYG{o}{\PYGZhy{}}\PYG{n}{get} \PYG{n}{install} \PYG{n}{git}
\PYG{n}{sudo} \PYG{n}{apt}\PYG{o}{\PYGZhy{}}\PYG{n}{get} \PYG{n}{install} \PYG{n}{python}\PYG{o}{\PYGZhy{}}\PYG{n}{sphinx}
\PYG{n}{sudo} \PYG{n}{apt}\PYG{o}{\PYGZhy{}}\PYG{n}{get} \PYG{n}{install} \PYG{n}{gfortran}
\PYG{n}{sudo} \PYG{n}{apt}\PYG{o}{\PYGZhy{}}\PYG{n}{get} \PYG{n}{install} \PYG{n}{openmpi}\PYG{o}{\PYGZhy{}}\PYG{n+nb}{bin}
\PYG{n}{sudo} \PYG{n}{apt}\PYG{o}{\PYGZhy{}}\PYG{n}{get} \PYG{n}{install} \PYG{n}{liblapack}\PYG{o}{\PYGZhy{}}\PYG{n}{dev}
\PYG{n}{sudo} \PYG{n}{apt}\PYG{o}{\PYGZhy{}}\PYG{n}{get} \PYG{n}{install} \PYG{n}{thunar}
\PYG{n}{sudo} \PYG{n}{apt}\PYG{o}{\PYGZhy{}}\PYG{n}{get} \PYG{n}{install} \PYG{n}{xfce4}\PYG{o}{\PYGZhy{}}\PYG{n}{terminal}

\PYG{c}{\PYGZsh{} some packages not installed on the VM }
\PYG{c}{\PYGZsh{} that you might want to add:}

\PYG{n}{sudo} \PYG{n}{apt}\PYG{o}{\PYGZhy{}}\PYG{n}{get} \PYG{n}{install} \PYG{n}{gitk}               \PYG{c}{\PYGZsh{} to view git history}
\PYG{n}{sudo} \PYG{n}{apt}\PYG{o}{\PYGZhy{}}\PYG{n}{get} \PYG{n}{install} \PYG{n}{xxdiff}             \PYG{c}{\PYGZsh{} to compare two files}
\PYG{n}{sudo} \PYG{n}{apt}\PYG{o}{\PYGZhy{}}\PYG{n}{get} \PYG{n}{install} \PYG{n}{python}\PYG{o}{\PYGZhy{}}\PYG{n}{sympy}       \PYG{c}{\PYGZsh{} symbolic python}
\PYG{n}{sudo} \PYG{n}{apt}\PYG{o}{\PYGZhy{}}\PYG{n}{get} \PYG{n}{install} \PYG{n}{imagemagick}        \PYG{c}{\PYGZsh{} so you can \PYGZdq{}display plot.png\PYGZdq{}}


\PYG{n}{sudo} \PYG{n}{apt}\PYG{o}{\PYGZhy{}}\PYG{n}{get} \PYG{n}{install} \PYG{n}{python}\PYG{o}{\PYGZhy{}}\PYG{n}{setuptools}  \PYG{c}{\PYGZsh{} so easy\PYGZus{}install is available}
\PYG{n}{sudo} \PYG{n}{easy\PYGZus{}install} \PYG{n}{nose}                  \PYG{c}{\PYGZsh{} unit testing framework}
\PYG{n}{sudo} \PYG{n}{easy\PYGZus{}install} \PYG{n}{StarCluster}           \PYG{c}{\PYGZsh{} to help manage clusters on AWS}

\end{Verbatim}

\item {} 
To setup the login screen edit the file Xresources so that the
greeting line says.:

\begin{Verbatim}[commandchars=\\\{\}]
\PYG{n}{xlogin}\PYG{o}{*}\PYG{n}{greeting}\PYG{p}{:} \PYG{n}{Login} \PYG{o+ow}{and} \PYG{n}{Password} \PYG{n}{are} \PYG{n}{uwhpsc}
\end{Verbatim}

\item {} 
Create the file uwhpscvm-shutdown.:

\end{enumerate}
\begin{quote}
\end{quote}
\begin{enumerate}
\setcounter{enumi}{10}
\item {} 
Save it at.:

\begin{Verbatim}[commandchars=\\\{\}]
\PYG{o}{/}\PYG{n}{usr}\PYG{o}{/}\PYG{n}{local}\PYG{o}{/}\PYG{n+nb}{bin}\PYG{o}{/}\PYG{n}{uwhpscvm}\PYG{o}{\PYGZhy{}}\PYG{n}{shutdown}
\end{Verbatim}

\item {} 
Execute the following command command.:

\begin{Verbatim}[commandchars=\\\{\}]
\PYGZdl{} sudo chmod +x /usr/local/bin/uwhpscvm\PYGZhy{}shutdown
\end{Verbatim}

\item {} 
Right click on the upper panel and select add new items and choose
to add a new launcher.

\item {} 
Name the new launcher something like shutdown and in the command
blank copy the following line.:

\begin{Verbatim}[commandchars=\\\{\}]
\PYG{n}{gksudo} \PYG{o}{/}\PYG{n}{usr}\PYG{o}{/}\PYG{n}{local}\PYG{o}{/}\PYG{n+nb}{bin}\PYG{o}{/}\PYG{n}{uwhpscvm}\PYG{o}{\PYGZhy{}}\PYG{n}{shutdown}
\end{Verbatim}

\item {} 
Go to preferred applications and select Thunar for file managment
and the xfce4 terminal.

\item {} 
Run jockey-gtk and install guest-additions.

\item {} 
Go to Applications then Settings then screensaver and select
``disable screen saver'' mode

\item {} 
In the settings menu select the general settings and hit the advanced
tab. Here you can set the clipboard and drag
and drop to allow Host To Guest.

\item {} 
Shutdown the machine and then go to the main virtualbox screen.
Click on the virtualmachine and then hit the settings button.

\item {} 
After, in the system settings click on the processor tab. This may let
you allow the virtual machine to use more than one processor (depending
on your computer). Choose a setting somewhere in the green section of
the Processors slider.

\end{enumerate}
\end{quote}


\subsubsection{About the VM}
\label{2013/vm:about-the-vm}
The class virtual machine is running XUbuntu 12.04, a variant of Ubuntu
Linux (\url{http://www.ubuntu.com}), which itself is an offshoot of
Debian GNU/Linux (\url{http://www.debian.org}).  XUbuntu is a
stripped-down, simplified version of Ubuntu suitable for running on
smaller systems (or virtual machines); it runs the \emph{xfce4} desktop
environment.


\subsubsection{Further reading}
\label{2013/vm:further-reading}
\phantomsection\label{2013/vm:id2}{\hyperref[biblio:virtualbox]{\crossref{{[}VirtualBox{]}}}}
\phantomsection\label{2013/vm:id3}{\hyperref[biblio:virtualbox\string-documentation]{\crossref{{[}VirtualBox-documentation{]}}}}


\subsection{Amazon Web Services EC2 AMI {[}2013 version{]}}
\label{2013/aws:aws}\label{2013/aws::doc}\label{2013/aws:amazon-web-services-ec2-ami-2013-version}
\begin{notice}{warning}{Warning:}
This is the 2013 version.  See \DUrole{xref,std,std-ref}{2014\_aws} for updated instructions.
\end{notice}

We are using a wide variety of software in this class, much of which is
probably not found on your computer.  It is all open source software (see
\DUrole{xref,std,std-ref}{licences}) and links/instructions
can be found in the section {\hyperref[software_installation:software\string-installation]{\crossref{\DUrole{std,std-ref}{Downloading and installing software for this class}}}}.
You can also use the {\hyperref[vm:vm]{\crossref{\DUrole{std,std-ref}{Virtual Machine for this class {[}2014 Edition{]}}}}}.

Another alternative is to write and run your programs ``in the cloud''
using Amazon Web Services (AWS) Elastic Cloud Computing (EC2).
You can start up an ``instance'' (your own private computer, or so it appears)
that is configured using an Amazon Machine Image (AMI) that has been
configured with the Linux operating system and containing
all the software needed for this class.

You must first sign up for an account  on the \href{http://aws.amazon.com/}{AWS main page}.  For this you will need a credit
card, but note that with an account you can get 750 hours per month of
free ``micro instance'' usage in the
\href{http://aws.amazon.com/free/}{free usage tier}.
A micro instance is a single processor (that you will probably be sharing
with others) so it's not suitable for trying out parallel computing, but
should be just fine for much of the programming work in this class.

You can start up more powerful instances with 2 or more processors for a cost
starting at about 14.5 cents per hour (the High CPU Medium on-demand
instance).  See the \href{http://aws.amazon.com/ec2/\#pricing}{pricing guide}.

For general information and guides to getting started:
\begin{itemize}
\item {} 
\href{http://docs.amazonwebservices.com/AWSEC2/latest/GettingStartedGuide/}{Getting started with EC2},
with tutorial to lead you through an example.

\item {} 
\href{http://aws.amazon.com/ec2/faqs}{EC2 FAQ}.

\item {} 
\href{http://aws.amazon.com/ec2/\#pricing}{Pricing}.  Note: you are charged
per hour for hours (or fraction thereof) that your instance is in
\titleref{running} mode, regardless of whether the CPU is being used.

\item {} 
\href{http://aws.amazon.com/hpc-applications/}{High Performance Computing on AWS}
with instructions on starting a cluster instance.

\item {} 
\href{http://escience.washington.edu/get-help-now/get-started-amazon-web-services}{UW eScience information on AWS}.

\end{itemize}


\subsubsection{Launching an instance with the \emph{uwhpsc} AMI}
\label{2013/aws:launching-an-instance-with-the-uwhpsc-ami}

\paragraph{Quick way}
\label{2013/aws:quick-way}
Navigate your browser to
\url{https://console.aws.amazon.com/ec2/home?region=us-west-2\#launchAmi=ami-b47feb84}

Then you can skip the next section and proceed to \DUrole{xref,std,std-ref}{aws\_select\_size}.


\paragraph{Search for AMI}
\label{2013/aws:search-for-ami}
Going through this part may be useful if you want to see how to search for
other AMI's in the future.

Once you have an AWS account, sign in to the
\href{https://console.aws.amazon.com/ec2/}{management console}
and click on the
EC2 tab, and then select Region US West (Oregon) from the menu
at the top right of the page, next to your user name.

You should now be on the page
\url{https://console.aws.amazon.com/ec2/v2/home?region=us-west-2}.

Click on the big ``Launch Instance'' button.

Select the ``Classic Wizard'' and ``Continue''.

On the next page you will see a list of Amazon Machine Images (AMIs) that
you can select from if you want to start with a fresh VM.  For this class
you don't want any of these.  Instead click on the ``Community AMIs'' tab and
wait a while for a list to load.
Make sure Viewing ``All images'' is selected from the drop-down menu.

After the list of AMIs loads, type \titleref{uwhpsc} in the search bar.
Select this image.


\paragraph{Select size and security group}
\label{2013/aws:select-size-and-security-group}\label{2013/aws:aws-select-size}
On the next page you can select what sort of instance you wish to start (larger
instances cost more per hour). T1-micro is the the size you can run free (as
long as you only have one running).

Click \titleref{Continue} on the next few screens through the ``instance details''
and eventually you get to one that
asks for a key pair.  If you don't already have one, create a new one and
select it here.

Click \titleref{Continue} and you will get a screen to set Security Groups.  Select
the \titleref{quicklaunch-1} option.  On the next screen click \titleref{Launch}.


\subsubsection{Logging on to your instance}
\label{2013/aws:logging-on-to-your-instance}
Click \titleref{Close} on the  page that appears to
go back to the Management Console.  Click on \titleref{Instances} on the left menu
and you should see a list of instance you
have created, in your case only one.  If the status is not yet \titleref{running}
then wait until it is (click on the \titleref{Refresh} button if necessary).

\emph{Click on the instance} and information about it should appear at the bottom
of the screen. Scroll down until you find the \titleref{Public DNS} information

Go into the directory where your key pair is stored, in a file with a name
like \titleref{rjlkey.pem} and you should be able to \titleref{ssh} into your instance using
the name of the public DNS, with format like:

\begin{Verbatim}[commandchars=\\\{\}]
\PYGZdl{} ssh \PYGZhy{}X \PYGZhy{}i KEYPAIR\PYGZhy{}FILE  ubuntu@DNS
\end{Verbatim}

where KEYPAIR-FILE and DNS must be replaced by the appropriate
things, e.g. for the above example:

\begin{Verbatim}[commandchars=\\\{\}]
\PYGZdl{} ssh \PYGZhy{}X \PYGZhy{}i rjlkey.pem ubuntu@ec2\PYGZhy{}50\PYGZhy{}19\PYGZhy{}75\PYGZhy{}229.compute\PYGZhy{}1.amazonaws.com
\end{Verbatim}

Note:
\begin{itemize}
\item {} 
You must include \titleref{-i keypair-file}

\item {} 
You must log in as user ubuntu.

\item {} 
Including -X in the ssh command allows X window forwarding, so that if you
give a command that opens a new window (e.g. plotting in Python) it will
appear on your local screen.

\item {} 
See the section {\hyperref[ssh:ssh]{\crossref{\DUrole{std,std-ref}{Using ssh to connect to remote computers}}}} for tips if you are using a Mac or Windows
machine.
If you use Windows, see also the Amazon notes on using \emph{putty} found at
\url{http://docs.aws.amazon.com/AWSEC2/latest/UserGuide/putty.html}.

\end{itemize}

Once you have logged into your instance, you are on Ubuntu Linux that has
software needed for this class pre-installed.

Other software is easily installed using \titleref{apt-get install}, as described
in {\hyperref[software_installation:software\string-installation]{\crossref{\DUrole{std,std-ref}{Downloading and installing software for this class}}}}.


\subsubsection{Transferring files to/from your instance}
\label{2013/aws:transferring-files-to-from-your-instance}
You can use \titleref{scp} to transfer files between a running instance and
the computer on which the ssh key is stored.

From your computer (not from the instance):

\begin{Verbatim}[commandchars=\\\{\}]
\PYGZdl{} scp \PYGZhy{}i KEYPAIR\PYGZhy{}FILE FILE\PYGZhy{}TO\PYGZhy{}SEND ubuntu@DNS:REMOTE\PYGZhy{}DIRECTORY
\end{Verbatim}

where DNS is the public DNS of the instance and \titleref{REMOTE-DIRECTORY} is
the path (relative to home directory)
where you want the file to end up.  You can leave off
\titleref{:REMOTE-DIRECTORY} if you want it to end up in your home directory.

Going the other way, you can download a file from your instance to
your own computer via:

\begin{Verbatim}[commandchars=\\\{\}]
\PYGZdl{} scp \PYGZhy{}i KEYPAIR\PYGZhy{}FILE ubuntu@DNS:FILE\PYGZhy{}TO\PYGZhy{}GET .
\end{Verbatim}

to retrieve the file named \titleref{FILE-TO-GET} (which might include a path
relative to the home directory) into the current directory.


\subsubsection{Stopping your instance}
\label{2013/aws:stopping-your-instance}
Once you are done computing for the day, you will probably want to stop your
instance so you won't be charged while it's sitting idle.  You can do this
by selecting the instance from the Management Console / Instances, and then
select \titleref{Stop} from the \titleref{Instance Actions} menu.

You can restart it later and it will be in the same state you left it in.
But note that it will probably have a new Public DNS!


\subsubsection{Creating your own AMI}
\label{2013/aws:creating-your-own-ami}
If you add additional software and want to save a disk image of your
improved virtual machine (e.g. in order to launch additional images in the
future to run multiple jobs at once), simply click on \titleref{Create Image (EBS
AMI)} from the \titleref{Instance Actions} menu.


\subsubsection{Viewing webpages directly from your instance}
\label{2013/aws:viewing-webpages-directly-from-your-instance}
An apache webserver should already be running in your instance,
but to allow people (including yourself) to view
webpages you will need to adjust the security settings.  Go back to the
Management Console and click on \titleref{Security Groups} on the left menu.  Select
\titleref{quick-start-1} and then click on \titleref{Inbound}.  You should see a list of ports
that only lists 22 (SSH).  You want to add port 80 (HTTP).  Select HTTP from
the drop-down menu that says \titleref{Custom TCP Rule} and type 80 for the \titleref{Port
range}.  Then click \titleref{Add Rule} and \titleref{Apply Rule Changes}.

Now you should be able to point your browser to \titleref{http://DNS} where \titleref{DNS} is
replaced by the Public DNS name of your instance, the same as used for the
\titleref{ssh} command.  So for the example above, this would be

\begin{Verbatim}[commandchars=\\\{\}]
\PYG{n}{http}\PYG{p}{:}\PYG{o}{/}\PYG{o}{/}\PYG{n}{ec2}\PYG{o}{\PYGZhy{}}\PYG{l+m+mi}{50}\PYG{o}{\PYGZhy{}}\PYG{l+m+mi}{19}\PYG{o}{\PYGZhy{}}\PYG{l+m+mi}{75}\PYG{o}{\PYGZhy{}}\PYG{l+m+mf}{229.}\PYG{n}{compute}\PYG{o}{\PYGZhy{}}\PYG{l+m+mf}{1.}\PYG{n}{amazonaws}\PYG{o}{.}\PYG{n}{com}
\end{Verbatim}

The page being displayed can be found in \titleref{/var/www/index.html} on your
instance.  Any files you want to be visible on the web should be in
\titleref{/var/www}, or it is sufficient to have a link from this directory to where
they are located (created with the \titleref{ln -s} command in linux).

So, for example, if you do the following:

\begin{Verbatim}[commandchars=\\\{\}]
\PYGZdl{} cd \PYGZdl{}HOME
\PYGZdl{} mkdir public      \PYGZsh{} create a directory for posting things
\PYGZdl{} chmod 755 public  \PYGZsh{} make it readable by others
\PYGZdl{} sudo ln \PYGZhy{}s \PYGZdl{}HOME/public /var/www/public
\end{Verbatim}

then you can see the contents of your \$HOME/public directory at:

\begin{Verbatim}[commandchars=\\\{\}]
\PYG{n}{http}\PYG{p}{:}\PYG{o}{/}\PYG{o}{/}\PYG{n}{ec2}\PYG{o}{\PYGZhy{}}\PYG{l+m+mi}{50}\PYG{o}{\PYGZhy{}}\PYG{l+m+mi}{19}\PYG{o}{\PYGZhy{}}\PYG{l+m+mi}{75}\PYG{o}{\PYGZhy{}}\PYG{l+m+mf}{229.}\PYG{n}{compute}\PYG{o}{\PYGZhy{}}\PYG{l+m+mf}{1.}\PYG{n}{amazonaws}\PYG{o}{.}\PYG{n}{com}\PYG{o}{/}\PYG{n}{public}
\end{Verbatim}

Remember to change the DNS above to the right thing for your own instance!


\subsection{Git {[}2013 version{]}}
\label{2013/git:git-2013-version}\label{2013/git:git}\label{2013/git::doc}
\begin{notice}{warning}{Warning:}
This page is from 2013.
See {\hyperref[git:git]{\crossref{\DUrole{std,std-ref}{Git}}}} for this year's version.
\end{notice}

See {\hyperref[versioncontrol:versioncontrol]{\crossref{\DUrole{std,std-ref}{Version Control Software}}}} and the links there
for a more general discussion of the concepts.


\subsubsection{Instructions for cloning the class repository}
\label{2013/git:classgit}\label{2013/git:instructions-for-cloning-the-class-repository}
All of the materials for this class, including homeworks, sample programs,
and the webpages you are now
reading (or at least the \emph{.rst} files used to create them, see
{\hyperref[sphinx:sphinx]{\crossref{\DUrole{std,std-ref}{Sphinx documentation}}}}), are in a Git repository hosted at Bitbucket, located
at
\url{http://bitbucket.org/rjleveque/uwhpsc/}.
In addition to viewing the files via the link above, you can also view
changesets, issues, etc. (see {\hyperref[bitbucket:bitbucket]{\crossref{\DUrole{std,std-ref}{Bitbucket repositories: viewing changesets, issue tracking}}}}).

To obtain a copy, simply move to the directory where you want your copy to
reside (assumed to be your home directory below)
and then \emph{clone} the repository:

\begin{Verbatim}[commandchars=\\\{\}]
\PYGZdl{} cd
\PYGZdl{} git clone https://rjleveque@bitbucket.org/rjleveque/uwhpsc.git
\end{Verbatim}

Note the following:
\begin{itemize}
\item {} 
It is assumed you have \emph{git} installed, see
{\hyperref[software_installation:software\string-installation]{\crossref{\DUrole{std,std-ref}{Downloading and installing software for this class}}}}.

\item {} 
The clone statement will download the entire repository as a new
subdirectory called \emph{uwhpsc}, residing in your home directory.  If you
want \emph{uwhpsc} to reside elsewhere, you should first \emph{cd} to that
directory.

\end{itemize}

It will be useful to set a Unix environment variable (see {\hyperref[unix:env]{\crossref{\DUrole{std,std-ref}{Environment variables}}}}) called
\emph{UWHPSC} to refer to the directory you have just created.  Assuming you are
using the bash shell (see {\hyperref[unix:bash]{\crossref{\DUrole{std,std-ref}{The bash shell}}}}), and that you cloned uwhpsc
into your home directory, you can do this via:

\begin{Verbatim}[commandchars=\\\{\}]
\PYGZdl{} export UWHPSC=\PYGZdl{}HOME/uwhpsc
\end{Verbatim}

This uses the standard environment variable \emph{HOME}, which is the full path
to your home directory.

If you put it somewhere else, you can instead do:

\begin{Verbatim}[commandchars=\\\{\}]
\PYGZdl{} cd uwhpsc
\PYGZdl{} export UWHPSC={}`pwd{}`
\end{Verbatim}

The syntax
\emph{{}`pwd{}`} means to run the \emph{pwd} command (print working directory) and insert the
output of this command into the export command.

Type:

\begin{Verbatim}[commandchars=\\\{\}]
\PYGZdl{} printenv UWHPSC
\end{Verbatim}

to make sure \emph{UWHPSC} is set properly. This should print the full path to the
new directory.

If you log out and log in again later, you will find that this environment
variable is no longer set.  Or if you set it in one terminal window, it
will not be set in others.  To have it set automatically every time a new
bash shell is created (e.g. whenever a new terminal window is opened), add a
line of the form:

\begin{Verbatim}[commandchars=\\\{\}]
export UWHPSC=\PYGZdl{}HOME/uwhpsc
\end{Verbatim}

to your \emph{.bashrc} file.  (See {\hyperref[unix:bashrc]{\crossref{\DUrole{std,std-ref}{.bashrc file}}}}).  This assumes it is in your
home directory.  If not, you will have to add a line of the form:

\begin{Verbatim}[commandchars=\\\{\}]
\PYG{n}{export} \PYG{n}{UWHPSC}\PYG{o}{=}\PYG{n}{full}\PYG{o}{\PYGZhy{}}\PYG{n}{path}\PYG{o}{\PYGZhy{}}\PYG{n}{to}\PYG{o}{\PYGZhy{}}\PYG{n}{uwhpsc}
\end{Verbatim}

where the full path is what was returned by the \emph{printenv} statement above.


\subsubsection{Updating your clone}
\label{2013/git:uwhpsc-update}\label{2013/git:updating-your-clone}
The files in the class repository will change as the quarter progresses ---
new notes, sample programs, and homeworks will be added.  In order
to bring these changes over to your cloned copy, all you need to do is:

\begin{Verbatim}[commandchars=\\\{\}]
\PYGZdl{} cd \PYGZdl{}UWHPSC
\PYGZdl{} git fetch origin
\PYGZdl{} git merge origin/master
\end{Verbatim}

Of course this assumes that \emph{UWHPSC} has been properly set, see above.

The \emph{git fetch} command instructs \emph{git} to fetch any changes from \emph{origin},
which points to the remote bitbucket repository that you originally cloned
from.  In the merge command, \titleref{origin/master} refers to the master branch
in this repository (which
is the only branch that exists for this particular repository).
This merges any changes retrieved into the files in your current working
directory.

The last two command can be combined as:

\begin{Verbatim}[commandchars=\\\{\}]
\PYGZdl{} git pull origin master
\end{Verbatim}

or simply:

\begin{Verbatim}[commandchars=\\\{\}]
\PYGZdl{} git pull
\end{Verbatim}

since \titleref{origin} and \titleref{master} are the defaults.


\subsubsection{Creating your own Bitbucket repository}
\label{2013/git:mygit}\label{2013/git:creating-your-own-bitbucket-repository}
In addition to using the class repository, students in AMath 483/583 are
also required to create their own repository on Bitbucket.  It is possible
to use \emph{git} for your own work without creating a repository on a hosted
site such as Bitbucket (see \DUrole{xref,std,std-ref}{newgit} below), but there are several
reasons for this requirement:
\begin{itemize}
\item {} 
You should learn how to use Bitbucket for more than just pulling changes.

\item {} 
You will use this repository to ``submit'' your solutions to homeworks.
You will give the instructor and TA permission to clone your repository so
that we can grade the homework (others will not be able to clone or view it
unless you also give them permission).

\item {} 
It is recommended that after the class ends you
continue to use your repository as a way to back up your important work on
another computer (with all the benefits of version control too!).
At that point, of course, you can change the permissions so the
instructor and TA no longer have access.

\end{itemize}

Below are the instructions for creating your own repository.  Note that
this should be a \emph{private repository} so nobody can view or clone it unless
you grant permission.

Anyone can create a free private repository on Bitbucket.
Note that you can also create an unlimited number of public repositories
free at Bitbucket, which you might want to do for open source software
projects, or for classes like this one.

Follow these directions exactly.  Doing so is part of \DUrole{xref,std,std-ref}{homework1}.
We will clone your repository and check that \emph{testfile.txt} has been created
and modified as directed below.
\begin{enumerate}
\item {} 
On the machine you're working on:

\begin{Verbatim}[commandchars=\\\{\}]
\PYGZdl{} git config \PYGZhy{}\PYGZhy{}global user.name \PYGZdq{}Your Name\PYGZdq{}
\PYGZdl{} git config \PYGZhy{}\PYGZhy{}global user.email you@example.com
\end{Verbatim}

These will be used when you commit changes.
If you don't do this, you might get a warning message
the first time you try to commit.

\item {} 
Go to \url{http://bitbucket.org/} and click on ``Sign up now'' if you don't
already have an account.

\item {} 
Fill in the form, make sure you remember your username and password.

\item {} 
You should then be taken to your account.  Click on ``Create'' next
to ``Repositories''.

\item {} 
You should now see a form where you can specify the name of a repository
and a description.  The repository name need not be the same as your user
name (a single user might have several repositories).  For example, the class
repository is named \emph{uwhpsc}, owned by user \emph{rjleveque}.
To avoid confusion, you should probably not name your repository
\emph{uwhpsc}.

\item {} 
Make sure you click on ``Private'' at the bottom.  Also turn ``Issue
tracking'' and ``Wiki'' on if you wish to use these features.

\item {} 
Click on ``Create repository''.

\item {} 
You should now see a page with instructions on how to \emph{clone} your
(currently empty) repository.  In a Unix window, \emph{cd} to the directory where
you want your cloned copy to reside, and perform the clone by typing in
the clone command shown.  This will create a new directory with the same
name as the repository.

\item {} 
You should now be able to \emph{cd} into the directory this created.

\item {} 
To keep track of where this directory is and get to it easily in the
future, create an environment variable \emph{MYHPSC} from inside this directory
by:

\begin{Verbatim}[commandchars=\\\{\}]
\PYGZdl{} export MYHPSC={}`pwd{}`
\end{Verbatim}

See the discussion above in section {\hyperref[git:classgit]{\crossref{\DUrole{std,std-ref}{Instructions for cloning the class repository}}}} for what this does.  You
will also probably want to add a line to your \emph{.bashrc} file to define
\emph{MYHPSC} similar to the line added for \emph{UWHPSC}.

\item {} 
The directory you are now in will appear empty if you simply do:

\begin{Verbatim}[commandchars=\\\{\}]
\PYGZdl{} ls
\end{Verbatim}

But try:

\begin{Verbatim}[commandchars=\\\{\}]
\PYGZdl{} ls \PYGZhy{}a
./  ../  .git/
\end{Verbatim}

the \emph{-a} option causes \emph{ls} to list files starting with a dot, which are
normally suppressed.  See \DUrole{xref,std,std-ref}{ls} for a discussion of \emph{./} and \emph{../}.
The directory \emph{.git} is the directory that stores all the information
about the contents of this directory and a complete history of every file
and every change ever committed.  You shouldn't touch or modify the files in
this directory, they are used by \emph{git}.

\item {} 
Add a new file to your directory:

\begin{Verbatim}[commandchars=\\\{\}]
\PYGZdl{} cat \PYGZgt{} testfile.txt
This is a new file
with only two lines so far.
\PYGZca{}D
\end{Verbatim}

The Unix \emph{cat} command simply redirects everything you type on the
following lines into a file called \emph{testfile.txt}.  This goes on until
you type a \textless{}ctrl\textgreater{}-d (the 4th line in the example
above).  After typing \textless{}ctrl\textgreater{}-d you should get the Unix
prompt back.  Alternatively, you could create the file testfile.txt using
your favorite text editor (see {\hyperref[editors:editors]{\crossref{\DUrole{std,std-ref}{Text editors}}}}).

\item {} 
Type:

\begin{Verbatim}[commandchars=\\\{\}]
\PYGZdl{} git status \PYGZhy{}s
\end{Verbatim}

The response should be:

\begin{Verbatim}[commandchars=\\\{\}]
?? testfile.txt
\end{Verbatim}

The ?? means that this file is not under revision control.
The \emph{-s} flag results in this \emph{short} status list.  Leave it off for more
information.

To put the file under revision control, type:

\begin{Verbatim}[commandchars=\\\{\}]
\PYGZdl{} git add testfile.txt
\PYGZdl{} git status \PYGZhy{}s
A  testfile.txt
\end{Verbatim}

The A means it has been added.  However, at this point \emph{git} is not
we have not yet taken a \emph{snapshot} of this version of the file.
To do so, type:

\begin{Verbatim}[commandchars=\\\{\}]
\PYGZdl{} git commit \PYGZhy{}m \PYGZdq{}My first commit of a test file.\PYGZdq{}
\end{Verbatim}

The string following the -m is a comment about this commit that may help
you in general remember why you committed new or changed files.

You should get a response like:

\begin{Verbatim}[commandchars=\\\{\}]
\PYG{p}{[}\PYG{n}{master} \PYG{p}{(}\PYG{n}{root}\PYG{o}{\PYGZhy{}}\PYG{n}{commit}\PYG{p}{)} \PYG{l+m+mi}{28}\PYG{n}{a4da5}\PYG{p}{]} \PYG{n}{My} \PYG{n}{first} \PYG{n}{commit} \PYG{n}{of} \PYG{n}{a} \PYG{n}{test} \PYG{n}{file}\PYG{o}{.}
 \PYG{l+m+mi}{1} \PYG{n}{file} \PYG{n}{changed}\PYG{p}{,} \PYG{l+m+mi}{2} \PYG{n}{insertions}\PYG{p}{(}\PYG{o}{+}\PYG{p}{)}
 \PYG{n}{create} \PYG{n}{mode} \PYG{l+m+mi}{100644} \PYG{n}{testfile}\PYG{o}{.}\PYG{n}{txt}
\end{Verbatim}

We can now see the status of our directory via:

\begin{Verbatim}[commandchars=\\\{\}]
\PYGZdl{} git status
\PYGZsh{} On branch master
nothing to commit (working directory clean)
\end{Verbatim}

Alternatively, you can check the status of a single file with:

\begin{Verbatim}[commandchars=\\\{\}]
\PYGZdl{} git status testfile.txt
\end{Verbatim}

You can get a list of all the commits you have made (only one so far)
using:

\begin{Verbatim}[commandchars=\\\{\}]
\PYGZdl{} git log

commit 28a4da5a0deb04b32a0f2fd08f78e43d6bd9e9dd
Author: Randy LeVeque \PYGZlt{}rjl@ned\PYGZgt{}
Date:   Tue Mar 5 17:44:22 2013 \PYGZhy{}0800

    My first commit of a test file.
\end{Verbatim}

The number 28a4da5a0deb04b32a0f2fd08f78e43d6bd9e9dd above is the ``name''
of this commit and you can always get back to the state of your files as
of this commit by using this number.  You don't have to remember it, you
can use commands like \emph{git log} to find it later.

Yes, this is a number... it is a 40 digit hexadecimal number, meaning it
is in base 16 so in addition to 0, 1, 2, ..., 9, there are 6 more digits
a, b, c, d, e, f representing 10 through 15.  This number is almost
certainly guaranteed to be unique among all commits you will ever
do (or anyone has ever done, for that matter).  It is computed based
on the state of all the files in this snapshot as a \href{http://en.wikipedia.org/wiki/SHA-1}{SHA-1
Cryptographic hash function},
called a SHA-1 Hash for short.

Now let's modify this file:

\begin{Verbatim}[commandchars=\\\{\}]
\PYGZdl{} cat \PYGZgt{}\PYGZgt{} testfile.txt
Adding a third line
\PYGZca{}D
\end{Verbatim}

Here the \emph{\textgreater{}\textgreater{}} tells \emph{cat} that we want to add on to the end of an
existing file rather than creating a new one.  (Or you can edit the file
with your favorite editor and add this third line.)

Now try the following:

\begin{Verbatim}[commandchars=\\\{\}]
\PYGZdl{} git status \PYGZhy{}s
 M testfile.txt
\end{Verbatim}

The M indicates this file has been modified relative to the most recent
version that was committed.

To see what changes have been made, try:

\begin{Verbatim}[commandchars=\\\{\}]
\PYGZdl{} git diff testfile.txt
\end{Verbatim}

This will produce something like:

\begin{Verbatim}[commandchars=\\\{\}]
diff \PYGZhy{}\PYGZhy{}git a/testfile.txt b/testfile.txt
index d80ef00..fe42584 100644
\PYGZhy{}\PYGZhy{}\PYGZhy{} a/testfile.txt
+++ b/testfile.txt
@@ \PYGZhy{}1,2 +1,3 @@
 This is a new file
 with only two lines so far
+Adding a third line
\end{Verbatim}

The + in front of the last line shows that it was added.
The two lines before it are printed to show the context.  If the
file were longer, \emph{git diff}
would only print a few lines around any change to indicate the context.

Now let's try to commit this changed file:

\begin{Verbatim}[commandchars=\\\{\}]
\PYGZdl{} git commit \PYGZhy{}m \PYGZdq{}added a third line to the test file\PYGZdq{}
\end{Verbatim}

This will fail!  You should get a response like this:

\begin{Verbatim}[commandchars=\\\{\}]
\PYG{c}{\PYGZsh{} On branch master}
\PYG{c}{\PYGZsh{} Changes not staged for commit:}
\PYG{c}{\PYGZsh{}   (use \PYGZdq{}git add \PYGZlt{}file\PYGZgt{}...\PYGZdq{} to update what will be committed)}
\PYG{c}{\PYGZsh{}   (use \PYGZdq{}git checkout \PYGZhy{}\PYGZhy{} \PYGZlt{}file\PYGZgt{}...\PYGZdq{} to discard changes in working}
\PYG{c}{\PYGZsh{}   directory)}
\PYG{c}{\PYGZsh{}}
\PYG{c}{\PYGZsh{}   modified:   testfile.txt}
\PYG{c}{\PYGZsh{}}
\PYG{n}{no} \PYG{n}{changes} \PYG{n}{added} \PYG{n}{to} \PYG{n}{commit} \PYG{p}{(}\PYG{n}{use} \PYG{l+s}{\PYGZdq{}}\PYG{l+s}{git add}\PYG{l+s}{\PYGZdq{}} \PYG{o+ow}{and}\PYG{o}{/}\PYG{o+ow}{or} \PYG{l+s}{\PYGZdq{}}\PYG{l+s}{git commit \PYGZhy{}a}\PYG{l+s}{\PYGZdq{}}\PYG{p}{)}
\end{Verbatim}

\emph{git} is saying that the file \emph{testfile.txt} is modified but that no
files have been \textbf{staged} for this commit.

If you are used to Mercurial, \emph{git} has an extra level of complexity (but
also flexibility):  you can choose which modified files will be included
in the next commit.  Since we only have one file, there will not be a
commit unless we add this to the \textbf{index} of files staged for the next
commit:

\begin{Verbatim}[commandchars=\\\{\}]
\PYGZdl{} git add testfile.txt
\end{Verbatim}

Note that the status is now:

\begin{Verbatim}[commandchars=\\\{\}]
\PYGZdl{} git status \PYGZhy{}s
M  testfile.txt
\end{Verbatim}

This is different in a subtle way from what we saw before: The \emph{M} is
in the first column rather than the second, meaning it has been both
modified and staged.

We can get more information if we leave off the \emph{-s} flag:

\begin{Verbatim}[commandchars=\\\{\}]
\PYGZdl{} git status

\PYGZsh{} On branch master
\PYGZsh{} Changes to be committed:
\PYGZsh{}   (use \PYGZdq{}git reset HEAD \PYGZlt{}file\PYGZgt{}...\PYGZdq{} to unstage)
\PYGZsh{}
\PYGZsh{}   modified:   testfile.txt
\PYGZsh{}
\end{Verbatim}

Now \emph{testfile.txt} is on the index of files staged for the next commit.

Now we can do the commit:

\begin{Verbatim}[commandchars=\\\{\}]
\PYGZdl{} git commit \PYGZhy{}m \PYGZdq{}added a third line to the test file\PYGZdq{}

[master 51918d7] added a third line to the test file
 1 file changed, 1 insertion(+)
\end{Verbatim}

Try doing \emph{git log} now and you should see something like:

\begin{Verbatim}[commandchars=\\\{\}]
\PYG{n}{commit} \PYG{l+m+mi}{51918}\PYG{n}{d7ea4a63da6ab42b3c03f661cbc1a560815}
\PYG{n}{Author}\PYG{p}{:} \PYG{n}{Randy} \PYG{n}{LeVeque} \PYG{o}{\PYGZlt{}}\PYG{n}{rjl}\PYG{n+nd}{@ned}\PYG{o}{\PYGZgt{}}
\PYG{n}{Date}\PYG{p}{:}   \PYG{n}{Tue} \PYG{n}{Mar} \PYG{l+m+mi}{5} \PYG{l+m+mi}{18}\PYG{p}{:}\PYG{l+m+mi}{11}\PYG{p}{:}\PYG{l+m+mi}{34} \PYG{l+m+mi}{2013} \PYG{o}{\PYGZhy{}}\PYG{l+m+mi}{0800}

    \PYG{n}{added} \PYG{n}{a} \PYG{n}{third} \PYG{n}{line} \PYG{n}{to} \PYG{n}{the} \PYG{n}{test} \PYG{n}{file}

\PYG{n}{commit} \PYG{l+m+mi}{28}\PYG{n}{a4da5a0deb04b32a0f2fd08f78e43d6bd9e9dd}
\PYG{n}{Author}\PYG{p}{:} \PYG{n}{Randy} \PYG{n}{LeVeque} \PYG{o}{\PYGZlt{}}\PYG{n}{rjl}\PYG{n+nd}{@ned}\PYG{o}{\PYGZgt{}}
\PYG{n}{Date}\PYG{p}{:}   \PYG{n}{Tue} \PYG{n}{Mar} \PYG{l+m+mi}{5} \PYG{l+m+mi}{17}\PYG{p}{:}\PYG{l+m+mi}{44}\PYG{p}{:}\PYG{l+m+mi}{22} \PYG{l+m+mi}{2013} \PYG{o}{\PYGZhy{}}\PYG{l+m+mi}{0800}

    \PYG{n}{My} \PYG{n}{first} \PYG{n}{commit} \PYG{n}{of} \PYG{n}{a} \PYG{n}{test} \PYG{n}{file}\PYG{o}{.}
\end{Verbatim}

If you want to revert your working directory back to the first snapshot
you could do:

\begin{Verbatim}[commandchars=\\\{\}]
\PYGZdl{} git checkout 28a4da5a0de
Note: checking out \PYGZsq{}28a4da5a0de\PYGZsq{}.

You are in \PYGZsq{}detached HEAD\PYGZsq{} state. You can look around, make experimental
changes and commit them, and you can discard any commits you make in this
state without impacting any branches by performing another checkout.

HEAD is now at 28a4da5... My first commit of a test file.
\end{Verbatim}

Take a look at the file, it should be back to the state with only two
lines.

Note that you don't need the full SHA-1 hash code, the first few digits
are enough to uniquely identify it.

You can go back to the most recent version with:

\begin{Verbatim}[commandchars=\\\{\}]
\PYGZdl{} git checkout master
Switched to branch \PYGZsq{}master\PYGZsq{}
\end{Verbatim}

We won't discuss branches, but unless you create a new branch, the
default name for your main branch is \emph{master} and this \emph{checkout} command
just goes back to the most recent commit.

\item {} 
So far you have been using \emph{git} to keep track of changes in your own
directory, on your computer.  None of these changes have been seen by
Bitbucket, so if someone else cloned your repository from there, they
would not see \emph{testfile.txt}.

Now let's \emph{push} these changes back to the Bitbucket repository:

First do:

\begin{Verbatim}[commandchars=\\\{\}]
\PYGZdl{} git status
\end{Verbatim}

to make sure there are no changes that have not been committed.  This
should print nothing.

Now do:

\begin{Verbatim}[commandchars=\\\{\}]
\PYGZdl{} git push \PYGZhy{}u origin master
\end{Verbatim}

This will prompt for your Bitbucket password and should then print
something indicating that it has uploaded these two commits to
your bitbucket repository.

Not only has it copied the 1 file over, it has added both changesets, so
the entire history of your commits is now stored in the repository.  If
someone else clones the repository, they get the entire commit history
and could revert to any previous version, for example.

To push future commits to bitbucket, you should only need to do:

\begin{Verbatim}[commandchars=\\\{\}]
\PYGZdl{} git push
\end{Verbatim}

and by default it will push your master branch (the only branch you
have, probably) to \titleref{origin}, which is the shorthand name for the
place you originally cloned the repository from.  To see where this
actually points to:

\begin{Verbatim}[commandchars=\\\{\}]
\PYGZdl{} git remote \PYGZhy{}v
\end{Verbatim}

This lists all \titleref{remotes}.
By default there is only one, the place you cloned the repository from.
(Or none if you had created a new repository using \titleref{git init} rather
than cloning an existing one.)

\item {} 
Check that the file is in your Bitbucket repository:  Go back to that web
page for your repository and click on the  ``Source'' tab at the top.  It
should display the files in your repository and show \emph{testfile.txt}.

Now click on the ``Commits'' tab at the top.  It should show that you
made two commits and display the comments you added with the \emph{-m} flag
with each commit.

If you click on the hex-string for a commit, it will show the
\emph{change set} for this commit.  What you
should see is the file in its final state, with three lines.  The third
line should be highlighted in green, indicating that this line was added
in this changeset.  A line highlighted in red would indicate a line deleted
in this changeset.  (See also {\hyperref[bitbucket:bitbucket]{\crossref{\DUrole{std,std-ref}{Bitbucket repositories: viewing changesets, issue tracking}}}}.)

\end{enumerate}

This is enough for now!

\DUrole{xref,std,std-ref}{homework1} instructs you to add some additional files to the Bitbucket
repository.

Feel free to experiment further with your repository at this point.


\subsubsection{Further reading}
\label{2013/git:further-reading}
Next see {\hyperref[bitbucket:bitbucket]{\crossref{\DUrole{std,std-ref}{Bitbucket repositories: viewing changesets, issue tracking}}}} and/or {\hyperref[git_more:git\string-more]{\crossref{\DUrole{std,std-ref}{More about Git}}}}.

Remember that you can get help with \emph{git} commands by typing, e.g.:

\begin{Verbatim}[commandchars=\\\{\}]
\PYGZdl{} git help
\PYGZdl{} git help diff  \PYGZsh{} or any other specific command name
\end{Verbatim}

Each command has lots of options!

{\hyperref[biblio:biblio\string-git]{\crossref{\DUrole{std,std-ref}{Git references}}}} contains references to other sources of information and
tutorials.


\chapter{Course materials -- 2014 Edition}
\label{index:course-materials}\label{index:course-materials-2014-edition}

\section{About these notes -- important disclaimers}
\label{about:about-these-notes-important-disclaimers}\label{about:about}\label{about::doc}
These note on high performance scientific computing are being developed for
the course \href{http://www.amath.washington.edu/}{Applied Mathematics} \href{http://www.amath.washington.edu/courses/583-spring-2014/index.html}{483/583}
at the \href{http://www.washington.edu}{University of Washington}, Spring Quarter, 2014.

They are very much a work in progress.  Many pages are not yet here and the
ones that are will mostly be modified and supplemented as the quarter
progresses.

It is not intended to be a complete textbook on the subject, by any means.
The goal is to get the student started with a few key concepts and
techniques and then encourage further reading elsewhere.
So it is a collection of brief introductions to various important topics
with pointers to books, websites, and other references for more details.
See the {\hyperref[biblio:biblio]{\crossref{\DUrole{std,std-ref}{Bibliography and further reading}}}} for more references and other suggested readings.

There are many pointers to Wikipedia pages sprinkled through the notes and
in the bibliography, simply because these pages often give a good overview
of issues without getting into too much detail.  They are not necessarily
definitive sources of accurate information.

These notes are mostly written in Sphinx (see {\hyperref[sphinx:sphinx]{\crossref{\DUrole{std,std-ref}{Sphinx documentation}}}}) and the input
files are available using git (see {\hyperref[git:classgit]{\crossref{\DUrole{std,std-ref}{Instructions for cloning the class repository}}}}).


\subsection{License}
\label{about:id2}\label{about:license}
These notes are being made freely available and are released under the
\href{http://creativecommons.org/}{Creative Commons} \href{http://creativecommons.org/licenses/by/3.0/}{CC BY license}.
This means that you are welcome to use them and quote from them
as long as you give appropriate attribution.

Of course you should always do this with any material you find on the web or
elsewhere, provided you are allowed to reuse it at all.

You should also give some thought to licensing issues whenever you post your
own work on the web, including computer code.
Whether or not you want others to be able to make use
of it for various purposes, make your intentions known.


\section{Class Format}
\label{class_format:class-format}\label{class_format::doc}\label{class_format:id1}
This class is being taught differently in 2014 than in past years.
\begin{itemize}
\item {} 
Students will be required to watch 3 hours of lectures a week that will be
available on the web (videos recorded in Spring, 2013). These follow the
slides and lecture notes from last year, but some new material will also be
developed.

Enrolled students can find the videos on the
\href{https://canvas.uw.edu/courses/893991/assignments/syllabus}{Canvas Syllabus Page} or on
your Canvas Calendar on the corresponding date.  Click on the Lecture
number link to find the videos, slides, and the associated quiz.

\item {} 
There is a short quiz associated with each lecture to encourage you
to watch the lectures.
All the quizzes can also be found on the \href{https://canvas.uw.edu/courses/893991/quizzes}{Canvas Quizzes Page}.

\item {} 
Each quiz is worth 5 points (with the lowest 3
scores dropped, out of the 28 Lecture Quizzes).
The quiz for all three lectures of each calendar week must be completed
by 11:00pm PDT on the following Monday.

\item {} 
Slides to accompany all the lectures, along with a summary of the topics
covered in each lecture, can be found at
{\color{red}\bfseries{}{}`http://faculty.washington.edu/rjl/classes/am583s2013/slides/{}`\_}

\item {} 
For on-campus students, the class meets for a lab session
T-Th, 2:30 - 3:20 in OUG 136. This
is an \href{http://www.lib.washington.edu/ougl/learning-spaces/active-learning-classrooms}{Active Learning Classroom}
and class time will be used primarily for
working together with other students on exercises designed to illustrate and
reinforce the material. Some additional new material will also be presented.
For on-campus students, attendance will be required at these sessions,
and each lab will contain some work that is turned in for grading.
(With the lowest two scores dropped in case you miss a lab or two.)

\item {} 
Undergraduates should enroll in AMath 483 and graduate students in 583. The
lectures and classroom sessions are identical, but the 583 course will have
some additional and/or more advanced assignments.

\item {} 
Section 583B is for online Masters degree students only and is not
available to on-campus students. Students in this section will view the same
video lectures as on-campus students. In addition, parts of the T-Th
lab sessions will be taped and available to watch, along with the
exercises tackled by students in these sessions.

\item {} 
Homework and a final project will consist primarily of programming
assignments.   There will be 4 homeworks worth 100 points each and
the project will be worth 200 points.

\end{itemize}


\section{Overview and syllabus}
\label{outline::doc}\label{outline:outline}\label{outline:overview-and-syllabus}
This course will cover a large number of topics in a somewhat superficial
manner.  The main goals are to:
\begin{itemize}
\item {} 
Introduce a number of concepts related to machine architecture,
programming languages, etc. that necessary to understand if one plans to
write computer programs beyond simple exercises, where it is important
that they run efficiently.

\item {} 
Introduce a variety of software tools that are useful to programmers
working in scientific computing.

\item {} 
Gain a bit of hands on experience with these tools so that at the end
of the quarter students will be able to continue working with them and
be in a good position to learn more on their own from available
resources.

\end{itemize}


\subsection{Some topics}
\label{outline:some-topics}
Many topics will be covered in a non-linear fashion, jumping around between
topics to tie things together.
\begin{itemize}
\item {} 
Using the Virtual Machine

\item {} 
Unix / Linux

\item {} 
Version control systems

\item {} 
Using Git and Bitbucket.

\item {} 
Basic Python

\item {} 
Ipython and the IPython notebook

\item {} 
NumPy and Scipy

\item {} 
Debugging Python

\item {} 
Compiled vs. interpreted languages

\item {} 
Introduction to Fortran 90

\item {} 
Makefiles

\item {} 
Computer architecture: CPU, memory access, cache hierachy, pipelining

\item {} 
Optimizing Fortran

\item {} 
BLAS and LAPACK routines

\item {} 
Parallel computing

\item {} 
OpenMP with Fortran

\item {} 
MPI with Fortran

\item {} 
Parallel Python

\item {} 
Graphics and visualization

\item {} 
I/O, Binary output

\item {} 
Mixed language programming

\end{itemize}


\section{Notes to accompany lab sessions}
\label{labs/index:labs}\label{labs/index:notes-to-accompany-lab-sessions}\label{labs/index::doc}

\subsection{Lab 1: Tuesday April 1, 2014}
\label{labs/lab1:lab1}\label{labs/lab1::doc}\label{labs/lab1:lab-1-tuesday-april-1-2014}

\subsubsection{Brief overview of the class structure}
\label{labs/lab1:brief-overview-of-the-class-structure}
See:
\begin{itemize}
\item {} 
{\hyperref[class_format:class\string-format]{\crossref{\DUrole{std,std-ref}{Class Format}}}}

\item {} 
{\hyperref[computing_options:computing\string-options]{\crossref{\DUrole{std,std-ref}{Computing Options {[}2014 version{]}}}}}

\item {} 
\href{https://canvas.uw.edu/courses/893991}{Canvas page}

\item {} 
\href{https://canvas.uw.edu/courses/893991/wiki/honor-code}{honor code}

\end{itemize}


\subsubsection{Things to try in groups:}
\label{labs/lab1:things-to-try-in-groups}
Please work together in groups of 2 or 3.
\begin{itemize}
\item {} 
Log on to \href{https://cloud.sagemath.com/}{SageMathCloud}.
See \DUrole{xref,std,std-ref}{smc}.

\item {} 
Create a new project and share it with your group members.

\item {} 
Open a terminal window.  If people in your group aren't comfortable with
Unix, try out commands like \emph{ls}, \emph{pwd}, \emph{mkdir}, \emph{mv}, \emph{cp}, etc.
See {\hyperref[unix:unix]{\crossref{\DUrole{std,std-ref}{Unix, Linux, and OS X}}}}.

\item {} 
Create and edit a new file named \emph{hello.py} by typing:

\begin{Verbatim}[commandchars=\\\{\}]
\PYGZdl{} open hello.py
\end{Verbatim}

at the terminal prompt \titleref{\$}.

\item {} 
Add one line to this file and then click Save:

\begin{Verbatim}[commandchars=\\\{\}]
\PYG{n+nb}{print} \PYG{l+s}{\PYGZdq{}}\PYG{l+s}{Hello World!}\PYG{l+s}{\PYGZdq{}}
\end{Verbatim}

\item {} 
In the terminal window, type:

\begin{Verbatim}[commandchars=\\\{\}]
\PYGZdl{} python hello.py
\end{Verbatim}

In should print out Hello World!

\item {} 
Edit the file to add the lines:

\begin{Verbatim}[commandchars=\\\{\}]
\PYG{k+kn}{import} \PYG{n+nn}{matplotlib}        \PYG{c}{\PYGZsh{} python plotting package}
\PYG{n}{matplotlib}\PYG{o}{.}\PYG{n}{use}\PYG{p}{(}\PYG{l+s}{\PYGZdq{}}\PYG{l+s}{Agg}\PYG{l+s}{\PYGZdq{}}\PYG{p}{)}    \PYG{c}{\PYGZsh{} so plot commands work in this script}
\PYG{k+kn}{from} \PYG{n+nn}{pylab} \PYG{k}{import} \PYG{o}{*}      \PYG{c}{\PYGZsh{} imports lots of things like linspace, sin, plot}

\PYG{n}{x} \PYG{o}{=} \PYG{n}{linspace}\PYG{p}{(}\PYG{l+m+mi}{0}\PYG{p}{,}\PYG{l+m+mi}{1}\PYG{p}{,}\PYG{l+m+mi}{1001}\PYG{p}{)}
\PYG{n}{y} \PYG{o}{=} \PYG{n}{sin}\PYG{p}{(}\PYG{l+m+mi}{10} \PYG{o}{*} \PYG{n}{pi} \PYG{o}{*} \PYG{n}{x}\PYG{o}{*}\PYG{o}{*}\PYG{l+m+mi}{2}\PYG{p}{)}
\PYG{n}{plot}\PYG{p}{(}\PYG{n}{x}\PYG{p}{,}\PYG{n}{y}\PYG{p}{)}

\PYG{n}{fname} \PYG{o}{=} \PYG{l+s}{\PYGZdq{}}\PYG{l+s}{myplot.png}\PYG{l+s}{\PYGZdq{}}
\PYG{n}{savefig}\PYG{p}{(}\PYG{n}{fname}\PYG{p}{)}
\PYG{n+nb}{print} \PYG{l+s}{\PYGZdq{}}\PYG{l+s}{Saved }\PYG{l+s}{\PYGZdq{}}\PYG{p}{,}\PYG{n}{fname}
\end{Verbatim}

\item {} 
Save the file and rerun it at the terminal prompt, then view the file:

\begin{Verbatim}[commandchars=\\\{\}]
\PYGZdl{} python hello.py
\PYGZdl{} open myplot.png
\end{Verbatim}

\item {} 
Open an IPython Notebook from the New menu.

\item {} 
Follow the instructions for ``making a plot in IPython'' from the page
\DUrole{xref,std,std-ref}{smc}.

\end{itemize}


\subsection{Lab 8: Thursday April 24, 2014}
\label{labs/lab8:lab8}\label{labs/lab8::doc}\label{labs/lab8:lab-8-thursday-april-24-2014}
The code for this lab can be found in \titleref{\$UWHPSC/labs/lab8}.

Lab started with a demo of using compiler flags and \titleref{gdb} to debug Fortran
code, using the code in  \titleref{\$UWHPSC/labs/lab8/demo}.

Running it without any compiler flags gives no error by there is a \titleref{NaN}
value in the results:

\begin{Verbatim}[commandchars=\\\{\}]
\PYGZdl{} cd \PYGZdl{}UWHPSC/labs/lab8
\PYGZdl{} make run

 The max value of y is   0.99973335466585400
 x(501) is    0.0000000000000000
 y(501) is                        NaN
\end{Verbatim}

Running it with compiler flags to catch floating point exceptions:

\begin{Verbatim}[commandchars=\\\{\}]
\PYGZdl{} make debug
make: *** [debug] Floating point exception: 8
\end{Verbatim}

Once it's compiled with the flags specified in the Makefile for the \titleref{debug}
target, the debugger \titleref{gdb} can be used to run
the code and figure out where it died:

\begin{Verbatim}[commandchars=\\\{\}]
\PYGZdl{} gdb ./a.out
(gdb) run

Program received signal SIGFPE, Arithmetic exception.
0x0000000000400a6d in demo () at demo.f90:12
12          y(j) = sin(x(j)) / x(j)

(gdb) p j
\PYGZdl{}1 = 501

(gdb) p x(j)
\PYGZdl{}2 = 0
\end{Verbatim}

Many commands are available in \titleref{gdb}, see for example
\href{http://www.sourceware.org/gdb/current/onlinedocs/gdb.html}{this documentation}.

\textbf{Note:} On the Mac with the Mavericks OS, \titleref{gdb} has been replaced by
\titleref{lldb}.  See \url{http://lldb.llvm.org/index.html} for more information.

The page \href{http://lldb.llvm.org/lldb-gdb.html}{GDB TO LLDB COMMAND MAP}
gives a good summary of both \titleref{gdb} and \titleref{lldb} commands and the relation
between them.


\subsubsection{Debugging compile-time errors}
\label{labs/lab8:debugging-compile-time-errors}
The code in \titleref{\$UWHPSC/labs/lab8/problem1} does not compile.  See if you can
find and fix all the errors.  See \titleref{\$UWHPSC/labs/lab8/problem1b}
for a corrected version (and see the \titleref{README.txt} file in that directory for
some comments).


\subsubsection{Debugging run-time errors}
\label{labs/lab8:debugging-run-time-errors}
The code \titleref{\$UWHPSC/labs/lab8/problem2/array1.f90} does not run properly.
See if you can find and fix all the errors.  See
\titleref{\$UWHPSC/labs/lab8/problem2/array1b.f90} for a corrected version and use
\titleref{diff} to see the differences.


\subsection{Lab 9: Tuesday April 29, 2014}
\label{labs/lab9:lab9}\label{labs/lab9::doc}\label{labs/lab9:lab-9-tuesday-april-29-2014}

\subsubsection{Programming problem}
\label{labs/lab9:programming-problem}
Work on this in groups!

Write a Fortran program to compute the roots of a quadratic equation and
determine the absolute and relative error in the roots computed.  The
program should do the following:
\begin{itemize}
\item {} 
Prompt the user and then read in the desired roots \titleref{x1true} and \titleref{x2true}
(See the example below).

\item {} 
Determine the coefficients \titleref{a, b, c} so that the quadratic equation
\(a x^2 + bx + c =0\) has the desired roots (you can set \titleref{a=1}).

\item {} 
Using \titleref{a,b,c}, compute the roots \titleref{x1} and \titleref{x2} by using the quadratic
formula.

\item {} 
Print out the ``true'' and computed values for each root along with the
absolute and relative error in each.

\end{itemize}

So you should be able to do something like this:

\begin{Verbatim}[commandchars=\\\{\}]
\PYGZdl{} gfortran quadratic.f90
\PYGZdl{} ./a.out
 input x1true, x2true:
2.5, 3

Coefficients:  a =     0.100000E+01  b =    \PYGZhy{}0.550000E+01  c = 0.750000E+01

Root x1  computed:  0.250000000000000E+01  true:  0.250000000000000E+01
         absolute error:     0.000000E+00  relative error:     0.000000E+00

Root x2  computed:  0.300000000000000E+01  true:  0.300000000000000E+01
         absolute error:     0.000000E+00  relative error:     0.000000E+00
\end{Verbatim}

Don't worry too much about the formatting but you might want to print out 15
digits in the computed roots.

You might want to assume the values are entered with \titleref{x1true \textless{}= x2true} so
you know which root from the quadratic equations goes with which original
value. (And print out an informative error message otherwise.)


\subsubsection{Test it out}
\label{labs/lab9:test-it-out}
Test a variety of values to see that it's working.

Once it's working on reasonable values, try the following:
\begin{itemize}
\item {} 
\titleref{x1 = 1e-12,  x2 = 2}

\item {} 
\titleref{x1 = -2, x2 = 1e-12}

\end{itemize}

In each case you should find that one root is computed accurately
but the other root has a large relative error (few digits of accuracy).

Figure out why ``catastrophic cancellation'' is the problem.


\subsubsection{Improvements}
\label{labs/lab9:improvements}\begin{itemize}
\item {} 
Improve the code by noticing that if one root is calculated accurately, the
other root can be calculated from the fact that \titleref{x1 * x2 = c}.

\item {} 
Remove the assumption that \titleref{x1true \textless{}= x2true}.

\end{itemize}


\subsection{Lab 10: Tuesday May 1, 2014}
\label{labs/lab10:lab-10-tuesday-may-1-2014}\label{labs/lab10::doc}\label{labs/lab10:lab10}

\subsubsection{Programming problem}
\label{labs/lab10:programming-problem}
Work on this in groups!
\begin{enumerate}
\item {} 
The OpenMP code \titleref{\$UWHPSC/labs/lab10/array\_omp.f90} contains some bugs.
Find the bugs and fix them so that it runs and gives output like this:

\begin{Verbatim}[commandchars=\\\{\}]
\PYGZdl{} gfortran \PYGZhy{}fopenmp array\PYGZus{}omp.f90
\PYGZdl{} ./a.out
 nthreads =            6
 b and bt should be equal
 b=
    0.270000D+02
    0.330000D+02
    0.390000D+02
    0.450000D+02
    0.510000D+02
 bt=
    0.270000D+02
    0.330000D+02
    0.390000D+02
    0.450000D+02
    0.510000D+02
\end{Verbatim}

\item {} 
If \(A\) is an \(n \times n\) matrix and \(x\) is a vector of
length \(n\), then \(x^TAx\) is a scalar, a ``quadratic form''
since it is the sum of terms of the form \(a_{ij}x_ix_j\) that are
quadratic in the elements of \(x\) .

Write an OpenMP code to compute this for a given matrix and vector.  Write
out the matrix-vector multiplies as loops and use ``omp parallel do'' loops to
compute first the vector \(Ax\) and then the inner product of this with
the vector \titleref{x}.  Test your code using the \(10 \times 10\) identity matrix
for \(A\) and \(x_i = i\), in which case the correct answer can be
determined to be 385 from the formula
\begin{quote}

\(\sum_{i=1}^n i^2 = \frac{n(n+1)(2n+1)}6.\)
\end{quote}

\item {} 
There is a quiz for this lab.

\end{enumerate}


\subsection{Lab 11: Tuesday May 6, 2014}
\label{labs/lab11:lab-11-tuesday-may-6-2014}\label{labs/lab11::doc}\label{labs/lab11:lab11}\begin{itemize}
\item {} 
Note that there are several example codes in the class repository that
might be useful, e.g.
\begin{itemize}
\item {} 
\titleref{\$UWHPSC/codes/openmp}

\item {} 
\titleref{\$UWHPSC/codes/lapack/random}

\item {} 
\titleref{\$UWHPSC/2013/homeworks}

\item {} 
\titleref{\$UWHPSC/2013/solutions}

\end{itemize}

\item {} 
Discussion of random number generators.
See \titleref{UWHPSC/labs/lab11}

\item {} 
Questions about OpenMP?

\end{itemize}


\subsubsection{Programming problem}
\label{labs/lab11:programming-problem}
Work on this in groups!
\begin{enumerate}
\item {} 
Write a Fortran program to do the following:
\begin{itemize}
\item {} 
input \titleref{seed} and \titleref{n} from the command line

\item {} 
seed the random number generator using \titleref{init\_random\_seed} from the
\titleref{random\_util.90} module,

\item {} 
generate an array \titleref{x} of \titleref{n} random numbers

\item {} 
compute the mean of these values:  the sum of all elements of \titleref{x}
divided by \titleref{n}.  Do this with a \titleref{do} loop.

\end{itemize}

Since \titleref{random\_number} produces numbers that should be uniformly
distributed between 0 and 1, the mean should be approximately 0.5
for large \titleref{n}.  It can also be shown that for a uniform
distribution, the difference between the mean of a sample of \titleref{n} numbers
and the true mean of the distribution should decay to zero like
\(1/\sqrt{n}\) as \(n\longrightarrow\infty\).  Do you observe this?

\item {} 
Modify your code to use OpenMP by using an \titleref{omp parallel do} loop
with a suitable reduction to compute the mean.

\item {} 
There is a quiz for this lab.

\end{enumerate}


\subsection{Lab 12: Thursday May 8, 2014}
\label{labs/lab12::doc}\label{labs/lab12:lab12}\label{labs/lab12:lab-12-thursday-may-8-2014}

\subsubsection{Programming problem}
\label{labs/lab12:programming-problem}
Work on this in groups!
\begin{enumerate}
\item {} 
In Lab 11 you worked on a program to compute the mean of n random
numbers.  A sample solution can be found at \titleref{\$UWHPSC/labs/lab11/mean.f90}.

Write a Fortran program that runs over different values of \titleref{n},
and for each \titleref{n} generates a vector \titleref{x} containing \titleref{n} random numbers
and then computes the mean of these.  Also compute the fraction of the
numbers that lie in the first quartile (the fraction of \titleref{x(i)} values
that are between 0 and 0.25) and the fraction that lie in the fourth
quartile (between 0.75 and 1.0).  Since the \titleref{random\_number} routine
returns numbers uniformly distributed between 0 and 1, we expect  each of
these fractions to be about 0.25.

Use OpenMP to make the loop on \titleref{i} from 1 to \titleref{n} into a parallel do loop.

Running this code should give something like this if you take as the \titleref{n}
values \(n = 10^k\) for \(k=2,3,\ldots,8\):

\begin{Verbatim}[commandchars=\\\{\}]
 \PYG{n}{Number} \PYG{n}{of} \PYG{n}{threads}\PYG{p}{:}            \PYG{l+m+mi}{2}
 \PYG{n+nb}{input} \PYG{n}{seed}
\PYG{l+m+mi}{12345}
 \PYG{n}{seed1} \PYG{k}{for} \PYG{n}{random} \PYG{n}{number} \PYG{n}{generator}\PYG{p}{:}       \PYG{l+m+mi}{12345}

       \PYG{n}{n}          \PYG{n}{mean}        \PYG{n}{quartile} \PYG{l+m+mi}{1}    \PYG{n}{quartile} \PYG{l+m+mi}{4}
       \PYG{l+m+mi}{100}     \PYG{l+m+mf}{0.51902466}     \PYG{l+m+mf}{0.22000000}     \PYG{l+m+mf}{0.24000000}
      \PYG{l+m+mi}{1000}     \PYG{l+m+mf}{0.47476778}     \PYG{l+m+mf}{0.27800000}     \PYG{l+m+mf}{0.22500000}
     \PYG{l+m+mi}{10000}     \PYG{l+m+mf}{0.49606601}     \PYG{l+m+mf}{0.25670000}     \PYG{l+m+mf}{0.25190000}
    \PYG{l+m+mi}{100000}     \PYG{l+m+mf}{0.50121669}     \PYG{l+m+mf}{0.24815000}     \PYG{l+m+mf}{0.25130000}
   \PYG{l+m+mi}{1000000}     \PYG{l+m+mf}{0.50001034}     \PYG{l+m+mf}{0.24986300}     \PYG{l+m+mf}{0.24979800}
  \PYG{l+m+mi}{10000000}     \PYG{l+m+mf}{0.49998532}     \PYG{l+m+mf}{0.24994350}     \PYG{l+m+mf}{0.24992770}
 \PYG{l+m+mi}{100000000}     \PYG{l+m+mf}{0.49995944}     \PYG{l+m+mf}{0.25003764}     \PYG{l+m+mf}{0.24995608}
\end{Verbatim}

\item {} 
If you haven't already, study the code in
\titleref{\$UWHPSC/codes/openmp/pisum2.f90}
and make sure you understand how this coarse grain parallelism works.
Discuss with your neighbors.

\item {} 
If you have time, try to follow this model to make your code that
computes the mean and quartiles work in a similar manner, where you
break up the different values of \titleref{n} to be tested between different
threads, e.g. in the above example one thread would take the
first three values of \titleref{n} and the second thread would take the final
two values of \titleref{n}.

\item {} 
Discuss with your neighbors whether this is a sensible way to try
to use two threads on this problem.

\item {} 
There is a quiz for this lab.

\item {} 
Sample solutions can now be found in \titleref{\$UWHPSC/labs/lab12}.

\end{enumerate}


\subsection{Lab 13: Tuesday May 13, 2014}
\label{labs/lab13:lab13}\label{labs/lab13::doc}\label{labs/lab13:lab-13-tuesday-may-13-2014}

\subsubsection{Announcements}
\label{labs/lab13:announcements}\begin{itemize}
\item {} 
Homework 4 will be posted soon and due on Tuesday, May 27 at 11pm PDT.

\item {} 
IPython 2.0 was released in April and has some cool new features.
It is \textbf{not} available yet on SageMathCloud.
See {\hyperref[ipython_notebook:ipython\string-notebook]{\crossref{\DUrole{std,std-ref}{IPython\_notebook}}}} for some references.

\end{itemize}


\subsubsection{Demos}
\label{labs/lab13:demos}\begin{itemize}
\item {} 
IPython 2.0 with interactive widgets:

The notebook in \titleref{\$UWHPSC/labs/lab13/widgets\_demo.ipynb} illustrates this.
(Does not run on SMC.)

\item {} 
Plots you can zoom in on in IPython notebooks:
\href{https://github.com/jakevdp/mpld3}{mpld3}.
The notebook \titleref{\$UWHPSC/labs/lab13/mpld3\_demo.ipynb}
can be run on SMC.

\end{itemize}


\subsubsection{Gamblers' Ruin problem}
\label{labs/lab13:gamblers-ruin-problem}\begin{itemize}
\item {} 
The notebook \titleref{\$UWHPSC/labs/lab13/GamblersRuin.ipynb}
illustrates in Python a problem that you will tackle in Homework 4
using Fortran with OpenMP and MPI.

\end{itemize}

\textbf{There is a quiz for Lab 13}


\subsection{Lab 14: Thursday May 15, 2014}
\label{labs/lab14:lab14}\label{labs/lab14::doc}\label{labs/lab14:lab-14-thursday-may-15-2014}

\subsubsection{Demos}
\label{labs/lab14:demos}\begin{itemize}
\item {} 
Python dictionaries: \titleref{\$UWHPSC/labs/lab14/dictionary\_demo.ipynb}
illustrates this.

\item {} 
Towers of Hanoi: \titleref{\$UWHPSC/labs/lab14/Towers\_of\_Hanoi.ipynb}
illustrates this.

\item {} 
One solution to the lab problem: \titleref{\$UWHPSC/labs/lab14/Towers\_of\_Hanoi\_plots.ipynb}
You can view it at
\url{http://nbviewer.ipython.org/url/faculty.washington.edu/rjl/notebooks/Towers\_of\_Hanoi\_plots.ipynb}

\end{itemize}

\textbf{There is a quiz for Lab 14}


\subsection{Lab 15: Tuesday May 20, 2014}
\label{labs/lab15:lab15}\label{labs/lab15::doc}\label{labs/lab15:lab-15-tuesday-may-20-2014}
Install JSAnimation:  See {\hyperref[animation:animation]{\crossref{\DUrole{std,std-ref}{Animation in Python}}}}.


\subsubsection{Demos}
\label{labs/lab15:demos}\begin{itemize}
\item {} 
\titleref{\$UWHPSC/labs/lab15/JSAnimation\_demo.ipynb}

View at
\url{http://nbviewer.ipython.org/url/faculty.washington.edu/rjl/notebooks/JSAnimation\_demo.ipynb}

\item {} 
\titleref{\$UWHPSC/labs/lab15/WavePacket.ipynb}

View at
\url{http://nbviewer.ipython.org/url/faculty.washington.edu/rjl/notebooks/WavePacket.ipynb}

\item {} 
\titleref{\$UWHPSC/labs/lab15/WavePacket.py}  Script version

\item {} 
\titleref{\$UWHPSC/labs/lab15/Towers\_of\_Hanoi\_animation.ipynb}

View at
\url{http://nbviewer.ipython.org/url/faculty.washington.edu/rjl/notebooks/Towers\_of\_Hanoi\_animation.ipynb}

\end{itemize}

Note: These use \titleref{\$UWHPSC/labs/lab15/JSAnimation\_frametools.py}.


\subsubsection{Problem to solve}
\label{labs/lab15:problem-to-solve}
Create an animation similar to
\url{http://faculty.washington.edu/rjl/classes/am583s2014/Square.html}.

Hints:
\begin{itemize}
\item {} 
The following matplotlib commands may be useful:

\begin{Verbatim}[commandchars=\\\{\}]
\PYG{n}{fill}\PYG{p}{(}\PYG{n}{x}\PYG{p}{,}\PYG{n}{y}\PYG{p}{,}\PYG{l+s}{\PYGZsq{}}\PYG{l+s}{b}\PYG{l+s}{\PYGZsq{}}\PYG{p}{)}   \PYG{c}{\PYGZsh{} fill polygon specified by arrays x and y with blue}
\PYG{n}{axis}\PYG{p}{(}\PYG{l+s}{\PYGZsq{}}\PYG{l+s}{scaled}\PYG{l+s}{\PYGZsq{}}\PYG{p}{)}   \PYG{c}{\PYGZsh{} scale x and y axes the same way}
\end{Verbatim}

\item {} 
Recall that to rotate a point \$(x,y)\$ through angle \$theta\$ you can
can compute

\(\hat x = \cos(\theta)x + \sin(\theta)y\)

\(\hat y = -\sin(\theta)x + \cos(\theta)y\)

\end{itemize}

\textbf{There is no quiz for Lab 15}


\subsection{Lab 16: Thursday May 22, 2014}
\label{labs/lab16:lab16}\label{labs/lab16::doc}\label{labs/lab16:lab-16-thursday-may-22-2014}

\subsubsection{Problem to solve}
\label{labs/lab16:problem-to-solve}\begin{itemize}
\item {} 
Adapt the program \titleref{\$UWHPSC/codes/lapack/randomsys3.f90} to use a specific matrix A in place of
the random matrix used in the original code.  The matrix to use is the Hilbert matrix defined by
\begin{quote}

\(a_{i,j} = \frac{1}{i+j-1}\)
\end{quote}

This is a notorious matrix since it is always nonsingular but is very ill-conditioned even for
moderately small values of \titleref{n}.

For more discussion of this matrix, and a formula for how the condition number grows with \titleref{n},
see this \href{http://blogs.mathworks.com/cleve/2013/02/02/hilbert-matrices/\#73083b00-1b97-4570-a516-31796a031dc4}{Cleve's Corner blog post}.

\item {} 
Note that in order to create an executable for your program, in the linking step you will need
to make sure \titleref{gfortran} also links in the BLAS and LAPACK library.  See the \titleref{LFLAGS} set in
{\color{red}\bfseries{}{}`}\$UWHPSC/codes/lapack/Makefile for the arguments you need to add to the linking step.

\item {} 
Instead of using the \titleref{random\_number} subroutine to generate a random \titleref{x} for checking the
relative error, as is done in \titleref{\$UWHPSC/codes/lapack/randomsys3.f90}, try taking \titleref{x} to be
a vector of all 1's.  (And as in the original code compute \(b = Ax\) usint \titleref{matmul} and
then solve the system to recover \titleref{x}.)  Print out the computed \titleref{x} as well as computing the
relative error in the 1-norm as in the original code.  How well does it do?  How does the
accuracy relate to the condition number?

\item {} 
You might want to look at the \href{http://www.netlib.no/netlib/lapack/double/dgecon.f}{dgecon documentation}.

\item {} 
Try different values of \titleref{n} with your program to see if it gives the expected behavior.
Note that the LAPACK function \titleref{dgecon} does not compute the exact condition number but only
estimates it.  Also note that the program estimates the 1-norm condition  number, while the
approximate formula is for the 2-norm condition number (but they grow in a similar exponential
fashion).

\end{itemize}

\textbf{If you have time to do more...}
\begin{itemize}
\item {} 
Modify your code by creating a Fortran function \titleref{hilbert\_condition} that returns the condition number
estimate for a given value of \titleref{n}.

Then write a main program that loops over \titleref{n} from 1 to 20, computes the estimate for each \titleref{n},
and writes a text file with two columns \titleref{n} and the estimate.  These statements might be
useful:

\begin{Verbatim}[commandchars=\\\{\}]
   open(21, file=\PYGZsq{}cond.txt\PYGZsq{},status=\PYGZsq{}unknown\PYGZsq{})

   do n=1,20
       cond = hilbert\PYGZus{}condition(n)
       ! print *, \PYGZdq{}cond = \PYGZdq{},cond
       write(21, 210) n,cond
210    format(i4,e16.6)
       enddo
\end{Verbatim}

\item {} 
The text file produced should be readable by the Python script
\titleref{\$UWHPSC/labs/lab16/plot\_cond.py}, which plots the results on a logarithmic scale, along with
what the formula predicts.

\item {} 
For the function version you do not need to solve a linear system, so you don't need to call
\titleref{dgesv}, but you do need to compute the LU factorization of A before calling \titleref{dgecon}.
The could be done by calling \titleref{dgetrf} instead of \titleref{dgesv}.
You might want to look at the \href{http://www.netlib.no/netlib/lapack/double/dgetrf.f}{dgetrf documentation}.

\end{itemize}

\textbf{There is quiz for Lab 16}


\subsection{Lab 17: Tuesday May 27, 2014}
\label{labs/lab17:lab17}\label{labs/lab17:lab-17-tuesday-may-27-2014}\label{labs/lab17::doc}

\subsubsection{Announcements}
\label{labs/lab17:announcements}\begin{itemize}
\item {} 
Due tonight: Homework 4, lecture quizzes, lab quiz.

\item {} 
Office hours today:

\end{itemize}
\begin{itemize}
\item {} 
In Odegaard after class, and then in Lewis 328 until 5pm,

\item {} 
Online 5-6pm, look for announcement on Canvas, but may use GoToMeeting

\end{itemize}
\begin{itemize}
\item {} 
Final project:  Part 1 will be posted soon, discussed on Thursday.

Due Wednesday, June 11.  (But don't wait to the last minute!)

\end{itemize}


\subsubsection{Today's lab}
\label{labs/lab17:today-s-lab}\begin{itemize}
\item {} 
Go through the notebook \titleref{\$UWHPSC/labs/lab17/Tridiagonal.ipynb}, also
visible at
\url{http://nbviewer.ipython.org/url/faculty.washington.edu/rjl/notebooks/Tridiagonal.ipynb}.

\item {} 
Work in pairs on writing a Fortran program to solve the same
tridiagonal linear system considered in the notebook with:

\begin{Verbatim}[commandchars=\\\{\}]
\PYG{n}{A} \PYG{o}{=}
  \PYG{p}{[}\PYG{p}{[}   \PYG{l+m+mf}{1.}  \PYG{l+m+mf}{200.}    \PYG{l+m+mf}{0.}    \PYG{l+m+mf}{0.}    \PYG{l+m+mf}{0.}\PYG{p}{]}
   \PYG{p}{[}  \PYG{l+m+mf}{10.}    \PYG{l+m+mf}{2.}  \PYG{l+m+mf}{300.}    \PYG{l+m+mf}{0.}    \PYG{l+m+mf}{0.}\PYG{p}{]}
   \PYG{p}{[}   \PYG{l+m+mf}{0.}   \PYG{l+m+mf}{20.}    \PYG{l+m+mf}{3.}  \PYG{l+m+mf}{400.}    \PYG{l+m+mf}{0.}\PYG{p}{]}
   \PYG{p}{[}   \PYG{l+m+mf}{0.}    \PYG{l+m+mf}{0.}   \PYG{l+m+mf}{30.}    \PYG{l+m+mf}{4.}  \PYG{l+m+mf}{500.}\PYG{p}{]}
   \PYG{p}{[}   \PYG{l+m+mf}{0.}    \PYG{l+m+mf}{0.}    \PYG{l+m+mf}{0.}   \PYG{l+m+mf}{40.}    \PYG{l+m+mf}{5.}\PYG{p}{]}\PYG{p}{]}

\PYG{n}{b} \PYG{o}{=}
  \PYG{p}{[} \PYG{l+m+mf}{201.}  \PYG{l+m+mf}{312.}  \PYG{l+m+mf}{423.}  \PYG{l+m+mf}{534.}   \PYG{l+m+mf}{45.}\PYG{p}{]}
\end{Verbatim}

\item {} 
Use the LAPACK surbroutine \titleref{dgtsv}.  See the
documentation at \url{http://www.netlib.no/netlib/lapack/double/dgtsv.f}.

\item {} 
\textbf{There is quiz for Lab 17}

\end{itemize}


\subsection{Lab 18: Thursday May 29, 2014}
\label{labs/lab18:lab-18-thursday-may-29-2014}\label{labs/lab18::doc}\label{labs/lab18:lab18}\begin{itemize}
\item {} 
We will go through the notebook \titleref{\$UWHPSC/homeworks/project/BVP.ipynb}, also
visible at
\url{http://nbviewer.ipython.org/url/faculty.washington.edu/rjl/notebooks/BVP.ipynb}.
This outlines a recursive \href{https://www.google.com/search?q=domain+decomposition\&rlz=1C5CHFA\_enUS534US534\&espv=2\&source=lnms\&tbm=isch\&sa=X\&ei=R4GHU4uFKI-XyAT4t4Bo\&ved=0CAYQ\_AUoAQ\&biw=1440\&bih=779}{domain decomposition}  approach to solving a
boundary value problem.  Part 1 of the project is to convert this into
Fortran with OpenMP.

\item {} 
Working in pairs, copy this notebook to \titleref{BVP2.ipynb} and modify it to solve
a \href{http://en.wikipedia.org/wiki/Helmholtz\_equation}{Helmholtz equation}
(in one dimension) of the form:
\begin{quote}

\(u''(x) + k^2 u(x) = -f(x)\)
\end{quote}

on the interval \(0<x<1\) with specified boundary conditions.

As an exact solution, consider the case \(f(x)=0\) in which case
the general solution to \(u''(x) = -k^2 u(x)\) is
\(u(x) = c_1 \cos(kx) + c_2 \sin(kx)\).

The boundary value problem has a unique exact solution for any boundary
values \(u(0)\) and \(u(1)\) provided that \(k\) is not an
integer multiple of \(\pi\).  (Insert \(x=0\) and \(x=1\) into
the general solution and determine \(c_1\) and \(c_2\) so that the
boundary conditions are satisfied.)

You might try values such as:

\begin{Verbatim}[commandchars=\\\{\}]
\PYG{n}{k} \PYG{o}{=} \PYG{l+m+mf}{15.}
\PYG{n}{u\PYGZus{}left} \PYG{o}{=} \PYG{l+m+mf}{2.}
\PYG{n}{u\PYGZus{}right} \PYG{o}{=} \PYG{o}{\PYGZhy{}}\PYG{l+m+mf}{1.}
\end{Verbatim}

You will need to use at least 40 or 50 grid points to get a solution that
looks at all reasonable.
If you make \(k\) larger, the solution will be more oscillatory and
you will need even more grid points to get a reasonable approximation.

\item {} 
Work through as much of the notebook as you can, adjusting things to
solve the Helmholtz equation.  The main objective is to work through the
notebook and understand what is being done.

Some tips:
\begin{itemize}
\item {} 
Add another parameter \titleref{k} to the \titleref{solve\_BVP\_*} functions.

\item {} 
In setting up the tridiagonal matrix in \titleref{solve\_BVP\_direct}, you will need
to modify the diagonal terms for the difference equation that
approximates the Helmholtz equation,

\(\frac{U_{i-1} - 2U_i + U_{i+1}}{\Delta x^2} + k^2 U_i = -f(x_i)\)

This gives the linear system to be solved:

\(U_{i-1} + (k^2\Delta x^2 - 2) U_i + U_{i+1} = -\Delta x^2 f(x_i)\)

along with the boundary conditions.

\item {} 
If you get to the divide-and-conquer approach, you will have to modify
the equation for the mismatch to take into account the modification to
the linear system being solved.

\end{itemize}

\item {} 
There is now a sample solution in the repository, visible at
\url{http://nbviewer.ipython.org/url/faculty.washington.edu/rjl/notebooks/BVP\_helmholtz.ipynb}.

\item {} 
\textbf{There is quiz for Lab 18}

\end{itemize}


\subsection{Lab 19: Tuesday June 3, 2014}
\label{labs/lab19::doc}\label{labs/lab19:lab19}\label{labs/lab19:lab-19-tuesday-june-3-2014}\begin{itemize}
\item {} 
We will go through the notebook \titleref{\$UWHPSC/homeworks/project/Heat\_Equation.ipynb}, also
visible at
\url{http://nbviewer.ipython.org/url/faculty.washington.edu/rjl/notebooks/Heat\_Equation.ipynb}.

This notebook gives a brief introduction to the heat equation and
two numerical methods for its solution, an explicit method and the
more stable implicit Crank-Nicolson method.

\end{itemize}


\subsubsection{Some things to try}
\label{labs/lab19:some-things-to-try}\begin{itemize}
\item {} 
You might want to make a copy of this notebook before you start playing
with it.

\item {} 
Experiment with different initial conditions for the heat equation.

\item {} 
Create an animation (in the notebook) of the numerical solution to the
heat equation along with the true solution.

\item {} 
The Crank-Nicolson method is \emph{second order accurate}: the error should
go to zero like \({\cal O}(\Delta t^2 + \Delta x^2)\) as the grid is
refined.  So increasing both \titleref{n} (the number of spatial points) and
\titleref{nsteps} (the number of time steps) by a factor of 2 should reduce the
error by a factor of 4.  Test this out.

\item {} 
Compute or look up the Fourier sine series for some interesting function
and try this as initial conditions for the heat equation.  Compare the
true solution with the numerical solution (where the ``true solution'' might
be estimated by adding up a large but finite number of terms in the
Fourier series).

\item {} 
Try using Sympy to compute the coefficients in the Fourier sine series.

\end{itemize}

\textbf{There is quiz for Lab 19}


\subsection{Lab 20: Thursday June 5, 2014}
\label{labs/lab20:lab-20-thursday-june-5-2014}\label{labs/lab20:lab20}\label{labs/lab20::doc}\begin{itemize}
\item {} 
The directory \titleref{\$UWHPSC/labs/lab20} contains a Fortran code
\titleref{fourier\_sum.f90} that computes terms in the Fourier sine series
for the function \(f(x) = e^x\) and prints them out to the file
\titleref{frames.txt} along with partial sums of the series.
It also creates a second file \titleref{xf.txt} that contains the values of \titleref{x}
and \titleref{f(x)} on the finite grid where the terms and sum are computed.

The Python script \titleref{animate.py} reads in these two files and produces
an animation, which can be viewed at
\url{http://faculty.washington.edu/rjl/FourierSum.html}.

We will go through these codes.

\item {} 
You may have to install JSAnimation on your SMC project, see
{\hyperref[animation:animation]{\crossref{\DUrole{std,std-ref}{Animation in Python}}}}.

\end{itemize}


\subsubsection{Work on in pairs:}
\label{labs/lab20:work-on-in-pairs}\begin{itemize}
\item {} 
The subdirectory \titleref{\$UWHPSC/labs/lab20/gamblers\_ruin} contains a modified
version of the solution to Homework 4, part 2,  and does a single random
walk.  The \titleref{gamblers.f90} code has been modified to print \titleref{n1, n2} each
step to the file \titleref{walk.txt}.

Produce a script \titleref{animate.py} and modify \titleref{Makefile1} so that you can do:

\begin{Verbatim}[commandchars=\\\{\}]
\PYGZdl{} make movie \PYGZhy{}f Makefile1 n1=10 n2=10 seed=1234
\end{Verbatim}

and generate a movie like the one shown at
\url{http://faculty.washington.edu/rjl/RandomWalk.html}.

\item {} 
Don't worry about the Makefile at first, get \titleref{animate.py} working.

\item {} 
For testing, you might want to use smaller \titleref{n1} and \titleref{n2} so there are
fewer \titleref{png} files to generate.

\end{itemize}

\textbf{There is quiz for Lab 20}

See also the codes in \titleref{\$UWHPSC/labs}.


\section{Homework}
\label{homeworks:homeworks}\label{homeworks::doc}\label{homeworks:homework}
Homeworks assigned in 2013 can be found in {\hyperref[2013/homeworks:homeworks]{\crossref{\DUrole{std,std-ref}{2013 Homework}}}} and the
code they refer to can be found in \titleref{\$UWHPSC/2013/homeworks}.  Some
solutions to these homeworks are in the directory \titleref{\$UWHPSC/2013/solutions}.
These homeworks may be referenced in some of the lectures.


\subsection{2014 Homework assignments}
\label{homeworks:homework-assignments}
There will be 4 homeworks during the quarter with
tentative due dates listed below:
\begin{itemize}
\item {} 
\DUrole{xref,std,std-ref}{homework1}: Thursday of Week 2, April 10

\item {} 
\DUrole{xref,std,std-ref}{homework2}: Thursday of Week 4, April 24

\item {} 
\DUrole{xref,std,std-ref}{homework3}: Thursday of Week 6, May 8 -- \DUrole{xref,std,std-ref}{homework3\_solution}

\item {} 
\DUrole{xref,std,std-ref}{homework4}: Tuesday of Week 9, May 27 -- \DUrole{xref,std,std-ref}{homework4\_solution}

\item {} 
\DUrole{xref,std,std-ref}{project}: Wednesday of Week 11, June 11 -- \DUrole{xref,std,std-ref}{project\_hints}

\end{itemize}

There will be a ``final project'' tentatively due on Wednesday, June 11.
This will count twice as much as a homework and will be similar in
spirit but longer and tying together several things from the quarter
into a more interesting computing assignment.


\section{Computing Options {[}2014 version{]}}
\label{computing_options:computing-options-2014-version}\label{computing_options::doc}\label{computing_options:computing-options}
All of the software we will use this year is open source, so in principle
you can install it all on your own computer.  See {\hyperref[software_installation:software\string-installation]{\crossref{\DUrole{std,std-ref}{Downloading and installing software for this class}}}}
for some tips on doing so.

However, there are several reasons you might want to use a different
computing environment for this class:
\begin{itemize}
\item {} 
To avoid having to install many packages yourself,

\item {} 
To make sure you have the same computing environment as fellow students
and the TAs,

\item {} 
To have access to a multi-core machine if your own computer is has a
single processor, since much of the course material concerns parallel computing.

\item {} 
To work together during lab sessions.

\end{itemize}


\subsection{SageMathCloud}
\label{computing_options:options-smc}\label{computing_options:sagemathcloud}
This is the recommended computing platform  and what we will mostly use
during the T-Th lab sessions.  SageMathCloud is a freely
available cloud computing resource developed by the Sage Team, led by
Prof. William Stein in the UW Mathematics Department.  You can easily create
an account at \href{https://cloud.sagemath.com/}{SageMathCloud}.

You should create an account using your UW email address \titleref{netid@uw.edu}.
This will make it easiest for us to add you as a collaborator on projects.

All of the software needed this quarter is installed on SageMathCloud.

Use of SageMathCloud will be demonstrated during the first Lab session on
Tuesday April 1, 2014.

For some tips on using it, see \DUrole{xref,std,std-ref}{smc}.


\subsection{Virtual Machine}
\label{computing_options:options-vm}\label{computing_options:virtual-machine}
If you want to be able to compute on your own computer but don't want to
try installing all the necessary software packages
individually, another option is to
run a \emph{virtual machine} using the VirtualBox software.  See {\hyperref[vm:vm]{\crossref{\DUrole{std,std-ref}{Virtual Machine for this class {[}2014 Edition{]}}}}}
for more information.


\subsection{Amazon Web Services}
\label{computing_options:amazon-web-services}
Another possibility for cloud computing is to use Amazon Web Services.
An Amazon Machine Image (a virtual machine) has been created that has all
the software needed for this class.  For more detail on how to launch an
instance running this AMI, see {\hyperref[aws:aws]{\crossref{\DUrole{std,std-ref}{Amazon Web Services EC2 AMI {[}2014 version{]}}}}}.


\section{Downloading and installing software for this class}
\label{software_installation::doc}\label{software_installation:software-installation}\label{software_installation:downloading-and-installing-software-for-this-class}
In 2014 we will use
\href{https://cloud.sagemath.com/}{SageMathCloud}
for much of the computing in this class (see \DUrole{xref,std,std-ref}{smc}), but
if you want to install software on your own computer, this page gives some
hints.

Another option for computing in the cloud is to follow the instructions in
the section {\hyperref[aws:aws]{\crossref{\DUrole{std,std-ref}{Amazon Web Services EC2 AMI {[}2014 version{]}}}}}.

Rather than downloading and installing all of this software individually,
you might want to
consider using the {\hyperref[vm:vm]{\crossref{\DUrole{std,std-ref}{Virtual Machine for this class {[}2014 Edition{]}}}}}, which already contains everything you need.

It is assumed that you are on a Unix-like machine (e.g Linux or Mac OS
X).  For some flavors of Unix it is easy to download and install some
of the required packages using \emph{apt-get} (see {\hyperref[software_installation:apt\string-get]{\crossref{\DUrole{std,std-ref}{Software available through apt-get}}}}),
or your system's package
manager.  Many Python packages can also be installed using
\emph{easy\_install} (see {\hyperref[software_installation:easy\string-install]{\crossref{\DUrole{std,std-ref}{Software available through easy\_install}}}}).

If you must use a Windows PC, then you should
download and install \phantomsection\label{software_installation:id1}{\hyperref[biblio:virtualbox]{\crossref{{[}VirtualBox{]}}}} for Windows and then
run the {\hyperref[vm:vm]{\crossref{\DUrole{std,std-ref}{Virtual Machine for this class {[}2014 Edition{]}}}}} to provide a Linux environment.  Some of the software
we'll use is available on Windows, but we will assume you are using a
Unix-like environment and learning to do so is part of the goal of this class.

If you are using a Mac and want to install the necessary software, you also
need to install \href{https://developer.apple.com/xcode/}{Xcode} developer
tools, which includes necessary compilers and \emph{make}, for example.

Some of this software may already be available on your machine.  The \emph{which}
command in Unix will tell you if a command is found on your \emph{search path},
e.g.:

\begin{Verbatim}[commandchars=\\\{\}]
\PYGZdl{} which python
/usr/bin/python
\end{Verbatim}

tells me that when I type the python command it runs the program located in
the file listed. Often executables are stored in directories named \emph{bin},
which is short for \emph{binary}, since they are often binary machine code files.

If \emph{which} doesn't print anything, or says something like:

\begin{Verbatim}[commandchars=\\\{\}]
\PYGZdl{} which xyz
/usr/bin/which: no xyz in (/usr/bin:/usr/local/bin)
\end{Verbatim}

then the command cannot be found on the \emph{search path}.  So either the
software is not installed or it has been installed in a directory that isn't
searched by the shell program (see {\hyperref[shells:shells]{\crossref{\DUrole{std,std-ref}{Shells}}}}) when it tries to interpret
your command.  See {\hyperref[unix:unix\string-path]{\crossref{\DUrole{std,std-ref}{PATH and other search paths}}}} for more information.


\subsection{Versions}
\label{software_installation:versions}
Often there is more than one version of software packages in use.  Newer
versions may have more features than older versions and perhaps even behave
differently with respect to common features.  For some of what we do it will
be important to have a version that is sufficiently current.

For example, Python has changed dramatically in recent years.  Everything we
need (I think!) for this class can be found in
Version 2.X.Y for any \(X \geq 4\).

Major changes were made to Python in going to Python 3.0, which has not been
broadly adopted by the community yet (because much code would have to be
rewritten).  In this class we are \emph{not} using Python 3.X.  (See
\phantomsection\label{software_installation:id2}{\hyperref[biblio:python\string-3\string-0\string-tutorial]{\crossref{{[}Python-3.0-tutorial{]}}}} for more information.)

To determine what version of software
you have installed, often the command can be issued with the \code{-{-}version}
flag, e.g.:

\begin{Verbatim}[commandchars=\\\{\}]
\PYGZdl{} python \PYGZhy{}\PYGZhy{}version
Python 2.5.4
\end{Verbatim}


\subsection{Individual packages}
\label{software_installation:individual-packages}

\subsubsection{Python}
\label{software_installation:python}\label{software_installation:installing-python}
If the version of Python on your computer is older than 2.7.0 (see above),
you should upgrade.

See \url{http://www.python.org/download/} or consider the Enthought Python
Distribution (EPD) or Anaconda CE described below.


\subsubsection{SciPy Superpack}
\label{software_installation:scipy-superpack}\label{software_installation:installing-superpack}
On Mac OSX, you can often install gfortran and all the Python packages we'll
need using the \href{http://fonnesbeck.github.com/ScipySuperpack/}{SciPy Superpack}.


\subsubsection{Enthought Python Distribution (EPD)}
\label{software_installation:enthought-python-distribution-epd}\label{software_installation:installing-epd}
You might consider installing
\href{http://www.enthought.com/products/epd\_free.php}{EPD free}

This includes a recent version of Python 2.X as well as many of the other
Python packages listed below (IPython, NumPy, SciPy, matplotlib, mayavi).

EPD works well on Windows machines too.


\subsubsection{Anaconda CE}
\label{software_installation:installing-anaconda}\label{software_installation:anaconda-ce}
\href{https://store.continuum.io/cshop/anaconda}{Anaconda}
is a new collection of Python tools distributed by
\href{http://www.continuum.io/index.html}{Continuum Analytics}
The ``community edition'' Anaconda CE is free and contains most of the tools
we'll be using, including IPython, NumPy, SciPy, matplotlib,
and many others.  The full Anaconda is also free for academic users.


\subsubsection{IPython}
\label{software_installation:ipython}\label{software_installation:installing-ipython}
The IPython shell is much nicer to use than the standard Python shell (see
{\hyperref[shells:shells]{\crossref{\DUrole{std,std-ref}{Shells}}}} and \DUrole{xref,std,std-ref}{ipython}).
(Included in EPD, Anaconda, and the SciPy Superpack.)

See \url{http://ipython.scipy.org/moin/}


\subsubsection{NumPy and SciPy}
\label{software_installation:installing-numpy}\label{software_installation:numpy-and-scipy}
Used for numerical computing in Python (see {\hyperref[numerical_python:numerical\string-python]{\crossref{\DUrole{std,std-ref}{Numerics in Python}}}}).
(Included in EPD, Anaconda, and the SciPy Superpack.)

See \url{http://www.scipy.org/Installing\_SciPy}


\subsubsection{Matplotlib}
\label{software_installation:matplotlib}
Matlab-like plotting package for 1d and 2d plots in Python.
(Included in EPD, Anaconda, and the SciPy Superpack.)

See \url{http://matplotlib.sourceforge.net/}


\subsubsection{Git}
\label{software_installation:git}\label{software_installation:installing-git}
Version control system (see {\hyperref[git:git]{\crossref{\DUrole{std,std-ref}{Git}}}}).

See \href{http://git-scm.com/downloads}{downloads}.


\subsubsection{Sphinx}
\label{software_installation:installing-sphinx}\label{software_installation:sphinx}
Documentation system used to create these class notes pages (see
{\hyperref[sphinx:sphinx]{\crossref{\DUrole{std,std-ref}{Sphinx documentation}}}}).

See \url{http://sphinx.pocoo.org/}


\subsubsection{gfortran}
\label{software_installation:installing-gfortran}\label{software_installation:gfortran}
GNU fortran compiler (see {\hyperref[fortran:fortran]{\crossref{\DUrole{std,std-ref}{Fortran}}}}).

You may already have this installed, try:

\begin{Verbatim}[commandchars=\\\{\}]
\PYGZdl{} which gfortran
\end{Verbatim}

See \url{http://gcc.gnu.org/wiki/GFortran}


\subsubsection{OpenMP}
\label{software_installation:openmp}\label{software_installation:installing-openmp}
Included with gfortran (see {\hyperref[openmp:openmp]{\crossref{\DUrole{std,std-ref}{OpenMP}}}}).


\subsubsection{Open MPI}
\label{software_installation:installing-mpi}\label{software_installation:open-mpi}
Message Passing Interface software for parallel computing (see {\hyperref[mpi:mpi]{\crossref{\DUrole{std,std-ref}{MPI}}}}).

See \url{http://www.open-mpi.org/}

Some instructions for installing from source on a Mac can be found at
\href{https://sites.google.com/site/dwhipp/tutorials/installing-open-mpi-on-mac-os-x}{here}.


\subsubsection{LAPack}
\label{software_installation:lapack}\label{software_installation:installing-lapack}
Linear Algebra Package, a standard library of highly optimized linear
algebra subroutines.  LAPack depends on the BLAS (Basic Linear Algebra
Subroutines); it is distributed with a reference BLAS implementation,
but more highly optimized BLAS are available for most systems.

See \url{http://www.netlib.org/lapack/}


\subsection{Software available through \emph{apt-get}}
\label{software_installation:apt-get}\label{software_installation:software-available-through-apt-get}
On a recent Debian or Ubuntu Linux system, most of the software for
this class can be installed through \emph{apt-get}.  To install, type the
command:

\begin{Verbatim}[commandchars=\\\{\}]
\PYGZdl{} sudo apt\PYGZhy{}get install PACKAGE
\end{Verbatim}

where the appropriate PACKAGE to install comes from the list below.

NOTE: You will only be able to do this on your own machine, the VM described
at {\hyperref[vm:vm]{\crossref{\DUrole{std,std-ref}{Virtual Machine for this class {[}2014 Edition{]}}}}}, or a computer on which you have super user privileges to
install software in the sytsem files.  (See {\hyperref[unix:sudo]{\crossref{\DUrole{std,std-ref}{sudo}}}})

You can also install
these packages using a graphical package manager such as Synaptic
instead of \emph{apt-get}.  If you are able to install all of these
packages, you do not need to install the Enthought Python
Distribution.

\begin{tabulary}{\linewidth}{|L|L|}
\hline
\textsf{\relax 
Software
} & \textsf{\relax 
Package
}\\
\hline
Python
 & 
python
\\
\hline
IPython
 & 
ipython
\\
\hline
NumPy
 & 
python-numpy
\\
\hline
SciPy
 & 
python-scipy
\\
\hline
Matplotlib
 & 
python-matplotlib
\\
\hline
Python development files
 & 
python-dev
\\
\hline
Git
 & 
git
\\
\hline
Sphinx
 & 
python-sphinx
\\
\hline
gfortran
 & 
gfortran
\\
\hline
OpenMPI libraries
 & 
libopenmpi-dev
\\
\hline
OpenMPI executables
 & 
openmpi-bin
\\
\hline
LAPack
 & 
liblapack-dev
\\
\hline\end{tabulary}


Many of these packages depend on other packages; answer ``yes'' when
\emph{apt-get} asks you if you want to download them.  Some of them, such
as Python, are probably already installed on your system, in which
case \emph{apt-get} will tell you that they are already installed and do
nothing.

The script below was used to install software on the Ubuntu VM:
\begin{quote}

\begin{Verbatim}[commandchars=\\\{\}]
\PYG{c}{\PYGZsh{} \PYGZdl{}UWHPSC/notes/install.sh}
\PYG{c}{\PYGZsh{} }
\PYG{n}{sudo} \PYG{n}{apt}\PYG{o}{\PYGZhy{}}\PYG{n}{get} \PYG{n}{update}
\PYG{n}{sudo} \PYG{n}{apt}\PYG{o}{\PYGZhy{}}\PYG{n}{get} \PYG{n}{upgrade}
\PYG{n}{sudo} \PYG{n}{apt}\PYG{o}{\PYGZhy{}}\PYG{n}{get} \PYG{n}{install} \PYG{n}{xfce4}
\PYG{n}{sudo} \PYG{n}{apt}\PYG{o}{\PYGZhy{}}\PYG{n}{get} \PYG{n}{install} \PYG{n}{jockey}\PYG{o}{\PYGZhy{}}\PYG{n}{gtk}
\PYG{n}{sudo} \PYG{n}{apt}\PYG{o}{\PYGZhy{}}\PYG{n}{get} \PYG{n}{install} \PYG{n}{xdm}
\PYG{n}{sudo} \PYG{n}{apt}\PYG{o}{\PYGZhy{}}\PYG{n}{get} \PYG{n}{install} \PYG{n}{ipython}
\PYG{n}{sudo} \PYG{n}{apt}\PYG{o}{\PYGZhy{}}\PYG{n}{get} \PYG{n}{install} \PYG{n}{python}\PYG{o}{\PYGZhy{}}\PYG{n}{numpy}
\PYG{n}{sudo} \PYG{n}{apt}\PYG{o}{\PYGZhy{}}\PYG{n}{get} \PYG{n}{install} \PYG{n}{python}\PYG{o}{\PYGZhy{}}\PYG{n}{scipy}
\PYG{n}{sudo} \PYG{n}{apt}\PYG{o}{\PYGZhy{}}\PYG{n}{get} \PYG{n}{install} \PYG{n}{python}\PYG{o}{\PYGZhy{}}\PYG{n}{matplotlib}
\PYG{n}{sudo} \PYG{n}{apt}\PYG{o}{\PYGZhy{}}\PYG{n}{get} \PYG{n}{install} \PYG{n}{python}\PYG{o}{\PYGZhy{}}\PYG{n}{dev}
\PYG{n}{sudo} \PYG{n}{apt}\PYG{o}{\PYGZhy{}}\PYG{n}{get} \PYG{n}{install} \PYG{n}{git}
\PYG{n}{sudo} \PYG{n}{apt}\PYG{o}{\PYGZhy{}}\PYG{n}{get} \PYG{n}{install} \PYG{n}{python}\PYG{o}{\PYGZhy{}}\PYG{n}{sphinx}
\PYG{n}{sudo} \PYG{n}{apt}\PYG{o}{\PYGZhy{}}\PYG{n}{get} \PYG{n}{install} \PYG{n}{gfortran}
\PYG{n}{sudo} \PYG{n}{apt}\PYG{o}{\PYGZhy{}}\PYG{n}{get} \PYG{n}{install} \PYG{n}{openmpi}\PYG{o}{\PYGZhy{}}\PYG{n+nb}{bin}
\PYG{n}{sudo} \PYG{n}{apt}\PYG{o}{\PYGZhy{}}\PYG{n}{get} \PYG{n}{install} \PYG{n}{liblapack}\PYG{o}{\PYGZhy{}}\PYG{n}{dev}
\PYG{n}{sudo} \PYG{n}{apt}\PYG{o}{\PYGZhy{}}\PYG{n}{get} \PYG{n}{install} \PYG{n}{thunar}
\PYG{n}{sudo} \PYG{n}{apt}\PYG{o}{\PYGZhy{}}\PYG{n}{get} \PYG{n}{install} \PYG{n}{xfce4}\PYG{o}{\PYGZhy{}}\PYG{n}{terminal}

\PYG{c}{\PYGZsh{} some packages not installed on the VM }
\PYG{c}{\PYGZsh{} that you might want to add:}

\PYG{n}{sudo} \PYG{n}{apt}\PYG{o}{\PYGZhy{}}\PYG{n}{get} \PYG{n}{install} \PYG{n}{gitk}               \PYG{c}{\PYGZsh{} to view git history}
\PYG{n}{sudo} \PYG{n}{apt}\PYG{o}{\PYGZhy{}}\PYG{n}{get} \PYG{n}{install} \PYG{n}{xxdiff}             \PYG{c}{\PYGZsh{} to compare two files}
\PYG{n}{sudo} \PYG{n}{apt}\PYG{o}{\PYGZhy{}}\PYG{n}{get} \PYG{n}{install} \PYG{n}{python}\PYG{o}{\PYGZhy{}}\PYG{n}{sympy}       \PYG{c}{\PYGZsh{} symbolic python}
\PYG{n}{sudo} \PYG{n}{apt}\PYG{o}{\PYGZhy{}}\PYG{n}{get} \PYG{n}{install} \PYG{n}{imagemagick}        \PYG{c}{\PYGZsh{} so you can \PYGZdq{}display plot.png\PYGZdq{}}


\PYG{n}{sudo} \PYG{n}{apt}\PYG{o}{\PYGZhy{}}\PYG{n}{get} \PYG{n}{install} \PYG{n}{python}\PYG{o}{\PYGZhy{}}\PYG{n}{setuptools}  \PYG{c}{\PYGZsh{} so easy\PYGZus{}install is available}
\PYG{n}{sudo} \PYG{n}{easy\PYGZus{}install} \PYG{n}{nose}                  \PYG{c}{\PYGZsh{} unit testing framework}
\PYG{n}{sudo} \PYG{n}{easy\PYGZus{}install} \PYG{n}{StarCluster}           \PYG{c}{\PYGZsh{} to help manage clusters on AWS}

\end{Verbatim}
\end{quote}


\subsection{Software available through \emph{easy\_install}}
\label{software_installation:easy-install}\label{software_installation:software-available-through-easy-install}
\emph{easy\_install} is a Python utility that can automatically download and
install many Python packages.  It is part of the Python \emph{setuptools}
package, available from \url{http://pypi.python.org/pypi/setuptools},
and requires Python to already be installed on your system.  Once this
package is installed, you can install Python packages on a Unix system
by typing:

\begin{Verbatim}[commandchars=\\\{\}]
\PYGZdl{} sudo easy\PYGZus{}install PACKAGE
\end{Verbatim}

where the PACKAGE to install comes from the list below.  Note that
these packages are redundant with the ones available from \emph{apt-get};
use \emph{apt-get} if it's available.

\begin{tabulary}{\linewidth}{|L|L|}
\hline
\textsf{\relax 
Software
} & \textsf{\relax 
Package
}\\
\hline
IPython
 & 
IPython{[}kernel,security{]}
\\
\hline
NumPy
 & 
numpy
\\
\hline
SciPy
 & 
scipy
\\
\hline
Matplotlib
 & 
matplotlib
\\
\hline
Mayavi
 & 
mayavi
\\
\hline
Git
 & 
git
\\
\hline
Sphinx
 & 
sphinx
\\
\hline\end{tabulary}


If these packages fail to build, you may need to install the Python
headers.


\section{Virtual Machine for this class {[}2014 Edition{]}}
\label{vm:virtual-machine-for-this-class-2014-edition}\label{vm::doc}\label{vm:vm}
We are using a wide variety of software in this class, much of which is
probably not found on your computer.  It is all open source software (see
\DUrole{xref,std,std-ref}{licences}) and links/instructions
can be found in the section {\hyperref[software_installation:software\string-installation]{\crossref{\DUrole{std,std-ref}{Downloading and installing software for this class}}}}.

An alternative, which many will find more convenient, is to download and
install the \phantomsection\label{vm:id1}{\hyperref[biblio:virtualbox]{\crossref{{[}VirtualBox{]}}}} software and then download a Virtual Machine (VM)
that has been built specifically for this course.  VirtualBox will run this
machine, which will emulate a specific version of Linux that already has
installed all of the software packages that will be used in this course.

You can find the VM on the \href{http://faculty.washington.edu/rjl/classes/am583s2014/}{class
webpage}
in the file \href{http://faculty.washington.edu/rjl/classes/am583s2014/uwhpsc.zip}{uwhpsc.zip}.

Note that the file is quite
large (approximately 803 MB compressed), and if possible you should
download it from on-campus to shorten the download time.  The TA's will also
have the VM on memory sticks for transferring.


\subsection{System requirements}
\label{vm:system-requirements}
The VM is around 2 GB in size, uncompressed, and the virtual disk
image may expand to up to 8 GB, depending on how much data you store
in the VM.  Make sure you have enough free space available before
installing.  You can set how much RAM is available to the VM when
configuring it, but it is recommended that you give it at least 512
MB; since your computer must host your own operating system at the
same time, it is recommended that you have at least 1 GB of total RAM.


\subsection{Setting up the VM in VirtualBox}
\label{vm:setting-up-the-vm-in-virtualbox}
Once you have downloaded and uncompressed the virtual machine disk
image from the class web site, you can set it up in VirtualBox, by
doing the following:
\begin{enumerate}
\item {} 
Start VirtualBox

\item {} 
Click the \emph{New} button near the upper-left corner

\item {} 
Click \emph{Next} at the starting page

\item {} 
Enter a name for the VM (put in whatever you like); for \emph{OS Type},
select ``Linux'', and for \emph{Version}, select ``Ubuntu''.  Click \emph{Next}.

\item {} 
Enter the amount of memory to give the VM, in megabytes.
512 MB is the recommended minimum.  Click \emph{Next}.

\item {} 
Click \emph{Use existing hard disk}, then click the folder icon next to
the disk list.  In the Virtual Media Manager that appears, click
\emph{Add}, then select the virtual machine disk image you downloaded
from the class web site.  Ignore the message about the recommended
size of the boot disk, and leave the box labeled ``Boot Hard Disk
(Primary Master)'' checked.  Once you have selected the disk image,
click \emph{Next}.

\item {} 
Review the summary VirtualBox gives you, then click \emph{Finish}.  Your
new virtual machine should appear on the left side of the VirtualBox
window.

\end{enumerate}


\subsection{Starting the VM}
\label{vm:starting-the-vm}
Once you have configured the VM in VirtualBox, you can start it by
double-clicking it in the list of VM's on your system.  The virtual
machine will take a little time to start up; as it does, VirtualBox
will display a few messages explaining about mouse pointer and
keyboard capturing, which you should read.

After the VM has finished booting, it will present you with a login
screen; the login and password are both \code{uwhpsc}.  (We would have
liked to set up a VM with no password, but many things in Linux assume
you have one.)

Note that you will also need this password to quit the VM.


\subsection{Running programs}
\label{vm:running-programs}
You can access the programs on the virtual machine through the Applications
Menu (the mouse on an \emph{X} symbol in the upper-left corner of the
screen), or by clicking the quick-launch icons next to the menu
button.  By default, you will have quick-launch icons for a command
prompt window (also known as a \emph{terminal window}), a text editor, and
a web browser.  After logging in for the first time, you should start
the web browser to make sure your network connection is working.


\subsection{Fixing networking issues}
\label{vm:fixing-networking-issues}
When a Linux VM is moved to a new computer, it sometimes doesn't
realize that the previous computer's network adaptor is no longer
available.

Also, if you move your computer from one wireless network to another while
the VM is running, it may lose connection with the internet.

If this happens, it should be sufficient to shut down the VM (with the 0/1
button on the top right corner) and then restart it.
On shutdown, a script is automatically run that does the following, which in
earlier iterations of the VM had to be done manually...
\begin{quote}

\$ sudo rm /etc/udev/rules.d/70-persistent-net.rules
\end{quote}

This will remove the incorrect settings; Linux should then autodetect
and correctly configure the network interface it boots.


\subsection{Shutting down}
\label{vm:shutting-down}
When you are done using the virtual machine, you can shut it down by
clicking the 0/1 button on the top-right corner of the VM.
You will need the password \titleref{uwhpsc}.


\subsection{Cutting and pasting}
\label{vm:cutting-and-pasting}
If you want to cut text from one window in the VM and paste it into another,
you should be able to highlight the text and then type ctrl-c (or in a
terminal window, ctrl-shift-C, since ctrl-c is the interrupt signal). To
paste, type ctrl-v (or ctrl-shift-V in a terminal window).

If you want to be able to cut and paste between a window in the VM and a
window on your host machine, click on Machine from the main VitualBox menu
(or \titleref{Settings} in the Oracle VM VirtualBox Manager window), then click on
\titleref{General} and then \titleref{Advanced}.  Select \titleref{Bidirectional} from the \titleref{Shared
Clipboard} menu.


\subsection{Shared Folders}
\label{vm:shared-folders}
If you create a file on the VM that you want to move to the file system of
the host machine, or vice versa, you can create a ``shared folder'' that is
seen by both.

First create a folder (i.e. directory) on the host machine, e.g. via:

\begin{Verbatim}[commandchars=\\\{\}]
\PYGZdl{} mkdir \PYGZti{}/uwhpsc\PYGZus{}shared
\end{Verbatim}

This creates a new subdirectory in your home directory on the host machine.

In the VirtualBox menu click on \titleref{Devices}, then click on
\titleref{Shared Folders}.  Click the + button on the right side and then type in the
full path to the folder you want to share under \titleref{Folder Path}, including the
folder name, and then the folder name itself under \titleref{Folder name}.
If you click on \titleref{Auto-mount} then this will be mounted every time you start
the VM.

Then click \titleref{OK} twice.

Then, in the VM (at the linux prompt), type the following commands:

\begin{Verbatim}[commandchars=\\\{\}]
sharename=uwhpsc\PYGZus{}shared   \PYGZsh{} or whatever name the folder has
sudo mkdir /mnt/\PYGZdl{}sharename
sudo chmod 777 /mnt/\PYGZdl{}sharename
sudo mount \PYGZhy{}t vboxsf \PYGZhy{}o uid=1000,gid=1000 \PYGZdl{}sharename /mnt/\PYGZdl{}sharename
\end{Verbatim}

You may need the password \titleref{uwhpsc} for the first \titleref{sudo} command.

The folder should now be found in the VM in \titleref{/mnt/\$sharename}.
(Note \titleref{\$sharename} is a variable set in the first command above.)

If auto-mounting doesn't work properly, you may need to repeat the final
\titleref{sudo mount ...} command  each time you start the VM.


\subsection{Enabling more processors}
\label{vm:enabling-more-processors}
If you have a reasonably new computer with a multi-core
processor and want to be able to run parallel programs across multiple
cores, you can tell VirtualBox to allow the VM to use additional
cores.  To do this, open the VirtualBox
\emph{Settings}.  Under \emph{System}, click the \emph{Processor}
tab, then use the slider to set the number of processors the VM will
see.  Note that some older multi-core processors do not support the
necessary extensions for this, and on these machines you will only be
able to run the VM on a single core.


\subsection{Problems enabling multiple processors...}
\label{vm:problems-enabling-multiple-processors}
Users may encounter several problems with enabling mutliple processors. Some users may not
be able to change this setting (it will be greyed out). Other users when may find no improved performance after enabling multiple processors. Still others may encounter an error such as:

\begin{Verbatim}[commandchars=\\\{\}]
\PYG{n}{VD}\PYG{p}{:} \PYG{n}{error} \PYG{n}{VERR\PYGZus{}NOT\PYGZus{}SUPPORTED}
\end{Verbatim}

All of these problems indicate that virtualization has not been enabled on your processors.

Fortunately this has an easy fix. You just have to enable virtualization in your BIOS
settings.

1. To  access the BIOS settings you must restart your computer and press a certain
button on startup. This button will depend on the company that manufactures your computer
(for example for Lenovo's it appears to be the f1 key).

2. Next you must locate a setting that will refer to either virtualization, VT, or VT-x.
Again the exact specifications will depend on the computer's manufacturer, however
it should be found in the Security section (or the Performance section if you are using a Dell).

3. Enable this setting,
then save and exit the bios settings.After your computer reboots you should be able to start the VM using multiple processors now.

4. If your BIOS does not have any settings like this it is possible that your BIOS is set up to hide this option from you, and you
may need to follow the advice here: \url{http://mathy.vanvoorden.be/blog/2010/01/enable-vt-x-on-dell-laptop/}

Note: Unfortunately some older hardware does not support virtualization, and so if these solutions don't work for you it may
be that this is the case for your processors. There also may be other possible problems...so don't be afraid to ask the TAs for help!


\subsection{Changing guest resolution/VM window size}
\label{vm:changing-guest-resolution-vm-window-size}

\strong{See also:}


The section {\hyperref[vm:vm\string-additions]{\crossref{\DUrole{std,std-ref}{Guest Additions}}}}, which makes this easier.



It's possible that the size of the VM's window may be too large for
your display; resizing it in the normal way will result in not all of
the VM desktop being displayed, which may not be the ideal way to
work.  Alternately, if you are working on a high-resolution display,
you may want to \emph{increase} the size of the VM's desktop to take
advantage of it.  In either case, you can change the VM's display size
by going to the Applications menu in the upper-left corner, pointing to
\emph{Settings}, then clicking \emph{Display}.  Choose a resolution from the
drop-down list, then click \emph{Apply}.


\subsection{Setting the host key}
\label{vm:setting-the-host-key}

\strong{See also:}


The section {\hyperref[vm:vm\string-additions]{\crossref{\DUrole{std,std-ref}{Guest Additions}}}}, which makes this easier.



When you click on the VM window, it will capture your mouse and future mouse
actions will apply to the windows in the VM.  To uncapture the mouse you
need to hit some control key, called the \emph{host key}.  It should give you a
message about this.  If it says the host key is Right Control, for example,
that means the Control key on the right side of your keyboard (it does \emph{not}
mean to click the right mouse button).

On some systems, the host key that transfers input focus between the
VM and the host operating system may be a key that you want to use in
the VM for other purposes.  To fix this, you can
change the host key in VirtualBox.  In the main VirtualBox window (not
the VM's window; in fact, the VM doesn't need to be running to do
this), go to the \emph{File} menu, then click \emph{Settings}.  Under \emph{Input},
click the box marked ``Host Key'', then press the key you want to use.


\subsection{Guest Additions}
\label{vm:guest-additions}\label{vm:vm-additions}
While we have installed the VirtualBox guest additions on the class
VM, the guest additions sometimes stop working when the VM is moved to
a different computer, so you may need to reinstall them.
Do the following so that the VM will automatically capture and uncapture
your mouse depending on whether you click in the VM window or outside it,
and to make it easier to resize the VM window to fit your display.
\begin{enumerate}
\item {} 
Boot the VM, and log in.

\item {} 
In the VirtualBox menu bar on your host system, select Devices --\textgreater{}
Install Guest Additions...  (Note: click on the window for the class
VM itself to get this menu, not on the main ``Sun VirtualBox'' window.)

\item {} 
A CD drive should appear on the VM's desktop, along with a popup
window.  (If it doesn't, see the additional instructions below.)
Select ``Allow Auto-Run'' in the popup window.  Then enter the
password you use to log in.

\item {} 
The Guest Additions will begin to install, and a window will appear,
displaying the progress of the installation.  When the installation is done,
the window will tell you to press `Enter' to close it.

\item {} 
Right-click the CD drive on the desktop, and select `Eject'.

\item {} 
Restart the VM.

\end{enumerate}

If step 3 doesn't work the first time, you might need to:
\begin{quote}
\begin{description}
\item[{Alternative Step 3:}] \leavevmode\begin{enumerate}
\item {} 
Reboot the VM.

\item {} 
Mount the CD image by right-clicking the CD drive icon, and clicking
`Mount'.

\item {} 
Double click the CD image to open it.

\item {} 
Double click `autorun.sh'.

\item {} 
Enter the VM password to install.

\end{enumerate}

\end{description}
\end{quote}


\subsection{How This Virtual Machine was made}
\label{vm:how-this-virtual-machine-was-made}\begin{quote}
\begin{enumerate}
\item {} 
Download Ubuntu 12.04 PC (Intel x86) alternate install ISO from
\url{http://cdimage.ubuntu.com/xubuntu/releases/12.04.2/release/xubuntu-12.04.2-alternate-i386.iso}

\item {} 
Create a new virtual box

\item {} 
Set the system as Ubuntu

\item {} 
Use defualt options

\item {} 
After that double click on your new virtual machine...a dropdown
box should appear where you can select
your ubuntu iso

\item {} 
As you are installing...at the first menu hit F4 and install a
command line system

\item {} 
Let the install proceed following the instructions as given. On most
options the default answer will be appropriate.
When it comes time to format the hard drive, choose the manual option.
Format all the free space and set it as the mount
point. From the next list choose root (you dont need a swap space).

\item {} 
Install the necessary packages

\begin{Verbatim}[commandchars=\\\{\}]
\PYG{c}{\PYGZsh{} \PYGZdl{}UWHPSC/notes/install.sh}
\PYG{c}{\PYGZsh{} }
\PYG{n}{sudo} \PYG{n}{apt}\PYG{o}{\PYGZhy{}}\PYG{n}{get} \PYG{n}{update}
\PYG{n}{sudo} \PYG{n}{apt}\PYG{o}{\PYGZhy{}}\PYG{n}{get} \PYG{n}{upgrade}
\PYG{n}{sudo} \PYG{n}{apt}\PYG{o}{\PYGZhy{}}\PYG{n}{get} \PYG{n}{install} \PYG{n}{xfce4}
\PYG{n}{sudo} \PYG{n}{apt}\PYG{o}{\PYGZhy{}}\PYG{n}{get} \PYG{n}{install} \PYG{n}{jockey}\PYG{o}{\PYGZhy{}}\PYG{n}{gtk}
\PYG{n}{sudo} \PYG{n}{apt}\PYG{o}{\PYGZhy{}}\PYG{n}{get} \PYG{n}{install} \PYG{n}{xdm}
\PYG{n}{sudo} \PYG{n}{apt}\PYG{o}{\PYGZhy{}}\PYG{n}{get} \PYG{n}{install} \PYG{n}{ipython}
\PYG{n}{sudo} \PYG{n}{apt}\PYG{o}{\PYGZhy{}}\PYG{n}{get} \PYG{n}{install} \PYG{n}{python}\PYG{o}{\PYGZhy{}}\PYG{n}{numpy}
\PYG{n}{sudo} \PYG{n}{apt}\PYG{o}{\PYGZhy{}}\PYG{n}{get} \PYG{n}{install} \PYG{n}{python}\PYG{o}{\PYGZhy{}}\PYG{n}{scipy}
\PYG{n}{sudo} \PYG{n}{apt}\PYG{o}{\PYGZhy{}}\PYG{n}{get} \PYG{n}{install} \PYG{n}{python}\PYG{o}{\PYGZhy{}}\PYG{n}{matplotlib}
\PYG{n}{sudo} \PYG{n}{apt}\PYG{o}{\PYGZhy{}}\PYG{n}{get} \PYG{n}{install} \PYG{n}{python}\PYG{o}{\PYGZhy{}}\PYG{n}{dev}
\PYG{n}{sudo} \PYG{n}{apt}\PYG{o}{\PYGZhy{}}\PYG{n}{get} \PYG{n}{install} \PYG{n}{git}
\PYG{n}{sudo} \PYG{n}{apt}\PYG{o}{\PYGZhy{}}\PYG{n}{get} \PYG{n}{install} \PYG{n}{python}\PYG{o}{\PYGZhy{}}\PYG{n}{sphinx}
\PYG{n}{sudo} \PYG{n}{apt}\PYG{o}{\PYGZhy{}}\PYG{n}{get} \PYG{n}{install} \PYG{n}{gfortran}
\PYG{n}{sudo} \PYG{n}{apt}\PYG{o}{\PYGZhy{}}\PYG{n}{get} \PYG{n}{install} \PYG{n}{openmpi}\PYG{o}{\PYGZhy{}}\PYG{n+nb}{bin}
\PYG{n}{sudo} \PYG{n}{apt}\PYG{o}{\PYGZhy{}}\PYG{n}{get} \PYG{n}{install} \PYG{n}{liblapack}\PYG{o}{\PYGZhy{}}\PYG{n}{dev}
\PYG{n}{sudo} \PYG{n}{apt}\PYG{o}{\PYGZhy{}}\PYG{n}{get} \PYG{n}{install} \PYG{n}{thunar}
\PYG{n}{sudo} \PYG{n}{apt}\PYG{o}{\PYGZhy{}}\PYG{n}{get} \PYG{n}{install} \PYG{n}{xfce4}\PYG{o}{\PYGZhy{}}\PYG{n}{terminal}

\PYG{c}{\PYGZsh{} some packages not installed on the VM }
\PYG{c}{\PYGZsh{} that you might want to add:}

\PYG{n}{sudo} \PYG{n}{apt}\PYG{o}{\PYGZhy{}}\PYG{n}{get} \PYG{n}{install} \PYG{n}{gitk}               \PYG{c}{\PYGZsh{} to view git history}
\PYG{n}{sudo} \PYG{n}{apt}\PYG{o}{\PYGZhy{}}\PYG{n}{get} \PYG{n}{install} \PYG{n}{xxdiff}             \PYG{c}{\PYGZsh{} to compare two files}
\PYG{n}{sudo} \PYG{n}{apt}\PYG{o}{\PYGZhy{}}\PYG{n}{get} \PYG{n}{install} \PYG{n}{python}\PYG{o}{\PYGZhy{}}\PYG{n}{sympy}       \PYG{c}{\PYGZsh{} symbolic python}
\PYG{n}{sudo} \PYG{n}{apt}\PYG{o}{\PYGZhy{}}\PYG{n}{get} \PYG{n}{install} \PYG{n}{imagemagick}        \PYG{c}{\PYGZsh{} so you can \PYGZdq{}display plot.png\PYGZdq{}}


\PYG{n}{sudo} \PYG{n}{apt}\PYG{o}{\PYGZhy{}}\PYG{n}{get} \PYG{n}{install} \PYG{n}{python}\PYG{o}{\PYGZhy{}}\PYG{n}{setuptools}  \PYG{c}{\PYGZsh{} so easy\PYGZus{}install is available}
\PYG{n}{sudo} \PYG{n}{easy\PYGZus{}install} \PYG{n}{nose}                  \PYG{c}{\PYGZsh{} unit testing framework}
\PYG{n}{sudo} \PYG{n}{easy\PYGZus{}install} \PYG{n}{StarCluster}           \PYG{c}{\PYGZsh{} to help manage clusters on AWS}

\end{Verbatim}

\item {} 
To setup the login screen edit the file Xresources so that the
greeting line says.:

\begin{Verbatim}[commandchars=\\\{\}]
\PYG{n}{xlogin}\PYG{o}{*}\PYG{n}{greeting}\PYG{p}{:} \PYG{n}{Login} \PYG{o+ow}{and} \PYG{n}{Password} \PYG{n}{are} \PYG{n}{uwhpsc}
\end{Verbatim}

\item {} 
Create the file uwhpscvm-shutdown.:

\end{enumerate}
\begin{quote}

\begin{Verbatim}[commandchars=\\\{\}]
\PYG{c}{\PYGZsh{}!/bin/sh}
\PYG{c}{\PYGZsh{}Finish up Script\PYGZhy{}\PYGZhy{}does some cleanup then shuts down}

\PYG{c}{\PYGZsh{}prevent network problems}
\PYG{n}{rm} \PYG{o}{\PYGZhy{}}\PYG{n}{f} \PYG{o}{/}\PYG{n}{etc}\PYG{o}{/}\PYG{n}{udev}\PYG{o}{/}\PYG{n}{rules}\PYG{o}{.}\PYG{n}{d}\PYG{o}{/}\PYG{l+m+mi}{70}\PYG{o}{\PYGZhy{}}\PYG{n}{persistent}\PYG{o}{\PYGZhy{}}\PYG{n}{net}\PYG{o}{.}\PYG{n}{rules}

\PYG{c}{\PYGZsh{}shutdown}
\PYG{n}{shutdown} \PYG{o}{\PYGZhy{}}\PYG{n}{h} \PYG{n}{now}
\end{Verbatim}
\end{quote}
\begin{enumerate}
\setcounter{enumi}{10}
\item {} 
Save it at.:

\begin{Verbatim}[commandchars=\\\{\}]
\PYG{o}{/}\PYG{n}{usr}\PYG{o}{/}\PYG{n}{local}\PYG{o}{/}\PYG{n+nb}{bin}\PYG{o}{/}\PYG{n}{uwhpscvm}\PYG{o}{\PYGZhy{}}\PYG{n}{shutdown}
\end{Verbatim}

\item {} 
Execute the following command command.:

\begin{Verbatim}[commandchars=\\\{\}]
\PYGZdl{} sudo chmod +x /usr/local/bin/uwhpscvm\PYGZhy{}shutdown
\end{Verbatim}

\item {} 
Right click on the upper panel and select add new items and choose
to add a new launcher.

\item {} 
Name the new launcher something like shutdown and in the command
blank copy the following line.:

\begin{Verbatim}[commandchars=\\\{\}]
\PYG{n}{gksudo} \PYG{o}{/}\PYG{n}{usr}\PYG{o}{/}\PYG{n}{local}\PYG{o}{/}\PYG{n+nb}{bin}\PYG{o}{/}\PYG{n}{uwhpscvm}\PYG{o}{\PYGZhy{}}\PYG{n}{shutdown}
\end{Verbatim}

\item {} 
Go to preferred applications and select Thunar for file managment
and the xfce4 terminal.

\item {} 
Run jockey-gtk and install guest-additions.

\item {} 
Go to Applications then Settings then screensaver and select
``disable screen saver'' mode

\item {} 
In the settings menu select the general settings and hit the advanced
tab. Here you can set the clipboard and drag
and drop to allow Host To Guest.

\item {} 
Shutdown the machine and then go to the main virtualbox screen.
Click on the virtualmachine and then hit the settings button.

\item {} 
After, in the system settings click on the processor tab. This may let
you allow the virtual machine to use more than one processor (depending
on your computer). Choose a setting somewhere in the green section of
the Processors slider.

\end{enumerate}
\end{quote}


\subsection{About the VM}
\label{vm:about-the-vm}
The class virtual machine is running XUbuntu 12.04, a variant of Ubuntu
Linux (\url{http://www.ubuntu.com}), which itself is an offshoot of
Debian GNU/Linux (\url{http://www.debian.org}).  XUbuntu is a
stripped-down, simplified version of Ubuntu suitable for running on
smaller systems (or virtual machines); it runs the \emph{xfce4} desktop
environment.


\subsection{Further reading}
\label{vm:further-reading}
\phantomsection\label{vm:id2}{\hyperref[biblio:virtualbox]{\crossref{{[}VirtualBox{]}}}}
\phantomsection\label{vm:id3}{\hyperref[biblio:virtualbox\string-documentation]{\crossref{{[}VirtualBox-documentation{]}}}}


\section{Amazon Web Services EC2 AMI {[}2014 version{]}}
\label{aws:aws}\label{aws:amazon-web-services-ec2-ami-2014-version}\label{aws::doc}
We are using a wide variety of software in this class, much of which is
probably not found on your computer.  It is all open source software (see
\DUrole{xref,std,std-ref}{licences}) and links/instructions
can be found in the section {\hyperref[software_installation:software\string-installation]{\crossref{\DUrole{std,std-ref}{Downloading and installing software for this class}}}}.
You can also use the {\hyperref[vm:vm]{\crossref{\DUrole{std,std-ref}{Virtual Machine for this class {[}2014 Edition{]}}}}}.

Another alternative is to write and run your programs ``in the cloud''
using Amazon Web Services (AWS) Elastic Cloud Computing (EC2).
You can start up an ``instance'' (your own private computer, or so it appears)
that is configured using an Amazon Machine Image (AMI) that has been
configured with the Linux operating system and containing
all the software needed for this class.

You must first sign up for an account  on the \href{http://aws.amazon.com/}{AWS main page}.  For this you will need a credit
card, but note that with an account you can get 750 hours per month of
free ``micro instance'' usage in the
\href{http://aws.amazon.com/free/}{free usage tier}.
A micro instance is a single processor (that you will probably be sharing
with others) so it's not suitable for trying out parallel computing, but
should be just fine for much of the programming work in this class.

You can start up more powerful instances with 2 or more processors for a cost
starting at less than 3 cents per hour (the m3.large on-demand
instance).  See the \href{http://aws.amazon.com/ec2/pricing/}{pricing guide}.

For general information and guides to getting started:
\begin{itemize}
\item {} 
\href{http://docs.amazonwebservices.com/AWSEC2/latest/GettingStartedGuide/}{Getting started with EC2},
with tutorial to lead you through an example.

\item {} 
\href{http://aws.amazon.com/ec2/faqs}{EC2 FAQ}.

\item {} 
\href{http://aws.amazon.com/ec2/pricing}{Pricing}.  Note: you are charged
per hour for hours (or fraction thereof) that your instance is in
\titleref{running} mode, regardless of whether the CPU is being used.

\item {} 
\href{http://aws.amazon.com/hpc-applications/}{High Performance Computing on AWS}
with instructions on starting a cluster instance.

\item {} 
\href{http://escience.washington.edu/get-help-now/get-started-amazon-web-services}{UW eScience information on AWS}.

\end{itemize}


\subsection{Launching an instance with the \emph{uwhpsc} AMI}
\label{aws:launching-an-instance-with-the-uwhpsc-ami}

\subsubsection{Quick way}
\label{aws:quick-way}
Navigate your browser to
\url{https://console.aws.amazon.com/ec2/home?region=us-west-2\#launchAmi=ami-501a7260}

You should then be on a page where you see you are on Step 2 of 7 at the top
of the page, ``Choose instance type''.

Then you can skip the next section and proceed to {\hyperref[aws:aws\string-instance\string-type]{\crossref{\DUrole{std,std-ref}{Choose instance type}}}}.


\subsubsection{Search for AMI}
\label{aws:search-for-ami}
\textbf{Skip this section} if you followed the ``quick way'' instructions above.

Going through this part may be useful if you want to see how to search for
other AMI's in the future.

Once you have an AWS account, sign in to the
\href{https://console.aws.amazon.com/ec2/}{management console}
and click on the
EC2 tab, and then select Region US West (Oregon) from the menu
at the top right of the page, next to your user name.

You should now be on the page
\url{https://console.aws.amazon.com/ec2/v2/home?region=us-west-2}.

Click on the big ``Launch Instance'' button.

On the next page, you will see a list of ``Quick start''
Amazon Machine Images (AMIs) that
you can select from if you want to start with a fresh VM.  For this class
you don't want any of these.  Instead click on the ``Community AMIs'' tab and
then type \titleref{uwhpsc2014} in the search bar.  Select this image.

You will then be taken to Step 2, ``Choose instance type''.


\subsubsection{Choose instance type}
\label{aws:choose-instance-type}\label{aws:aws-instance-type}
On the next page you can select what sort of instance you wish to start (larger
instances cost more per hour). t1-micro is the the size you can run free (as
long as you only have one running).

Click \titleref{Continue} on the next few screens through the ``instance details''
and eventually you get to one that
asks for a key pair.  If you don't already have one, create a new one and
select it here.

You can now skip over steps 3-6 and jump directly to Step 7, ``Review and
Launch''.


\subsection{Launch instance / create key pair}
\label{aws:launch-instance-create-key-pair}
When you click on ``Launch'', you will get a page that asks you to ``Select and
existing key pair or create a new one''.  If you don't already have a key
pair, select ``Create a new pair'' from the menu and follow instructions.  If
you give the name \titleref{mykey}, for example, then this will download a file
\titleref{mykey.pem}.   Store this file in a directory where you can find it again,
since you will need this key in order to log in to your instance once it is
running.

You also need to change the permissions on this file so that is readable only
by the account user.  In the directory where this file lives, give the
command:

\begin{Verbatim}[commandchars=\\\{\}]
\PYG{n}{chmod} \PYG{l+m+mi}{400} \PYG{n}{mykey}\PYG{o}{.}\PYG{n}{pem}
\end{Verbatim}

If the file is more widely readable then you will not be able to use this
key to log into your instance.


\subsection{Logging on to your instance}
\label{aws:logging-on-to-your-instance}
Click \titleref{View Instances} on the  page that appears to
go back to the Management Console.  Click on \titleref{Instances} on the left menu
and you should see a list of instance you
have created, in your case only one.  If the status is not yet \titleref{running}
then wait until it is (click on the \titleref{Refresh} button if necessary).

\emph{Click on the instance} and information about it should appear at the bottom
of the screen. Scroll down until you find the \titleref{Public DNS} information

Go into the directory where your key pair is stored, in a file with a name
like \titleref{mykey.pem} and you should be able to \titleref{ssh} into your instance using
the name of the public DNS, with format like:

\begin{Verbatim}[commandchars=\\\{\}]
\PYGZdl{} ssh \PYGZhy{}Y \PYGZhy{}i KEYPAIR\PYGZhy{}FILE  ubuntu@DNS
\end{Verbatim}

where KEYPAIR-FILE and DNS must be replaced by the appropriate
things, e.g. something like this:

\begin{Verbatim}[commandchars=\\\{\}]
\PYGZdl{} ssh \PYGZhy{}Y \PYGZhy{}i mykey.pem ubuntu@ec2\PYGZhy{}50\PYGZhy{}19\PYGZhy{}75\PYGZhy{}229.compute\PYGZhy{}1.amazonaws.com
\end{Verbatim}

Note:
\begin{itemize}
\item {} 
You must include \titleref{-i keypair-file}

\item {} 
You must log in as user ubuntu.

\item {} 
Including -Y in the ssh command allows X window forwarding, so that if you
give a command that opens a new window (e.g. plotting in Python) it will
appear on your local screen.

\item {} 
See the section {\hyperref[ssh:ssh]{\crossref{\DUrole{std,std-ref}{Using ssh to connect to remote computers}}}} for tips if you are using a Mac or Windows
machine.
If you use Windows, see also the Amazon notes on using \emph{putty} found at
\url{http://docs.aws.amazon.com/AWSEC2/latest/UserGuide/putty.html}.

\end{itemize}

Once you have logged into your instance, you are on Ubuntu Linux that has
software needed for this class pre-installed.  See the file \titleref{install.sh} in
the running instance to see the commands that were used to install software.

Other software is easily installed using \titleref{apt-get install}, as described
in {\hyperref[software_installation:software\string-installation]{\crossref{\DUrole{std,std-ref}{Downloading and installing software for this class}}}}.


\subsection{Transferring files to/from your instance}
\label{aws:transferring-files-to-from-your-instance}
You can use \titleref{scp} to transfer files between a running instance and
the computer on which the ssh key is stored.

From your computer (not from the instance):

\begin{Verbatim}[commandchars=\\\{\}]
\PYGZdl{} scp \PYGZhy{}i KEYPAIR\PYGZhy{}FILE FILE\PYGZhy{}TO\PYGZhy{}SEND ubuntu@DNS:REMOTE\PYGZhy{}DIRECTORY
\end{Verbatim}

where DNS is the public DNS of the instance and \titleref{REMOTE-DIRECTORY} is
the path (relative to home directory)
where you want the file to end up.  You can leave off
\titleref{:REMOTE-DIRECTORY} if you want it to end up in your home directory.

Going the other way, you can download a file from your instance to
your own computer via:

\begin{Verbatim}[commandchars=\\\{\}]
\PYGZdl{} scp \PYGZhy{}i KEYPAIR\PYGZhy{}FILE ubuntu@DNS:FILE\PYGZhy{}TO\PYGZhy{}GET .
\end{Verbatim}

to retrieve the file named \titleref{FILE-TO-GET} (which might include a path
relative to the home directory) into the current directory.


\subsection{Stopping your instance}
\label{aws:stopping-your-instance}
Once you are done computing for the day, you will probably want to stop your
instance so you won't be charged while it's sitting idle.  You can do this
by selecting the instance from the Management Console / Instances, and then
select \titleref{Stop} from the \titleref{Instance Actions} menu.

You can restart it later and it will be in the same state you left it in.
But note that it will probably have a new Public DNS!


\subsection{Creating your own AMI}
\label{aws:creating-your-own-ami}
If you add additional software and want to save a disk image of your
improved virtual machine (e.g. in order to launch additional images in the
future to run multiple jobs at once), simply click on \titleref{Create Image (EBS
AMI)} from the \titleref{Instance Actions} menu.


\subsection{Viewing webpages directly from your instance}
\label{aws:viewing-webpages-directly-from-your-instance}
An apache webserver should already be running in your instance,
but to allow people (including yourself) to view
webpages you will need to adjust the security settings.  Go back to the
Management Console and click on \titleref{Security Groups} on the left menu.  Select
\titleref{launch-wizard-1} and then click on \titleref{Inbound}.  Click on \titleref{+Add rule}.
You should see a list of ports
that only lists 22 (SSH).  You want to add port 80 (HTTP).  Select HTTP from
the drop-down menu that says \titleref{Custom TCP Rule} and then click on \titleref{+Add rule}
and \titleref{Apply Rule Change}.

Now you should be able to point your browser to \titleref{http://DNS} where \titleref{DNS} is
replaced by the Public DNS name of your instance, the same as used for the
\titleref{ssh} command.  So for the example above, this would be

\begin{Verbatim}[commandchars=\\\{\}]
\PYG{n}{http}\PYG{p}{:}\PYG{o}{/}\PYG{o}{/}\PYG{n}{ec2}\PYG{o}{\PYGZhy{}}\PYG{l+m+mi}{50}\PYG{o}{\PYGZhy{}}\PYG{l+m+mi}{19}\PYG{o}{\PYGZhy{}}\PYG{l+m+mi}{75}\PYG{o}{\PYGZhy{}}\PYG{l+m+mf}{229.}\PYG{n}{compute}\PYG{o}{\PYGZhy{}}\PYG{l+m+mf}{1.}\PYG{n}{amazonaws}\PYG{o}{.}\PYG{n}{com}
\end{Verbatim}

The page being displayed can be found in \titleref{/var/www/index.html} on your
instance.  Any files you want to be visible on the web should be in
\titleref{/var/www}, or it is sufficient to have a link from this directory to where
they are located (created with the \titleref{ln -s} command in linux).

So, for example, you could do the following (this has already been done if
you start with the uwhpsc2104 AMI):

\begin{Verbatim}[commandchars=\\\{\}]
\PYGZdl{} cd \PYGZdl{}HOME
\PYGZdl{} mkdir public      \PYGZsh{} create a directory for posting things
\PYGZdl{} chmod 755 public  \PYGZsh{} make it readable by others
\PYGZdl{} sudo ln \PYGZhy{}s \PYGZdl{}HOME/public /var/www/public
\end{Verbatim}

then you can see the contents of your \$HOME/public directory at:

\begin{Verbatim}[commandchars=\\\{\}]
\PYG{n}{http}\PYG{p}{:}\PYG{o}{/}\PYG{o}{/}\PYG{n}{ec2}\PYG{o}{\PYGZhy{}}\PYG{l+m+mi}{50}\PYG{o}{\PYGZhy{}}\PYG{l+m+mi}{19}\PYG{o}{\PYGZhy{}}\PYG{l+m+mi}{75}\PYG{o}{\PYGZhy{}}\PYG{l+m+mf}{229.}\PYG{n}{compute}\PYG{o}{\PYGZhy{}}\PYG{l+m+mf}{1.}\PYG{n}{amazonaws}\PYG{o}{.}\PYG{n}{com}\PYG{o}{/}\PYG{n}{public}
\end{Verbatim}

Remember to change the DNS above to the right thing for your own instance!


\section{Software Carpentry}
\label{software_carpentry:software-carpentry}\label{software_carpentry::doc}\label{software_carpentry:id1}
\href{http://software-carpentry.org}{Sofware Carpentry}
started out as a course at the University of Toronto
taught by Greg Wilson.  He has since left the university and turned this
into a major effort to help researchers be more productive by teaching them
basic computing skills.

Their ``boot camps'' are taught at many universities and other insitutions
around the world, check the \href{http://software-carpentry.org/bootcamps/index.html\#calendar}{calendar} to see if
one is coming up near you.

Some sections particularly relevant to this course, with links to videos:
\begin{itemize}
\item {} 
\href{http://software-carpentry.org/4\_0/shell/}{The Shell}

\item {} 
\href{http://software-carpentry.org/4\_0/vc/}{Version Control}
(using Subversion -- svn)

\item {} 
\href{http://software-carpentry.org/4\_0/test/}{Testing}

\item {} 
\href{http://software-carpentry.org/4\_0/python/}{Python}

\item {} 
\href{http://software-carpentry.org/4\_0/sysprog/}{Systems Programming}
(from Python)

\item {} 
\href{http://software-carpentry.org/4\_0/oop/}{Classes and Objects}
(object oriented programming)

\item {} 
\href{http://software-carpentry.org/4\_0/make/}{Make}

\item {} 
\href{http://software-carpentry.org/4\_0/matrix/}{Matrix programming} (linear
algebra, NumPy)

\item {} 
\href{http://software-carpentry.org/4\_0/regexp/}{Regular expressions}

\end{itemize}


\chapter{Technical Topics}
\label{index:toc-technical-topics}\label{index:technical-topics}

\section{Shells}
\label{shells:id1}\label{shells::doc}\label{shells:shells}
A shell is a program that allows you to interact with the computer's
operating system or some software package by typing in commands.  The shell
interprets (parses) the commands and typically immediately performs some
action as a result.  Sometimes a shell is called a \emph{command line interface}
(CLI), as opposed to a \emph{graphical user interface} (GUI), which generally is
more point-and-click.

On a Windows system, most people use the point-and-click approach,
though it is also possible to open a window in command-line mode
for its DOS operating system. Note that DOS is different from Unix, and
we will \emph{not} be using DOS.  Using \DUrole{xref,std,std-ref}{cygwin} is one way to get a
unix-like environment on Windows, but if have a Windows PC, we
recommend that you use one of the other options listed in
{\hyperref[software_installation:software\string-installation]{\crossref{\DUrole{std,std-ref}{Downloading and installing software for this class}}}}.

On a Unix or Linux computer, people normally use a shell in a ``terminal
window'' to interact with the computer, although most flavors of Linux also
have point-and-click interfaces depending on what ``Window manager'' is being
used.

On a Mac there is also the option of using a Unix shell in a terminal window
(go to Applications --\textgreater{} Untilities --\textgreater{} Terminal to open a terminal).
The Mac OS X operating system (also known as Leopard, Lion,
etc. depending on version) is essentially a flavor of Unix.


\subsection{Unix shells}
\label{shells:unix-shells}
See also the Software Carpentry lectures on \href{http://software-carpentry.org/4\_0/shell/index.html}{The Shell}.

When a terminal opens, it shows a \emph{prompt} to indicate that it is waiting
for input. In these notes a single \$ will generally be used to indicate a
Unix prompt, though your system might give something different.  Often the
name of the computer appears in the prompt.   (See {\hyperref[unix:prompt]{\crossref{\DUrole{std,std-ref}{Setting the prompt}}}} for
information on how you can change the Unix prompt to your liking.)

Type a command at the prompt and hit return, and in general you should get
some response followed by a new prompt.  For example:

\begin{Verbatim}[commandchars=\\\{\}]
\PYGZdl{} pwd
/Users/rjl/
\PYGZdl{}
\end{Verbatim}

In Unix the \emph{pwd} command means ``print working directory'', and the result is
the full path to the directory you are currently working in.  (Directories
are called ``folders'' on windows systems.)  The output above shows that on my
computer at the top level there is a directory named \emph{/Users} that has a
subdirectory for each distinct user of the computer.  The directory
\emph{/Users/rjl} is where Randy LeVeque's files are stored, and within this we
are several levels down.

To see what files are in the current working directory, the \emph{ls} (list)
command can be used:

\begin{Verbatim}[commandchars=\\\{\}]
\PYGZdl{} ls
\end{Verbatim}

For more about Unix commands, see the section {\hyperref[unix:unix]{\crossref{\DUrole{std,std-ref}{Unix, Linux, and OS X}}}}.

There are actually several different shells that have been developed for
Unix, which have somewhat different command names and capabilities.  Basic
commands like \emph{pwd} and \emph{ls} (and many others) are the same for any Unix
shell, but they more complicated things may differ.

In this class, we will assume you are using the bash shell (see {\hyperref[unix:bash]{\crossref{\DUrole{std,std-ref}{The bash shell}}}}).
See {\hyperref[unix:unix]{\crossref{\DUrole{std,std-ref}{Unix, Linux, and OS X}}}} for more Unix commands.


\subsubsection{Matlab shell}
\label{shells:matlab-shell}
If you have used Matlab before, you are familiar with the Matlab shell,
which uses the prompt \textgreater{}\textgreater{}.  If you use the GUI version of Matlab then this
shell is running in the ``Command window''.  You can also run Matlab from the
command line in Unix, resulting in the Matlab prompt simply showing up in
your terminal window.  To start it this way, use the \emph{-nojvm} option:

\begin{Verbatim}[commandchars=\\\{\}]
\PYGZdl{} matlab \PYGZhy{}nojvm
\PYGZgt{}\PYGZgt{}
\end{Verbatim}


\subsubsection{Python shell}
\label{shells:python-shell}
We will use Python extensively in this class.  For more information see the
section {\hyperref[python:python]{\crossref{\DUrole{std,std-ref}{Python}}}}.

Most Unix (Linux, OSX) computers have Python available by default, invoked by:

\begin{Verbatim}[commandchars=\\\{\}]
\PYGZdl{} python
Python 2.7.3 (default, Aug 28 2012, 13:37:53)
[GCC 4.2.1 Compatible Apple Clang 4.0 ((tags/Apple/clang\PYGZhy{}421.0.60))] on darwin
Type \PYGZdq{}help\PYGZdq{}, \PYGZdq{}copyright\PYGZdq{}, \PYGZdq{}credits\PYGZdq{} or \PYGZdq{}license\PYGZdq{} for more information.
\PYGZgt{}\PYGZgt{}\PYGZgt{}
\end{Verbatim}

This prints out some information about the version of Python and then gives
the standard Python prompt, \textgreater{}\textgreater{}\textgreater{}.  At this point you are in the Python shell
and any commands you type will be interpreted by this shell rather than the
Unix shell.  You can now type Python commands, e.g.:

\begin{Verbatim}[commandchars=\\\{\}]
\PYG{g+gp}{\PYGZgt{}\PYGZgt{}\PYGZgt{} }\PYG{n}{x} \PYG{o}{=} \PYG{l+m+mi}{3}\PYG{o}{+}\PYG{l+m+mi}{4}
\PYG{g+gp}{\PYGZgt{}\PYGZgt{}\PYGZgt{} }\PYG{n}{x}
\PYG{g+go}{7}
\PYG{g+gp}{\PYGZgt{}\PYGZgt{}\PYGZgt{} }\PYG{n}{x}\PYG{o}{+}\PYG{l+m+mi}{2}
\PYG{g+go}{9}
\PYG{g+gp}{\PYGZgt{}\PYGZgt{}\PYGZgt{} }\PYG{l+m+mi}{4}\PYG{o}{/}\PYG{l+m+mi}{3}
\PYG{g+go}{1}
\end{Verbatim}

The last line might be cause for concern, since 4/3 is not 1.  For more
about this, see {\hyperref[numerical_python:numerical\string-python]{\crossref{\DUrole{std,std-ref}{Numerics in Python}}}}.  The problem is that since 4 and 3 are
both integers, Python gives an integer result.  To get a better result,
express 4 and 3 as real numbers (called {\color{red}\bfseries{}*}float*s in Python) by adding
decimal points:

\begin{Verbatim}[commandchars=\\\{\}]
\PYG{g+gp}{\PYGZgt{}\PYGZgt{}\PYGZgt{} }\PYG{l+m+mf}{4.}\PYG{o}{/}\PYG{l+m+mf}{3.}
\PYG{g+go}{1.3333333333333333}
\end{Verbatim}

The standard Python shell is very basic; you can type in Python commands and
it will interpret them, but it doesn't do much else.


\subsection{IPython shell}
\label{shells:id4}\label{shells:ipython-shell}
A much better shell for Python is the \emph{IPython shell}, which has
extensive documentation at \phantomsection\label{shells:id5}{\hyperref[biblio:ipython\string-documentation]{\crossref{{[}IPython-documentation{]}}}}.

Note that IPython has a different sort of prompt:

\begin{Verbatim}[commandchars=\\\{\}]
\PYGZdl{} ipython

Python 2.7.2 (default, Jun 20 2012, 16:23:33)
Type \PYGZdq{}copyright\PYGZdq{}, \PYGZdq{}credits\PYGZdq{} or \PYGZdq{}license\PYGZdq{} for more information.

IPython 0.14.dev \PYGZhy{}\PYGZhy{} An enhanced Interactive Python.
?         \PYGZhy{}\PYGZgt{} Introduction and overview of IPython\PYGZsq{}s features.
\PYGZpc{}quickref \PYGZhy{}\PYGZgt{} Quick reference.
help      \PYGZhy{}\PYGZgt{} Python\PYGZsq{}s own help system.
object?   \PYGZhy{}\PYGZgt{} Details about \PYGZsq{}object\PYGZsq{}, use \PYGZsq{}object??\PYGZsq{} for extra details.

In [1]: x = 4./3.

In [2]: x
Out[2]: 1.3333333333333333

In [3]:
\end{Verbatim}

The prompt has the form \emph{In {[}n{]}} and any output is preceeded by
by \emph{Out {[}n{]}}.  IPython stores all the inputs and outputs in an array of
strings, allowing you to later reuse expressions.

For more about some handy features of this shell, see \DUrole{xref,std,std-ref}{ipython}.

The IPython shell also is programmed to recognize many commands that are not
Python commands, making it easier to do many things.  For example, IPython
recognizes \emph{pwd}, \emph{ls} and various other Unix commands, e.g. to print out
the working directory you are in while in IPython, just do:

\begin{Verbatim}[commandchars=\\\{\}]
\PYG{n}{In} \PYG{p}{[}\PYG{l+m+mi}{3}\PYG{p}{]}\PYG{p}{:} \PYG{n}{pwd}
\end{Verbatim}

Note that IPython is not installed by default on most computers, you will
have to download it and install yourself (see \phantomsection\label{shells:id6}{\hyperref[biblio:ipython\string-documentation]{\crossref{{[}IPython-documentation{]}}}}).  It
is installed on the {\hyperref[vm:vm]{\crossref{\DUrole{std,std-ref}{Virtual Machine for this class {[}2014 Edition{]}}}}}.

If you get hooked on the IPython shell, you can even use it as a Unix shell,
see \href{http://ipython.scipy.org/doc/rel-0.10/html/interactive/shell.html}{documentation}.


\subsection{Further reading}
\label{shells:further-reading}
See \phantomsection\label{shells:id7}{\hyperref[biblio:ipython\string-documentation]{\crossref{{[}IPython-documentation{]}}}}


\section{Unix, Linux, and OS X}
\label{unix:unix}\label{unix::doc}\label{unix:unix-linux-and-os-x}
A brief introduction to Unix shells appears in the section {\hyperref[shells:shells]{\crossref{\DUrole{std,std-ref}{Shells}}}}.
Please read that first before continuing here.

There are many Unix commands and most of them have many optional arguments
(generally specified by adding something like -x after the command name,
where x is some letter).   Only a few important commands are reviewed here.
See the references (e.g. \phantomsection\label{unix:id1}{\hyperref[biblio:wikipedia\string-unix\string-utilities]{\crossref{{[}Wikipedia-unix-utilities{]}}}})
for links with many more details.


\subsection{pwd and cd}
\label{unix:pwd-and-cd}
The command name \emph{pwd} stands for ``print working directory'' and tells you
the full path to the directory you are currently working in, e.g.:

\begin{Verbatim}[commandchars=\\\{\}]
\PYGZdl{} pwd
/Users/rjl/uwhpsc
\end{Verbatim}

To change directories, use the \emph{cd} command, either relative to the current
directory, or as an absolute path (starting with a ``/'', like the output of
the above pwd command).  To go up one level:

\begin{Verbatim}[commandchars=\\\{\}]
\PYGZdl{} cd ..
\PYGZdl{} pwd
/Users/rjl
\end{Verbatim}


\subsection{ls}
\label{unix:ls}
\emph{ls} is used to list the contents of the current working directory.
As with many commands, \emph{ls} can take both \emph{arguments} and \emph{options}.  An
option tells the shell more about what you want the command to do, and is
preceded by a dash, e.g.:

\begin{Verbatim}[commandchars=\\\{\}]
\PYGZdl{} ls \PYGZhy{}l
\end{Verbatim}

The \emph{-l} option tells \emph{ls} to give a long listing that contains additional
information about each file, such as how large it is, who owns it, when it
was last modified, etc.   The first 10 mysterious characters tell who has
permission to read, write, or execute the file, see \href{http://en.wikipedia.org/wiki/File\_system\_permissions}{{[}Wikipedia{]}}.

Commands often also take \emph{arguments}, just like a function takes an argument
and the function value depends on the argument supplied.  In the case of
\emph{ls}, you can specify a list of files to apply \emph{ls} to. For example, if we
only want to list the information about a specific file:

\begin{Verbatim}[commandchars=\\\{\}]
\PYGZdl{} ls \PYGZhy{}l fname
\end{Verbatim}

You can also use the \emph{wildcard} * character to match more than one file:

\begin{Verbatim}[commandchars=\\\{\}]
\PYGZdl{} ls *.x
\end{Verbatim}

If you type
\begin{quote}

\$ ls -F
\end{quote}

then directories will show up with a trailing / and executable files with a
trailing asterisk, useful in distinguishing these from ordinary files.

When you type \emph{ls} with no arguments it generally shows most files and
subdirectories, but may not show them all.  By default it does not show
files or directories that start with a period (dot).  These are ``hidden''
files such as \emph{.bashrc} described in Section {\hyperref[unix:bashrc]{\crossref{\DUrole{std,std-ref}{.bashrc file}}}}.

To list these hidden files use:

\begin{Verbatim}[commandchars=\\\{\}]
\PYGZdl{} ls \PYGZhy{}aF
\end{Verbatim}

Note that this will also list two directories \emph{./} and \emph{../}  These appear
in every directory and always refer to the current directory and the parent
directory up one level.  The latter is frequently used to move up one level
in the directory structure via:

\begin{Verbatim}[commandchars=\\\{\}]
\PYGZdl{} cd ..
\end{Verbatim}

For more about \emph{ls}, try:

\begin{Verbatim}[commandchars=\\\{\}]
\PYGZdl{} man ls
\end{Verbatim}

Note that this invokes the \emph{man} command (manual pages) with the argument
\emph{ls}, and causes Unix to print out some user manual information about \emph{ls}.

If you try this, you will probably get one page of information with a `:' at
the bottom.  At this point you are in a different shell, one designed to
let you scroll or search through a long file within a terminal.  The `:' is
the prompt for this shell.  The commands you can type at this point are
different than those in the Unix shell.  The most useful are:

\begin{Verbatim}[commandchars=\\\{\}]
\PYG{p}{:} \PYG{n}{q}  \PYG{p}{[}\PYG{n}{to} \PYG{n}{quit} \PYG{n}{out} \PYG{n}{of} \PYG{n}{this} \PYG{n}{shell} \PYG{o+ow}{and} \PYG{k}{return} \PYG{n}{to} \PYG{n}{Unix}\PYG{p}{]}
\PYG{p}{:} \PYG{o}{\PYGZlt{}}\PYG{n}{SPACE}\PYG{o}{\PYGZgt{}}   \PYG{p}{[}\PYG{n}{tap} \PYG{n}{the} \PYG{n}{Spacebar} \PYG{n}{to} \PYG{n}{display} \PYG{n}{the} \PYG{n+nb}{next} \PYG{n}{screenfull}\PYG{p}{]}
\PYG{p}{:} \PYG{n}{b}  \PYG{p}{[}\PYG{n}{go} \PYG{n}{back} \PYG{n}{to} \PYG{n}{the} \PYG{n}{previous} \PYG{n}{screenfull}\PYG{p}{]}
\end{Verbatim}


\subsection{more, less, cat, head, tail}
\label{unix:more-less-cat-head-tail}
The same technique to paging through a long file can be applied to your own
files using the \emph{less} command (an improvement over the original \emph{more}
command of Unix), e.g.:

\begin{Verbatim}[commandchars=\\\{\}]
\PYGZdl{} less filename
\end{Verbatim}

will display the first screenfull of the file and give the : prompt.

The \emph{cat} command prints the entire file rather than a page at a time.
\emph{cat} stands for ``catenate'' and \emph{cat} can also be used to combine multiple
files into a single file:

\begin{Verbatim}[commandchars=\\\{\}]
\PYGZdl{} cat file1 file2 file3 \PYGZgt{} bigfile
\end{Verbatim}

The contents of all three files, in the order given, will now be in
\emph{bigfile}.  If you leave off the ``\textgreater{} bigfile'' then the results go to the
screen instead of to a new file, so ``cat file1'' just prints file1 on the
screen.  If you leave off file names before ``\textgreater{}'' it takes input from the
screen until you type \textless{}ctrl\textgreater{}-d, as used in the example at \DUrole{xref,std,std-ref}{myhg}.

Sometimes you want to just see the first 10 lines or the last 5 lines of a
file, for example.  Try:

\begin{Verbatim}[commandchars=\\\{\}]
\PYGZdl{} head \PYGZhy{}10 filename
\PYGZdl{} tail \PYGZhy{}5 filename
\end{Verbatim}


\subsection{removing, moving, copying files}
\label{unix:removing-moving-copying-files}
If you want to get rid of a file named \emph{filename}, use \emph{rm}:

\begin{Verbatim}[commandchars=\\\{\}]
\PYGZdl{} rm \PYGZhy{}i filename
remove filename?
\end{Verbatim}

The -i flags forces \emph{rm} to ask before deleting, a good precaution.  Many
systems are set up so this is the default, possibly by including the
following line in the {\hyperref[unix:bashrc]{\crossref{\DUrole{std,std-ref}{.bashrc file}}}}:

\begin{Verbatim}[commandchars=\\\{\}]
\PYG{n}{alias} \PYG{n}{rm}\PYG{o}{=}\PYG{l+s}{\PYGZsq{}}\PYG{l+s}{rm \PYGZhy{}i}\PYG{l+s}{\PYGZsq{}}
\end{Verbatim}

If you want to force removal without asking (useful if you're removing a
bunch of files at once and you're sure about what you're doing), use the -f
flag.

To rename a file or move to a different place (e.g. a different directory):

\begin{Verbatim}[commandchars=\\\{\}]
\PYGZdl{} mv oldfile newfile
\end{Verbatim}

each can be a full or relative path to a location outside the current
working directory.

To copy a file use \emph{cp}:

\begin{Verbatim}[commandchars=\\\{\}]
\PYGZdl{} cp oldfile newfile
\end{Verbatim}

The original \emph{oldfile} still exists.
To copy an entire directory structure recursively (copying all files in it
and any subdirectories the same way), use ``cp -r'':

\begin{Verbatim}[commandchars=\\\{\}]
\PYGZdl{} cp \PYGZhy{}r olddir newdir
\end{Verbatim}


\subsection{background and foreground jobs}
\label{unix:background-and-foreground-jobs}
When you run a program that will take a long time to execute, you might want
to run it in \emph{background} so that you can continue to use the Unix command
line to do other things while it runs.  For example, suppose
\emph{fortrancode.exe} is a Fortran executable in your current directory
that is going to run for a long time.  You can do:

\begin{Verbatim}[commandchars=\\\{\}]
\PYGZdl{} ./fortrancode.exe \PYGZam{}
[1] 15442
\end{Verbatim}

if you now hit return you should get the Unix prompt back and can continue
working.

The ./ before the command in the example above is
to tell Unix to run the executable in this
directory (see \DUrole{xref,std,std-ref}{paths}), and the \& at the end of the line tells it to
run in background.  The ``{[}1{]} 15442'' means that it is background job number 1
run from this shell and that it has the \emph{processor id} 15442.

If you want to find out what jobs you have running in background and their
pid's, try:

\begin{Verbatim}[commandchars=\\\{\}]
\PYGZdl{} jobs \PYGZhy{}l
[1]+ 15443 Running                 ./fortrancode.exe \PYGZam{}
\end{Verbatim}

You can bring the job back to the foreground with:

\begin{Verbatim}[commandchars=\\\{\}]
\PYGZdl{} fg \PYGZpc{}1
\end{Verbatim}

Now you won't get a Unix prompt back until the job finishes (or you put it
back into background as described below). The \%1 refers to job 1.  In this
example \emph{fg} alone would suffice since there's only one job running, but
more generally you may have several in background.

To put a job that is foreground into background, you can often type
\textless{}ctrl\textgreater{}-z, which will pause the job and give you the prompt back:

\begin{Verbatim}[commandchars=\\\{\}]
\PYGZca{}Z
[1]+  Stopped                 ./fortrancode.exe
\PYGZdl{}
\end{Verbatim}

Note that the job is not running in background now, it is stopped.  To get
it running again in background, type:

\begin{Verbatim}[commandchars=\\\{\}]
\PYGZdl{} bg \PYGZpc{}1
\end{Verbatim}

Or you could get it running in foreground with ``fg \%1''.


\subsection{nice and top}
\label{unix:nice-and-top}
If you are running a code that will run for a long time you might want to
make sure it doesn't slow down other things you are doing.  You can do this
with the \emph{nice} command, e.g.:

\begin{Verbatim}[commandchars=\\\{\}]
\PYGZdl{} nice \PYGZhy{}n 19 ./fortrancode.exe \PYGZam{}
\end{Verbatim}

gives the job lowest priority (nice values between 1 and 19 can be used) so
it won't hog the CPU if you're also trying to edit a file at the same time,
for example.

You can change the priority of a job running in background with \emph{renice},
e.g.:

\begin{Verbatim}[commandchars=\\\{\}]
\PYGZdl{} renice \PYGZhy{}n 19 15443
\end{Verbatim}

where the last number is the process id.

Another useful command is \emph{top}.  This will fill your window with a page of
information about the jobs running on your computer that are using the most
resources currently.  See {\hyperref[top:topcommand]{\crossref{\DUrole{std,std-ref}{Unix top command}}}} for some examples.


\subsection{killing jobs}
\label{unix:killing-jobs}\label{unix:kill}
Sometimes you need to kill a job that's running, perhaps because you realize
it's going to run for too long, or you gave it or the wrong input data.  Or
you may be running a program like the IPython shell and it freezes up on you
with no way to get control back.  (This sometimes happens when plotting when
you give the \emph{pylab.show()} command, for example.)

Many programs can be killed with \textless{}ctrl\textgreater{}-c.  For this to work the job must be
running in the foreground, so you might need to first give the \emph{fg} command.

Sometimes this doesn't work, like when IPython freezes.  Then try stopping
it with \textless{}ctrl\textgreater{}-z (which should work), find out its PID, and use the \emph{kill}
command:

\begin{Verbatim}[commandchars=\\\{\}]
\PYGZdl{} jobs \PYGZhy{}l
[1]+ 15841 Suspended               ipython

\PYGZdl{} kill 15841
\end{Verbatim}

Hit return again you with luck you will see:

\begin{Verbatim}[commandchars=\\\{\}]
\PYGZdl{}
[1]+ Terminated              ipython
\PYGZdl{}
\end{Verbatim}

If not, more drastic action is needed with the -9 flag:

\begin{Verbatim}[commandchars=\\\{\}]
\PYGZdl{} kill \PYGZhy{}9 15841
\end{Verbatim}

This almost always kills a process.  Be careful what you kill.


\subsection{sudo}
\label{unix:sudo}\label{unix:id2}
A command like:

\begin{Verbatim}[commandchars=\\\{\}]
\PYGZdl{} sudo rm 70\PYGZhy{}persistent\PYGZhy{}net.rules
\end{Verbatim}

found in the section {\hyperref[vm:vm]{\crossref{\DUrole{std,std-ref}{Virtual Machine for this class {[}2014 Edition{]}}}}} means to do the remove command as super user.
You will be prompted for your password at this point.

You cannot do this unless you are registered on a list of super users. You
can do this on the VM because the \emph{amath583} account has sudo privileges. The
reason this is needed is that the file being removed here is a system file
that ordinary users are not allowed to modify or delete.

Another example is seen at {\hyperref[software_installation:apt\string-get]{\crossref{\DUrole{std,std-ref}{Software available through apt-get}}}}, where only those with super user
permission can install software on to the system.


\subsection{The bash shell}
\label{unix:bash}\label{unix:the-bash-shell}
There are several popular shells for Unix.  The examples given in these
notes assume the bash shell is used.  If you think your shell is different,
you can probably just type:

\begin{Verbatim}[commandchars=\\\{\}]
\PYGZdl{} bash
\end{Verbatim}

which will start a new bash shell and give you the bash prompt.

For more information on bash, see for example
\phantomsection\label{unix:id3}{\hyperref[biblio:bash\string-beginners\string-guide]{\crossref{{[}Bash-Beginners-Guide{]}}}}, \phantomsection\label{unix:id4}{\hyperref[biblio:gnu\string-bash]{\crossref{{[}gnu-bash{]}}}}, \phantomsection\label{unix:id5}{\hyperref[biblio:wikipedia\string-bash]{\crossref{{[}Wikipedia-bash{]}}}}.


\subsection{.bashrc file}
\label{unix:bashrc}\label{unix:bashrc-file}
Everytime you start a new bash shell, e.g. by the command above, or when you
first log in or open a new window (assuming bash is the default), a file
named ''.bashrc'' in your home directory is executed as a bash script.  You
can place in this file anything you want to have executed on startup, such
as exporting environment variables, setting paths, defining aliases, setting
your prompt the way you like it, etc.  See below for more about these
things.


\subsection{Environment variables}
\label{unix:environment-variables}\label{unix:env}
The command \emph{printenv} will print out any environment variables you have
set, e.g.:

\begin{Verbatim}[commandchars=\\\{\}]
\PYGZdl{} printenv
USER=rjl
HOME=/Users/rjl
PWD=/Users/rjl/uwhpsc/sphinx
FC=gfortran
PYTHONPATH=/Users/rjl/claw4/trunk/python:/Applications/visit1.11.2/src/lib:
PATH=/opt/local/bin:/opt/local/sbin:/Users/rjl/bin
etc.
\end{Verbatim}

You can also print just one variable by, e.g.:

\begin{Verbatim}[commandchars=\\\{\}]
\PYGZdl{} printenv HOME
/Users/rjl
\end{Verbatim}

or:

\begin{Verbatim}[commandchars=\\\{\}]
\PYGZdl{} echo \PYGZdl{}HOME
/Users/rjl
\end{Verbatim}

The latter form has \$HOME instead of HOME because we are actually \emph{using}
the variable in an echo command rather than just printing its value.  This
particular variable is useful for things like
\begin{quote}

\$ cd \$HOME/uwhpsc
\end{quote}

which will go to the uwhpsc subdirectory of your home directory no
matter where you start.

As part of Homework 1 you are instructed to define a new environment
variable to make this even easier,  for example by:

\begin{Verbatim}[commandchars=\\\{\}]
\PYGZdl{} export UWHPSC=\PYGZdl{}HOME/uwhpsc
\end{Verbatim}

Note there are no spaces around the =.   This defines a new environment
variable and \emph{exports} it, so that it can be used by other programs you
might run from this shell (not so important for our purposes, but sometimes
necessary).

You can now just do:

\begin{Verbatim}[commandchars=\\\{\}]
\PYGZdl{} cd \PYGZdl{}UWHPSC
\end{Verbatim}

to go to this directory.

Note that I have set an environment variable FC as:

\begin{Verbatim}[commandchars=\\\{\}]
\PYGZdl{} printenv FC
gfortran
\end{Verbatim}

This environment variable is used in some Makefiles (see {\hyperref[makefiles:makefiles]{\crossref{\DUrole{std,std-ref}{Makefiles}}}})
to determine which Fortran compiler to use in compiling Fortran codes.


\subsection{PATH and other search paths}
\label{unix:path-and-other-search-paths}\label{unix:unix-path}
Whenever you type a command at the Unix prompt, the shell looks for a
program to run.  This is true of built-in commands and also new commands you
might define or programs that have been installed.  To figure out where to
look for such programs, the shell searches through the directories specified
by the PATH variable (see {\hyperref[unix:env]{\crossref{\DUrole{std,std-ref}{Environment variables}}}} above).  This might look something
like:

\begin{Verbatim}[commandchars=\\\{\}]
\PYGZdl{} printenv PATH
PATH=/usr/local/bin:/usr/bin:/Users/rjl/bin
\end{Verbatim}

This gives a list of directories to search through, in order, separated by
'':''.   The PATH is usually longer than this, but in the above example there
are 3 directories on the path.  The first two are general system-wide
repositories and the last one is my own \emph{bin} directory (bin stands for
binary since the executables are often binary files, though often the bin
directory also contains shell scripts or other programs in text).


\subsection{which}
\label{unix:id6}\label{unix:which}
The \emph{which} command is useful for finding out the full path to the code that
is actually being executed when you type a command, e.g.:

\begin{Verbatim}[commandchars=\\\{\}]
\PYGZdl{} which gfortran
/usr/bin/gfortran

\PYGZdl{} which f77
\PYGZdl{}
\end{Verbatim}

In the latter case it found no program called f77 in the search path, either
because it is not installed or because the proper diretory is not on the
PATH.

Some programs require their own path to be set if it needs to search for
input files.  For example, you can set MATLABPATH or PYTHONPATH
to be a list of directories (separated by '':'') to search for .m files
when you execute a command in Matlab, or for .py files when you import
a module in Python.


\subsection{Setting the prompt}
\label{unix:prompt}\label{unix:setting-the-prompt}
If you don't like the prompt bash is using you can change it by changing the
environment variable PS1, e.g.:

\begin{Verbatim}[commandchars=\\\{\}]
\PYGZdl{} PS1=\PYGZsq{}myprompt* \PYGZsq{}
myprompt*
\end{Verbatim}

This is now your prompt.  There are various special characters you can use
to specify things in your prompts, for example:

\begin{Verbatim}[commandchars=\\\{\}]
\PYGZdl{} PS1=\PYGZsq{}[\PYGZbs{}W] \PYGZbs{}h\PYGZpc{} \PYGZsq{}
[sphinx] aspen\PYGZpc{}
\end{Verbatim}

tells me that I'm currently in a directory named sphinx on a computer named
aspen.  This is handy to keep track of where you are, and what machine the
shell is running on if you might be using ssh to connect to remote machines
in some windows.

Once you find something you like, you can put this command in your .bashrc
file.


\subsection{Further reading}
\label{unix:further-reading}
\phantomsection\label{unix:id7}{\hyperref[biblio:wikipedia\string-unix\string-utilities]{\crossref{{[}Wikipedia-unix-utilities{]}}}}


\section{Unix \texttt{top} command}
\label{top:unix-top-command}\label{top:topcommand}\label{top::doc}
The Unix \code{top} command is a very useful way to see what programs are
currently running on the system and how heavily they are using system
resources.  (The command is named ``top'' because it shows the top users
of the system.)  It is a good idea to run \code{top} to check for other
users running large programs before running one yourself on a shared
computer (such as the Applied Math servers), and you can also use it
to monitor your own programs.


\subsection{Running \texttt{top}}
\label{top:running-top}
To run \code{top}, simply type:

\begin{Verbatim}[commandchars=\\\{\}]
\PYGZdl{} top
\end{Verbatim}

at the command line.  \code{top} will fill your terminal window with a
real-time display of system status, with a summary area displaying
general information about memory and processor usage in the first few
lines, followed by a list of processes and information about them.  An
example display is shown below, with the process list truncated for
brevity:

\begin{Verbatim}[commandchars=\\\{\}]
\PYG{n}{top} \PYG{o}{\PYGZhy{}} \PYG{l+m+mi}{14}\PYG{p}{:}\PYG{l+m+mi}{45}\PYG{p}{:}\PYG{l+m+mi}{34} \PYG{n}{up}  \PYG{l+m+mi}{6}\PYG{p}{:}\PYG{l+m+mi}{32}\PYG{p}{,}  \PYG{l+m+mi}{2} \PYG{n}{users}\PYG{p}{,}  \PYG{n}{load} \PYG{n}{average}\PYG{p}{:} \PYG{l+m+mf}{0.78}\PYG{p}{,} \PYG{l+m+mf}{0.61}\PYG{p}{,} \PYG{l+m+mf}{0.59}
\PYG{n}{Tasks}\PYG{p}{:} \PYG{l+m+mi}{110} \PYG{n}{total}\PYG{p}{,}   \PYG{l+m+mi}{3} \PYG{n}{running}\PYG{p}{,} \PYG{l+m+mi}{106} \PYG{n}{sleeping}\PYG{p}{,}   \PYG{l+m+mi}{0} \PYG{n}{stopped}\PYG{p}{,}   \PYG{l+m+mi}{1} \PYG{n}{zombie}
\PYG{n}{Cpu}\PYG{p}{(}\PYG{n}{s}\PYG{p}{)}\PYG{p}{:} \PYG{l+m+mf}{75.0}\PYG{o}{\PYGZpc{}}\PYG{n}{us}\PYG{p}{,}  \PYG{l+m+mf}{0.6}\PYG{o}{\PYGZpc{}}\PYG{n}{sy}\PYG{p}{,}  \PYG{l+m+mf}{0.0}\PYG{o}{\PYGZpc{}}\PYG{n}{ni}\PYG{p}{,} \PYG{l+m+mf}{24.4}\PYG{o}{\PYGZpc{}}\PYG{n+nb}{id}\PYG{p}{,}  \PYG{l+m+mf}{0.0}\PYG{o}{\PYGZpc{}}\PYG{n}{wa}\PYG{p}{,}  \PYG{l+m+mf}{0.0}\PYG{o}{\PYGZpc{}}\PYG{n}{hi}\PYG{p}{,}  \PYG{l+m+mf}{0.0}\PYG{o}{\PYGZpc{}}\PYG{n}{si}\PYG{p}{,}
\PYG{l+m+mf}{0.0}\PYG{o}{\PYGZpc{}}\PYG{n}{st}
\PYG{n}{Mem}\PYG{p}{:}    \PYG{l+m+mi}{507680}\PYG{n}{k} \PYG{n}{total}\PYG{p}{,}   \PYG{l+m+mi}{491268}\PYG{n}{k} \PYG{n}{used}\PYG{p}{,}    \PYG{l+m+mi}{16412}\PYG{n}{k} \PYG{n}{free}\PYG{p}{,}    \PYG{l+m+mi}{24560}\PYG{n}{k} \PYG{n}{buffers}
\PYG{n}{Swap}\PYG{p}{:}        \PYG{l+m+mi}{0}\PYG{n}{k} \PYG{n}{total}\PYG{p}{,}        \PYG{l+m+mi}{0}\PYG{n}{k} \PYG{n}{used}\PYG{p}{,}        \PYG{l+m+mi}{0}\PYG{n}{k} \PYG{n}{free}\PYG{p}{,}   \PYG{l+m+mi}{316368}\PYG{n}{k} \PYG{n}{cached}

  \PYG{n}{PID} \PYG{n}{USER}      \PYG{n}{PR}  \PYG{n}{NI}  \PYG{n}{VIRT}  \PYG{n}{RES}  \PYG{n}{SHR} \PYG{n}{S} \PYG{o}{\PYGZpc{}}\PYG{n}{CPU} \PYG{o}{\PYGZpc{}}\PYG{n}{MEM}    \PYG{n}{TIME}\PYG{o}{+}  \PYG{n}{COMMAND}
 \PYG{l+m+mi}{2342} \PYG{n}{uwhpsc}    \PYG{l+m+mi}{20}   \PYG{l+m+mi}{0}  \PYG{l+m+mi}{8380} \PYG{l+m+mi}{1212}  \PYG{l+m+mi}{796} \PYG{n}{R}  \PYG{l+m+mi}{300}  \PYG{l+m+mf}{0.2}   \PYG{l+m+mi}{0}\PYG{p}{:}\PYG{l+m+mf}{28.55} \PYG{n}{jacobi2d}\PYG{o}{.}\PYG{n}{exe}
  \PYG{l+m+mi}{842} \PYG{n}{root}      \PYG{l+m+mi}{20}   \PYG{l+m+mi}{0}  \PYG{l+m+mi}{100}\PYG{n}{m}  \PYG{l+m+mi}{24}\PYG{n}{m} \PYG{l+m+mi}{9780} \PYG{n}{R}    \PYG{l+m+mi}{2}  \PYG{l+m+mf}{5.0} \PYG{l+m+mi}{178}\PYG{p}{:}\PYG{l+m+mf}{27.50} \PYG{n}{Xorg}
 \PYG{l+m+mi}{1051} \PYG{n}{uwhpsc}    \PYG{l+m+mi}{20}   \PYG{l+m+mi}{0} \PYG{l+m+mi}{40236}  \PYG{l+m+mi}{11}\PYG{n}{m} \PYG{l+m+mi}{8904} \PYG{n}{S}    \PYG{l+m+mi}{1}  \PYG{l+m+mf}{2.3}   \PYG{l+m+mi}{0}\PYG{p}{:}\PYG{l+m+mf}{01.53} \PYG{n}{xfce4}\PYG{o}{\PYGZhy{}}\PYG{n}{terminal}
    \PYG{l+m+mi}{1} \PYG{n}{root}      \PYG{l+m+mi}{20}   \PYG{l+m+mi}{0}  \PYG{l+m+mi}{3528} \PYG{l+m+mi}{1864} \PYG{l+m+mi}{1304} \PYG{n}{S}    \PYG{l+m+mi}{0}  \PYG{l+m+mf}{0.4}   \PYG{l+m+mi}{0}\PYG{p}{:}\PYG{l+m+mf}{01.12} \PYG{n}{init}
    \PYG{l+m+mi}{2} \PYG{n}{root}      \PYG{l+m+mi}{20}   \PYG{l+m+mi}{0}     \PYG{l+m+mi}{0}    \PYG{l+m+mi}{0}    \PYG{l+m+mi}{0} \PYG{n}{S}    \PYG{l+m+mi}{0}  \PYG{l+m+mf}{0.0}   \PYG{l+m+mi}{0}\PYG{p}{:}\PYG{l+m+mf}{00.00} \PYG{n}{kthreadd}
\end{Verbatim}


\subsection{Summary area}
\label{top:summary-area}
The summary area shows a great deal of information about the general
state of the system.  The main highlights are the third through fifth
lines, which show CPU and memory usage.

CPU is given as percentages spent doing various tasks.  The
abbreviations have the following meanings:
\begin{itemize}
\item {} 
\code{us} -- time spent running non-niced user processes (most of your
programs will fall into this category)

\item {} 
\code{sy} -- time spent running the Linux kernel and its processes

\item {} 
\code{ni} -- time spent running niced use processes

\item {} 
\code{id} -- time spent idle

\item {} 
\code{wa} -- time spent waiting for I/O

\end{itemize}

The fourth and fifth lines show the usage of physical (i.e. actual RAM
chips) and virtual memory.  The first three fields are mostly
self-explanatory, though on Linux the \code{used} memory includes disk
cache.  (Linux uses RAM that's not allocated by programs to cache data
from the disk, which can improve the computer's performance because
RAM is much faster than disk.)  \code{buffers} gives the amount
of memory used for I/O buffering, and \code{cached} is the amount used by
the disk cache.  Programs' memory allocation takes priority over
buffering and caching, so the total amount of memory available is the
sum \code{free + buffers + cached}.


\subsection{Task area}
\label{top:task-area}
The task area gives a sorted list of the processes currently running
on your system.  By default, the list is sorted in descending order of
CPU usage; the example display above is sorted by memory usage
instead.  Many different fields can be displayed in the task area; the
default fields are:
\begin{itemize}
\item {} 
\code{PID}: Process ID number.  Useful for killing processes.

\item {} 
\code{USER}: Name of the user running the process.

\item {} 
\code{PR}: Relative priority of the process.

\item {} 
\code{NI}: Nice value of the process.  Negative nice values give a
process higher priority, while positive ones give it lower
priority.

\item {} 
\code{VIRT}: Total amount of memory used by the process.  This
includes all code, data, and shared libraries being used.  The
field name comes from the fact that this includes memory that has
been swapped to disk, so it measures the total virtual memory used,
not necessarily how much of it is currently present in RAM.

\item {} 
\code{RES}: Total amount of physical (``resident'') memory used by the
process.

\item {} 
\code{S}: Process status.  The most common values are \code{S} for
``Sleeping'' or \code{R} for ``Running''.

\item {} 
\code{\%CPU}: Percent of CPU time being used by the process.  This is
relative to a single CPU; in a multiprocessor system, it may be
higher than 100\%.

\item {} 
\code{\%MEM}: Percent of available RAM being used by the process.  This
does not include data that has been swapped to disk.

\item {} 
\code{TIME+}: Total CPU time used by the process since it started.
This only counts the time the process has spent using the CPU, not
time spent sleeping.

\item {} 
\code{COMMAND}: The name of the program.

\end{itemize}


\subsection{Interactive commands}
\label{top:interactive-commands}
There are many useful commands you can issue to \code{top}; only a few of
them are listed here.  For more information, see the \code{top} manual
page.

\begin{tabulary}{\linewidth}{|L|L|}
\hline
\textsf{\relax 
Command
} & \textsf{\relax 
Meaning
}\\
\hline
\code{q}
 & 
Quit
\\
\hline
\code{?}
 & 
Help
\\
\hline
\code{u}
 & 
Select processes belonging to a particular user.  Useful on shared systems.
\\
\hline
\code{k}
 & 
Kill a process
\\
\hline
\code{F}
 & 
Select which field to sort by
\\
\hline\end{tabulary}



\subsection{Further reading}
\label{top:further-reading}
For more information, see the quick reference guide at
\url{http://www.oreillynet.com/linux/cmd/cmd.csp?path=t/top}, or type
\code{man top} in a Unix terminal window to read the manual page.


\section{Using ssh to connect to remote computers}
\label{ssh:using-ssh-to-connect-to-remote-computers}\label{ssh::doc}\label{ssh:ssh}
Some computers allow you to remotely log and start a Unix shell running
using ssh (secure shell).  To do so you generally type something like:

\begin{Verbatim}[commandchars=\\\{\}]
\PYGZdl{} ssh username@host
\end{Verbatim}

where username is your account name on the machine you are trying to connect
to and host is the host name.

On Linux or a
Mac, the \titleref{ssh} command should work fine in a terminal.  On Windows, you may
need to install something like \href{http://www.putty.org/}{putty}.


\subsection{X-window forwarding}
\label{ssh:ssh-x}\label{ssh:x-window-forwarding}
If you plan on running a program remotely that might pop up its own
X-window, e.g. when doing plotting in Python or Matlab, you should use:

\begin{Verbatim}[commandchars=\\\{\}]
\PYGZdl{} ssh \PYGZhy{}X username@host
\end{Verbatim}

In order for X-windows forwarding to work you must be running
a X-window server on your machine.  If you are running on a linux machine
this is generally not an issue.  On a Mac you need to install the \emph{Xcode
developer tools} (which you will need anyway).

On Windows you will need something like \href{http://sourceforge.net/projects/xming/}{xming}.  A variety of tutorials on
using \emph{putty} and \emph{xming} together can be found by googling ``putty and
xming''.


\subsection{scp}
\label{ssh:scp}
To transfer files you can use \titleref{scp}, similar to the copy command \titleref{cp} but used
when the source and destination are on different computers:

\begin{Verbatim}[commandchars=\\\{\}]
\PYGZdl{} scp somefile username@host:somedirectory
\end{Verbatim}

which would copy \titleref{somefile} in your local directory to \titleref{somedirectory}
on the remote \titleref{host}, which is an address like \titleref{homer.u.washington.edu},
for example.

Going  in the other direction, you could copy a remote file to your local
machine via:

\begin{Verbatim}[commandchars=\\\{\}]
\PYGZdl{} scp username@host:somedirectory/somefile .
\end{Verbatim}

The last ''.'' means ``this directory''.  You could instead give the path to a
different local directory.

You will have to type your password on the remote host each
time you do this, unless you have remote ssh access set up, see for example
\href{http://www.debian.org/devel/passwordlessssh}{this page}.


\section{Text editors}
\label{editors:text-editors}\label{editors::doc}\label{editors:editors}

\subsection{Leafpad}
\label{editors:leafpad}
A simple and lightweight editor, installed on the course VM.

To use it, click on the green leaf in the bottom menu, or at the bash
prompt:

\begin{Verbatim}[commandchars=\\\{\}]
\PYGZdl{} leafpad \PYGZlt{}filename\PYGZgt{}
\end{Verbatim}


\subsection{gedit}
\label{editors:gedit}
\code{gedit} is a simple, easy-to-use text editor with support for syntax
highlighting for a variety of programming languages, including Python
and Fortran.  It is installed by default on most GNOME-based Linux
systems, such as Ubuntu.  It can be installed on the course VM via:

\begin{Verbatim}[commandchars=\\\{\}]
\PYGZdl{} sudo apt\PYGZhy{}get install gedit
\end{Verbatim}


\subsection{NEdit}
\label{editors:nedit}
Another easy-to-use editor, available from \url{http://www.nedit.org/}.
NEdit is available for almost all Unix systems, including Linux and
Mac OS X; OS X support requires an X Windows server.


\subsection{XCode}
\label{editors:xcode}
XCode is a free integrated development environment available for Mac
OS X from Apple at
\url{http://developer.apple.com/technologies/tools/xcode.html}.  It
should be more than adequate for the needs of this class.


\subsection{TextWrangler}
\label{editors:textwrangler}
TextWrangler is another free programmer's text editor for Mac OS X,
available at \url{http://www.barebones.com/products/TextWrangler/}.


\subsection{vi or vim}
\label{editors:vi-or-vim}
\code{vi} (``Visual Interface'') / \code{vim} (``\code{vi} iMproved'') is a fast,
powerful text editor.  Its interface is extremely different from
modern editors, and can be difficult to get used to, but \code{vi} can
offer substantially higher productivity for an experienced user.  It
is available for all operating systems; see \url{http://www.vim.org/}
for downloads and documentation.

A command line version of \code{vi} is already installed in the class VM.


\subsection{emacs}
\label{editors:emacs}
\code{emacs} (``Editing Macros'') is another powerful text editor.  Its
interface may be slightly easier to get used to than that of \code{vi},
but it is still extremely different from modern editors, and is also
extremely different from \code{vi}.  It offers similar productivity
benefits to \code{vi}.  See \url{http://www.gnu.org/software/emacs/} for
downloads and documentation.  On a Debian or Ubuntu Linux system, such
as the class VM, you can install it via:

\begin{Verbatim}[commandchars=\\\{\}]
\PYGZdl{} sudo apt\PYGZhy{}get install emacs
\end{Verbatim}


\section{Reproducibility}
\label{reproducibility:reproducibility}\label{reproducibility::doc}\label{reproducibility:id1}
Whenever you do a computation you should strive to make it reproducible, so
that if you want to produce the same result again at a later time you will
be able to.  This is particularly true if the result is important to
your research and will ultimately be used in a paper or thesis
you are writing, for example.

This may take a bit more time and effort initially, but doing so routinely
can save you an enormous amount of time in the future if you need to
slightly modify what you did, or if you need to reconstruct what you did
in order to convince yourself or others that you did it properly, or as the
basis for future work in the same direction.

In addition to the potential benefits to your own productivity in the
future, reproducibility of scientific results is a cornerstone of the
scientific method --- it should be possible for other researchers to
independently verify scientific claims by repeating experiments.  If it is
impossible to explain exactly what you did, then it will be impossible for
others to repeat your computational experiments.  Unfortunately most
publications in computational science fall short when it comes to
reproducibility.  Algorithms and data analysis techniques are usually
inadequately described in publications, and readers would find it very hard
to reproduce the results from scratch.  An obvious solution is to make the
actual computer code available.  At the very least, the authors of the paper
should maintain a complete copy of the code that was used to produce the
results.  Unfortunately even this is often not the case.

There is growing concern in many computational science communities with the
lack of reproducibility, not only because of the negative effect on
productivity and scientific progress, but also because the results of
simulations are increasingly important in making policy decisions that can
have far-reaching consequences.

Various efforts underway to develop better tools
for assisting in doing computational work in a reproducible manner and to
help provide better access to code and data that is used to obtain
research results.

Those interested in learning more about this topic might start with the
resources listed in {\hyperref[biblio:biblio\string-repro]{\crossref{\DUrole{std,std-ref}{Reproducibility references}}}}.

In this class we will concentrate on version control as one easy way to
greatly improve the reproducibility of your own work.  If you create a \emph{git}
repository for each project and are diligent about checking in versions of
the code that create important research products, with suitable commit
messages, then you will find it much easier in the long run to reconstruct
the version used to produce any particular result.


\subsection{Scripting vs. using a shell or GUI}
\label{reproducibility:scripting-vs-using-a-shell-or-gui}
Many graphics packages let you type commands in at a shell prompt and/or
click buttons on a GUI to generate data or adjust plot parameters.  While
very convenient, this makes it hard to reproduce the resulting plot unless
you remember exactly what you did.
It is a very good idea to instead write a script that produces the plot and
that sets all the appropriate plotting parameters in such a way that
rerunning the script gives exactly the same plot.  With some plotting tools
such a script can be automatically generated after you've come up with the
optimal plot by fiddling around with the GUI or by typing commands at the
prompt.  It's worth figuring out how to do this most easily for your own
tools and work style.  If you always create a script for each figure, and
then check that it works properly and commit it to the \emph{git} repository
for the project, then you will be able to easily reproduce the figure again
later.

Even if you are not concerned with allowing others to create the same plot,
it is in your own best interest to make it easy for you to do so again in
the future.  For example, it often happens that the referees of a paper or
members of a thesis committee will suggest improving a figure by plotting
something differently, perhaps as simple as increasing the font size so that
the labels on the axes can be read.  If you have the code that produced the
plot this is easy to do in a few minutes.  If you don't, it may take days
(or longer) to figure out again exactly how you produced that plot to begin
with.


\subsection{IPython history command}
\label{reproducibility:ipython-history-command}
If you are using IPython and typing in commands at the prompt,
the \emph{history} command can be used to print out a list of all the commands
entered, or something like:

\begin{Verbatim}[commandchars=\\\{\}]
\PYG{n}{In}\PYG{p}{[}\PYG{l+m+mi}{32}\PYG{p}{]}\PYG{p}{:}  \PYG{n}{history} \PYG{l+m+mi}{19}\PYG{o}{\PYGZhy{}}\PYG{l+m+mi}{31}
\end{Verbatim}

to print out commands \emph{In{[}19{]}} through \emph{In{[}31{]}}.


\section{Version Control Software}
\label{versioncontrol::doc}\label{versioncontrol:versioncontrol}\label{versioncontrol:version-control-software}
In this class we will use \emph{git}.  See the section {\hyperref[git:git]{\crossref{\DUrole{std,std-ref}{Git}}}}
for more information on using \emph{git} and the repositories required for this
class.

There are many other version control systems that are currently popular,
such as cvs, Subversion, Mercurial, and Bazaar.
See \phantomsection\label{versioncontrol:id1}{\hyperref[biblio:wikipedia\string-revision\string-control\string-software]{\crossref{{[}wikipedia-revision-control-software{]}}}} for a much longer list with
links.
See \phantomsection\label{versioncontrol:id2}{\hyperref[biblio:wikipedia\string-revision\string-control]{\crossref{{[}wikipedia-revision-control{]}}}} for a general discussion of such systems.

Version control systems were originally developed to aid in the development
of large software projects with many authors working on inter-related
pieces.  The basic idea is that you want to work on a file (one piece of the
code), you check it out of a repository, make changes, and then check it
back in when you're satisfied.  The repository keeps track of all changes
(and who made them) and can restore any previous version of a single file or
of the state of the whole project.  It does not keep a full copy of every
file ever checked in, it keeps track of differences ({\color{red}\bfseries{}*}diff*s) between
versions, so if you check in a version that only has one line changed from
the previous version, only the characters that actually changed are kept
track of.

It sounds like a hassle to be checking files in and out, but there are a
number of advantages to this system that make version control an
extremely useful tool even for use with you own projects if you are the only
one working on something.  Once you get comfortable with it you may wonder
how you ever lived without it.

Advantages include:
\begin{itemize}
\item {} 
You can revert to a previous version of a file if you decide the changes
you made are incorrect.  You can also easily compare different versions
to see what changes you made, e.g. where a bug was introduced.

\item {} 
If you use a computer program and some set of data to produce some
results for a publication, you can check in exactly the code and data
used.  If you later want to modify the code or data to produce new results,
as generally happens with computer programs, you still have access to the
first version without having to archive a full copy of all files for
every experiment you do.  Working in this manner is crucial if you want
to be able to later reproduce earlier results, as if often necessary if
you need to tweak the plots for to some journal's specifications or if a
reader of your paper wants to know exactly what parameter choices you
made to get a certain set of results.   This is an important aspect of
doing \emph{reproducible research}, as should be required in science.  (See
Section {\hyperref[reproducibility:reproducibility]{\crossref{\DUrole{std,std-ref}{Reproducibility}}}}).  If nothing else you can save yourself
hours of headaches down the road trying to figure out how you got your
own results.

\item {} 
If you work on more than one machine, e.g. a desktop and laptop, version
control systems are one way to keep your projects synched up between
machines.

\end{itemize}


\subsection{Client-server systems}
\label{versioncontrol:client-server-systems}
The original version control systems all used a client-server model, in
which there is one computer that contains \textbf{the repository} and everyone
else checks code into and out of that repository.

Systems such as CVS and Subversion (svn) have this form.
An important feature of these systems is that only the repository has the
full history of all changes made.

There is a \href{http://software-carpentry.org/4\_0/vc/}{software-carpentry webpage on version control} that gives a brief overview
of client-server systems.


\subsection{Distributed systems}
\label{versioncontrol:distributed-systems}
Git, and other systems such as Mercurial and Bazaar, use a distributed
system in which there is not necessarily a ``master repository''.  Any working
copy contains the full history of changes made to this copy.

The best way to get a feel for how \emph{git} works is to use it, for example
by following the instructions in Section {\hyperref[git:git]{\crossref{\DUrole{std,std-ref}{Git}}}}.


\subsection{Further reading}
\label{versioncontrol:further-reading}
See the {\hyperref[biblio:biblio\string-vcs]{\crossref{\DUrole{std,std-ref}{Version control systems references}}}} section of the bibliography.


\section{Git}
\label{git:git}\label{git::doc}\label{git:id1}
See {\hyperref[versioncontrol:versioncontrol]{\crossref{\DUrole{std,std-ref}{Version Control Software}}}} and the links there
for a more general discussion of the concepts.


\subsection{Instructions for cloning the class repository}
\label{git:classgit}\label{git:instructions-for-cloning-the-class-repository}
All of the materials for this class, including homeworks, sample programs,
and the webpages you are now
reading (or at least the \emph{.rst} files used to create them, see
{\hyperref[sphinx:sphinx]{\crossref{\DUrole{std,std-ref}{Sphinx documentation}}}}), are in a Git repository hosted at Bitbucket, located
at
\url{http://bitbucket.org/rjleveque/uwhpsc/}.
In addition to viewing the files via the link above, you can also view
changesets, issues, etc. (see {\hyperref[bitbucket:bitbucket]{\crossref{\DUrole{std,std-ref}{Bitbucket repositories: viewing changesets, issue tracking}}}}).

\textbf{Note:} This repository contains all of the files used during the 2013
version of this course, some of which are referred to in the Lecture Videos.
The repository also contains some new material for 2014 and will continue to
be added to during this quarter. See the overview at the top of the page
\url{http://bitbucket.org/rjleveque/uwhpsc/}
for an outline of the directory structure.  (This overview is showing the
file README.md that is in the top level directory of the repository.)

To obtain a copy, simply move to the directory where you want your copy to
reside (assumed to be your home directory below)
and then \emph{clone} the repository:

\begin{Verbatim}[commandchars=\\\{\}]
\PYGZdl{} cd
\PYGZdl{} git clone https://rjleveque@bitbucket.org/rjleveque/uwhpsc.git
\end{Verbatim}

Note the following:
\begin{itemize}
\item {} 
It is assumed you have \emph{git} installed, see
{\hyperref[software_installation:software\string-installation]{\crossref{\DUrole{std,std-ref}{Downloading and installing software for this class}}}}.

\item {} 
The clone statement will download the entire repository as a new
subdirectory called \emph{uwhpsc}, residing in your home directory.  If you
want \emph{uwhpsc} to reside elsewhere, you should first \emph{cd} to that
directory.

\end{itemize}

It will be useful to set a Unix environment variable (see {\hyperref[unix:env]{\crossref{\DUrole{std,std-ref}{Environment variables}}}}) called
\emph{UWHPSC} to refer to the directory you have just created.  Assuming you are
using the bash shell (see {\hyperref[unix:bash]{\crossref{\DUrole{std,std-ref}{The bash shell}}}}), and that you cloned uwhpsc
into your home directory, you can do this via:

\begin{Verbatim}[commandchars=\\\{\}]
\PYGZdl{} export UWHPSC=\PYGZdl{}HOME/uwhpsc
\end{Verbatim}

This uses the standard environment variable \emph{HOME}, which is the full path
to your home directory.

If you put it somewhere else, you can instead do:

\begin{Verbatim}[commandchars=\\\{\}]
\PYGZdl{} cd uwhpsc
\PYGZdl{} export UWHPSC={}`pwd{}`
\end{Verbatim}

The syntax
\emph{{}`pwd{}`} means to run the \emph{pwd} command (print working directory) and insert the
output of this command into the export command.

Type:

\begin{Verbatim}[commandchars=\\\{\}]
\PYGZdl{} printenv UWHPSC
\end{Verbatim}

to make sure \emph{UWHPSC} is set properly. This should print the full path to the
new directory.

If you log out and log in again later, you will find that this environment
variable is no longer set.  Or if you set it in one terminal window, it
will not be set in others.  To have it set automatically every time a new
bash shell is created (e.g. whenever a new terminal window is opened), add a
line of the form:

\begin{Verbatim}[commandchars=\\\{\}]
export UWHPSC=\PYGZdl{}HOME/uwhpsc
\end{Verbatim}

to your \emph{.bashrc} file.  (See {\hyperref[unix:bashrc]{\crossref{\DUrole{std,std-ref}{.bashrc file}}}}).  This assumes it is in your
home directory.  If not, you will have to add a line of the form:

\begin{Verbatim}[commandchars=\\\{\}]
\PYG{n}{export} \PYG{n}{UWHPSC}\PYG{o}{=}\PYG{n}{full}\PYG{o}{\PYGZhy{}}\PYG{n}{path}\PYG{o}{\PYGZhy{}}\PYG{n}{to}\PYG{o}{\PYGZhy{}}\PYG{n}{uwhpsc}
\end{Verbatim}

where the full path is what was returned by the \emph{printenv} statement above.


\subsection{Updating your clone}
\label{git:uwhpsc-update}\label{git:updating-your-clone}
The files in the class repository will change as the quarter progresses ---
new notes, sample programs, and homeworks will be added.  In order
to bring these changes over to your cloned copy, all you need to do is:

\begin{Verbatim}[commandchars=\\\{\}]
\PYGZdl{} cd \PYGZdl{}UWHPSC
\PYGZdl{} git fetch origin
\PYGZdl{} git merge origin/master
\end{Verbatim}

Of course this assumes that \emph{UWHPSC} has been properly set, see above.

The \emph{git fetch} command instructs \emph{git} to fetch any changes from \emph{origin},
which points to the remote bitbucket repository that you originally cloned
from.  In the merge command, \titleref{origin/master} refers to the master branch
in this repository (which
is the only branch that exists for this particular repository).
This merges any changes retrieved into the files in your current working
directory.

The last two command can be combined as:

\begin{Verbatim}[commandchars=\\\{\}]
\PYGZdl{} git pull origin master
\end{Verbatim}

or simply:

\begin{Verbatim}[commandchars=\\\{\}]
\PYGZdl{} git pull
\end{Verbatim}

since \titleref{origin} and \titleref{master} are the defaults.


\subsection{Creating your own Bitbucket repository}
\label{git:mygit}\label{git:creating-your-own-bitbucket-repository}
In addition to using the class repository, students in AMath 483/583 are
also required to create their own repository on Bitbucket.  It is possible
to use \emph{git} for your own work without creating a repository on a hosted
site such as Bitbucket (see \DUrole{xref,std,std-ref}{newgit} below), but there are several
reasons for this requirement:
\begin{itemize}
\item {} 
You should learn how to use Bitbucket for more than just pulling changes.

\item {} 
You will use this repository to ``submit'' your solutions to homeworks.
You will give the instructor and TA permission to clone your repository so
that we can grade the homework (others will not be able to clone or view it
unless you also give them permission).

\item {} 
It is recommended that after the class ends you
continue to use your repository as a way to back up your important work on
another computer (with all the benefits of version control too!).
At that point, of course, you can change the permissions so the
instructor and TA no longer have access.

\end{itemize}

Below are the instructions for creating your own repository.  Note that
this should be a \emph{private repository} so nobody can view or clone it unless
you grant permission.

Anyone can create a free private repository on Bitbucket.
Note that you can also create an unlimited number of public repositories
free at Bitbucket, which you might want to do for open source software
projects, or for classes like this one.

(To make free open access repositories that can be viewed by anyone,
\href{https://github.com/}{GitHub}
is recommended, which allows an unlimited number of
open repositories and is widely used for open source projects.)

To get started, follow the directions below exactly.
We will work through this as part of Lab 2 (Thursday, April 3), but feel
free to start earlier.

Doing this will be part of \DUrole{xref,std,std-ref}{homework1}.
We will clone your repository and check that \emph{testfile.txt} has been created
and modified as directed below.
\begin{enumerate}
\item {} 
On the machine you're working on:

\begin{Verbatim}[commandchars=\\\{\}]
\PYGZdl{} git config \PYGZhy{}\PYGZhy{}global user.name \PYGZdq{}Your Name\PYGZdq{}
\PYGZdl{} git config \PYGZhy{}\PYGZhy{}global user.email you@example.com
\end{Verbatim}

These will be used when you commit changes.
If you don't do this, you might get a warning message
the first time you try to commit.

\item {} 
Go to \url{http://bitbucket.org/} and click on ``Sign up now'' if you don't
already have an account.

\item {} 
Fill in the form, make sure you remember your username and password.

\item {} 
You should then be taken to your account.  Click on ``Create'' next
to ``Repositories''.

\item {} 
You should now see a form where you can specify the name of a repository
and a description.  The repository name need not be the same as your user
name (a single user might have several repositories).  For example, the class
repository is named \emph{uwhpsc}, owned by user \emph{rjleveque}.
To avoid confusion, you should probably not name your repository
\emph{uwhpsc}.

You should stick to lower case letters and numbers in your repository
name, e.g. \emph{myhpsc} or \emph{amath583} might be good choices.  Upper case and
special symbols such as underscore sometimes get modified by bitbucket
and the repository name you try to paste into the homework submission
form might not agree with what bitbucket expects.

Don't name your repository \emph{homework1} because you will be using the
same repository for other homeworks later in the quarter.

\item {} 
Make sure you click on ``Private'' at the bottom.  Also turn ``Issue
tracking'' and ``Wiki'' on if you wish to use these features.

\item {} 
Click on ``Create repository''.

\item {} 
You should now see a page with instructions on how to \emph{clone} your
(currently empty) repository.  In a Unix window, \emph{cd} to the directory where
you want your cloned copy to reside, and perform the clone by typing in
the clone command shown.  This will create a new directory with the same
name as the repository.

\item {} 
You should now be able to \emph{cd} into the directory this created.

\item {} 
To keep track of where this directory is and get to it easily in the
future, create an environment variable \emph{MYHPSC} from inside this directory
by:

\begin{Verbatim}[commandchars=\\\{\}]
\PYGZdl{} export MYHPSC={}`pwd{}`
\end{Verbatim}

See the discussion above in section {\hyperref[git:classgit]{\crossref{\DUrole{std,std-ref}{Instructions for cloning the class repository}}}} for what this does.  You
will also probably want to add a line to your \emph{.bashrc} file to define
\emph{MYHPSC} similar to the line added for \emph{UWHPSC}.

\item {} 
The directory you are now in will appear empty if you simply do:

\begin{Verbatim}[commandchars=\\\{\}]
\PYGZdl{} ls
\end{Verbatim}

But try:

\begin{Verbatim}[commandchars=\\\{\}]
\PYGZdl{} ls \PYGZhy{}a
./  ../  .git/
\end{Verbatim}

the \emph{-a} option causes \emph{ls} to list files starting with a dot, which are
normally suppressed.  See \DUrole{xref,std,std-ref}{ls} for a discussion of \emph{./} and \emph{../}.
The directory \emph{.git} is the directory that stores all the information
about the contents of this directory and a complete history of every file
and every change ever committed.  You shouldn't touch or modify the files in
this directory, they are used by \emph{git}.

\item {} 
Add a new file to your directory:

\begin{Verbatim}[commandchars=\\\{\}]
\PYGZdl{} cat \PYGZgt{} testfile.txt
This is a new file
with only two lines so far.
\PYGZca{}D
\end{Verbatim}

The Unix \emph{cat} command simply redirects everything you type on the
following lines into a file called \emph{testfile.txt}.  This goes on until
you type a \textless{}ctrl\textgreater{}-d (the 4th line in the example
above).  After typing \textless{}ctrl\textgreater{}-d you should get the Unix
prompt back.  Alternatively, you could create the file testfile.txt using
your favorite text editor (see {\hyperref[editors:editors]{\crossref{\DUrole{std,std-ref}{Text editors}}}}).

\item {} 
Type:

\begin{Verbatim}[commandchars=\\\{\}]
\PYGZdl{} git status \PYGZhy{}s
\end{Verbatim}

The response should be:

\begin{Verbatim}[commandchars=\\\{\}]
?? testfile.txt
\end{Verbatim}

The ?? means that this file is not under revision control.
The \emph{-s} flag results in this \emph{short} status list.  Leave it off for more
information.

To put the file under revision control, type:

\begin{Verbatim}[commandchars=\\\{\}]
\PYGZdl{} git add testfile.txt
\PYGZdl{} git status \PYGZhy{}s
A  testfile.txt
\end{Verbatim}

The A means it has been added.  However, at this point \emph{git} is not
we have not yet taken a \emph{snapshot} of this version of the file.
To do so, type:

\begin{Verbatim}[commandchars=\\\{\}]
\PYGZdl{} git commit \PYGZhy{}m \PYGZdq{}My first commit of a test file.\PYGZdq{}
\end{Verbatim}

The string following the -m is a comment about this commit that may help
you in general remember why you committed new or changed files.

You should get a response like:

\begin{Verbatim}[commandchars=\\\{\}]
\PYG{p}{[}\PYG{n}{master} \PYG{p}{(}\PYG{n}{root}\PYG{o}{\PYGZhy{}}\PYG{n}{commit}\PYG{p}{)} \PYG{l+m+mi}{28}\PYG{n}{a4da5}\PYG{p}{]} \PYG{n}{My} \PYG{n}{first} \PYG{n}{commit} \PYG{n}{of} \PYG{n}{a} \PYG{n}{test} \PYG{n}{file}\PYG{o}{.}
 \PYG{l+m+mi}{1} \PYG{n}{file} \PYG{n}{changed}\PYG{p}{,} \PYG{l+m+mi}{2} \PYG{n}{insertions}\PYG{p}{(}\PYG{o}{+}\PYG{p}{)}
 \PYG{n}{create} \PYG{n}{mode} \PYG{l+m+mi}{100644} \PYG{n}{testfile}\PYG{o}{.}\PYG{n}{txt}
\end{Verbatim}

We can now see the status of our directory via:

\begin{Verbatim}[commandchars=\\\{\}]
\PYGZdl{} git status
\PYGZsh{} On branch master
nothing to commit (working directory clean)
\end{Verbatim}

Alternatively, you can check the status of a single file with:

\begin{Verbatim}[commandchars=\\\{\}]
\PYGZdl{} git status testfile.txt
\end{Verbatim}

You can get a list of all the commits you have made (only one so far)
using:

\begin{Verbatim}[commandchars=\\\{\}]
\PYGZdl{} git log

commit 28a4da5a0deb04b32a0f2fd08f78e43d6bd9e9dd
Author: Randy LeVeque \PYGZlt{}rjl@ned\PYGZgt{}
Date:   Tue Mar 5 17:44:22 2013 \PYGZhy{}0800

    My first commit of a test file.
\end{Verbatim}

The number 28a4da5a0deb04b32a0f2fd08f78e43d6bd9e9dd above is the ``name''
of this commit and you can always get back to the state of your files as
of this commit by using this number.  You don't have to remember it, you
can use commands like \emph{git log} to find it later.

Yes, this is a number... it is a 40 digit hexadecimal number, meaning it
is in base 16 so in addition to 0, 1, 2, ..., 9, there are 6 more digits
a, b, c, d, e, f representing 10 through 15.  This number is almost
certainly guaranteed to be unique among all commits you will ever
do (or anyone has ever done, for that matter).  It is computed based
on the state of all the files in this snapshot as a \href{http://en.wikipedia.org/wiki/SHA-1}{SHA-1
Cryptographic hash function},
called a SHA-1 Hash for short.

Now let's modify this file:

\begin{Verbatim}[commandchars=\\\{\}]
\PYGZdl{} cat \PYGZgt{}\PYGZgt{} testfile.txt
Adding a third line
\PYGZca{}D
\end{Verbatim}

Here the \emph{\textgreater{}\textgreater{}} tells \emph{cat} that we want to add on to the end of an
existing file rather than creating a new one.  (Or you can edit the file
with your favorite editor and add this third line.)

Now try the following:

\begin{Verbatim}[commandchars=\\\{\}]
\PYGZdl{} git status \PYGZhy{}s
 M testfile.txt
\end{Verbatim}

The M indicates this file has been modified relative to the most recent
version that was committed.

To see what changes have been made, try:

\begin{Verbatim}[commandchars=\\\{\}]
\PYGZdl{} git diff testfile.txt
\end{Verbatim}

This will produce something like:

\begin{Verbatim}[commandchars=\\\{\}]
diff \PYGZhy{}\PYGZhy{}git a/testfile.txt b/testfile.txt
index d80ef00..fe42584 100644
\PYGZhy{}\PYGZhy{}\PYGZhy{} a/testfile.txt
+++ b/testfile.txt
@@ \PYGZhy{}1,2 +1,3 @@
 This is a new file
 with only two lines so far
+Adding a third line
\end{Verbatim}

The + in front of the last line shows that it was added.
The two lines before it are printed to show the context.  If the
file were longer, \emph{git diff}
would only print a few lines around any change to indicate the context.

Now let's try to commit this changed file:

\begin{Verbatim}[commandchars=\\\{\}]
\PYGZdl{} git commit \PYGZhy{}m \PYGZdq{}added a third line to the test file\PYGZdq{}
\end{Verbatim}

This will fail!  You should get a response like this:

\begin{Verbatim}[commandchars=\\\{\}]
\PYG{c}{\PYGZsh{} On branch master}
\PYG{c}{\PYGZsh{} Changes not staged for commit:}
\PYG{c}{\PYGZsh{}   (use \PYGZdq{}git add \PYGZlt{}file\PYGZgt{}...\PYGZdq{} to update what will be committed)}
\PYG{c}{\PYGZsh{}   (use \PYGZdq{}git checkout \PYGZhy{}\PYGZhy{} \PYGZlt{}file\PYGZgt{}...\PYGZdq{} to discard changes in working}
\PYG{c}{\PYGZsh{}   directory)}
\PYG{c}{\PYGZsh{}}
\PYG{c}{\PYGZsh{}   modified:   testfile.txt}
\PYG{c}{\PYGZsh{}}
\PYG{n}{no} \PYG{n}{changes} \PYG{n}{added} \PYG{n}{to} \PYG{n}{commit} \PYG{p}{(}\PYG{n}{use} \PYG{l+s}{\PYGZdq{}}\PYG{l+s}{git add}\PYG{l+s}{\PYGZdq{}} \PYG{o+ow}{and}\PYG{o}{/}\PYG{o+ow}{or} \PYG{l+s}{\PYGZdq{}}\PYG{l+s}{git commit \PYGZhy{}a}\PYG{l+s}{\PYGZdq{}}\PYG{p}{)}
\end{Verbatim}

\emph{git} is saying that the file \emph{testfile.txt} is modified but that no
files have been \textbf{staged} for this commit.

If you are used to Mercurial, \emph{git} has an extra level of complexity (but
also flexibility):  you can choose which modified files will be included
in the next commit.  Since we only have one file, there will not be a
commit unless we add this to the \textbf{index} of files staged for the next
commit:

\begin{Verbatim}[commandchars=\\\{\}]
\PYGZdl{} git add testfile.txt
\end{Verbatim}

Note that the status is now:

\begin{Verbatim}[commandchars=\\\{\}]
\PYGZdl{} git status \PYGZhy{}s
M  testfile.txt
\end{Verbatim}

This is different in a subtle way from what we saw before: The \emph{M} is
in the first column rather than the second, meaning it has been both
modified and staged.

We can get more information if we leave off the \emph{-s} flag:

\begin{Verbatim}[commandchars=\\\{\}]
\PYGZdl{} git status

\PYGZsh{} On branch master
\PYGZsh{} Changes to be committed:
\PYGZsh{}   (use \PYGZdq{}git reset HEAD \PYGZlt{}file\PYGZgt{}...\PYGZdq{} to unstage)
\PYGZsh{}
\PYGZsh{}   modified:   testfile.txt
\PYGZsh{}
\end{Verbatim}

Now \emph{testfile.txt} is on the index of files staged for the next commit.

Now we can do the commit:

\begin{Verbatim}[commandchars=\\\{\}]
\PYGZdl{} git commit \PYGZhy{}m \PYGZdq{}added a third line to the test file\PYGZdq{}

[master 51918d7] added a third line to the test file
 1 file changed, 1 insertion(+)
\end{Verbatim}

Try doing \emph{git log} now and you should see something like:

\begin{Verbatim}[commandchars=\\\{\}]
\PYG{n}{commit} \PYG{l+m+mi}{51918}\PYG{n}{d7ea4a63da6ab42b3c03f661cbc1a560815}
\PYG{n}{Author}\PYG{p}{:} \PYG{n}{Randy} \PYG{n}{LeVeque} \PYG{o}{\PYGZlt{}}\PYG{n}{rjl}\PYG{n+nd}{@ned}\PYG{o}{\PYGZgt{}}
\PYG{n}{Date}\PYG{p}{:}   \PYG{n}{Tue} \PYG{n}{Mar} \PYG{l+m+mi}{5} \PYG{l+m+mi}{18}\PYG{p}{:}\PYG{l+m+mi}{11}\PYG{p}{:}\PYG{l+m+mi}{34} \PYG{l+m+mi}{2013} \PYG{o}{\PYGZhy{}}\PYG{l+m+mi}{0800}

    \PYG{n}{added} \PYG{n}{a} \PYG{n}{third} \PYG{n}{line} \PYG{n}{to} \PYG{n}{the} \PYG{n}{test} \PYG{n}{file}

\PYG{n}{commit} \PYG{l+m+mi}{28}\PYG{n}{a4da5a0deb04b32a0f2fd08f78e43d6bd9e9dd}
\PYG{n}{Author}\PYG{p}{:} \PYG{n}{Randy} \PYG{n}{LeVeque} \PYG{o}{\PYGZlt{}}\PYG{n}{rjl}\PYG{n+nd}{@ned}\PYG{o}{\PYGZgt{}}
\PYG{n}{Date}\PYG{p}{:}   \PYG{n}{Tue} \PYG{n}{Mar} \PYG{l+m+mi}{5} \PYG{l+m+mi}{17}\PYG{p}{:}\PYG{l+m+mi}{44}\PYG{p}{:}\PYG{l+m+mi}{22} \PYG{l+m+mi}{2013} \PYG{o}{\PYGZhy{}}\PYG{l+m+mi}{0800}

    \PYG{n}{My} \PYG{n}{first} \PYG{n}{commit} \PYG{n}{of} \PYG{n}{a} \PYG{n}{test} \PYG{n}{file}\PYG{o}{.}
\end{Verbatim}

If you want to revert your working directory back to the first snapshot
you could do:

\begin{Verbatim}[commandchars=\\\{\}]
\PYGZdl{} git checkout 28a4da5a0de
Note: checking out \PYGZsq{}28a4da5a0de\PYGZsq{}.

You are in \PYGZsq{}detached HEAD\PYGZsq{} state. You can look around, make experimental
changes and commit them, and you can discard any commits you make in this
state without impacting any branches by performing another checkout.

HEAD is now at 28a4da5... My first commit of a test file.
\end{Verbatim}

Take a look at the file, it should be back to the state with only two
lines.

Note that you don't need the full SHA-1 hash code, the first few digits
are enough to uniquely identify it.

You can go back to the most recent version with:

\begin{Verbatim}[commandchars=\\\{\}]
\PYGZdl{} git checkout master
Switched to branch \PYGZsq{}master\PYGZsq{}
\end{Verbatim}

We won't discuss branches, but unless you create a new branch, the
default name for your main branch is \emph{master} and this \emph{checkout} command
just goes back to the most recent commit.

\item {} 
So far you have been using \emph{git} to keep track of changes in your own
directory, on your computer.  None of these changes have been seen by
Bitbucket, so if someone else cloned your repository from there, they
would not see \emph{testfile.txt}.

Now let's \emph{push} these changes back to the Bitbucket repository:

First do:

\begin{Verbatim}[commandchars=\\\{\}]
\PYGZdl{} git status
\end{Verbatim}

to make sure there are no changes that have not been committed.  This
should print nothing.

Now do:

\begin{Verbatim}[commandchars=\\\{\}]
\PYGZdl{} git push \PYGZhy{}u origin master
\end{Verbatim}

This will prompt for your Bitbucket password and should then print
something indicating that it has uploaded these two commits to
your bitbucket repository.

Not only has it copied the 1 file over, it has added both changesets, so
the entire history of your commits is now stored in the repository.  If
someone else clones the repository, they get the entire commit history
and could revert to any previous version, for example.

To push future commits to bitbucket, you should only need to do:

\begin{Verbatim}[commandchars=\\\{\}]
\PYGZdl{} git push
\end{Verbatim}

and by default it will push your master branch (the only branch you
have, probably) to \titleref{origin}, which is the shorthand name for the
place you originally cloned the repository from.  To see where this
actually points to:

\begin{Verbatim}[commandchars=\\\{\}]
\PYGZdl{} git remote \PYGZhy{}v
\end{Verbatim}

This lists all \titleref{remotes}.
By default there is only one, the place you cloned the repository from.
(Or none if you had created a new repository using \titleref{git init} rather
than cloning an existing one.)

\item {} 
Check that the file is in your Bitbucket repository:  Go back to that web
page for your repository and click on the  ``Source'' tab at the top.  It
should display the files in your repository and show \emph{testfile.txt}.

Now click on the ``Commits'' tab at the top.  It should show that you
made two commits and display the comments you added with the \emph{-m} flag
with each commit.

If you click on the hex-string for a commit, it will show the
\emph{change set} for this commit.  What you
should see is the file in its final state, with three lines.  The third
line should be highlighted in green, indicating that this line was added
in this changeset.  A line highlighted in red would indicate a line deleted
in this changeset.  (See also {\hyperref[bitbucket:bitbucket]{\crossref{\DUrole{std,std-ref}{Bitbucket repositories: viewing changesets, issue tracking}}}}.)

\end{enumerate}

This is enough for now!

\DUrole{xref,std,std-ref}{homework1} instructs you to add some additional files to the Bitbucket
repository.

Feel free to experiment further with your repository at this point.


\subsection{Further reading}
\label{git:further-reading}
Next see {\hyperref[bitbucket:bitbucket]{\crossref{\DUrole{std,std-ref}{Bitbucket repositories: viewing changesets, issue tracking}}}} and/or {\hyperref[git_more:git\string-more]{\crossref{\DUrole{std,std-ref}{More about Git}}}}.

Remember that you can get help with \emph{git} commands by typing, e.g.:

\begin{Verbatim}[commandchars=\\\{\}]
\PYGZdl{} git help
\PYGZdl{} git help diff  \PYGZsh{} or any other specific command name
\end{Verbatim}

Each command has lots of options!

{\hyperref[biblio:biblio\string-git]{\crossref{\DUrole{std,std-ref}{Git references}}}} contains references to other sources of information and
tutorials.


\section{Bitbucket repositories: viewing changesets, issue tracking}
\label{bitbucket:bitbucket-repositories-viewing-changesets-issue-tracking}\label{bitbucket:bitbucket}\label{bitbucket::doc}
The directions in Section {\hyperref[git:git]{\crossref{\DUrole{std,std-ref}{Git}}}} explain how to use the class Bitbucket
repository to download these class notes and other resources used in the
class, and also how to set up your own repository there.

See also \href{https://confluence.atlassian.com/display/BITBUCKET/Bitbucket+101}{Bitbucket 101}.

In addition to providing a hosting site for repositories to allow them to be
easily shared, Bitbucket provides other web-based resources for working with
a repository.  (So do other sites such as
\href{http://github.com/repositories}{github}, for
example.)

To get a feel for what's possible, take a look at one of the major software
projects hosted on bitbucket, for example
\url{http://bitbucket.org/birkenfeld/sphinx/} which is the repository for the
Sphinx software used for these class notes pages (see {\hyperref[sphinx:sphinx]{\crossref{\DUrole{std,std-ref}{Sphinx documentation}}}}).  You
will see that software is being actively developed.

If you click on the ``Source'' tab at the top of the page you can browse
through the source code repository in its current state.

If you click on the ``Changesets'' tab you can see all the changes ever
committed, with the message that was written following the -m flag when the
commit was made.  If you click on one of these messages, it will show all
the changes in the form of the lines of the files changed, highlighted in
green for things added or red for things deleted.

If you click on the ``Issues'' tab, you will see the issue-tracking page.  If
someone notices a bug that should be fixed, or thinks of an improvement that
should be made, a new issue can be created (called a ``ticket'' in some systems).

If you want to try creating a ticket, \textbf{don't} do it on the Sphinx page,
the developers won't appreciate it.  Instead try doing it on your own
bitbucket repository that you set up following \DUrole{xref,std,std-ref}{myhg}.

You might also want to look at the bitbucket page for this class repository,
at \url{http://bitbucket.org/rjleveque/uwhpsc/} to keep track
of changes made to notes or code available.


\section{More about Git}
\label{git_more:more-about-git}\label{git_more::doc}\label{git_more:git-more}

\subsection{Using \emph{git} to stay synced up on multiple computers}
\label{git_more:using-git-to-stay-synced-up-on-multiple-computers}
If you want to use your \emph{git} repository on two or more computers, staying
in sync via bitbucket should work well. To avoid having merge conflicts or
missing something on one computer because you didn't push it from the other,
here are some tips:
\begin{itemize}
\item {} 
When you finish working on one machine, make sure your directory is
``clean'' (using ``git status'') and if not, add and commit any changes.

\item {} 
Use ``git push'' to make sure all commits are pushed to bitbucket.

\item {} 
When you start working on a different machine, make sure you are up to
date by doing:

\begin{Verbatim}[commandchars=\\\{\}]
\PYGZdl{} git fetch origin          \PYGZsh{} fetch changes from bitbucket
\PYGZdl{} git merge origin/master   \PYGZsh{} merge into your working directory
\end{Verbatim}

These can probably be replaced by simply doing:

\begin{Verbatim}[commandchars=\\\{\}]
\PYGZdl{} git pull
\end{Verbatim}

but for more complicated merges it's recommended that you do the steps
separately to have more control over what's being done, and perhaps to
inspect what was fetched before merging.

If you do this in a clean directory that was pushed since you made any
changes, then this merge should go fine without any conflicts.

\end{itemize}


\section{Sphinx documentation}
\label{sphinx:sphinx}\label{sphinx:sphinx-documentation}\label{sphinx::doc}
Sphinx is a Python-based documentation system that allows writing
documentation in a simple mark-up language called ReStructuredText, which
can then be converted to html or to latex files (and then to pdf or
postscript).  See \phantomsection\label{sphinx:id1}{\hyperref[biblio:sphinx]{\crossref{{[}sphinx{]}}}} for general information,
\phantomsection\label{sphinx:id2}{\hyperref[biblio:sphinx\string-documentation]{\crossref{{[}sphinx-documentation{]}}}} for a
complete manual, and \phantomsection\label{sphinx:id3}{\hyperref[biblio:sphinx\string-rst]{\crossref{{[}sphinx-rst{]}}}} or \phantomsection\label{sphinx:id4}{\hyperref[biblio:rst\string-documentation]{\crossref{{[}rst-documentation{]}}}}
for a primer on ReStructuredText.
See also \phantomsection\label{sphinx:id5}{\hyperref[biblio:sphinx\string-cheatsheet]{\crossref{{[}sphinx-cheatsheet{]}}}}.

Although originally designed for aiding in documentation of Python software,
it is now being used for documentation of packages in other languages as
well.  See \phantomsection\label{sphinx:id6}{\hyperref[biblio:sphinx\string-examples]{\crossref{{[}sphinx-examples{]}}}} for a list of other projects that use Sphinx.

It can also be used for things beyond software documentation.  In
particular, it has been used to create the class note webpages that you are
now reading.  It was chosen for two main reasons:
\begin{enumerate}
\item {} 
It is a convenient way to create a set of hyper-linked web pages on a variety
of topics without having to write raw html or worry much about formatting.

\item {} 
Writing good documentation is a crucial aspect of high performance
scientific computing and students in this class should learn ways to
simplify this task.  For this reason many homework assignments must be
``submitted'' in the form of Sphinx pages.

\end{enumerate}

The easiest way to learn how to create Sphinx pages is to read some of the
documentation
and then examine Sphinx pages such as the one you are now reading to see how
it is written.  You can do this with these class notes or you might look at
one of the other \phantomsection\label{sphinx:id7}{\hyperref[biblio:sphinx\string-examples]{\crossref{{[}sphinx-examples{]}}}} to see other styles.

Note that any time you are reading a page of these class notes (or other
things created with Sphinx) there is generally a menu item \emph{Show Source} (on
this page it's on the left, under the heading \emph{This Page}) and clicking on
this link shows the raw text file as originally written.  \emph{Try this now}.

It is possible to customize Sphinx so the pages look very different, as
you'll see if you visit some other projects listed at \phantomsection\label{sphinx:id8}{\hyperref[biblio:sphinx\string-examples]{\crossref{{[}sphinx-examples{]}}}}.


\subsection{Using Sphinx for your own project}
\label{sphinx:using-sphinx-for-your-own-project}
If you want to create your own set of Sphinx pages for some project, it is
easy to get started following the instructions at \phantomsection\label{sphinx:id9}{\hyperref[biblio:sphinx]{\crossref{{[}sphinx{]}}}}, or for a quick
start with a different look and feel, see \phantomsection\label{sphinx:id10}{\hyperref[biblio:sphinx\string-sampledoc]{\crossref{{[}sphinx-sampledoc{]}}}}.


\subsection{Using Sphinx to create these webpages}
\label{sphinx:using-sphinx-to-create-these-webpages}
If you clone the git repository for this class (see {\hyperref[git:git]{\crossref{\DUrole{std,std-ref}{Git}}}}), you will find
a subdirectory called \emph{notes}, containing a number of files with the
extension \emph{.rst}, one for each webpage, containing the ReStructuredText
input.

To create the html pages, first make sure you have Sphinx installed via:

\begin{Verbatim}[commandchars=\\\{\}]
\PYGZdl{} which sphinx\PYGZhy{}build
\end{Verbatim}

(see {\hyperref[software_installation:software\string-installation]{\crossref{\DUrole{std,std-ref}{Downloading and installing software for this class}}}}) and then type:

\begin{Verbatim}[commandchars=\\\{\}]
\PYGZdl{} make html
\end{Verbatim}

This should process all the files (see {\hyperref[makefiles:makefiles]{\crossref{\DUrole{std,std-ref}{Makefiles}}}}) and create a
subdirectory of \emph{notes} called \emph{\_build/html}.  The file
\emph{\_build/html/index.html} contains the main Table of Contents page.
Navigate your browser to this page using \emph{Open File} on the \emph{File} menu of
your browser.

Or you may be able to direct Firefox directly to this page via:

\begin{Verbatim}[commandchars=\\\{\}]
\PYGZdl{} firefox \PYGZus{}build/html/index.html
\end{Verbatim}


\subsection{Making a pdf version}
\label{sphinx:making-a-pdf-version}
If you type:

\begin{Verbatim}[commandchars=\\\{\}]
\PYGZdl{} make latex
\end{Verbatim}

then Sphinx will create a subdirectory \emph{\_build/latex} containing latex
files.  If you have latex installed, you can then do:

\begin{Verbatim}[commandchars=\\\{\}]
\PYGZdl{} cd \PYGZus{}build/latex
\PYGZdl{} make all\PYGZhy{}pdf
\end{Verbatim}

to run latex and create a pdf version of all the class notes.


\subsection{Further reading}
\label{sphinx:further-reading}
See \phantomsection\label{sphinx:id11}{\hyperref[biblio:sphinx]{\crossref{{[}sphinx{]}}}} for general information, \phantomsection\label{sphinx:id12}{\hyperref[biblio:sphinx\string-documentation]{\crossref{{[}sphinx-documentation{]}}}} for a
complete manual, and \phantomsection\label{sphinx:id13}{\hyperref[biblio:sphinx\string-rst]{\crossref{{[}sphinx-rst{]}}}} or \phantomsection\label{sphinx:id14}{\hyperref[biblio:rst\string-documentation]{\crossref{{[}rst-documentation{]}}}}.
See also \phantomsection\label{sphinx:id15}{\hyperref[biblio:sphinx\string-cheatsheet]{\crossref{{[}sphinx-cheatsheet{]}}}}.

See \phantomsection\label{sphinx:id16}{\hyperref[biblio:sphinx\string-sampledoc]{\crossref{{[}sphinx-sampledoc{]}}}} to get started on your own project.


\section{Binary/metric prefixes for computer size, speed, etc.}
\label{metrics:metrics}\label{metrics:binary-metric-prefixes-for-computer-size-speed-etc}\label{metrics::doc}
Computers are often described in terms such as megahertz, gigabytes,
teraflops, etc.  This section has a brief summary of the meaning of these
terms.


\subsection{Prefixes}
\label{metrics:prefixes}\label{metrics:id1}
Numbers associated with computers are often very large or small and so
standard scientific prefixes are used to denote powers of 10. E.g. a
kilobyte is 1000 bytes and a megabyte is a million bytes.  These prefixes
are listed below, where 1e3 for example means \(10^3\):

\begin{Verbatim}[commandchars=\\\{\}]
\PYG{n}{kilo}  \PYG{o}{=} \PYG{l+m+mi}{1}\PYG{n}{e3}
\PYG{n}{mega}  \PYG{o}{=} \PYG{l+m+mi}{1}\PYG{n}{e6}
\PYG{n}{giga}  \PYG{o}{=} \PYG{l+m+mi}{1}\PYG{n}{e9}
\PYG{n}{tera}  \PYG{o}{=} \PYG{l+m+mi}{1}\PYG{n}{e12}
\PYG{n}{peta}  \PYG{o}{=} \PYG{l+m+mi}{1}\PYG{n}{e15}
 \PYG{n}{exa}  \PYG{o}{=} \PYG{l+m+mi}{1}\PYG{n}{e18}
\end{Verbatim}

Note, however, that in some computer contexts (e.g. size of main memory)
these prefixes refer to nearby numbers that are exactly powers of 2.
However, recent standards have given these new names:

\begin{Verbatim}[commandchars=\\\{\}]
\PYG{n}{kibi} \PYG{o}{=} \PYG{l+m+mi}{2}\PYG{o}{\PYGZca{}}\PYG{p}{\PYGZob{}}\PYG{l+m+mi}{10}\PYG{p}{\PYGZcb{}} \PYG{o}{=} \PYG{l+m+mi}{1024}
\PYG{n}{mebi} \PYG{o}{=} \PYG{l+m+mi}{2}\PYG{o}{\PYGZca{}}\PYG{p}{\PYGZob{}}\PYG{l+m+mi}{20}\PYG{p}{\PYGZcb{}} \PYG{o}{=} \PYG{l+m+mi}{1048576}
\PYG{n}{etc}\PYG{o}{.}
\end{Verbatim}

So \(2^{20}\) bytes is a \emph{mebibyte}, abbreviated MiB.
For a more detailed discussion of this (and additional prefixes)
see \href{http://en.wikipedia.org/wiki/Binary\_prefix}{{[}wikipedia{]}}.

For numbers that are much smaller than 1 a different set of prefixes are
used, e.g. a millisecond is 1/1000 = 1e-3 second:

\begin{Verbatim}[commandchars=\\\{\}]
\PYG{n}{mille} \PYG{o}{=} \PYG{l+m+mi}{1}\PYG{n}{e}\PYG{o}{\PYGZhy{}}\PYG{l+m+mi}{3}
\PYG{n}{micro} \PYG{o}{=} \PYG{l+m+mi}{1}\PYG{n}{e}\PYG{o}{\PYGZhy{}}\PYG{l+m+mi}{6}
 \PYG{n}{nano} \PYG{o}{=} \PYG{l+m+mi}{1}\PYG{n}{e}\PYG{o}{\PYGZhy{}}\PYG{l+m+mi}{9}
 \PYG{n}{pico} \PYG{o}{=} \PYG{l+m+mi}{1}\PYG{n}{e}\PYG{o}{\PYGZhy{}}\PYG{l+m+mi}{12}
\PYG{n}{femto} \PYG{o}{=} \PYG{l+m+mi}{1}\PYG{n}{e}\PYG{o}{\PYGZhy{}}\PYG{l+m+mi}{15}
\end{Verbatim}


\subsection{Units of memory, storage}
\label{metrics:units-of-memory-storage}
The amount of memory or disk space on a computer is normally measured in
bytes (1 byte = 8 bits) since almost everything we want to store on a
computer requires some integer number of bytes (e.g. an ASCII character can
be stored in 1 byte, a standard integer in 4 bytes, see \DUrole{xref,std,std-ref}{storage}).

Memory on a computer is generally split up into different types of memory
implemented with different technologies.  There is generally a large quantity
of fairly slow memory (slow to move into the processor to operate on it) and
a much smaller quantity of faster memory (used for the data and programs
that are actively being processed).  Fast memory is much more expensive than
slow memeory, consumes more power, and produces more heat.

The \emph{hard disk} on a computer is used to store data for long periods of
time and is generally slow to access (i.e. to move into the core memory for
processing), but may be large, hundreds of gigabytes (hundreds of
billions of bytes).

The \emph{main memory} or \emph{core memory} might only be 1GB or a few GB.
Computers also have a smaller amount of


\subsection{Units of speed}
\label{metrics:units-of-speed}
The speed of a processor is often measured in Hertz (cycles per second)
or some multiple such as Gigahertz (billions of cycles per second).  This
tells how many \emph{clock cycles} are executed each second.  Each instruction
that a computer can execute takes some integer number of clock cycles.
Different instructions may take different numbers of clock cycles.  An
instruction like ``add the contents of registers 1 and 2 and store the result
in register 3'' will typically take only 2 clock cycles.  On the other hand
the instruction ``load x into register 1'' can take a varying number of clock
cycles depending on where x is stored. If x is in cache because it has been
recently used, this instruction may take only a few cycles.  If it is not in
cache and it must be loaded from main memory, it might take 100 cycles.  If
the data set used by the program is so huge that it doesn't all fit in
memory and x must be retrieved from the hard disk, it can be orders of
magnitude slower.

So knowing how fast your computer's processor is in Hertz does not
necessarily directly tell you how quickly a given program will execute.  It
depends on what the program is doing and also on other factors such as how
fast the memory accesses are.


\subsection{Flops}
\label{metrics:flops}\label{metrics:id2}
In scientific computing we frequently write programs that perform many
\emph{floating point operations} such as multiplication or addition of two
floating point numbers.  A floating point operation is often called
a \textbf{flop}.

For many algorithms it is relatively easy to estimate how many flops are
required.  For example, multiplying an \(n\times n\)
matrix by a vector of length n
requires roughly \(n^2\) flops.  So it is convenient to know roughly how
many floating point operations the computer can perform in one second.  In
this context \textbf{flops} often stands for \emph{floating point operations per
second}.  For example, a computer with a peak speed of 100 Mflops can
perform up to 100 million floating point operations per second.  As in the
above discussion on clock speed, the actual performance on a particular problem
typically depends very much on factors other than the peak speed, which is
generally measured assuming that all the data needed for the floating point
operations is already in cache so there is no time wasted fetching data.  On
a real problem there may be many more clock cycles used on memory accesses
than on the actual floating point operations.
Because of this, counting flops is no longer a reliable way to determine
how long a program will take to execute.  It's also not possible to
compare the relative merits of two algorithms for the same problem simply by
counting how many arithmetic operations are required.  For large problems,
the way that data is accessed can be at least as important.


\section{Computer Architecture}
\label{computer_arch:computer-architecture}\label{computer_arch:computer-arch}\label{computer_arch::doc}
This page is currently a very brief survey of a few issues.  See the links
below for more details.

Not so long ago, most computers consisted of a single central processing unit
(CPU) together with some a few registers (high speed storage locations for
data currently being processed) and \emph{main memory}, typically a hard disk
from which data is moved in and out of the registers as needed (a slower
operation).  Computer speed, at least for doing large scale scientific
computations, was generally limited by the speed of the CPU.  Processor
speeds are often quoted in terms of \emph{Hertz} the number of \emph{machine cycles
per second} (or these days in terms of GigaHertz, GHz, in billions of cycles
per second).  A single operation like adding or multiplying two numbers
might take more than one cycle, how many depends on the architecture and
other factors we'll get to later.  For scientific codes, the most important
metric was often how many floating point operations could be done per
second, measured in flops (or Gigaflops, etc.).
See {\hyperref[metrics:flops]{\crossref{\DUrole{std,std-ref}{Flops}}}} for more about this.


\subsection{Moore's Law}
\label{computer_arch:moore-s-law}\label{computer_arch:moores-law}
For many years computers kept getting faster primarily because the clock
cycle was reduced and so the CPU was made faster.  In 1965, Gordon Moore
(co-founder of Intel) predicted that the transistor density (and hence the
speed) of chips would double every 18 months for the forseeable future.
This is know as \textbf{Moore's law}
This proved remarkably accurate for more than 40 years, see the graphs at
\href{http://en.wikipedia.org/wiki/Moore\%27s\_law}{{[}wikipedia-moores-law{]}}.
Note that doubling every 18 months means an increase by a factor of 4096
every 14 years.

Unfortunately, the days of just waiting for a faster computer in order to do
larger calculations has come to an end.  Two primary considerations are:
\begin{itemize}
\item {} 
The limit has nearly been reached of how densely transistors can be
packed and how fast a single processor can be made.

\item {} 
Even current processors can generally do computations much more quickly
than sufficient quantities of data can be moved between memory and
the CPU.  If you are doing 1 billion meaningful multiplies per second
you need to move lots of data around.

\end{itemize}

There is a hard limit imposed by the speed of light.  A 2 GHz
processor has a clock cycle of 0.5e-9 seconds.  The speed of light is
about 3e8 meters per second. So in one clock cycle information cannot
possibly travel more than 0.15 meters.  (A light year is a long distance but
a light nanosecond is only about 1 foot.)  If you're trying to move billions
of bits of information each second then you have a problem.

Another major problem is power consumption.  Doubling the clock speed of a
processor takes roughly 8 times as much power and also produces much more
heat.  By contrast, doubling the number of processors only takes twice as
much power.

There are ways to continue improving computing power in the future, but they
must include two things:
\begin{itemize}
\item {} 
Increasing the number of cores or CPUs that are being used simultaneously
(i.e., parallel computing)

\item {} 
Using memory hierachies to improve the ability to have large amounts of
data available to the CPU when it needs it.

\end{itemize}


\subsection{Further reading}
\label{computer_arch:further-reading}
See the {\hyperref[biblio:biblio\string-computer\string-arch]{\crossref{\DUrole{std,std-ref}{Computer architecture references}}}} section of the bibliography.

See also the slides from \phantomsection\label{computer_arch:id1}{\hyperref[biblio:yelick\string-ucb]{\crossref{{[}Yelick-UCB{]}}}}, \phantomsection\label{computer_arch:id2}{\hyperref[biblio:gropp\string-uiuc]{\crossref{{[}Gropp-UIUC{]}}}} and other courses
listed in the bibliography.


\section{Storing information in binary}
\label{memory:storing-information-in-binary}\label{memory::doc}\label{memory:memory}
All information stored in a computer must somehow be encoded as a sequence
of 0's and 1's, because all
storage devices consist of a set of locations that can have one of two possible
states.  One state represents 0, the other state represents 1.  For example,
on a CD or DVD there are billions of locations where either small pit has
been created by a laser beam (representing a 1) or no pit exists
(representing a 0).  An old magnetic tape (such as a audio cassette tape or
VHS video tape) consisted of a sequence of locations that could be
magnetized with an upward or downward polarization, representing 0 or 1.

A single storage location stores a single bit (binary digit) of information.
A set of 8 bits is a byte and this is generally the smallest unit of
information a computer deals with.  A byte can store \(2^8 = 256\) different
patterns of 0's and 1's and these different patterns might represent
different things, depending on the context.


\subsection{Integers}
\label{memory:integers}
If we want to store an integer
then it makes sense to store the binary representation of the integer, and
in one byte we could store any of the numbers 0 through 255, with the usual
binary representation:

\begin{Verbatim}[commandchars=\\\{\}]
\PYG{l+m+mi}{00000000} \PYG{o}{=} \PYG{l+m+mi}{0}
\PYG{l+m+mi}{00000001} \PYG{o}{=} \PYG{l+m+mi}{1}
\PYG{l+m+mi}{00000010} \PYG{o}{=} \PYG{l+m+mi}{2}
\PYG{l+m+mi}{00000011} \PYG{o}{=} \PYG{l+m+mi}{3}
\PYG{l+m+mi}{00000100} \PYG{o}{=} \PYG{l+m+mi}{4}
\PYG{n}{etc}\PYG{o}{.}
\PYG{l+m+mi}{11111111} \PYG{o}{=} \PYG{l+m+mi}{255}
\end{Verbatim}

Of course many practical problems involve integers larger than 256, and possibly
negative integers as well as positive.  So in practice a single integer is
generally stored using more than 1 byte.  The default for most computer
languages is to use 4 bytes for an integer, which is 32 bits.  One bit is
used to indicate whether the number is positive of negative, leaving 31 bits
to represent the values from 0 to \(2^{31} = 2147483648\) as well as the
negatives of these values.  Actually it's a bit more complicated (no pun
intended) since the scheme just described allows storing both +0 and -0  and
a more complicated system allows storing one more integer, and in practice
the two's complement representation is used, as described at
\href{http://en.wikipedia.org/wiki/Two\%27s\_complement}{{[}wikipedia{]}}, which
shows a table of how the numbers -128 through 127 would actually be
represented in one byte.


\subsection{Real numbers}
\label{memory:real-numbers}
If we are dealing with real numbers rather than integers, a more complicated
system is needed for storing arbitrary numbers over a fairly wide range
in a fixed number of bytes.

Note that fractions can be represented in binary using inverse powers of 2.
For example, the decimal number 5.625 can be expressed as
\(4+1+\frac 1 2 + \frac 1 8\), i.e.:

\begin{Verbatim}[commandchars=\\\{\}]
\PYG{l+m+mf}{5.625} \PYG{o}{=} \PYG{l+m+mi}{1}\PYG{o}{*}\PYG{l+m+mi}{2}\PYG{o}{\PYGZca{}}\PYG{l+m+mi}{2} \PYG{o}{+} \PYG{l+m+mi}{0}\PYG{o}{*}\PYG{l+m+mi}{2}\PYG{o}{\PYGZca{}}\PYG{l+m+mi}{1} \PYG{o}{+} \PYG{l+m+mi}{1}\PYG{o}{*}\PYG{l+m+mi}{2}\PYG{o}{\PYGZca{}}\PYG{l+m+mi}{0} \PYG{o}{+} \PYG{l+m+mi}{1}\PYG{o}{*}\PYG{l+m+mi}{2}\PYG{o}{\PYGZca{}}\PYG{p}{\PYGZob{}}\PYG{o}{\PYGZhy{}}\PYG{l+m+mi}{1}\PYG{p}{\PYGZcb{}} \PYG{o}{+} \PYG{l+m+mi}{0}\PYG{o}{*}\PYG{l+m+mi}{2}\PYG{o}{\PYGZca{}}\PYG{p}{\PYGZob{}}\PYG{o}{\PYGZhy{}}\PYG{l+m+mi}{2}\PYG{p}{\PYGZcb{}} \PYG{o}{+} \PYG{l+m+mi}{1}\PYG{o}{*}\PYG{l+m+mi}{2}\PYG{o}{\PYGZca{}}\PYG{p}{\PYGZob{}}\PYG{o}{\PYGZhy{}}\PYG{l+m+mi}{3}\PYG{p}{\PYGZcb{}}
\end{Verbatim}

and hence as 101.101 in binary.

Early computers used \emph{fixed point} notation in
which it was always assumed that a certain number of bits to the right of
the decimal point were stored.  This does not work well for most scientific
computations, however.  Instead computers now store real numbers as
\emph{floating point numbers}, by storing a \emph{mantissa} and an \emph{exponent}.

The decimal number 5.625 could be written as \(0.5625 \times 10^1\) in
normalized form, with mantissa 0.5625 and exponent 1.

Similarly, the binary number 101.101 in floating point form has mantissa
0.101101 and exponent 10 (the number 2 in binary, since the mantissa must be
multiplied by \(2^2 = 4\) to shift the binary point by two spaces).

Most scientific computation is done using 8-byte representation of real
numbers (64 bits) in which 52 bits are used for the mantissa and 11 bits for
the exponent (and one for the sign).
See \href{http://en.wikipedia.org/wiki/Floating\_point}{{[}wikipedia{]}} for more
details.  This is the standard for objects of type \emph{float} in Python.  In
Fortran this is sometimes called \emph{double precision} because 4-byte floating
point numbers (\emph{single precision}) were commonly used for non-scientific
applications.  8 byte floats are generally inadequate for most scientific
computing problems, but there are some problems for which higher precision
(e.g. \emph{quad precision}, 16 bytes) is required.

Before the 1980's, different computer manufacturers came up with their own
conventions for how to store real numbers, often handling computer
arithmetic poorly and leading to severe problems in portability of computer
codes between machines.  The IEEE standards have largely solved this
problem.
See \href{http://en.wikipedia.org/wiki/IEEE\_floating\_point}{{[}wikipedia{]}} for more
details.


\subsection{Text}
\label{memory:text}
If we are storing text, such as the words you are now reading, the
characters must also be encoded as strings of 0's and 1's.  Since a single
byte can represent 256 different things, text is generally encoded using one
byte for each character.  In the English language we need 52 different
patterns to represent all the possible letters (one for each lower case
letter and a distinct pattern for the corresponding upper case letter).  We
also need 10 patterns for the digits and a fairly large number of other
patterns to represent each other symbol (e.g. punctuation symbols, dollar
signs, etc.) that might be used.

A standard 8-bit encoding is UTF-8
\href{http://en.wikipedia.org/wiki/UTF-8}{{[}wikipedia{]}}.
This is an extension of the earlier standard called ASCII
\href{http://en.wikipedia.org/wiki/ASCII}{{[}wikipedia{]}}, which only used 7 bits.
For encoding a wider variety of symbols and characters (such as
Chinese, Arabic, etc.) there are standard encodings UTF-16 and
UTF-32 using more bits for each character.

See {\hyperref[punchcard:punchcard]{\crossref{\DUrole{std,std-ref}{Punch cards}}}} for an example of how the ASCII character are
represented on an punch card, an early form of computer memory.

Obviously, in order to interpret a byte stored in the computer, such as
01001011 properly, the computer needs to know whether it represents a UTF-8
character, a 1-byte integer, or something else.


\subsection{Colors}
\label{memory:colors}\label{memory:id5}
Another thing a string of 0's and 1's
might represent is a color, for example one pixel in an image that is
being displayed.  Each pixel is one dot of light and a string of 0's and 1's
must be used to indicate what color each pixel should be.  There are various
possible ways to specify a color.  One that is often used is to specify an
RGB triple, three integers that indicate the amount of Red, Green, and Blue
in the desired color.  Often each value is allowed to range from 0
(indicating none) to 255 (maximal amount). These values can all be stored in
1 byte of data, so with this system 3 bytes (24 bits) are used to store the
color of a single pixel.  The color red, for example, has maximal R and 0
for G and B and hence has the first byte 256 and the next two bytes 0 and 0.
Here is red a few other colors in their RGB and binary representations:

\begin{Verbatim}[commandchars=\\\{\}]
\PYG{p}{[}\PYG{l+m+mi}{255}\PYG{p}{,}  \PYG{l+m+mi}{0}\PYG{p}{,}  \PYG{l+m+mi}{0}\PYG{p}{]}  \PYG{o}{=} \PYG{l+m+mi}{11111111} \PYG{l+m+mi}{00000000} \PYG{l+m+mi}{00000000} \PYG{o}{=} \PYG{n}{red}
\PYG{p}{[}  \PYG{l+m+mi}{0}\PYG{p}{,}  \PYG{l+m+mi}{0}\PYG{p}{,}  \PYG{l+m+mi}{0}\PYG{p}{]}  \PYG{o}{=} \PYG{l+m+mi}{00000000} \PYG{l+m+mi}{00000000} \PYG{l+m+mi}{00000000} \PYG{o}{=} \PYG{n}{black}
\PYG{p}{[}\PYG{l+m+mi}{255}\PYG{p}{,}\PYG{l+m+mi}{255}\PYG{p}{,}\PYG{l+m+mi}{255}\PYG{p}{]}  \PYG{o}{=} \PYG{l+m+mi}{11111111} \PYG{l+m+mi}{11111111} \PYG{l+m+mi}{11111111} \PYG{o}{=} \PYG{n}{white}
\PYG{p}{[} \PYG{l+m+mi}{57}\PYG{p}{,} \PYG{l+m+mi}{39}\PYG{p}{,} \PYG{l+m+mi}{91}\PYG{p}{]}  \PYG{o}{=} \PYG{l+m+mi}{00111001} \PYG{l+m+mi}{00100111} \PYG{l+m+mi}{01011011} \PYG{o}{=} \PYG{n}{official} \PYG{n}{Husky} \PYG{n}{purple}
\PYG{p}{[}\PYG{l+m+mi}{240}\PYG{p}{,}\PYG{l+m+mi}{213}\PYG{p}{,}\PYG{l+m+mi}{118}\PYG{p}{]}  \PYG{o}{=} \PYG{l+m+mi}{11110000} \PYG{l+m+mi}{11010101} \PYG{l+m+mi}{01110110} \PYG{o}{=} \PYG{n}{official} \PYG{n}{Husky} \PYG{n}{gold}
\end{Verbatim}

Colors in html documents and elsewhere are often specified by writing out
exactly what each values each of the bytes should have.  Writing out the
bits as above is generally awkward for humans, so instead graphics languages
like Matlab or Matplotlib in Python generally allow you to specify the
RGB triple in terms of fractions between 0 and 1 (divide the RGB values
above by 255 to get the fractions):

\begin{Verbatim}[commandchars=\\\{\}]
\PYG{p}{[}\PYG{l+m+mf}{1.0}\PYG{p}{,} \PYG{l+m+mf}{0.0}\PYG{p}{,} \PYG{l+m+mf}{0.0}\PYG{p}{]}  \PYG{o}{=} \PYG{n}{red}
\PYG{p}{[}\PYG{l+m+mf}{0.0}\PYG{p}{,} \PYG{l+m+mf}{0.0}\PYG{p}{,} \PYG{l+m+mf}{0.0}\PYG{p}{]}  \PYG{o}{=} \PYG{n}{black}
\PYG{p}{[}\PYG{l+m+mf}{1.0}\PYG{p}{,} \PYG{l+m+mf}{1.0}\PYG{p}{,} \PYG{l+m+mf}{1.0}\PYG{p}{]}  \PYG{o}{=} \PYG{n}{white}
\PYG{p}{[} \PYG{l+m+mf}{0.22352941}\PYG{p}{,}  \PYG{l+m+mf}{0.15294118}\PYG{p}{,}  \PYG{l+m+mf}{0.35686275}\PYG{p}{]} \PYG{o}{=} \PYG{n}{official} \PYG{n}{Husky} \PYG{n}{purple}
\PYG{p}{[} \PYG{l+m+mf}{0.94117647}\PYG{p}{,}  \PYG{l+m+mf}{0.83529412}\PYG{p}{,}  \PYG{l+m+mf}{0.4627451} \PYG{p}{]} \PYG{o}{=} \PYG{n}{official} \PYG{n}{Husky} \PYG{n}{gold}
\end{Verbatim}

Another way is common in html documents (and also allowed in Matplotlib),
where the color red is denoted by the string:

\begin{Verbatim}[commandchars=\\\{\}]
\PYG{l+s}{\PYGZsq{}}\PYG{l+s}{\PYGZsh{}ff0000}\PYG{l+s}{\PYGZsq{}} \PYG{o}{=} \PYG{n}{red}
\PYG{l+s}{\PYGZsq{}}\PYG{l+s}{\PYGZsh{}000000}\PYG{l+s}{\PYGZsq{}} \PYG{o}{=} \PYG{n}{black}
\PYG{l+s}{\PYGZsq{}}\PYG{l+s}{\PYGZsh{}ffffff}\PYG{l+s}{\PYGZsq{}} \PYG{o}{=} \PYG{n}{white}
\PYG{l+s}{\PYGZsq{}}\PYG{l+s}{\PYGZsh{}39275b}\PYG{l+s}{\PYGZsq{}} \PYG{o}{=} \PYG{n}{official} \PYG{n}{Husky} \PYG{n}{purple}
\PYG{l+s}{\PYGZsq{}}\PYG{l+s}{\PYGZsh{}f0d576}\PYG{l+s}{\PYGZsq{}} \PYG{o}{=} \PYG{n}{official} \PYG{n}{Husky} \PYG{n}{gold}
\end{Verbatim}

This string is written using \emph{hexadecimal} notation (see below).
The \# sign just
indicates that it's a hexidecimal string and it is followed by 3 2-digit
hexidecimal numbers, e.g. for red they are ff, 00, and 00.

The latter two colors can be seen in the header of this webpage.

To find the hex string for a desired color, or view the color for a given
string, try the \href{http://acko.net/dev/farbtastic}{farbtastic demo}.
See also \href{http://en.wikipedia.org/wiki/Web\_colors}{{[}wikipedia-web-colors{]}}
or \href{http://www.webdiner.com/annexe/hexcode/hexcode.htm}{this page of colors}
showing RGB triples and hexcodes.


\subsection{Hexidecimal numbers}
\label{memory:hex}\label{memory:hexidecimal-numbers}
A hexidecimal number is in base 16, e.g. the hexidecimal 345 represents:

\begin{Verbatim}[commandchars=\\\{\}]
\PYG{l+m+mi}{345} \PYG{o}{=} \PYG{l+m+mi}{3}\PYG{o}{*}\PYG{p}{(}\PYG{l+m+mi}{16}\PYG{p}{)}\PYG{o}{*}\PYG{o}{*}\PYG{l+m+mi}{2} \PYG{o}{+} \PYG{l+m+mi}{4}\PYG{o}{*}\PYG{p}{(}\PYG{l+m+mi}{16}\PYG{p}{)}\PYG{o}{*}\PYG{o}{*}\PYG{l+m+mi}{1} \PYG{o}{+} \PYG{l+m+mi}{5}
\end{Verbatim}

which is 837 in decimal notation.  Each hexidecimal digit can take one of 16
values and so in addition to the digits 0, 1, ..., 9 we need symbols to
represent 10, 11, 12, 13, 14, and 15.  For these the letters a,b,c,d,e,f are
used, so for example:

\begin{Verbatim}[commandchars=\\\{\}]
\PYG{n}{a4e} \PYG{o}{=} \PYG{l+m+mi}{10}\PYG{o}{*}\PYG{p}{(}\PYG{l+m+mi}{16}\PYG{p}{)}\PYG{o}{*}\PYG{o}{*}\PYG{l+m+mi}{2} \PYG{o}{+} \PYG{l+m+mi}{4}\PYG{o}{*}\PYG{p}{(}\PYG{l+m+mi}{16}\PYG{p}{)} \PYG{o}{+} \PYG{l+m+mi}{14}
\end{Verbatim}

which is 2638 in decimal notation.

Hex notation is a convenient way to express binary numbers because there is
a very simple way to translate between hex and binary.  To convert the hex
number a4e to binary, for example, just translate each hex digit separately
into binary, a = 1010, 4 = 0100, e = 1110, and then catenate these together,
so:

\begin{Verbatim}[commandchars=\\\{\}]
\PYG{n}{a4e} \PYG{o}{=} \PYG{l+m+mi}{101001001110}
\end{Verbatim}

in binary.  Conversely, to convert a binary number such as 100101101 to hex,
group the bits in groups of 4 (starting at the right, adding 0's to the left
if necessary) and convert each group into a hex digit:

\begin{Verbatim}[commandchars=\\\{\}]
\PYG{l+m+mi}{1100101101} \PYG{o}{=} \PYG{l+m+mi}{0011} \PYG{l+m+mi}{0010} \PYG{l+m+mi}{1101} \PYG{o}{=} \PYG{l+m+mi}{32}\PYG{n}{d} \PYG{o+ow}{in} \PYG{n+nb}{hex}\PYG{o}{.}
\end{Verbatim}

Returning to the hex notation for colors, we can see that `\#ff0000'
corresponds to 111111110000000000000000, the binary string representing pure
red.


\subsection{Machine instructions}
\label{memory:machine-instructions}
In addition to storing data (numbers, text, colors, etc.) as strings of
bits, we must also store the computer instructions that make up a computer
program as a string of bits.  This may seem obvious now, but was actually a
revolutionary idea when it was first proposed by the eminant mathematician
(and one of the earliest ``computer scientists'') John von Neumann.  The idea
of a ``von Neumann'' machine in which the computer program is stored in the
same way the data is, and can be modified at will to cause the computer to
do different things, originated in 1930s and 40s through work of von
Neumann, Turing, and Zuse (see \href{http://en.wikipedia.org/wiki/Von\_Neumann\_architecture}{{[}wikipedia{]}}).
Prior to this time, computers were designed to do one particular set of
operations, executing one algorithm, and only the data could be easily
changed.  To ``reprogram'' the computer to do something different required
physically rewiring the circuits.

Current computers store a sequence of instructions to be executed, as binary
strings.  These strings are different than the strings that might represent
the instructions when the human-readable program is converted to ASCII to
store as a text file.

See \DUrole{xref,std,std-ref}{machine\_code\_and\_assembly} for more about this.


\section{Punch cards}
\label{punchcard:punch-cards}\label{punchcard:punchcard}\label{punchcard::doc}
Once upon a time (through the 1970s) many computer programs were written on
punch cards of the type shown here {[}\href{http://en.wikipedia.org/wiki/File:FortranCardPROJ039.agr.jpg}{image source}{]}:

\includegraphics[width=20cm]{{FortranCardPROJ039-agr}.jpg}

This is a form of binary memory (see {\hyperref[memory:memory]{\crossref{\DUrole{std,std-ref}{Storing information in binary}}}})
where a specific set of locations each has a
hole punched (representing 1) or not (representing 0).
Each character typed on the top line is represented by the bits reading down
the corresponding column, e.g. the first character Z is represented by
001000000001 since there are only two holes punched in this column.

When programs were written on cards, one card was required for each line of
the program.  The early conventions of the {\hyperref[fortran:fortran]{\crossref{\DUrole{std,std-ref}{Fortran}}}} programming
language are related to the columns on a punch card.
Only the first 72 columns were used for the program statements.  The last 8
columns could be used to print a card number, so that the unlucky programmer
who dropped a deck of cards had some chance of reordering them properly.
Many Fortran 77  compilers still ignore any characters beyond column 72 in a
line of a program, leading to bugs that can be hard to find if care is not
taken, e.g. a called \emph{xnew} might be truncated to \emph{x} if the \emph{x} is in
column 72.  If the program also uses a variable called \emph{x}, this will not be
caught by the compiler.

A program would require a \emph{punch card deck} as shown in this photo
{[}\href{http://en.wikipedia.org/wiki/File:PunchCardDecks.agr.jpg}{image source}{]}:

\includegraphics[width=15cm]{{PunchCardDecks-agr}.jpg}

Punch cards were a great step forward from \emph{punched tape} \href{http://en.wikipedia.org/wiki/Punched\_tape}{{[}wikipedia{]}} where a long strip of paper
was used to store the entire program (and making a mistake required retyping
more than just one card).


\section{Python and Fortran}
\label{python_and_fortran:python-and-fortran}\label{python_and_fortran::doc}\label{python_and_fortran:id1}
In this class we will use both Python and Fortran.  Why these two?
\begin{itemize}
\item {} 
They are different types of languages that allow illustrating the
difference between interperted vs. compiled languages, object oriented
vs. procedural languages, etc.

\item {} 
Each is very useful for certain common tasks in scientific computing.
They can easily be combined to take advantage of the best of both worlds.

\item {} 
Many scientific programs are written in one of these languages so you
should be familiar with them.

\item {} 
Both are freely available and students can set up laptops and desktops to
use them beyond this class.

\end{itemize}

Learning two new languages in a quarter along with the many other topics we
will be covering may seem overwhelming, but in many ways it makes sense to
learn them together.  By comparing features in the two languages it may help
clarify what the essential concepts are, and what is truly different about
the languages vs. simply different syntax or choices of convention.

We will not be writing extensive programs from scratch in this class.
Instead, many homework sets will require making relatively small changes to
Python of Fortran programs that have been provided as templates.  This
obviously isn't enough to become an expert programmer in either language,
but the goal in this course is to get you started down that path.


\subsection{Further reading}
\label{python_and_fortran:further-reading}

\chapter{Python}
\label{index:python}\label{index:toc-python}

\section{Python}
\label{python:python}\label{python::doc}\label{python:id1}
These notes only scratch the surface of Python, with discussion of a few
features of the language that are most important to getting started and to
appreciating how Python can be used in computational science.

See the references below or the {\hyperref[biblio:biblio\string-python]{\crossref{\DUrole{std,std-ref}{Python:}}}} section of the
{\hyperref[biblio:biblio]{\crossref{\DUrole{std,std-ref}{Bibliography and further reading}}}} for more detailed references.

See also the slides from lectures.


\subsection{Interactive Python}
\label{python:interactive-python}
The IPython shell is generally recommended for interactive work in Python
(see \url{http://ipython.org/documentation.html}), but for most examples we'll display the \textgreater{}\textgreater{}\textgreater{} prompt of the
standard Python shell.

Normally multiline Python statements are best written in a text file rather
than typing them at the prompt, but some of the short examples below are
done at the prompt.  If type a line that Python recognizes as an unfinished
block, it will give a line starting with three dots, like:

\begin{Verbatim}[commandchars=\\\{\}]
\PYG{g+gp}{\PYGZgt{}\PYGZgt{}\PYGZgt{} }\PYG{k}{if} \PYG{l+m+mi}{1}\PYG{o}{\PYGZgt{}}\PYG{l+m+mi}{2}\PYG{p}{:}
\PYG{g+gp}{... }   \PYG{n+nb}{print} \PYG{l+s}{\PYGZdq{}}\PYG{l+s}{oops!}\PYG{l+s}{\PYGZdq{}}
\PYG{g+gp}{... }\PYG{k}{else}\PYG{p}{:}
\PYG{g+gp}{... }   \PYG{n+nb}{print} \PYG{l+s}{\PYGZdq{}}\PYG{l+s}{this is what we expect}\PYG{l+s}{\PYGZdq{}}
\PYG{g+gp}{...}
\PYG{g+go}{this is what we expect}
\PYG{g+go}{\PYGZgt{}\PYGZgt{}\PYGZgt{}}
\end{Verbatim}

Once done with the full command, typing \textless{}return\textgreater{} alone at the ... prompt
tells Python we are done and it executes the command.


\subsection{Indentation}
\label{python:indentation}
Most computer languages have some form of begin-end structure, or
opening and closing braces, or some such thing to clearly delinieate
what piece of code is in a loop, or in different parts of an
if-then-else structure like what's shown above.  Good programmers
generally also indent their code so it is easier for a reader to
see what is inside a loop, particularly if there are multiple nested
loops.  But in most languages this is indentation is just a matter
of style and the begin-end structure of the language determines how it is
actually interpreted by the computer.

\textbf{In Python, indentation is everything}.  There are no begin-end's, only
indentation.  Everything that is supposed to be at one level of a loop must
be indented to that level.  Once the loop is done the indentation must go
back out to the previous level.  There are some other rules you need to
learn, such as that the ``else'' in and if-else block like the above has to be
indented exactly the same as as the ``if''.  See \DUrole{xref,std,std-ref}{if\_else} for more about
this.

How many spaces to indent each level is a matter of style, but you must be
consistent within a single code.  The standard is often 4 spaces.


\subsection{Wrapping lines}
\label{python:wrapping-lines}
In Python normally each statement is one line, and there is no need to use
separators such as the semicolon used in some languages to end a line.  One
the other hand you can use a semicolon to put several short statements on a
single line, such as:

\begin{Verbatim}[commandchars=\\\{\}]
\PYG{g+gp}{\PYGZgt{}\PYGZgt{}\PYGZgt{} }\PYG{n}{x} \PYG{o}{=} \PYG{l+m+mi}{5}\PYG{p}{;} \PYG{n+nb}{print} \PYG{n}{x}
\PYG{g+go}{5}
\end{Verbatim}

It is easiest to read codes if you avoid this in most cases.

If a line of code is too long to fit on a single line, you can break it into
multiple lines by putting a backslash at the end of a line:

\begin{Verbatim}[commandchars=\\\{\}]
\PYG{g+gp}{\PYGZgt{}\PYGZgt{}\PYGZgt{} }\PYG{n}{y} \PYG{o}{=} \PYG{l+m+mi}{3} \PYG{o}{+} \PYGZbs{}
\PYG{g+gp}{... }    \PYG{l+m+mi}{4}
\PYG{g+gp}{\PYGZgt{}\PYGZgt{}\PYGZgt{} }\PYG{n}{y}
\PYG{g+go}{7}
\end{Verbatim}


\subsection{Comments}
\label{python:comments}
Anything following a \# in a line is ignored as a comment (unless of course
the \# appears in a string):

\begin{Verbatim}[commandchars=\\\{\}]
\PYG{g+gp}{\PYGZgt{}\PYGZgt{}\PYGZgt{} }\PYG{n}{s} \PYG{o}{=} \PYG{l+s}{\PYGZdq{}}\PYG{l+s}{This \PYGZsh{} is part of the string}\PYG{l+s}{\PYGZdq{}}  \PYG{c}{\PYGZsh{} this is a comment}
\PYG{g+gp}{\PYGZgt{}\PYGZgt{}\PYGZgt{} }\PYG{n}{s}
\PYG{g+go}{\PYGZsq{}This \PYGZsh{} is part of the string\PYGZsq{}}
\end{Verbatim}

There is another form of comment, the \emph{docstring}, discussed below following
an introduction to strings.


\subsection{Strings}
\label{python:strings}
Strings are specified using either single or double quotes:

\begin{Verbatim}[commandchars=\\\{\}]
\PYG{g+gp}{\PYGZgt{}\PYGZgt{}\PYGZgt{} }\PYG{n}{s} \PYG{o}{=} \PYG{l+s}{\PYGZsq{}}\PYG{l+s}{some text}\PYG{l+s}{\PYGZsq{}}
\PYG{g+gp}{\PYGZgt{}\PYGZgt{}\PYGZgt{} }\PYG{n}{s} \PYG{o}{=} \PYG{l+s}{\PYGZdq{}}\PYG{l+s}{some text}\PYG{l+s}{\PYGZdq{}}
\end{Verbatim}

are the same.  This is useful if you want strings that themselves contain
quotes of a different type.

You can also use triple double quotes, which have the advantage that they
such strings can span multiple lines:

\begin{Verbatim}[commandchars=\\\{\}]
\PYG{g+gp}{\PYGZgt{}\PYGZgt{}\PYGZgt{} }\PYG{n}{s} \PYG{o}{=} \PYG{l+s}{\PYGZdq{}\PYGZdq{}\PYGZdq{}}\PYG{l+s}{Note that a }\PYG{l+s}{\PYGZsq{}}\PYG{l+s}{ doesn}\PYG{l+s}{\PYGZsq{}}\PYG{l+s}{t end}
\PYG{g+gp}{... }\PYG{l+s}{this string and that it spans two lines}\PYG{l+s}{\PYGZdq{}\PYGZdq{}\PYGZdq{}}

\PYG{g+gp}{\PYGZgt{}\PYGZgt{}\PYGZgt{} }\PYG{n}{s}
\PYG{g+go}{\PYGZdq{}Note that a \PYGZsq{} doesn\PYGZsq{}t end\PYGZbs{}nthis string and that it spans two lines\PYGZdq{}}

\PYG{g+gp}{\PYGZgt{}\PYGZgt{}\PYGZgt{} }\PYG{n+nb}{print} \PYG{n}{s}
\PYG{g+go}{Note that a \PYGZsq{} doesn\PYGZsq{}t end}
\PYG{g+go}{this string and that it spans two lines}
\end{Verbatim}

When it prints, the carriage return at the end of the line show up as ``n''.
This is what is actually stored. When we ``print s'' it gets printed as a
carriage return again.

You can put ``n'' in your strings as another way to break lines:

\begin{Verbatim}[commandchars=\\\{\}]
\PYG{g+gp}{\PYGZgt{}\PYGZgt{}\PYGZgt{} }\PYG{n+nb}{print} \PYG{l+s}{\PYGZdq{}}\PYG{l+s}{This spans }\PYG{l+s+se}{\PYGZbs{}n}\PYG{l+s}{ two lines}\PYG{l+s}{\PYGZdq{}}
\PYG{g+go}{This spans}
\PYG{g+go}{ two lines}
\end{Verbatim}

See {\hyperref[python_strings:python\string-strings]{\crossref{\DUrole{std,std-ref}{Python strings}}}} for more about strings.


\subsection{Docstrings}
\label{python:docstrings}
Often the first thing you will see in a Python script or module, or in a
function or class defined in a module, is a brief description that is
enclosed in triple quotes.   Although ordinarily this would just be a
string, in this special position it is interpreted by Python as a comment
and is not part of the code.  It is called the \emph{docstring} because it is
part of the documentation and some Python tools automatically use the
docstring in various ways.  See \DUrole{xref,std,std-ref}{ipython} for one example.  Also the
documentation formatting program Sphinx that is used to create these class
notes can automatically take a Python module and create html or latex
documentation for it by using the docstrings, the original purpose for which
Sphinx was developed. See {\hyperref[sphinx:sphinx]{\crossref{\DUrole{std,std-ref}{Sphinx documentation}}}} for more about this.

It's a good idea to get in the habit of putting a docstring at the top of
every Python file and function you write.


\subsection{Running Python scripts}
\label{python:running-python-scripts}
Most Python programs are written in text files ending with the .py
extension.
Some of these are simple \emph{scripts} that are just a set of Python
instructions to be executed, the same things you might type at the \textgreater{}\textgreater{}\textgreater{}
prompt but collected in a file (which makes it much easier to modify or
reuse later).  Such a script can be run at the Unix command
line simply by typing ``python'' followed by the file name.

See {\hyperref[python_scripts_modules:python\string-scripts\string-modules]{\crossref{\DUrole{std,std-ref}{Python scripts and modules}}}} for some examples.
The section {\hyperref[python_scripts_modules:importing\string-modules]{\crossref{\DUrole{std,std-ref}{Importing modules}}}}
also contains important information on how to ``import'' modules,
and how to set the path of directories that are searched for modules when
you try to import a module.


\subsection{Python objects}
\label{python:id2}\label{python:python-objects}
Python is an object-oriented language, which just means that virtually
everything you encounter in Python (variables, functions, modules, etc.) is
an \emph{object} of some \emph{class}.  There are many classes of objects built into
Python and in this course we will primarily be using these pre-defined
classes.  For large-scale programming projects you would probably define
some new classes, which is easy to do.  (Maybe an example to come...)

The \emph{type} command can be used to reveal the type of an object:

\begin{Verbatim}[commandchars=\\\{\}]
\PYG{g+gp}{\PYGZgt{}\PYGZgt{}\PYGZgt{} }\PYG{k+kn}{import} \PYG{n+nn}{numpy} \PYG{k}{as} \PYG{n+nn}{np}
\PYG{g+gp}{\PYGZgt{}\PYGZgt{}\PYGZgt{} }\PYG{n+nb}{type}\PYG{p}{(}\PYG{n}{np}\PYG{p}{)}
\PYG{g+go}{\PYGZlt{}type \PYGZsq{}module\PYGZsq{}\PYGZgt{}}

\PYG{g+gp}{\PYGZgt{}\PYGZgt{}\PYGZgt{} }\PYG{n+nb}{type}\PYG{p}{(}\PYG{n}{np}\PYG{o}{.}\PYG{n}{pi}\PYG{p}{)}
\PYG{g+go}{\PYGZlt{}type \PYGZsq{}float\PYGZsq{}\PYGZgt{}}

\PYG{g+gp}{\PYGZgt{}\PYGZgt{}\PYGZgt{} }\PYG{n+nb}{type}\PYG{p}{(}\PYG{n}{np}\PYG{o}{.}\PYG{n}{cos}\PYG{p}{)}
\PYG{g+go}{\PYGZlt{}type \PYGZsq{}numpy.ufunc\PYGZsq{}\PYGZgt{}}
\end{Verbatim}

We see that \emph{np} is a module, \emph{np.pi} is a floating point real number, and
\emph{np.cos} is of a special class that's defined in the numpy module.

The \emph{linspace} command creates a numerical array that is also a special
numpy class:

\begin{Verbatim}[commandchars=\\\{\}]
\PYG{g+gp}{\PYGZgt{}\PYGZgt{}\PYGZgt{} }\PYG{n}{x} \PYG{o}{=} \PYG{n}{np}\PYG{o}{.}\PYG{n}{linspace}\PYG{p}{(}\PYG{l+m+mi}{0}\PYG{p}{,} \PYG{l+m+mi}{5}\PYG{p}{,} \PYG{l+m+mi}{6}\PYG{p}{)}
\PYG{g+gp}{\PYGZgt{}\PYGZgt{}\PYGZgt{} }\PYG{n}{x}
\PYG{g+go}{array([ 0.,  1.,  2.,  3.,  4.,  5.])}
\PYG{g+gp}{\PYGZgt{}\PYGZgt{}\PYGZgt{} }\PYG{n+nb}{type}\PYG{p}{(}\PYG{n}{x}\PYG{p}{)}
\PYG{g+go}{\PYGZlt{}type \PYGZsq{}numpy.ndarray\PYGZsq{}\PYGZgt{}}
\end{Verbatim}

Objects of a particular class generally have certain operations that are
defined on them as part of the class definition.  For example, NumPy
numerical arrays have a \emph{max} method defined, which we can use on \emph{x} in one
of two ways:

\begin{Verbatim}[commandchars=\\\{\}]
\PYG{g+gp}{\PYGZgt{}\PYGZgt{}\PYGZgt{} }\PYG{n}{np}\PYG{o}{.}\PYG{n}{max}\PYG{p}{(}\PYG{n}{x}\PYG{p}{)}
\PYG{g+go}{5.0}
\PYG{g+gp}{\PYGZgt{}\PYGZgt{}\PYGZgt{} }\PYG{n}{x}\PYG{o}{.}\PYG{n}{max}\PYG{p}{(}\PYG{p}{)}
\PYG{g+go}{5.0}
\end{Verbatim}

The first way applies the method \emph{max} defined in the \emph{numpy} module to \emph{x}.
The second way uses the fact that \emph{x}, by virtue of being of type
\emph{numpy.ndarray}, automatically has a \emph{max} method which can be invoked (on
itself) by calling the function \emph{x.max()} with no argument.  Which way is
better depends in part on what you're doing.

Here's another example:

\begin{Verbatim}[commandchars=\\\{\}]
\PYG{g+gp}{\PYGZgt{}\PYGZgt{}\PYGZgt{} }\PYG{n}{L} \PYG{o}{=} \PYG{p}{[}\PYG{l+m+mi}{0}\PYG{p}{,} \PYG{l+m+mi}{1}\PYG{p}{,} \PYG{l+m+mi}{2}\PYG{p}{]}
\PYG{g+gp}{\PYGZgt{}\PYGZgt{}\PYGZgt{} }\PYG{n+nb}{type}\PYG{p}{(}\PYG{n}{L}\PYG{p}{)}
\PYG{g+go}{\PYGZlt{}type \PYGZsq{}list\PYGZsq{}\PYGZgt{}}

\PYG{g+gp}{\PYGZgt{}\PYGZgt{}\PYGZgt{} }\PYG{n}{L}\PYG{o}{.}\PYG{n}{append}\PYG{p}{(}\PYG{l+m+mi}{4}\PYG{p}{)}
\PYG{g+gp}{\PYGZgt{}\PYGZgt{}\PYGZgt{} }\PYG{n}{L}
\PYG{g+go}{[0, 1, 2, 4]}
\end{Verbatim}

\emph{L} is a list (a standard Python class) and so has a method \emph{append} that
can be used to append an item to the end of the list.


\subsection{Declaring variables?}
\label{python:declaring-variables}
In many languages, such as Fortran, you must generally declare variables before
you can use them and once you've specified that \emph{x} is a real number, say,
that is the only type of things you can store in \emph{x}, and a statement like
\emph{x = `string'} would not be allowed.

In Python you don't declare variables, you can just type, for example:

\begin{Verbatim}[commandchars=\\\{\}]
\PYG{g+gp}{\PYGZgt{}\PYGZgt{}\PYGZgt{} }\PYG{n}{x} \PYG{o}{=} \PYG{l+m+mf}{3.4}
\PYG{g+gp}{\PYGZgt{}\PYGZgt{}\PYGZgt{} }\PYG{l+m+mi}{2}\PYG{o}{*}\PYG{n}{x}
\PYG{g+go}{6.7999999999999998}

\PYG{g+gp}{\PYGZgt{}\PYGZgt{}\PYGZgt{} }\PYG{n}{x} \PYG{o}{=} \PYG{l+s}{\PYGZsq{}}\PYG{l+s}{string}\PYG{l+s}{\PYGZsq{}}
\PYG{g+gp}{\PYGZgt{}\PYGZgt{}\PYGZgt{} }\PYG{l+m+mi}{2}\PYG{o}{*}\PYG{n}{x}
\PYG{g+go}{\PYGZsq{}stringstring\PYGZsq{}}

\PYG{g+gp}{\PYGZgt{}\PYGZgt{}\PYGZgt{} }\PYG{n}{x} \PYG{o}{=} \PYG{p}{[}\PYG{l+m+mi}{4}\PYG{p}{,} \PYG{l+m+mi}{5}\PYG{p}{,} \PYG{l+m+mi}{6}\PYG{p}{]}
\PYG{g+gp}{\PYGZgt{}\PYGZgt{}\PYGZgt{} }\PYG{l+m+mi}{2}\PYG{o}{*}\PYG{n}{x}
\PYG{g+go}{[4, 5, 6, 4, 5, 6]}
\end{Verbatim}

Here \emph{x} is first used for a real number, then for a character string, then
for a list.  Note, by the way,
that multiplication behaves differently for objects of
different type (which has been specified as part of the definition of each
class of objects).

In Fortran if you declare \emph{x} to be a real variable then it sets aside a
particular 8 bytes of memory for \emph{x}, enough to hold one floating point
number.  There's no way to store 6 characters or a list of 3 integers in
these 8 bytes.

In Python it is often better to think of \emph{x} as simply being a pointer
that points to some object.  When you type ``x = 3.4'' Python creates an
object of type \emph{float} holding one real number and points \emph{x} to that.  When
you type \emph{x = `string'} it creates a new object of type \emph{str} and now points \emph{x}
to that, and so on.


\subsection{Lists}
\label{python:id3}\label{python:lists}
We have already seen lists in the example above.

Note that indexing in Python always starts at 0:

\begin{Verbatim}[commandchars=\\\{\}]
\PYG{g+gp}{\PYGZgt{}\PYGZgt{}\PYGZgt{} }\PYG{n}{L} \PYG{o}{=} \PYG{p}{[}\PYG{l+m+mi}{4}\PYG{p}{,}\PYG{l+m+mi}{5}\PYG{p}{,}\PYG{l+m+mi}{6}\PYG{p}{]}
\PYG{g+gp}{\PYGZgt{}\PYGZgt{}\PYGZgt{} }\PYG{n}{L}\PYG{p}{[}\PYG{l+m+mi}{0}\PYG{p}{]}
\PYG{g+go}{4}
\PYG{g+gp}{\PYGZgt{}\PYGZgt{}\PYGZgt{} }\PYG{n}{L}\PYG{p}{[}\PYG{l+m+mi}{1}\PYG{p}{]}
\PYG{g+go}{5}
\end{Verbatim}

Elements of a list need not all have the same type.  For example, here's a
list with 5 elements:

\begin{Verbatim}[commandchars=\\\{\}]
\PYG{g+gp}{\PYGZgt{}\PYGZgt{}\PYGZgt{} }\PYG{n}{L} \PYG{o}{=} \PYG{p}{[}\PYG{l+m+mi}{5}\PYG{p}{,} \PYG{l+m+mf}{2.3}\PYG{p}{,} \PYG{l+s}{\PYGZsq{}}\PYG{l+s}{abc}\PYG{l+s}{\PYGZsq{}}\PYG{p}{,} \PYG{p}{[}\PYG{l+m+mi}{4}\PYG{p}{,}\PYG{l+s}{\PYGZsq{}}\PYG{l+s}{b}\PYG{l+s}{\PYGZsq{}}\PYG{p}{]}\PYG{p}{,} \PYG{n}{np}\PYG{o}{.}\PYG{n}{cos}\PYG{p}{]}
\end{Verbatim}

Here's a way to see what each element of the list is, and its type:

\begin{Verbatim}[commandchars=\\\{\}]
\PYG{g+gp}{\PYGZgt{}\PYGZgt{}\PYGZgt{} }\PYG{k}{for} \PYG{n}{index}\PYG{p}{,}\PYG{n}{value} \PYG{o+ow}{in} \PYG{n+nb}{enumerate}\PYG{p}{(}\PYG{n}{L}\PYG{p}{)}\PYG{p}{:}
\PYG{g+gp}{... }    \PYG{n+nb}{print} \PYG{l+s}{\PYGZsq{}}\PYG{l+s}{L[}\PYG{l+s}{\PYGZpc{}}\PYG{l+s}{s] is }\PYG{l+s}{\PYGZpc{}}\PYG{l+s}{16s     }\PYG{l+s}{\PYGZpc{}}\PYG{l+s}{s}\PYG{l+s}{\PYGZsq{}} \PYG{o}{\PYGZpc{}} \PYG{p}{(}\PYG{n}{index}\PYG{p}{,}\PYG{n}{value}\PYG{p}{,}\PYG{n+nb}{type}\PYG{p}{(}\PYG{n}{value}\PYG{p}{)}\PYG{p}{)}
\PYG{g+gp}{...}
\PYG{g+go}{L[0] is                5     \PYGZlt{}type \PYGZsq{}int\PYGZsq{}\PYGZgt{}}
\PYG{g+go}{L[1] is              2.3     \PYGZlt{}type \PYGZsq{}float\PYGZsq{}\PYGZgt{}}
\PYG{g+go}{L[2] is              abc     \PYGZlt{}type \PYGZsq{}str\PYGZsq{}\PYGZgt{}}
\PYG{g+go}{L[3] is         [4, \PYGZsq{}b\PYGZsq{}]     \PYGZlt{}type \PYGZsq{}list\PYGZsq{}\PYGZgt{}}
\PYG{g+go}{L[4] is    \PYGZlt{}ufunc \PYGZsq{}cos\PYGZsq{}\PYGZgt{}     \PYGZlt{}type \PYGZsq{}numpy.ufunc\PYGZsq{}\PYGZgt{}}
\end{Verbatim}

Note that \emph{L{[}3{]}} is itself a list containing an integer and a string and
that \emph{L{[}4{]}} is a function.

One nice feature of Python is that you can also index backwards from the
end:  since \emph{L{[}0{]}} is the first item, \emph{L{[}-1{]}} is what you get going one to
the left of this, and wrapping around (periodic boundary conditions in math
terms):

\begin{Verbatim}[commandchars=\\\{\}]
\PYG{g+gp}{\PYGZgt{}\PYGZgt{}\PYGZgt{} }\PYG{k}{for} \PYG{n}{index} \PYG{o+ow}{in} \PYG{p}{[}\PYG{o}{\PYGZhy{}}\PYG{l+m+mi}{1}\PYG{p}{,} \PYG{o}{\PYGZhy{}}\PYG{l+m+mi}{2}\PYG{p}{,} \PYG{o}{\PYGZhy{}}\PYG{l+m+mi}{3}\PYG{p}{,} \PYG{o}{\PYGZhy{}}\PYG{l+m+mi}{4}\PYG{p}{,} \PYG{o}{\PYGZhy{}}\PYG{l+m+mi}{5}\PYG{p}{]}\PYG{p}{:}
\PYG{g+gp}{... }    \PYG{n+nb}{print} \PYG{l+s}{\PYGZsq{}}\PYG{l+s}{L[}\PYG{l+s}{\PYGZpc{}}\PYG{l+s}{s] is }\PYG{l+s}{\PYGZpc{}}\PYG{l+s}{16s}\PYG{l+s}{\PYGZsq{}} \PYG{o}{\PYGZpc{}} \PYG{p}{(}\PYG{n}{index}\PYG{p}{,} \PYG{n}{L}\PYG{p}{[}\PYG{n}{index}\PYG{p}{]}\PYG{p}{)}
\PYG{g+gp}{...}
\PYG{g+go}{L[\PYGZhy{}1] is    \PYGZlt{}ufunc \PYGZsq{}cos\PYGZsq{}\PYGZgt{}}
\PYG{g+go}{L[\PYGZhy{}2] is         [4, \PYGZsq{}b\PYGZsq{}]}
\PYG{g+go}{L[\PYGZhy{}3] is              abc}
\PYG{g+go}{L[\PYGZhy{}4] is              2.3}
\PYG{g+go}{L[\PYGZhy{}5] is                5}
\end{Verbatim}

In particular, \emph{L{[}-1{]}} always refers to the \emph{last} item in list \emph{L}.


\subsection{Copying objects}
\label{python:copying-objects}
One implication of the fact that variables are just pointers to
objects is that two names can point to the same object, which can sometimes
cause confusion.  Consider this example:

\begin{Verbatim}[commandchars=\\\{\}]
\PYG{g+gp}{\PYGZgt{}\PYGZgt{}\PYGZgt{} }\PYG{n}{x} \PYG{o}{=} \PYG{p}{[}\PYG{l+m+mi}{4}\PYG{p}{,}\PYG{l+m+mi}{5}\PYG{p}{,}\PYG{l+m+mi}{6}\PYG{p}{]}
\PYG{g+gp}{\PYGZgt{}\PYGZgt{}\PYGZgt{} }\PYG{n}{y} \PYG{o}{=} \PYG{n}{x}
\PYG{g+gp}{\PYGZgt{}\PYGZgt{}\PYGZgt{} }\PYG{n}{y}
\PYG{g+go}{[4, 5, 6]}

\PYG{g+gp}{\PYGZgt{}\PYGZgt{}\PYGZgt{} }\PYG{n}{y}\PYG{o}{.}\PYG{n}{append}\PYG{p}{(}\PYG{l+m+mi}{9}\PYG{p}{)}
\PYG{g+gp}{\PYGZgt{}\PYGZgt{}\PYGZgt{} }\PYG{n}{y}
\PYG{g+go}{[4, 5, 6, 9]}
\end{Verbatim}

So far nothing too surprising.  We initialized \emph{y} to be \emph{x} and then we
appended another list element to \emph{y}.  But take a look at \emph{x}:

\begin{Verbatim}[commandchars=\\\{\}]
\PYG{g+gp}{\PYGZgt{}\PYGZgt{}\PYGZgt{} }\PYG{n}{x}
\PYG{g+go}{[4, 5, 6, 9]}
\end{Verbatim}

We didn't really append 9 to \emph{y}, we appended it to the object \emph{y} points
to, which is the same object \emph{x} points to!

Failing to pay attention to this sort of thing can lead to programming
nightmares.

What if we really want \emph{y} to be a different object that happens to be
initialized by copying \emph{x}?  We can do this by:

\begin{Verbatim}[commandchars=\\\{\}]
\PYG{g+gp}{\PYGZgt{}\PYGZgt{}\PYGZgt{} }\PYG{n}{x} \PYG{o}{=} \PYG{p}{[}\PYG{l+m+mi}{4}\PYG{p}{,}\PYG{l+m+mi}{5}\PYG{p}{,}\PYG{l+m+mi}{6}\PYG{p}{]}
\PYG{g+gp}{\PYGZgt{}\PYGZgt{}\PYGZgt{} }\PYG{n}{y} \PYG{o}{=} \PYG{n+nb}{list}\PYG{p}{(}\PYG{n}{x}\PYG{p}{)}
\PYG{g+gp}{\PYGZgt{}\PYGZgt{}\PYGZgt{} }\PYG{n}{y}
\PYG{g+go}{[4, 5, 6]}

\PYG{g+gp}{\PYGZgt{}\PYGZgt{}\PYGZgt{} }\PYG{n}{y}\PYG{o}{.}\PYG{n}{append}\PYG{p}{(}\PYG{l+m+mi}{9}\PYG{p}{)}
\PYG{g+gp}{\PYGZgt{}\PYGZgt{}\PYGZgt{} }\PYG{n}{y}
\PYG{g+go}{[4, 5, 6, 9]}

\PYG{g+gp}{\PYGZgt{}\PYGZgt{}\PYGZgt{} }\PYG{n}{x}
\PYG{g+go}{[4, 5, 6]}
\end{Verbatim}

This is what we want.  Here \emph{list(x)} creates a new object, that is a list,
using the elements of the list \emph{x} to initialize it, and \emph{y} points to this
new object.  Changing this object doesn't change the one \emph{x} pointed to.

You could also use the \emph{copy} module, which works in general for any
objects:

\begin{Verbatim}[commandchars=\\\{\}]
\PYG{g+gp}{\PYGZgt{}\PYGZgt{}\PYGZgt{} }\PYG{k+kn}{import} \PYG{n+nn}{copy}
\PYG{g+gp}{\PYGZgt{}\PYGZgt{}\PYGZgt{} }\PYG{n}{y} \PYG{o}{=} \PYG{n}{copy}\PYG{o}{.}\PYG{n}{copy}\PYG{p}{(}\PYG{n}{x}\PYG{p}{)}
\end{Verbatim}

Sometimes it is more complicated, if the list \emph{x}
itself contains other objects.  See
\url{http://docs.python.org/library/copy.html} for more information.

There are some objects that cannot be changed once created (\emph{immutable
objects}, as described further below).
In particular, for  \emph{floats} and \emph{integers}, you can do things like:

\begin{Verbatim}[commandchars=\\\{\}]
\PYG{g+gp}{\PYGZgt{}\PYGZgt{}\PYGZgt{} }\PYG{n}{x} \PYG{o}{=} \PYG{l+m+mf}{3.4}
\PYG{g+gp}{\PYGZgt{}\PYGZgt{}\PYGZgt{} }\PYG{n}{y} \PYG{o}{=} \PYG{n}{x}
\PYG{g+gp}{\PYGZgt{}\PYGZgt{}\PYGZgt{} }\PYG{n}{y} \PYG{o}{=} \PYG{n}{y}\PYG{o}{+}\PYG{l+m+mi}{1}
\PYG{g+gp}{\PYGZgt{}\PYGZgt{}\PYGZgt{} }\PYG{n}{y}
\PYG{g+go}{4.4000000000000004}

\PYG{g+gp}{\PYGZgt{}\PYGZgt{}\PYGZgt{} }\PYG{n}{x}
\PYG{g+go}{3.3999999999999999}
\end{Verbatim}

Here changing \emph{y} did not change \emph{x}, luckily.
We don't have to explicitly make a copy of \emph{x} for \emph{y} in this case.  If we
did, writing any sort of numerical code in Python would be a nightmare.

We didn't because the command:

\begin{Verbatim}[commandchars=\\\{\}]
\PYG{g+gp}{\PYGZgt{}\PYGZgt{}\PYGZgt{} }\PYG{n}{y} \PYG{o}{=} \PYG{n}{y}\PYG{o}{+}\PYG{l+m+mi}{1}
\end{Verbatim}

above is not changing the object \emph{y} points to, instead it is creating a new
object that \emph{y} now points to, while \emph{x} still points to the old object.

For more about built-in data types in Python, see
\url{http://docs.python.org/release/2.5.2/ref/types.html}.


\subsection{Mutable and Immutable objects}
\label{python:mutable-and-immutable-objects}
Some objects can be changed after they have been created and others cannot
be.  Understanding the difference is key to understanding why the examples
above concerning copying objects behave as they do.

A list is a \emph{mutable} object.  The statement:

\begin{Verbatim}[commandchars=\\\{\}]
\PYGZdl{} x = [4,5,6]
\end{Verbatim}

above created an object that \emph{x} points to, and the data held in this object
can be changed without having to create a new object.   The statement
\begin{quote}

\$ y = x
\end{quote}

points \emph{y} at the same object, and since it can be changed, any change will
affect the object itself and be seen whether we access it using the pointer
\emph{x} or \emph{y}.

We can check this by:

\begin{Verbatim}[commandchars=\\\{\}]
\PYG{g+gp}{\PYGZgt{}\PYGZgt{}\PYGZgt{} }\PYG{n+nb}{id}\PYG{p}{(}\PYG{n}{x}\PYG{p}{)}
\PYG{g+go}{1823768}

\PYG{g+gp}{\PYGZgt{}\PYGZgt{}\PYGZgt{} }\PYG{n+nb}{id}\PYG{p}{(}\PYG{n}{y}\PYG{p}{)}
\PYG{g+go}{1823768}
\end{Verbatim}

The \emph{id} function just returns the location in memory where the object is
stored.  If you do something like \titleref{x{[}0{]} = 1}, you will find that the
objects' id's have not changed, they both point to the same object, but the
data stored in the object has changed.

Some data types correspond to \emph{immutable} objects that, once created,
cannot be changed.  Integers, floats, and strings are immutable:

\begin{Verbatim}[commandchars=\\\{\}]
\PYG{g+gp}{\PYGZgt{}\PYGZgt{}\PYGZgt{} }\PYG{n}{s} \PYG{o}{=} \PYG{l+s}{\PYGZdq{}}\PYG{l+s}{This is a string}\PYG{l+s}{\PYGZdq{}}

\PYG{g+gp}{\PYGZgt{}\PYGZgt{}\PYGZgt{} }\PYG{n}{s}\PYG{p}{[}\PYG{l+m+mi}{0}\PYG{p}{]}
\PYG{g+go}{\PYGZsq{}T\PYGZsq{}}

\PYG{g+gp}{\PYGZgt{}\PYGZgt{}\PYGZgt{} }\PYG{n}{s}\PYG{p}{[}\PYG{l+m+mi}{0}\PYG{p}{]} \PYG{o}{=} \PYG{l+s}{\PYGZsq{}}\PYG{l+s}{b}\PYG{l+s}{\PYGZsq{}}
\PYG{g+gt}{Traceback (most recent call last):}
  File \PYG{n+nb}{\PYGZdq{}\PYGZlt{}stdin\PYGZgt{}\PYGZdq{}}, line \PYG{l+m}{1}, in \PYG{n}{\PYGZlt{}module\PYGZgt{}}
\PYG{g+gr}{TypeError}: \PYG{n}{\PYGZsq{}str\PYGZsq{} object does not support item assignment}
\PYG{g+gt}{Traceback (most recent call last):}
  File \PYG{n+nb}{\PYGZdq{}\PYGZlt{}stdin\PYGZgt{}\PYGZdq{}}, line \PYG{l+m}{1}, in \PYG{n}{\PYGZlt{}module\PYGZgt{}}
\PYG{g+gr}{TypeError}: \PYG{n}{\PYGZsq{}str\PYGZsq{} object does not support item assignment}
\end{Verbatim}

You can index into a string, but you can't change a character in the string.
The only way to change \emph{s} is to redefine it as a new string (which will be
stored in a \textbf{new object}):

\begin{Verbatim}[commandchars=\\\{\}]
\PYG{g+gp}{\PYGZgt{}\PYGZgt{}\PYGZgt{} }\PYG{n+nb}{id}\PYG{p}{(}\PYG{n}{s}\PYG{p}{)}
\PYG{g+go}{1850368}

\PYG{g+gp}{\PYGZgt{}\PYGZgt{}\PYGZgt{} }\PYG{n}{s} \PYG{o}{=} \PYG{l+s}{\PYGZdq{}}\PYG{l+s}{New string}\PYG{l+s}{\PYGZdq{}}
\PYG{g+gp}{\PYGZgt{}\PYGZgt{}\PYGZgt{} }\PYG{n+nb}{id}\PYG{p}{(}\PYG{n}{s}\PYG{p}{)}
\PYG{g+go}{1850128}
\end{Verbatim}

What happened to the old object?  It depends on whether any other variable
was pointing to it.  If not, as in the example above, then Python's \emph{garbage
collection} would recognize it's no longer needed and free up the memory for
other uses.  But if any other variable is still pointing to it, the object
will still exist, e.g.

\begin{Verbatim}[commandchars=\\\{\}]
\PYG{g+gp}{\PYGZgt{}\PYGZgt{}\PYGZgt{} }\PYG{n}{s2} \PYG{o}{=} \PYG{n}{s}
\PYG{g+gp}{\PYGZgt{}\PYGZgt{}\PYGZgt{} }\PYG{n+nb}{id}\PYG{p}{(}\PYG{n}{s2}\PYG{p}{)}                     \PYG{c}{\PYGZsh{} same object as s above}
\PYG{g+go}{1850128}

\PYG{g+gp}{\PYGZgt{}\PYGZgt{}\PYGZgt{} }\PYG{n}{s} \PYG{o}{=} \PYG{l+s}{\PYGZdq{}}\PYG{l+s}{Yet another string}\PYG{l+s}{\PYGZdq{}}   \PYG{c}{\PYGZsh{} creates a new object}
\PYG{g+gp}{\PYGZgt{}\PYGZgt{}\PYGZgt{} }\PYG{n+nb}{id}\PYG{p}{(}\PYG{n}{s}\PYG{p}{)}                      \PYG{c}{\PYGZsh{} s now points to new object}
\PYG{g+go}{1813104}

\PYG{g+gp}{\PYGZgt{}\PYGZgt{}\PYGZgt{} }\PYG{n+nb}{id}\PYG{p}{(}\PYG{n}{s2}\PYG{p}{)}                     \PYG{c}{\PYGZsh{} s2 still points to the old one}
\PYG{g+go}{1850128}

\PYG{g+gp}{\PYGZgt{}\PYGZgt{}\PYGZgt{} }\PYG{n}{s2}
\PYG{g+go}{\PYGZsq{}New string\PYGZsq{}}
\end{Verbatim}


\subsection{Tuples}
\label{python:id4}\label{python:tuples}
We have seen that lists are mutable.  For some purposes we need something
like a list but that is immuatable (e.g. for dictionary keys, see below).  A
tuple is like a list but defined with parentheses \titleref{(..)} rather than square
brackets \titleref{{[}..{]}}:

\begin{Verbatim}[commandchars=\\\{\}]
\PYG{g+gp}{\PYGZgt{}\PYGZgt{}\PYGZgt{} }\PYG{n}{t} \PYG{o}{=} \PYG{p}{(}\PYG{l+m+mi}{4}\PYG{p}{,}\PYG{l+m+mi}{5}\PYG{p}{,}\PYG{l+m+mi}{6}\PYG{p}{)}

\PYG{g+gp}{\PYGZgt{}\PYGZgt{}\PYGZgt{} }\PYG{n}{t}\PYG{p}{[}\PYG{l+m+mi}{0}\PYG{p}{]}
\PYG{g+go}{4}

\PYG{g+gp}{\PYGZgt{}\PYGZgt{}\PYGZgt{} }\PYG{n}{t}\PYG{p}{[}\PYG{l+m+mi}{0}\PYG{p}{]} \PYG{o}{=} \PYG{l+m+mi}{9}
\PYG{g+gt}{Traceback (most recent call last):}
  File \PYG{n+nb}{\PYGZdq{}\PYGZlt{}stdin\PYGZgt{}\PYGZdq{}}, line \PYG{l+m}{1}, in \PYG{n}{\PYGZlt{}module\PYGZgt{}}
\PYG{g+gr}{TypeError}: \PYG{n}{\PYGZsq{}tuple\PYGZsq{} object does not support item assignment}
\PYG{g+gt}{Traceback (most recent call last):}
  File \PYG{n+nb}{\PYGZdq{}\PYGZlt{}stdin\PYGZgt{}\PYGZdq{}}, line \PYG{l+m}{1}, in \PYG{n}{\PYGZlt{}module\PYGZgt{}}
\PYG{g+gr}{TypeError}: \PYG{n}{\PYGZsq{}tuple\PYGZsq{} object does not support item assignment}
\end{Verbatim}


\subsection{Iterators}
\label{python:iterators}
We often want to iterate over a set of things.  In Python there are many
ways to do this, and it often takes the form:

\begin{Verbatim}[commandchars=\\\{\}]
\PYG{g+gp}{\PYGZgt{}\PYGZgt{}\PYGZgt{} }\PYG{k}{for} \PYG{n}{A} \PYG{o+ow}{in} \PYG{n}{B}\PYG{p}{:}
\PYG{g+gp}{... }    \PYG{c}{\PYGZsh{} do something, probably involving the current A}
\end{Verbatim}

In this construct \emph{B} is any Python object that is \emph{iterable}, meaning it
has a built-in way (when B's class was defined) of starting with one thing
in \emph{B} and progressing through the contents of \emph{B} in some hopefully logical
order.

Lists and tuples are
iterable in the obvious way: we step through it one element at a
time starting at the beginning:

\begin{Verbatim}[commandchars=\\\{\}]
\PYG{g+gp}{\PYGZgt{}\PYGZgt{}\PYGZgt{} }\PYG{k}{for} \PYG{n}{i} \PYG{o+ow}{in} \PYG{p}{[}\PYG{l+m+mi}{3}\PYG{p}{,} \PYG{l+m+mi}{7}\PYG{p}{,} \PYG{l+s}{\PYGZsq{}}\PYG{l+s}{b}\PYG{l+s}{\PYGZsq{}}\PYG{p}{]}\PYG{p}{:}
\PYG{g+gp}{... }    \PYG{n+nb}{print} \PYG{l+s}{\PYGZdq{}}\PYG{l+s}{i is now }\PYG{l+s}{\PYGZdq{}}\PYG{p}{,} \PYG{n}{i}
\PYG{g+gp}{...}
\PYG{g+go}{i is now  3}
\PYG{g+go}{i is now  7}
\PYG{g+go}{i is now  b}
\end{Verbatim}


\subsection{range}
\label{python:range}\label{python:id5}
In numerical work we often want to have i start at 0 and go up to some
number N, stepping by one.  We obviously don't want to have to construct the
list {[}0, 1, 2, 3, ..., N{]} by typing all the numbers
when \emph{N} is large, so Python has a way of doing this:

\begin{Verbatim}[commandchars=\\\{\}]
\PYG{g+gp}{\PYGZgt{}\PYGZgt{}\PYGZgt{} }\PYG{n+nb}{range}\PYG{p}{(}\PYG{l+m+mi}{7}\PYG{p}{)}
\PYG{g+go}{[0, 1, 2, 3, 4, 5, 6]}
\end{Verbatim}

NOTE:  The last element is 6, not 7.  The list has 7 elements but starts by
default at 0, just as Python indexing does.  This makes it convenient for
doing things like:

\begin{Verbatim}[commandchars=\\\{\}]
\PYG{g+gp}{\PYGZgt{}\PYGZgt{}\PYGZgt{} }\PYG{n}{L} \PYG{o}{=} \PYG{p}{[}\PYG{l+s}{\PYGZsq{}}\PYG{l+s}{a}\PYG{l+s}{\PYGZsq{}}\PYG{p}{,} \PYG{l+m+mi}{8}\PYG{p}{,} \PYG{l+m+mi}{12}\PYG{p}{]}
\PYG{g+gp}{\PYGZgt{}\PYGZgt{}\PYGZgt{} }\PYG{k}{for} \PYG{n}{i} \PYG{o+ow}{in} \PYG{n+nb}{range}\PYG{p}{(}\PYG{n+nb}{len}\PYG{p}{(}\PYG{n}{L}\PYG{p}{)}\PYG{p}{)}\PYG{p}{:}
\PYG{g+gp}{... }    \PYG{n+nb}{print} \PYG{l+s}{\PYGZdq{}}\PYG{l+s}{i = }\PYG{l+s}{\PYGZdq{}}\PYG{p}{,} \PYG{n}{i}\PYG{p}{,} \PYG{l+s}{\PYGZdq{}}\PYG{l+s}{  L[i] = }\PYG{l+s}{\PYGZdq{}}\PYG{p}{,} \PYG{n}{L}\PYG{p}{[}\PYG{n}{i}\PYG{p}{]}
\PYG{g+gp}{...}
\PYG{g+go}{i =  0   L[i] =  a}
\PYG{g+go}{i =  1   L[i] =  8}
\PYG{g+go}{i =  2   L[i] =  12}
\end{Verbatim}

Note that \emph{len(L)} returns the length of the list, so \emph{range(len(L))} is
always a list of all the valid indices for the list \emph{L}.


\subsection{enumerate}
\label{python:id6}\label{python:enumerate}
Another way to do this is:

\begin{Verbatim}[commandchars=\\\{\}]
\PYG{g+gp}{\PYGZgt{}\PYGZgt{}\PYGZgt{} }\PYG{k}{for} \PYG{n}{i}\PYG{p}{,}\PYG{n}{value} \PYG{o+ow}{in} \PYG{n+nb}{enumerate}\PYG{p}{(}\PYG{n}{L}\PYG{p}{)}\PYG{p}{:}
\PYG{g+gp}{... }    \PYG{n+nb}{print} \PYG{l+s}{\PYGZdq{}}\PYG{l+s}{i = }\PYG{l+s}{\PYGZdq{}}\PYG{p}{,}\PYG{n}{i}\PYG{p}{,} \PYG{l+s}{\PYGZdq{}}\PYG{l+s}{  L[i] = }\PYG{l+s}{\PYGZdq{}}\PYG{p}{,}\PYG{n}{value}
\PYG{g+gp}{...}
\PYG{g+go}{i =  0   L[i] =  a}
\PYG{g+go}{i =  1   L[i] =  8}
\PYG{g+go}{i =  2   L[i] =  12}
\end{Verbatim}

\emph{range} can be used with more arguments, for example if
you want to start at 2 and step by 3 up to 20:

\begin{Verbatim}[commandchars=\\\{\}]
\PYG{g+gp}{\PYGZgt{}\PYGZgt{}\PYGZgt{} }\PYG{n+nb}{range}\PYG{p}{(}\PYG{l+m+mi}{2}\PYG{p}{,}\PYG{l+m+mi}{20}\PYG{p}{,}\PYG{l+m+mi}{3}\PYG{p}{)}
\PYG{g+go}{[2, 5, 8, 11, 14, 17]}
\end{Verbatim}

Note that this doesn't go up to 20.  Just like \emph{range(7)} stops at 6, this
list stops one item short of what you might expect.

NumPy has a \emph{linspace} command that behaves like Matlab's, which is
sometimes more useful in numerical work, e.g.:

\begin{Verbatim}[commandchars=\\\{\}]
\PYG{g+gp}{\PYGZgt{}\PYGZgt{}\PYGZgt{} }\PYG{n}{np}\PYG{o}{.}\PYG{n}{linspace}\PYG{p}{(}\PYG{l+m+mi}{2}\PYG{p}{,}\PYG{l+m+mi}{20}\PYG{p}{,}\PYG{l+m+mi}{7}\PYG{p}{)}
\PYG{g+go}{array([  2.,   5.,   8.,  11.,  14.,  17.,  20.])}
\end{Verbatim}

This returns a NumPy array with 7 equally spaced points between 2 and 20,
including the endpoints.  Note that the elements are floats, not integers.
You could use this as an iterator too.

If you plan to iterate over a lot of values, say 1 million, it may
be inefficient to generate a list object with 1 million elements using
\emph{range}.  So there is another option called \emph{xrange}, that does the
iteration you want without explicitly creating and storing the list:

\begin{Verbatim}[commandchars=\\\{\}]
\PYG{k}{for} \PYG{n}{i} \PYG{o+ow}{in} \PYG{n}{xrange}\PYG{p}{(}\PYG{l+m+mi}{1000000}\PYG{p}{)}\PYG{p}{:}
    \PYG{c}{\PYGZsh{} do something}
\end{Verbatim}

does what we want.

Note that the elements in a list you're iterating on need not be numbers.
For example, the sample module \emph{myfcns} in \$UWHPSC/codes/python defines two
functions \emph{f1} and \emph{f2}.  If we want to evaluate each of them at x=3., we
could do:

\begin{Verbatim}[commandchars=\\\{\}]
\PYG{g+gp}{\PYGZgt{}\PYGZgt{}\PYGZgt{} }\PYG{k+kn}{from} \PYG{n+nn}{myfcns} \PYG{k}{import} \PYG{n}{f1}\PYG{p}{,} \PYG{n}{f2}
\PYG{g+gp}{\PYGZgt{}\PYGZgt{}\PYGZgt{} }\PYG{n+nb}{type}\PYG{p}{(}\PYG{n}{f1}\PYG{p}{)}
\PYG{g+go}{\PYGZlt{}type \PYGZsq{}function\PYGZsq{}\PYGZgt{}}

\PYG{g+gp}{\PYGZgt{}\PYGZgt{}\PYGZgt{} }\PYG{k}{for} \PYG{n}{f} \PYG{o+ow}{in} \PYG{p}{[}\PYG{n}{f1}\PYG{p}{,} \PYG{n}{f2}\PYG{p}{]}\PYG{p}{:}
\PYG{g+gp}{... }    \PYG{n+nb}{print} \PYG{n}{f}\PYG{p}{(}\PYG{l+m+mf}{3.}\PYG{p}{)}
\PYG{g+gp}{...}
\PYG{g+go}{5.0}
\PYG{g+go}{162754.791419}
\end{Verbatim}

This can be very handy if you want to perform some tests for a set of test
functions.


\subsection{Further reading}
\label{python:further-reading}
See the {\hyperref[biblio:biblio\string-python]{\crossref{\DUrole{std,std-ref}{Python:}}}} section of the {\hyperref[biblio:biblio]{\crossref{\DUrole{std,std-ref}{Bibliography and further reading}}}}.

In particular,
see the \phantomsection\label{python:id7}{\hyperref[biblio:python\string-2\string-5\string-tutorial]{\crossref{{[}Python-2.5-tutorial{]}}}}  or \phantomsection\label{python:id8}{\hyperref[biblio:python\string-2\string-7\string-tutorial]{\crossref{{[}Python-2.7-tutorial{]}}}} for good overviews
(these two versions of Python are very similar).

There are several introductory Python pages at the \phantomsection\label{python:id9}{\hyperref[biblio:software\string-carpentry]{\crossref{{[}software-carpentry{]}}}}
site.

For more on basic data structures:
\url{http://docs.python.org/2/tutorial/datastructures.html}


\section{Python scripts and modules}
\label{python_scripts_modules:python-scripts-modules}\label{python_scripts_modules::doc}\label{python_scripts_modules:python-scripts-and-modules}
A Python script is a collection of commands in a file designed to be
executed like a program.  The file can of course contain functions and
import various modules, but the idea is that it will be run or executed
from the command line or from within a Python interactive shell to perform a
specific task.  Often a script first contains a set of function definitions
and then has the \emph{main program} that might call the functions.

Consider this script,  found in \$UWHPSC/python/script1.py:

\textbf{script1.py}

\begin{Verbatim}[commandchars=\\\{\},numbers=left,firstnumber=1,stepnumber=1]
\PYG{l+s+sd}{\PYGZdq{}\PYGZdq{}\PYGZdq{}}
\PYG{l+s+sd}{\PYGZdl{}UWHPSC/codes/python/script1.py}

\PYG{l+s+sd}{Sample script to print values of a function at a few points.}
\PYG{l+s+sd}{\PYGZdq{}\PYGZdq{}\PYGZdq{}}
\PYG{k+kn}{import} \PYG{n+nn}{numpy} \PYG{k+kn}{as} \PYG{n+nn}{np}

\PYG{k}{def} \PYG{n+nf}{f}\PYG{p}{(}\PYG{n}{x}\PYG{p}{)}\PYG{p}{:}
    \PYG{l+s+sd}{\PYGZdq{}\PYGZdq{}\PYGZdq{}}
\PYG{l+s+sd}{    A quadratic function.}
\PYG{l+s+sd}{    \PYGZdq{}\PYGZdq{}\PYGZdq{}}
    \PYG{n}{y} \PYG{o}{=} \PYG{n}{x}\PYG{o}{*}\PYG{o}{*}\PYG{l+m+mi}{2} \PYG{o}{+} \PYG{l+m+mf}{1.}
    \PYG{k}{return} \PYG{n}{y}

\PYG{k}{print} \PYG{l+s}{\PYGZdq{}}\PYG{l+s}{     x        f(x)}\PYG{l+s}{\PYGZdq{}}
\PYG{k}{for} \PYG{n}{x} \PYG{o+ow}{in} \PYG{n}{np}\PYG{o}{.}\PYG{n}{linspace}\PYG{p}{(}\PYG{l+m+mi}{0}\PYG{p}{,}\PYG{l+m+mi}{4}\PYG{p}{,}\PYG{l+m+mi}{3}\PYG{p}{)}\PYG{p}{:}
    \PYG{k}{print} \PYG{l+s}{\PYGZdq{}}\PYG{l+s+si}{\PYGZpc{}8.3f}\PYG{l+s}{  }\PYG{l+s+si}{\PYGZpc{}8.3f}\PYG{l+s}{\PYGZdq{}} \PYG{o}{\PYGZpc{}} \PYG{p}{(}\PYG{n}{x}\PYG{p}{,} \PYG{n}{f}\PYG{p}{(}\PYG{n}{x}\PYG{p}{)}\PYG{p}{)}

\end{Verbatim}

The \emph{main program} starts with the print statement.

There are several ways to run a script contained in a file.

At the Unix prompt:

\begin{Verbatim}[commandchars=\\\{\}]
\PYGZdl{} python script1.py
    x        f(x)
  0.000     1.000
  2.000     5.000
  4.000    17.000
\end{Verbatim}

From within Python:

\begin{Verbatim}[commandchars=\\\{\}]
\PYG{g+gp}{\PYGZgt{}\PYGZgt{}\PYGZgt{} }\PYG{n}{execfile}\PYG{p}{(}\PYG{l+s}{\PYGZdq{}}\PYG{l+s}{script1.py}\PYG{l+s}{\PYGZdq{}}\PYG{p}{)}
\PYG{g+go}{[same output as above]}
\end{Verbatim}

From within IPython, using either \titleref{execfile} as above, or \titleref{run}:

\begin{Verbatim}[commandchars=\\\{\}]
\PYG{n}{In} \PYG{p}{[}\PYG{l+m+mi}{48}\PYG{p}{]}\PYG{p}{:} \PYG{n}{run} \PYG{n}{script1}\PYG{o}{.}\PYG{n}{py}
\PYG{p}{[}\PYG{n}{same} \PYG{n}{output} \PYG{k}{as} \PYG{n}{above}\PYG{p}{]}
\end{Verbatim}

Or, you can \titleref{import} the file as a module (see {\hyperref[python_scripts_modules:importing\string-modules]{\crossref{\DUrole{std,std-ref}{Importing modules}}}}
below for more about this):

\begin{Verbatim}[commandchars=\\\{\}]
\PYG{g+gp}{\PYGZgt{}\PYGZgt{}\PYGZgt{} }\PYG{k+kn}{import} \PYG{n+nn}{script1}
\PYG{g+go}{     x        f(x)}
\PYG{g+go}{   0.000     1.000}
\PYG{g+go}{   2.000     5.000}
\PYG{g+go}{   4.000    17.000}
\end{Verbatim}

Note that this also gives the same output.  Whenever a module is imported,
any statements that are in the main body of the module are executed when it
is imported.  In addition, any variables or functions defined in the file
are available as attributes of the module, e.g.,

\begin{Verbatim}[commandchars=\\\{\}]
\PYG{g+gp}{\PYGZgt{}\PYGZgt{}\PYGZgt{} }\PYG{n}{script1}\PYG{o}{.}\PYG{n}{f}\PYG{p}{(}\PYG{l+m+mi}{4}\PYG{p}{)}
\PYG{g+go}{17.0}

\PYG{g+gp}{\PYGZgt{}\PYGZgt{}\PYGZgt{} }\PYG{n}{script1}\PYG{o}{.}\PYG{n}{np}
\PYG{g+go}{\PYGZlt{}module \PYGZsq{}numpy\PYGZsq{} from}
\PYG{g+go}{\PYGZsq{}/Library/Python/2.5/site\PYGZhy{}packages/numpy\PYGZhy{}1.4.0.dev7064\PYGZhy{}py2.5\PYGZhy{}macosx\PYGZhy{}10.3\PYGZhy{}fat.egg/numpy/\PYGZus{}\PYGZus{}init\PYGZus{}\PYGZus{}.pyc\PYGZsq{}\PYGZgt{}}
\end{Verbatim}

Note there are some differences between executing the script and importing
it.  When it is executed as a script,
it is as if the commands were typed at the command line.  Hence:

\begin{Verbatim}[commandchars=\\\{\}]
\PYG{g+gp}{\PYGZgt{}\PYGZgt{}\PYGZgt{} }\PYG{n}{execfile}\PYG{p}{(}\PYG{l+s}{\PYGZsq{}}\PYG{l+s}{script1.py}\PYG{l+s}{\PYGZsq{}}\PYG{p}{)}
\PYG{g+go}{     x        f(x)}
\PYG{g+go}{   0.000     1.000}
\PYG{g+go}{   2.000     5.000}
\PYG{g+go}{   4.000    17.000}

\PYG{g+gp}{\PYGZgt{}\PYGZgt{}\PYGZgt{} }\PYG{n}{f}
\PYG{g+go}{\PYGZlt{}function f at 0x1c0430\PYGZgt{}}

\PYG{g+gp}{\PYGZgt{}\PYGZgt{}\PYGZgt{} }\PYG{n}{np}
\PYG{g+go}{\PYGZlt{}module \PYGZsq{}numpy\PYGZsq{} from}
\PYG{g+go}{\PYGZsq{}/Library/Python/2.5/site\PYGZhy{}packages/numpy\PYGZhy{}1.4.0.dev7064\PYGZhy{}py2.5\PYGZhy{}macosx\PYGZhy{}10.3\PYGZhy{}fat.egg/numpy/\PYGZus{}\PYGZus{}init\PYGZus{}\PYGZus{}.pyc\PYGZsq{}\PYGZgt{}}
\end{Verbatim}

In this case \titleref{f} and \titleref{np} are in the namespace of the interactive session as
if we had defined them at the prompt.


\subsection{Writing scripts for ease of importing}
\label{python_scripts_modules:writing-scripts-for-ease-of-importing}\label{python_scripts_modules:python-name-main}
The script used above as an example contains a function \titleref{f(x)} that we might
want to be able to import without necessarily running the \emph{main program}.
This can be arranged by modifying the script as follows:

\textbf{script2.py}

\begin{Verbatim}[commandchars=\\\{\},numbers=left,firstnumber=1,stepnumber=1]
\PYG{l+s+sd}{\PYGZdq{}\PYGZdq{}\PYGZdq{}}
\PYG{l+s+sd}{\PYGZdl{}UWHPSC/codes/python/script2.py}

\PYG{l+s+sd}{Sample script to print values of a function at a few points.}
\PYG{l+s+sd}{The printing is only done if the file is executed as a script, not if it is}
\PYG{l+s+sd}{imported as a module.}
\PYG{l+s+sd}{\PYGZdq{}\PYGZdq{}\PYGZdq{}}
\PYG{k+kn}{import} \PYG{n+nn}{numpy} \PYG{k+kn}{as} \PYG{n+nn}{np}

\PYG{k}{def} \PYG{n+nf}{f}\PYG{p}{(}\PYG{n}{x}\PYG{p}{)}\PYG{p}{:}
    \PYG{l+s+sd}{\PYGZdq{}\PYGZdq{}\PYGZdq{}}
\PYG{l+s+sd}{    A quadratic function.}
\PYG{l+s+sd}{    \PYGZdq{}\PYGZdq{}\PYGZdq{}}
    \PYG{n}{y} \PYG{o}{=} \PYG{n}{x}\PYG{o}{*}\PYG{o}{*}\PYG{l+m+mi}{2} \PYG{o}{+} \PYG{l+m+mf}{1.}
    \PYG{k}{return} \PYG{n}{y}

\PYG{k}{def} \PYG{n+nf}{print\PYGZus{}table}\PYG{p}{(}\PYG{p}{)}\PYG{p}{:}
    \PYG{k}{print} \PYG{l+s}{\PYGZdq{}}\PYG{l+s}{     x        f(x)}\PYG{l+s}{\PYGZdq{}}
    \PYG{k}{for} \PYG{n}{x} \PYG{o+ow}{in} \PYG{n}{np}\PYG{o}{.}\PYG{n}{linspace}\PYG{p}{(}\PYG{l+m+mi}{0}\PYG{p}{,}\PYG{l+m+mi}{4}\PYG{p}{,}\PYG{l+m+mi}{3}\PYG{p}{)}\PYG{p}{:}
        \PYG{k}{print} \PYG{l+s}{\PYGZdq{}}\PYG{l+s+si}{\PYGZpc{}8.3f}\PYG{l+s}{  }\PYG{l+s+si}{\PYGZpc{}8.3f}\PYG{l+s}{\PYGZdq{}} \PYG{o}{\PYGZpc{}} \PYG{p}{(}\PYG{n}{x}\PYG{p}{,} \PYG{n}{f}\PYG{p}{(}\PYG{n}{x}\PYG{p}{)}\PYG{p}{)}

\PYG{k}{if} \PYG{n}{\PYGZus{}\PYGZus{}name\PYGZus{}\PYGZus{}} \PYG{o}{==} \PYG{l+s}{\PYGZdq{}}\PYG{l+s}{\PYGZus{}\PYGZus{}main\PYGZus{}\PYGZus{}}\PYG{l+s}{\PYGZdq{}}\PYG{p}{:}
    \PYG{n}{print\PYGZus{}table}\PYG{p}{(}\PYG{p}{)}
\end{Verbatim}

When a file is imported or executed, an attribute \titleref{\_\_name\_\_} is
automatically set, and has the value \titleref{\_\_main\_\_} only if the file is executed
as a script, not if it is imported as a module.  So we see the following
behavior:

\begin{Verbatim}[commandchars=\\\{\}]
\PYGZdl{} python script2.py
    x        f(x)
  0.000     1.000
  2.000     5.000
  4.000    17.000
\end{Verbatim}

as with \titleref{script1.py}, but:

\begin{Verbatim}[commandchars=\\\{\}]
\PYG{g+gp}{\PYGZgt{}\PYGZgt{}\PYGZgt{} }\PYG{k+kn}{import} \PYG{n+nn}{script2}           \PYG{c}{\PYGZsh{} does not print table}

\PYG{g+gp}{\PYGZgt{}\PYGZgt{}\PYGZgt{} }\PYG{n}{script2}\PYG{o}{.}\PYG{n}{\PYGZus{}\PYGZus{}name\PYGZus{}\PYGZus{}}
\PYG{g+go}{\PYGZsq{}script2\PYGZsq{}                    \PYGZsh{} not \PYGZsq{}\PYGZus{}\PYGZus{}main\PYGZus{}\PYGZus{}\PYGZsq{}}

\PYG{g+gp}{\PYGZgt{}\PYGZgt{}\PYGZgt{} }\PYG{n}{script2}\PYG{o}{.}\PYG{n}{f}\PYG{p}{(}\PYG{l+m+mi}{4}\PYG{p}{)}
\PYG{g+go}{17.0}

\PYG{g+gp}{\PYGZgt{}\PYGZgt{}\PYGZgt{} }\PYG{n}{script2}\PYG{o}{.}\PYG{n}{print\PYGZus{}table}\PYG{p}{(}\PYG{p}{)}
\PYG{g+go}{     x        f(x)}
\PYG{g+go}{   0.000     1.000}
\PYG{g+go}{   2.000     5.000}
\PYG{g+go}{   4.000    17.000}
\end{Verbatim}


\subsection{Reloading modules}
\label{python_scripts_modules:python-reload}\label{python_scripts_modules:reloading-modules}
When you import a module, Python keeps track of the fact that it is imported
and if it encounters another statement to import the same module will not
bother to do so again (the list of modules already import is in
\titleref{sys.modules}).  This is convenient since loading a module can be
time consuming.  So if you're debugging a script using \titleref{execfile} or \titleref{run}
from an IPython shell, each time you change it and then re-execute it will
not reload \titleref{numpy}, for example.

Sometimes, however, you want to force reloading of a module, in particular
if it has changed (e.g. when we are debugging it).

Suppose, for example, that we modify \titleref{script2.py} so
that the quadratic function is changed from \titleref{y = x**2 + 1 {}` to
{}`y = x**2 + 10}.
If we make this change and then try the following (in the same Python
session as above, where \titleref{script2} was already imported as a module):

\begin{Verbatim}[commandchars=\\\{\}]
\PYG{g+gp}{\PYGZgt{}\PYGZgt{}\PYGZgt{} }\PYG{k+kn}{import} \PYG{n+nn}{script2}

\PYG{g+gp}{\PYGZgt{}\PYGZgt{}\PYGZgt{} }\PYG{n}{script2}\PYG{o}{.}\PYG{n}{print\PYGZus{}table}\PYG{p}{(}\PYG{p}{)}
\PYG{g+go}{     x        f(x)}
\PYG{g+go}{   0.000     1.000}
\PYG{g+go}{   2.000     5.000}
\PYG{g+go}{   4.000    17.000}
\end{Verbatim}

we get the same results as above, even though we changed \titleref{script2.py}.

We have to use the \titleref{reload} command to see the change we want:

\begin{Verbatim}[commandchars=\\\{\}]
\PYG{g+gp}{\PYGZgt{}\PYGZgt{}\PYGZgt{} }\PYG{n}{reload}\PYG{p}{(}\PYG{n}{script2}\PYG{p}{)}
\PYG{g+go}{\PYGZlt{}module \PYGZsq{}script2\PYGZsq{} from \PYGZsq{}script2.py\PYGZsq{}\PYGZgt{}}

\PYG{g+gp}{\PYGZgt{}\PYGZgt{}\PYGZgt{} }\PYG{n}{script2}\PYG{o}{.}\PYG{n}{print\PYGZus{}table}\PYG{p}{(}\PYG{p}{)}
\PYG{g+go}{     x        f(x)}
\PYG{g+go}{   0.000    10.000}
\PYG{g+go}{   2.000    14.000}
\PYG{g+go}{   4.000    26.000}
\end{Verbatim}


\subsection{Command line arguments}
\label{python_scripts_modules:python-argv}\label{python_scripts_modules:command-line-arguments}
We might want to make this script a bit fancier by adding an optional argument
to the \titleref{print\_table} function to print a different number of points, rather
than the 3 points shown above.

The next version has this change, and also has a modified version of the
main program that allows the user to specify this value \titleref{n} as a command
line argument:

\textbf{script3.py}

\begin{Verbatim}[commandchars=\\\{\},numbers=left,firstnumber=1,stepnumber=1]
\PYG{l+s+sd}{\PYGZdq{}\PYGZdq{}\PYGZdq{}}
\PYG{l+s+sd}{\PYGZdl{}UWHPSC/codes/python/script3.py}

\PYG{l+s+sd}{Modification of script2.py that allows a command line argument telling how}
\PYG{l+s+sd}{many points to plot in the table.}

\PYG{l+s+sd}{Usage example: To print table with 5 values:}
\PYG{l+s+sd}{   python script3 5}

\PYG{l+s+sd}{\PYGZdq{}\PYGZdq{}\PYGZdq{}}
\PYG{k+kn}{import} \PYG{n+nn}{numpy} \PYG{k+kn}{as} \PYG{n+nn}{np}

\PYG{k}{def} \PYG{n+nf}{f}\PYG{p}{(}\PYG{n}{x}\PYG{p}{)}\PYG{p}{:}
    \PYG{l+s+sd}{\PYGZdq{}\PYGZdq{}\PYGZdq{}}
\PYG{l+s+sd}{    A quadratic function.}
\PYG{l+s+sd}{    \PYGZdq{}\PYGZdq{}\PYGZdq{}}
    \PYG{n}{y} \PYG{o}{=} \PYG{n}{x}\PYG{o}{*}\PYG{o}{*}\PYG{l+m+mi}{2} \PYG{o}{+} \PYG{l+m+mf}{1.}
    \PYG{k}{return} \PYG{n}{y}

\PYG{k}{def} \PYG{n+nf}{print\PYGZus{}table}\PYG{p}{(}\PYG{n}{n}\PYG{o}{=}\PYG{l+m+mi}{3}\PYG{p}{)}\PYG{p}{:}
    \PYG{k}{print} \PYG{l+s}{\PYGZdq{}}\PYG{l+s}{     x        f(x)}\PYG{l+s}{\PYGZdq{}}
    \PYG{k}{for} \PYG{n}{x} \PYG{o+ow}{in} \PYG{n}{np}\PYG{o}{.}\PYG{n}{linspace}\PYG{p}{(}\PYG{l+m+mi}{0}\PYG{p}{,}\PYG{l+m+mi}{4}\PYG{p}{,}\PYG{n}{n}\PYG{p}{)}\PYG{p}{:}
        \PYG{k}{print} \PYG{l+s}{\PYGZdq{}}\PYG{l+s+si}{\PYGZpc{}8.3f}\PYG{l+s}{  }\PYG{l+s+si}{\PYGZpc{}8.3f}\PYG{l+s}{\PYGZdq{}} \PYG{o}{\PYGZpc{}} \PYG{p}{(}\PYG{n}{x}\PYG{p}{,} \PYG{n}{f}\PYG{p}{(}\PYG{n}{x}\PYG{p}{)}\PYG{p}{)}

\PYG{k}{if} \PYG{n}{\PYGZus{}\PYGZus{}name\PYGZus{}\PYGZus{}} \PYG{o}{==} \PYG{l+s}{\PYGZdq{}}\PYG{l+s}{\PYGZus{}\PYGZus{}main\PYGZus{}\PYGZus{}}\PYG{l+s}{\PYGZdq{}}\PYG{p}{:}
    \PYG{l+s+sd}{\PYGZdq{}\PYGZdq{}\PYGZdq{}}
\PYG{l+s+sd}{    What to do if the script is executed at command line.}
\PYG{l+s+sd}{    Note that sys.argv is a list of the tokens typed at the command line.}
\PYG{l+s+sd}{    \PYGZdq{}\PYGZdq{}\PYGZdq{}}
    \PYG{k+kn}{import} \PYG{n+nn}{sys}
    \PYG{k}{print} \PYG{l+s}{\PYGZdq{}}\PYG{l+s}{sys.argv is }\PYG{l+s}{\PYGZdq{}}\PYG{p}{,}\PYG{n}{sys}\PYG{o}{.}\PYG{n}{argv}
    \PYG{k}{if} \PYG{n+nb}{len}\PYG{p}{(}\PYG{n}{sys}\PYG{o}{.}\PYG{n}{argv}\PYG{p}{)} \PYG{o}{\PYGZgt{}} \PYG{l+m+mi}{1}\PYG{p}{:}
        \PYG{k}{try}\PYG{p}{:}
            \PYG{n}{n} \PYG{o}{=} \PYG{n+nb}{int}\PYG{p}{(}\PYG{n}{sys}\PYG{o}{.}\PYG{n}{argv}\PYG{p}{[}\PYG{l+m+mi}{1}\PYG{p}{]}\PYG{p}{)}
            \PYG{n}{print\PYGZus{}table}\PYG{p}{(}\PYG{n}{n}\PYG{p}{)}
        \PYG{k}{except}\PYG{p}{:}
            \PYG{k}{print} \PYG{l+s}{\PYGZdq{}}\PYG{l+s}{*** Error: expect an integer n as the argument}\PYG{l+s}{\PYGZdq{}}
    \PYG{k}{else}\PYG{p}{:}
        \PYG{n}{print\PYGZus{}table}\PYG{p}{(}\PYG{p}{)}
\end{Verbatim}

Note that:
\begin{itemize}
\item {} 
The function \titleref{sys.argv} from the \titleref{sys} module returns the arguments that
were present if the script is executed from the command line.  It is a
list of strings, with \titleref{sys.argv{[}0{]}} being the name of the script itself,
\titleref{sys.argv{[}1{]}} being the next thing on the line, etc. (if there were more
than one command line argument, separated by spaces).

\item {} 
We use \titleref{int(sys.argv{[}1{]})} to convert the argument, if present, from a
string to an integer.

\item {} 
We put this conversion in a try-except block in case the user gives an
invalid argument.

\end{itemize}

Sample output:

\begin{Verbatim}[commandchars=\\\{\}]
\PYGZdl{} python script3.py
     x        f(x)
   0.000     1.000
   2.000     5.000
   4.000    17.000

\PYGZdl{} python script3.py 5
     x        f(x)
   0.000     1.000
   1.000     2.000
   2.000     5.000
   3.000    10.000
   4.000    17.000

\PYGZdl{} python script3.py 5.2
*** Error: expect an integer n as the argument
\end{Verbatim}


\subsection{Importing modules}
\label{python_scripts_modules:importing-modules}\label{python_scripts_modules:id1}
When Python starts up there are a certain number of basic commands defined
along with the general syntax of the language, but most useful things needed
for specific purposes (such as working with webpages, or solving linear
systems) are in \emph{modules} that do not load by default.  Otherwise it would
take forever to start up Python, loading lots of things you don't plan to
use.  So when you start using Python, either interactively or at the top of
a script, often the  first thing you do is \emph{import} one or more modules.

A Python module is often defined simply by grouping a set of parameters and
functions together in a single .py file.
See {\hyperref[python_scripts_modules:python\string-scripts\string-modules]{\crossref{\DUrole{std,std-ref}{Python scripts and modules}}}} for some examples.

Two useful modules are \emph{os} and \emph{sys} that help you interact with the
operating system and the Python system that is running.  These are standard
modules that should be available with any Python implementation, so you
should be able to import them at the Python prompt:

\begin{Verbatim}[commandchars=\\\{\}]
\PYG{g+gp}{\PYGZgt{}\PYGZgt{}\PYGZgt{} }\PYG{k+kn}{import} \PYG{n+nn}{os}\PYG{o}{,} \PYG{n+nn}{sys}
\end{Verbatim}

Each module contains many different functions and parameters which are the
\emph{methods} and \emph{attributes} of the module.   Here we will only use a couple
of these.  The
\emph{getcwd} method of the os module is called to return the ``current working
directory''  (the same thing \emph{pwd} prints in Unix), e.g.:

\begin{Verbatim}[commandchars=\\\{\}]
\PYG{g+gp}{\PYGZgt{}\PYGZgt{}\PYGZgt{} }\PYG{n}{os}\PYG{o}{.}\PYG{n}{getcwd}\PYG{p}{(}\PYG{p}{)}
\PYG{g+go}{/home/uwhpsc/uwhpsc/codes/python}
\end{Verbatim}

Note that this function is called with no arguments, but you need the open
and close parens.  If you type ``os.getcwd'' without these, Python will
instead print what type of object this function is:

\begin{Verbatim}[commandchars=\\\{\}]
\PYG{g+gp}{\PYGZgt{}\PYGZgt{}\PYGZgt{} }\PYG{n}{os}\PYG{o}{.}\PYG{n}{getcwd}
\PYG{g+go}{\PYGZlt{}built\PYGZhy{}in function getcwd\PYGZgt{}}
\end{Verbatim}


\subsection{The Python Path}
\label{python_scripts_modules:the-python-path}
The \emph{sys} module has an attribute \emph{sys.path}, a variable that is set by
default to the search path for modules.  Whenever you perform an \emph{import},
this is the set of directories that Python searches through looking for a
file by that name (with a .py extension).  If you print this, you will see a
list of strings, each one of which is the full path to some directory.
Sometimes the first thing in this list is the empty string, which means ``the
current directory'', so it looks for a module in your working directory first
and if it doesn't find it, searches through the other directories in order:

\begin{Verbatim}[commandchars=\\\{\}]
\PYG{g+gp}{\PYGZgt{}\PYGZgt{}\PYGZgt{} }\PYG{n+nb}{print} \PYG{n}{sys}\PYG{o}{.}\PYG{n}{path}
\PYG{g+go}{[\PYGZsq{}\PYGZsq{}, \PYGZsq{}/usr/lib/python2.7\PYGZsq{}, ....]}
\end{Verbatim}

If you try to import a module and it doesn't find a file with this name on
the path, then you will get an import error:

\begin{Verbatim}[commandchars=\\\{\}]
\PYG{g+gp}{\PYGZgt{}\PYGZgt{}\PYGZgt{} }\PYG{k+kn}{import} \PYG{n+nn}{junkname}
\PYG{g+gt}{Traceback (most recent call last):}
  File \PYG{n+nb}{\PYGZdq{}\PYGZlt{}stdin\PYGZgt{}\PYGZdq{}}, line \PYG{l+m}{1}, in \PYG{n}{\PYGZlt{}module\PYGZgt{}}
\PYG{g+gr}{ImportError}: \PYG{n}{No module named junkname}
\PYG{g+gt}{Traceback (most recent call last):}
  File \PYG{n+nb}{\PYGZdq{}\PYGZlt{}stdin\PYGZgt{}\PYGZdq{}}, line \PYG{l+m}{1}, in \PYG{n}{\PYGZlt{}module\PYGZgt{}}
\PYG{g+gr}{ImportError}: \PYG{n}{No module named junkname}
\end{Verbatim}

When new Python software such as NumPy or SciPy is installed, the
installation script should modify the path appropriately so it can be found.
You can also add to the path if you have your own directory that you want
Python to look in, e.g.:

\begin{Verbatim}[commandchars=\\\{\}]
\PYG{g+gp}{\PYGZgt{}\PYGZgt{}\PYGZgt{} }\PYG{n}{sys}\PYG{o}{.}\PYG{n}{path}\PYG{o}{.}\PYG{n}{append}\PYG{p}{(}\PYG{l+s}{\PYGZdq{}}\PYG{l+s}{/home/uwhpsc/mypython}\PYG{l+s}{\PYGZdq{}}\PYG{p}{)}
\end{Verbatim}

will append the directory indicated to the path.  To avoid having to do this
each time you start Python, you can set a Unix environment variable that
is used to modify the path every time Python is started.  First print out
the current value of this variable:

\begin{Verbatim}[commandchars=\\\{\}]
\PYGZdl{} echo \PYGZdl{}PYTHONPATH
\end{Verbatim}

It will probably be blank unless you've set this before or have installed
software that sets this automatically.
To append the above example directory to this path:

\begin{Verbatim}[commandchars=\\\{\}]
\PYGZdl{} export PYTHONPATH=\PYGZdl{}PYTHONPATH:/home/uwhpsc/mypython
\end{Verbatim}

This appends another directory to the search path already specified (if any).
You can repeat this multiple times to add more directories, or put something
like:

\begin{Verbatim}[commandchars=\\\{\}]
export PYTHONPATH=\PYGZdl{}PYTHONPATH:dir1:dir2:dir3
\end{Verbatim}

in your \emph{.bashrc} file if there are the only 3 personal
directories you always want to search.


\subsection{Other forms of import}
\label{python_scripts_modules:other-forms-of-import}
If all we want to use from the \emph{os} module is \emph{getcwd}, then another option
is to do:

\begin{Verbatim}[commandchars=\\\{\}]
\PYG{g+gp}{\PYGZgt{}\PYGZgt{}\PYGZgt{} }\PYG{k+kn}{from} \PYG{n+nn}{os} \PYG{k}{import} \PYG{n}{getcwd}
\PYG{g+gp}{\PYGZgt{}\PYGZgt{}\PYGZgt{} }\PYG{n}{getcwd}\PYG{p}{(}\PYG{p}{)}
\PYG{g+go}{\PYGZsq{}/Users/rjl/uwhpsc/codes/python\PYGZsq{}}
\end{Verbatim}

In this case we only imported one method from the module, not the whole
thing.  Note that now \emph{getcwd} is called by just giving the name of the
method, not \emph{module.method}.  The name \emph{getcwd} is
now in our \emph{namespace}.  If we only imported \emph{getcwd} and tried typing
``os.getcwd()'' we'd get an error, since it wouldn't find \emph{os} in our
namespace.

You can rename things when you import them, which is sometimes useful if
different modules contain different objects with the same name.
For example, to compare how the \titleref{sqrt} function in the standard Python math
module compares to the numpy version:

\begin{Verbatim}[commandchars=\\\{\}]
\PYG{g+gp}{\PYGZgt{}\PYGZgt{}\PYGZgt{} }\PYG{k+kn}{from} \PYG{n+nn}{math} \PYG{k}{import} \PYG{n}{sqrt} \PYG{k}{as} \PYG{n}{sqrtm}
\PYG{g+gp}{\PYGZgt{}\PYGZgt{}\PYGZgt{} }\PYG{k+kn}{from} \PYG{n+nn}{numpy} \PYG{k}{import} \PYG{n}{sqrt} \PYG{k}{as} \PYG{n}{sqrtn}

\PYG{g+gp}{\PYGZgt{}\PYGZgt{}\PYGZgt{} }\PYG{n}{sqrtm}\PYG{p}{(}\PYG{o}{\PYGZhy{}}\PYG{l+m+mf}{1.}\PYG{p}{)}
\PYG{g+gt}{Traceback (most recent call last):}
  File \PYG{n+nb}{\PYGZdq{}\PYGZlt{}stdin\PYGZgt{}\PYGZdq{}}, line \PYG{l+m}{1}, in \PYG{n}{\PYGZlt{}module\PYGZgt{}}
\PYG{g+gr}{ValueError}: \PYG{n}{math domain error}

\PYG{g+gp}{\PYGZgt{}\PYGZgt{}\PYGZgt{} }\PYG{n}{sqrtn}\PYG{p}{(}\PYG{o}{\PYGZhy{}}\PYG{l+m+mf}{1.}\PYG{p}{)}
\PYG{g+go}{nan}
\PYG{g+gt}{Traceback (most recent call last):}
  File \PYG{n+nb}{\PYGZdq{}\PYGZlt{}stdin\PYGZgt{}\PYGZdq{}}, line \PYG{l+m}{1}, in \PYG{n}{\PYGZlt{}module\PYGZgt{}}
\PYG{g+gr}{ValueError}: \PYG{n}{math domain error}
\end{Verbatim}

The standard function gives an error whereas the \emph{numpy} version returns
\emph{nan}, a special \emph{numpy} object representing ``Not a Number''.

You can also import a module and give it a different name locally.  This is
particularly useful if you import a module with a long name, but even for
\emph{numpy} many examples you'll find on the web abbreviate this as \emph{np}
(see {\hyperref[numerical_python:numerical\string-python]{\crossref{\DUrole{std,std-ref}{Numerics in Python}}}}):

\begin{Verbatim}[commandchars=\\\{\}]
\PYG{g+gp}{\PYGZgt{}\PYGZgt{}\PYGZgt{} }\PYG{k+kn}{import} \PYG{n+nn}{numpy} \PYG{k}{as} \PYG{n+nn}{np}
\PYG{g+gp}{\PYGZgt{}\PYGZgt{}\PYGZgt{} }\PYG{n}{theta} \PYG{o}{=} \PYG{n}{np}\PYG{o}{.}\PYG{n}{linspace}\PYG{p}{(}\PYG{l+m+mf}{0.}\PYG{p}{,} \PYG{l+m+mi}{2}\PYG{o}{*}\PYG{n}{np}\PYG{o}{.}\PYG{n}{pi}\PYG{p}{,} \PYG{l+m+mi}{5}\PYG{p}{)}
\PYG{g+gp}{\PYGZgt{}\PYGZgt{}\PYGZgt{} }\PYG{n}{theta}
\PYG{g+go}{array([ 0.        ,  1.57079633,  3.14159265,  4.71238898,  6.28318531])}

\PYG{g+gp}{\PYGZgt{}\PYGZgt{}\PYGZgt{} }\PYG{n}{np}\PYG{o}{.}\PYG{n}{cos}\PYG{p}{(}\PYG{n}{theta}\PYG{p}{)}
\PYG{g+go}{array([  1.00000000e+00,   6.12323400e\PYGZhy{}17,  \PYGZhy{}1.00000000e+00, \PYGZhy{}1.83697020e\PYGZhy{}16,   1.00000000e+00])}
\end{Verbatim}

If you don't like having to type the module name repeatedly you can import
just the things you need into your namespace:

\begin{Verbatim}[commandchars=\\\{\}]
\PYG{g+gp}{\PYGZgt{}\PYGZgt{}\PYGZgt{} }\PYG{k+kn}{from} \PYG{n+nn}{numpy} \PYG{k}{import} \PYG{n}{pi}\PYG{p}{,} \PYG{n}{linspace}\PYG{p}{,} \PYG{n}{cos}
\PYG{g+gp}{\PYGZgt{}\PYGZgt{}\PYGZgt{} }\PYG{n}{theta} \PYG{o}{=} \PYG{n}{linspace}\PYG{p}{(}\PYG{l+m+mf}{0.}\PYG{p}{,} \PYG{l+m+mi}{2}\PYG{o}{*}\PYG{n}{pi}\PYG{p}{,} \PYG{l+m+mi}{5}\PYG{p}{)}
\PYG{g+gp}{\PYGZgt{}\PYGZgt{}\PYGZgt{} }\PYG{n}{theta}
\PYG{g+go}{array([ 0.        ,  1.57079633,  3.14159265,  4.71238898,  6.28318531])}
\PYG{g+gp}{\PYGZgt{}\PYGZgt{}\PYGZgt{} }\PYG{n}{cos}\PYG{p}{(}\PYG{n}{theta}\PYG{p}{)}
\PYG{g+go}{array([  1.00000000e+00,   6.12323400e\PYGZhy{}17,  \PYGZhy{}1.00000000e+00, \PYGZhy{}1.83697020e\PYGZhy{}16,   1.00000000e+00])}
\end{Verbatim}

If you're going to be using lots of things form \emph{numpy} you might want to
import everything into your namespace:

\begin{Verbatim}[commandchars=\\\{\}]
\PYG{g+gp}{\PYGZgt{}\PYGZgt{}\PYGZgt{} }\PYG{k+kn}{from} \PYG{n+nn}{numpy} \PYG{k}{import} \PYG{o}{*}
\end{Verbatim}

Then \emph{linspace}, \emph{pi}, \emph{cos}, and several hundred other things will be available
without the prefix.

When writing code it is often best to not do this, however, since then it is
not clear to the reader (or even to the programmer sometimes)
what methods or attributes are coming from which module if several
different modules are being used. (They may define methods with the same
names but that do very different things, for example.)

When using IPython, it is often convenient to start it with:

\begin{Verbatim}[commandchars=\\\{\}]
\PYGZdl{} ipython \PYGZhy{}\PYGZhy{}pylab
\end{Verbatim}

This automatically imports everything from \emph{numpy} into the namespace, and
also all of the plotting tools from \emph{matplotlib}.


\subsection{Further reading}
\label{python_scripts_modules:further-reading}
\href{http://docs.python.org/2/tutorial/modules.html}{Modules section of Python documentation}


\section{Python functions}
\label{python_functions:python-functions}\label{python_functions::doc}\label{python_functions:id1}
Functions are easily defined in Python using \emph{def}, for example:

\begin{Verbatim}[commandchars=\\\{\}]
\PYG{g+gp}{\PYGZgt{}\PYGZgt{}\PYGZgt{} }\PYG{k}{def} \PYG{n+nf}{myfcn}\PYG{p}{(}\PYG{n}{x}\PYG{p}{)}\PYG{p}{:}
\PYG{g+gp}{... }    \PYG{k+kn}{import} \PYG{n+nn}{numpy} \PYG{k}{as} \PYG{n+nn}{np}
\PYG{g+gp}{... }    \PYG{n}{y} \PYG{o}{=} \PYG{n}{np}\PYG{o}{.}\PYG{n}{cos}\PYG{p}{(}\PYG{n}{x}\PYG{p}{)} \PYG{o}{*} \PYG{n}{np}\PYG{o}{.}\PYG{n}{exp}\PYG{p}{(}\PYG{n}{x}\PYG{p}{)}
\PYG{g+gp}{... }    \PYG{k}{return} \PYG{n}{y}
\PYG{g+gp}{...}

\PYG{g+gp}{\PYGZgt{}\PYGZgt{}\PYGZgt{} }\PYG{n}{myfcn}\PYG{p}{(}\PYG{l+m+mf}{0.}\PYG{p}{)}
\PYG{g+go}{1.0}

\PYG{g+gp}{\PYGZgt{}\PYGZgt{}\PYGZgt{} }\PYG{n}{myfcn}\PYG{p}{(}\PYG{l+m+mf}{1.}\PYG{p}{)}
\PYG{g+go}{1.4686939399158851}
\end{Verbatim}

As elsewhere in Python, there is no begin-end notation except the
indentation.  If you are defining a function at the command line as above,
you need to input a blank line to indicate that you are done typing in the
function.


\subsection{Defining functions in modules}
\label{python_functions:defining-functions-in-modules}
Except for very simple functions, you do not want to type it in at the
command line in Python.  Normally you want to create a text file containing
your function and import the resulting module into your interactive session.

If you have a file named \emph{myfile.py} for example that contains:

\begin{Verbatim}[commandchars=\\\{\}]
\PYG{k}{def} \PYG{n+nf}{myfcn}\PYG{p}{(}\PYG{n}{x}\PYG{p}{)}\PYG{p}{:}
    \PYG{k+kn}{import} \PYG{n+nn}{numpy} \PYG{k}{as} \PYG{n+nn}{np}
    \PYG{n}{y} \PYG{o}{=} \PYG{n}{np}\PYG{o}{.}\PYG{n}{cos}\PYG{p}{(}\PYG{n}{x}\PYG{p}{)} \PYG{o}{*} \PYG{n}{np}\PYG{o}{.}\PYG{n}{exp}\PYG{p}{(}\PYG{n}{x}\PYG{p}{)}
    \PYG{k}{return} \PYG{n}{y}
\end{Verbatim}

and this file is in your Python search path (see \DUrole{xref,std,std-ref}{python\_path}), then
you can do:

\begin{Verbatim}[commandchars=\\\{\}]
\PYG{g+gp}{\PYGZgt{}\PYGZgt{}\PYGZgt{} }\PYG{k+kn}{from} \PYG{n+nn}{myfile} \PYG{k}{import} \PYG{n}{myfcn}
\PYG{g+gp}{\PYGZgt{}\PYGZgt{}\PYGZgt{} }\PYG{n}{myfcn}\PYG{p}{(}\PYG{l+m+mf}{0.}\PYG{p}{)}
\PYG{g+go}{1.0}
\PYG{g+gp}{\PYGZgt{}\PYGZgt{}\PYGZgt{} }\PYG{n}{myfcn}\PYG{p}{(}\PYG{l+m+mf}{1.}\PYG{p}{)}
\PYG{g+go}{1.4686939399158851}
\end{Verbatim}

In Python a function is an object that can be manipulated like any other
object.


\subsection{Lambda functions}
\label{python_functions:id2}\label{python_functions:lambda-functions}
Some functions can be easily defined in a single line of code, and it is
sometimes useful to be able to define a function ``on the fly'' using ``lambda''
notation.  To define a function that returns 2*x for any input x, rather
than:

\begin{Verbatim}[commandchars=\\\{\}]
\PYG{k}{def} \PYG{n+nf}{f}\PYG{p}{(}\PYG{n}{x}\PYG{p}{)}\PYG{p}{:}
    \PYG{k}{return} \PYG{l+m+mi}{2}\PYG{o}{*}\PYG{n}{x}
\end{Verbatim}

we could also define \emph{f} via:

\begin{Verbatim}[commandchars=\\\{\}]
\PYG{n}{f} \PYG{o}{=} \PYG{k}{lambda} \PYG{n}{x}\PYG{p}{:} \PYG{l+m+mi}{2}\PYG{o}{*}\PYG{n}{x}
\end{Verbatim}

You can also define functions of more than one variable, e.g.:

\begin{Verbatim}[commandchars=\\\{\}]
\PYG{n}{g} \PYG{o}{=} \PYG{k}{lambda} \PYG{n}{x}\PYG{p}{,}\PYG{n}{y}\PYG{p}{:} \PYG{l+m+mi}{2}\PYG{o}{*}\PYG{p}{(}\PYG{n}{x}\PYG{o}{+}\PYG{n}{y}\PYG{p}{)}
\end{Verbatim}


\subsection{Further reading}
\label{python_functions:further-reading}

\section{Python strings}
\label{python_strings::doc}\label{python_strings:python-strings}\label{python_strings:id1}
See this \href{http://docs.python.org/release/2.5.2/lib/string-methods.html}{List of methods applicable to strings}


\subsection{String formatting}
\label{python_strings:string-formatting}
Often you want to construct a string that incorporates the values of some
variables.  This can be done using the form \emph{format \% values} where \emph{format}
is a string that describes the desired format and \emph{values} is a single value
or tuple of values that go into various slots in the format.

See \href{http://docs.python.org/release/2.5.2/lib/typesseq-strings.html}{String Formatting Operations}

This is best learned from some examples:

\begin{Verbatim}[commandchars=\\\{\}]
\PYG{g+gp}{\PYGZgt{}\PYGZgt{}\PYGZgt{} }\PYG{n}{x} \PYG{o}{=} \PYG{l+m+mf}{45.6}
\PYG{g+gp}{\PYGZgt{}\PYGZgt{}\PYGZgt{} }\PYG{n}{s} \PYG{o}{=} \PYG{l+s}{\PYGZdq{}}\PYG{l+s}{The value of x is }\PYG{l+s}{\PYGZpc{}}\PYG{l+s}{s}\PYG{l+s}{\PYGZdq{}}  \PYG{o}{\PYGZpc{}} \PYG{n}{x}
\PYG{g+gp}{\PYGZgt{}\PYGZgt{}\PYGZgt{} }\PYG{n}{s}
\PYG{g+go}{\PYGZsq{}The value of x is 45.6\PYGZsq{}}
\end{Verbatim}

The \emph{\%s} in the format string means to convert x to a string and insert into
the format.  It will use as few spaces as possible.

\begin{Verbatim}[commandchars=\\\{\}]
\PYG{g+gp}{\PYGZgt{}\PYGZgt{}\PYGZgt{} }\PYG{n}{s} \PYG{o}{=} \PYG{l+s}{\PYGZdq{}}\PYG{l+s}{The value of x is }\PYG{l+s}{\PYGZpc{}}\PYG{l+s}{21.14e}\PYG{l+s}{\PYGZdq{}}  \PYG{o}{\PYGZpc{}} \PYG{n}{x}
\PYG{g+gp}{\PYGZgt{}\PYGZgt{}\PYGZgt{} }\PYG{n}{s}
\PYG{g+go}{\PYGZsq{}The value of x is  4.56000000000000e+01\PYGZsq{}}
\end{Verbatim}

In the case above, exponential notation is used with 14 digits to the right
of the decimal point, put into a field of 21 digits total.  (You need at
least 7 extra characters
to leave room for a possible minus sign as well as the first
digit, the decimal point, and the exponent such as \emph{e+01}.

\begin{Verbatim}[commandchars=\\\{\}]
\PYG{g+gp}{\PYGZgt{}\PYGZgt{}\PYGZgt{} }\PYG{n}{y} \PYG{o}{=} \PYG{o}{\PYGZhy{}}\PYG{l+m+mf}{0.324876}
\PYG{g+gp}{\PYGZgt{}\PYGZgt{}\PYGZgt{} }\PYG{n}{s} \PYG{o}{=} \PYG{l+s}{\PYGZdq{}}\PYG{l+s}{Now x is }\PYG{l+s}{\PYGZpc{}}\PYG{l+s}{8.3f and y is }\PYG{l+s}{\PYGZpc{}}\PYG{l+s}{8.3f}\PYG{l+s}{\PYGZdq{}} \PYG{o}{\PYGZpc{}} \PYG{p}{(}\PYG{n}{x}\PYG{p}{,}\PYG{n}{y}\PYG{p}{)}
\PYG{g+gp}{\PYGZgt{}\PYGZgt{}\PYGZgt{} }\PYG{n}{s}
\PYG{g+go}{\PYGZsq{}Now x is   45.600 and y is   \PYGZhy{}0.325\PYGZsq{}}
\end{Verbatim}

In this example, fixed notation is used instead of scientific notation, with
3 digits to the right of the decimal point, in a field 8 characters wide.
Note that \emph{y} has been rounded.

In the last example, two variables are inserted into the format string.


\subsection{Further reading}
\label{python_strings:further-reading}
See  also:
\begin{itemize}
\item {} 
\url{http://docs.python.org/tutorial/inputoutput.html}

\item {} 
\href{http://docs.python.org/release/2.5.2/lib/string-methods.html}{List of methods applicable to strings}

\item {} 
\href{http://docs.python.org/2/tutorial/inputoutput.html}{Input and Output documentation}

\end{itemize}


\section{Numerics in Python}
\label{numerical_python:numerical-python}\label{numerical_python::doc}\label{numerical_python:numerics-in-python}
Python is a general programming language and is used for many purposes that have
nothing to do with scientific computing or numerical methods.  However,
there are a number of modules that can be imported that provide a variety of
numerical methods and other tools that are very useful for scientific
computing.

The basic module needed for most numerical work is \emph{NumPy}, which provides in
particular the data structures needed for working with
arrays of real numbers representing vectors or matrices.  The module \emph{SciPy}
provides additional numerical methods and tools.

If you know Matlab, you will find that many of the things you are used to
doing in that language can be done using \emph{NumPy} and \emph{SciPy}, although the
syntax is often a bit different.  Matlab users will find web page
\phantomsection\label{numerical_python:id1}{\hyperref[biblio:numpy\string-for\string-matlab\string-users]{\crossref{{[}NumPy-for-Matlab-Users{]}}}} crucial for understanding the differences and
transitioning to Python, and this page is useful for all new users.  A
tutorial can be found at \phantomsection\label{numerical_python:id2}{\hyperref[biblio:numpy\string-tutorial]{\crossref{{[}NumPy-tutorial{]}}}}.


\subsection{Vectors and Matrices}
\label{numerical_python:vectors-and-matrices}
Python has lists as a built-in data type, e.g.:

\begin{Verbatim}[commandchars=\\\{\}]
\PYG{g+gp}{\PYGZgt{}\PYGZgt{}\PYGZgt{} }\PYG{n}{x} \PYG{o}{=} \PYG{p}{[}\PYG{l+m+mf}{1.}\PYG{p}{,} \PYG{l+m+mf}{2.}\PYG{p}{,} \PYG{l+m+mf}{3.}\PYG{p}{]}
\end{Verbatim}

defines a list that contains 3 real numbers and might be viewed as a vector.
However, you cannot easily do arithmetic on such lists the way you can with
vectors in Matlab, e.g.  2*x does not give {[}2., 4., 6.{]} as you might hope:

\begin{Verbatim}[commandchars=\\\{\}]
\PYG{g+gp}{\PYGZgt{}\PYGZgt{}\PYGZgt{} }\PYG{l+m+mi}{2}\PYG{o}{*}\PYG{n}{x}
\PYG{g+go}{[1.0, 2.0, 3.0, 1.0, 2.0, 3.0]}
\end{Verbatim}

instead it doubles the length of \emph{x}, and \emph{x+x} would give the same thing.
You also cannot apply \emph{sqrt} to a list to get a new list containing the
square root of each element, for example.

Two-dimensional arrays are also a bit clumsy in Python, as they have to be
specified as a list of lists, e.g. a 3x2 array with the elements 11,12 in
the first row, 21,22 in the second row, 31,32 in the third row would be
specified by:

\begin{Verbatim}[commandchars=\\\{\}]
\PYG{g+gp}{\PYGZgt{}\PYGZgt{}\PYGZgt{} }\PYG{n}{A} \PYG{o}{=} \PYG{p}{[}\PYG{p}{[}\PYG{l+m+mi}{11}\PYG{p}{,} \PYG{l+m+mi}{12}\PYG{p}{]}\PYG{p}{,} \PYG{p}{[}\PYG{l+m+mi}{21}\PYG{p}{,} \PYG{l+m+mi}{22}\PYG{p}{]}\PYG{p}{,} \PYG{p}{[}\PYG{l+m+mi}{31}\PYG{p}{,} \PYG{l+m+mi}{32}\PYG{p}{]}\PYG{p}{]}
\end{Verbatim}

Note that indexing always starts with 0 in Python, so we find for example
that:

\begin{Verbatim}[commandchars=\\\{\}]
\PYG{g+gp}{\PYGZgt{}\PYGZgt{}\PYGZgt{} }\PYG{n}{A}\PYG{p}{[}\PYG{l+m+mi}{0}\PYG{p}{]}
\PYG{g+go}{[11, 12]}

\PYG{g+gp}{\PYGZgt{}\PYGZgt{}\PYGZgt{} }\PYG{n}{A}\PYG{p}{[}\PYG{l+m+mi}{1}\PYG{p}{]}
\PYG{g+go}{[21, 22]}

\PYG{g+gp}{\PYGZgt{}\PYGZgt{}\PYGZgt{} }\PYG{n}{A}\PYG{p}{[}\PYG{l+m+mi}{1}\PYG{p}{]}\PYG{p}{[}\PYG{l+m+mi}{0}\PYG{p}{]}
\PYG{g+go}{21}
\end{Verbatim}

Here A{[}0{]} refers to the 0-index element of A, which is itself a list {[}11, 12{]}.
and A{[}1{]}{[}0{]} can be understood as the 0-index element of A{[}1{]} = {[}21, 22{]}.

You cannot work with A as a matrix, for example to multiply it by a vector,
except by writing code that loops over the elements explicitly.

NumPy was developed to make it easy to do the sorts of things we want to do
with matrices and vectors, and more generally n-dimensional arrays of real
numbers.

For example:

\begin{Verbatim}[commandchars=\\\{\}]
\PYG{g+gp}{\PYGZgt{}\PYGZgt{}\PYGZgt{} }\PYG{k+kn}{import} \PYG{n+nn}{numpy} \PYG{k}{as} \PYG{n+nn}{np}
\PYG{g+gp}{\PYGZgt{}\PYGZgt{}\PYGZgt{} }\PYG{n}{x} \PYG{o}{=} \PYG{n}{np}\PYG{o}{.}\PYG{n}{array}\PYG{p}{(}\PYG{p}{[}\PYG{l+m+mf}{1.}\PYG{p}{,} \PYG{l+m+mf}{2.}\PYG{p}{,} \PYG{l+m+mf}{3.}\PYG{p}{]}\PYG{p}{)}
\PYG{g+gp}{\PYGZgt{}\PYGZgt{}\PYGZgt{} }\PYG{n}{x}
\PYG{g+go}{array([ 1.,  2.,  3.])}

\PYG{g+gp}{\PYGZgt{}\PYGZgt{}\PYGZgt{} }\PYG{l+m+mi}{2}\PYG{o}{*}\PYG{n}{x}
\PYG{g+go}{array([ 2.,  4.,  6.])}

\PYG{g+gp}{\PYGZgt{}\PYGZgt{}\PYGZgt{} }\PYG{n}{np}\PYG{o}{.}\PYG{n}{sqrt}\PYG{p}{(}\PYG{n}{x}\PYG{p}{)}
\PYG{g+go}{array([ 1.        ,  1.41421356,  1.73205081])}
\end{Verbatim}

We see that we can multiply by a scalar or take component-wise square roots.

You may find it ugly to have to start numpy command with np., as necessary
here since we imported numpy as np. Instead you could do:

\begin{Verbatim}[commandchars=\\\{\}]
\PYG{g+gp}{\PYGZgt{}\PYGZgt{}\PYGZgt{} }\PYG{k+kn}{from} \PYG{n+nn}{numpy} \PYG{k}{import} \PYG{o}{*}
\end{Verbatim}

and then just use \emph{sqrt}, for example, and you will get the NumPy version.
But in these notes and many Python examples you'll see, the module is
explicitly listed so it is clear where a function is coming from.

For matrices, we can convert our list of lists into a \emph{NumPy} array as
follows (we specify \emph{dtype=float} to make sure the elements of A are stored
as floating point real number even though we type them here as integers):

\begin{Verbatim}[commandchars=\\\{\}]
\PYG{g+gp}{\PYGZgt{}\PYGZgt{}\PYGZgt{} }\PYG{n}{A} \PYG{o}{=} \PYG{n}{np}\PYG{o}{.}\PYG{n}{array}\PYG{p}{(}\PYG{p}{[}\PYG{p}{[}\PYG{l+m+mi}{11}\PYG{p}{,} \PYG{l+m+mi}{12}\PYG{p}{]}\PYG{p}{,} \PYG{p}{[}\PYG{l+m+mi}{21}\PYG{p}{,} \PYG{l+m+mi}{22}\PYG{p}{]}\PYG{p}{,} \PYG{p}{[}\PYG{l+m+mi}{31}\PYG{p}{,} \PYG{l+m+mi}{32}\PYG{p}{]}\PYG{p}{]}\PYG{p}{,} \PYG{n}{dtype}\PYG{o}{=}\PYG{n+nb}{float}\PYG{p}{)}
\PYG{g+gp}{\PYGZgt{}\PYGZgt{}\PYGZgt{} }\PYG{n}{A}
\PYG{g+go}{array([[ 11.,  12.],}
\PYG{g+go}{       [ 21.,  22.],}
\PYG{g+go}{       [ 31.,  32.]])}

\PYG{g+gp}{\PYGZgt{}\PYGZgt{}\PYGZgt{} }\PYG{n}{A}\PYG{p}{[}\PYG{l+m+mi}{0}\PYG{p}{,}\PYG{l+m+mi}{1}\PYG{p}{]}
\PYG{g+go}{12.}
\end{Verbatim}

Note that we can now index into the array as in matrix notation A{[}0,1{]}
(remembering that indexing starts at 0 in Python), so this the {[}0,1{]}
element of A means the first row and second column.

We can also do slicing operations, extracting a single row or column:

\begin{Verbatim}[commandchars=\\\{\}]
\PYG{g+gp}{\PYGZgt{}\PYGZgt{}\PYGZgt{} }\PYG{n}{A}\PYG{p}{[}\PYG{l+m+mi}{0}\PYG{p}{,}\PYG{p}{:}\PYG{p}{]}
\PYG{g+go}{array([11., 12.])}
\PYG{g+gp}{\PYGZgt{}\PYGZgt{}\PYGZgt{} }\PYG{n}{A}\PYG{p}{[}\PYG{p}{:}\PYG{p}{,}\PYG{l+m+mi}{0}\PYG{p}{]}
\PYG{g+go}{array([11., 21., 31.])}
\end{Verbatim}

Since A is a NumPy array object, there are certain methods automatically
defined on A, such as the transpose:

\begin{Verbatim}[commandchars=\\\{\}]
\PYG{g+gp}{\PYGZgt{}\PYGZgt{}\PYGZgt{} }\PYG{n}{A}\PYG{o}{.}\PYG{n}{T}
\PYG{g+go}{array([[11., 21., 31.],}
\PYG{g+go}{       [12., 22., 32.]])}
\end{Verbatim}

Seeing all the methods defined for A is easy if you use the IPython shell
(see \DUrole{xref,std,std-ref}{ipython}), just type A. followed by the tab key (you will find
there are 155 methods defined).

We can do matrix-vector or matrix-matrix multiplication using the NumPy dot
function:

\begin{Verbatim}[commandchars=\\\{\}]
\PYG{g+gp}{\PYGZgt{}\PYGZgt{}\PYGZgt{} }\PYG{n}{np}\PYG{o}{.}\PYG{n}{dot}\PYG{p}{(}\PYG{n}{A}\PYG{o}{.}\PYG{n}{T}\PYG{p}{,} \PYG{n}{x}\PYG{p}{)}
\PYG{g+go}{array([ 146.,  152.])}

\PYG{g+gp}{\PYGZgt{}\PYGZgt{}\PYGZgt{} }\PYG{n}{np}\PYG{o}{.}\PYG{n}{dot}\PYG{p}{(}\PYG{n}{A}\PYG{o}{.}\PYG{n}{T}\PYG{p}{,} \PYG{n}{A}\PYG{p}{)}
\PYG{g+go}{array([[1523., 1586.],}
\PYG{g+go}{       [1586., 1652.]])}
\end{Verbatim}

This looks somewhat less mathematical than Matlab notation A'{\color{red}\bfseries{}*}A, but the syntax
and data structures of Matlab were designed specifically for linear algebra,
whereas Python is a more general language and so doing linear algebra has to
be done in this framework.

Note that elements of a \emph{NumPy} array are always all of the same type, and
generally we want \emph{floats}, though integer arrays can also be defined.
This is different than Python lists, which can contain elements with
different types, e.g.:

\begin{Verbatim}[commandchars=\\\{\}]
\PYG{g+gp}{\PYGZgt{}\PYGZgt{}\PYGZgt{} }\PYG{n}{L} \PYG{o}{=} \PYG{p}{[}\PYG{l+m+mi}{2}\PYG{p}{,} \PYG{l+m+mf}{3.}\PYG{p}{,} \PYG{l+s}{\PYGZsq{}}\PYG{l+s}{xyz}\PYG{l+s}{\PYGZsq{}}\PYG{p}{,} \PYG{p}{[}\PYG{l+m+mi}{4}\PYG{p}{,}\PYG{l+m+mi}{5}\PYG{p}{]}\PYG{p}{]}

\PYG{g+gp}{\PYGZgt{}\PYGZgt{}\PYGZgt{} }\PYG{n+nb}{print} \PYG{n+nb}{type}\PYG{p}{(}\PYG{n}{L}\PYG{p}{[}\PYG{l+m+mi}{0}\PYG{p}{]}\PYG{p}{)}\PYG{p}{,} \PYG{n+nb}{type}\PYG{p}{(}\PYG{n}{L}\PYG{p}{[}\PYG{l+m+mi}{1}\PYG{p}{]}\PYG{p}{)}\PYG{p}{,} \PYG{n+nb}{type}\PYG{p}{(}\PYG{n}{L}\PYG{p}{[}\PYG{l+m+mi}{2}\PYG{p}{]}\PYG{p}{)}\PYG{p}{,} \PYG{n+nb}{type}\PYG{p}{(}\PYG{n}{L}\PYG{p}{[}\PYG{l+m+mi}{3}\PYG{p}{]}\PYG{p}{)}
\PYG{g+go}{\PYGZlt{}type \PYGZsq{}int\PYGZsq{}\PYGZgt{} \PYGZlt{}type \PYGZsq{}float\PYGZsq{}\PYGZgt{} \PYGZlt{}type \PYGZsq{}str\PYGZsq{}\PYGZgt{} \PYGZlt{}type \PYGZsq{}list\PYGZsq{}\PYGZgt{}}
\end{Verbatim}


\subsection{Component-wise operations}
\label{numerical_python:component-wise-operations}
One thing to watch out for if you are used to Matlab notation:  In Matlab
some operations (such as sqrt, sin, cos, exp, etc) can be applied to vectors
or matrices and will automatically be applied component-wise.  Other
operations like * and / (multiplication and division) attempt to do things
in terms of linear algebra, and so in Matlab, A*B gives the matrix product
and only makes sense if the number of columns of A agrees with the number of
rows of B.  If you want a component-wise product of A and B you must use .*
instead, with a period before the {\color{red}\bfseries{}*}.

In NumPy,  * and / are applied component-wise, like any other operation.  To
get a matrix-product you must use \emph{dot}:

\begin{Verbatim}[commandchars=\\\{\}]
\PYG{g+gp}{\PYGZgt{}\PYGZgt{}\PYGZgt{} }\PYG{n}{A} \PYG{o}{=} \PYG{n}{np}\PYG{o}{.}\PYG{n}{array}\PYG{p}{(}\PYG{p}{[}\PYG{p}{[}\PYG{l+m+mi}{1}\PYG{p}{,}\PYG{l+m+mi}{2}\PYG{p}{]}\PYG{p}{,} \PYG{p}{[}\PYG{l+m+mi}{3}\PYG{p}{,}\PYG{l+m+mi}{4}\PYG{p}{]}\PYG{p}{]}\PYG{p}{)}
\PYG{g+gp}{\PYGZgt{}\PYGZgt{}\PYGZgt{} }\PYG{n}{B} \PYG{o}{=} \PYG{n}{np}\PYG{o}{.}\PYG{n}{array}\PYG{p}{(}\PYG{p}{[}\PYG{p}{[}\PYG{l+m+mi}{5}\PYG{p}{,}\PYG{l+m+mi}{0}\PYG{p}{]}\PYG{p}{,} \PYG{p}{[}\PYG{l+m+mi}{0}\PYG{p}{,}\PYG{l+m+mi}{7}\PYG{p}{]}\PYG{p}{]}\PYG{p}{)}
\PYG{g+gp}{\PYGZgt{}\PYGZgt{}\PYGZgt{} }\PYG{n}{A}
\PYG{g+go}{array([[1, 2],}
\PYG{g+go}{       [3, 4]])}
\PYG{g+gp}{\PYGZgt{}\PYGZgt{}\PYGZgt{} }\PYG{n}{B}
\PYG{g+go}{array([[5, 0],}
\PYG{g+go}{       [0, 7]])}

\PYG{g+gp}{\PYGZgt{}\PYGZgt{}\PYGZgt{} }\PYG{n}{A}\PYG{o}{*}\PYG{n}{B}
\PYG{g+go}{array([[ 5,  0],}
\PYG{g+go}{       [ 0, 28]])}

\PYG{g+gp}{\PYGZgt{}\PYGZgt{}\PYGZgt{} }\PYG{n}{np}\PYG{o}{.}\PYG{n}{dot}\PYG{p}{(}\PYG{n}{A}\PYG{p}{,}\PYG{n}{B}\PYG{p}{)}
\PYG{g+go}{array([[ 5, 14],}
\PYG{g+go}{       [15, 28]])}
\end{Verbatim}

Many other linear algebra tools can be found in \emph{NumPy}.  For example, to
solve a linear system Ax = b using Gaussian Elimination, we can do:

\begin{Verbatim}[commandchars=\\\{\}]
\PYG{g+gp}{\PYGZgt{}\PYGZgt{}\PYGZgt{} }\PYG{n}{A}
\PYG{g+go}{array([[1, 2],}
\PYG{g+go}{       [3, 4]])}

\PYG{g+gp}{\PYGZgt{}\PYGZgt{}\PYGZgt{} }\PYG{n}{b} \PYG{o}{=} \PYG{n}{np}\PYG{o}{.}\PYG{n}{array}\PYG{p}{(}\PYG{p}{[}\PYG{l+m+mi}{2}\PYG{p}{,}\PYG{l+m+mi}{3}\PYG{p}{]}\PYG{p}{)}
\PYG{g+gp}{\PYGZgt{}\PYGZgt{}\PYGZgt{} }\PYG{n}{x} \PYG{o}{=} \PYG{n}{np}\PYG{o}{.}\PYG{n}{linalg}\PYG{o}{.}\PYG{n}{solve}\PYG{p}{(}\PYG{n}{A}\PYG{p}{,}\PYG{n}{b}\PYG{p}{)}

\PYG{g+gp}{\PYGZgt{}\PYGZgt{}\PYGZgt{} }\PYG{n}{x}
\PYG{g+go}{array([\PYGZhy{}1. ,  1.5])}
\end{Verbatim}

To find the eigenvalues and eigenvectors of A:

\begin{Verbatim}[commandchars=\\\{\}]
\PYG{g+gp}{\PYGZgt{}\PYGZgt{}\PYGZgt{} }\PYG{n}{evals}\PYG{p}{,} \PYG{n}{evecs} \PYG{o}{=} \PYG{n}{np}\PYG{o}{.}\PYG{n}{linalg}\PYG{o}{.}\PYG{n}{eig}\PYG{p}{(}\PYG{n}{A}\PYG{p}{)}

\PYG{g+gp}{\PYGZgt{}\PYGZgt{}\PYGZgt{} }\PYG{n}{evals}
\PYG{g+go}{array([\PYGZhy{}0.37228132,  5.37228132])}

\PYG{g+gp}{\PYGZgt{}\PYGZgt{}\PYGZgt{} }\PYG{n}{evecs}
\PYG{g+go}{array([[\PYGZhy{}0.82456484, \PYGZhy{}0.41597356],}
\PYG{g+go}{       [ 0.56576746, \PYGZhy{}0.90937671]])}
\end{Verbatim}

Note: You may be tempted to use the variable name \emph{lambda} for the eigenvalues
of a matrix, but this isn't allowed in Python because \emph{lambda} is a keyword
of the language, see {\hyperref[python_functions:lambda\string-functions]{\crossref{\DUrole{std,std-ref}{Lambda functions}}}}.


\subsection{Further reading}
\label{numerical_python:further-reading}
Be sure to visit
\begin{itemize}
\item {} 
\url{http://www.scipy.org/Tentative\_NumPy\_Tutorial}

\item {} 
\url{http://www.scipy.org/NumPy\_for\_Matlab\_Users}

\end{itemize}

See also \phantomsection\label{numerical_python:id7}{\hyperref[biblio:numpy\string-pros\string-cons]{\crossref{{[}NumPy-pros-cons{]}}}} for more about differences with other
mathematical languages.


\section{IPython\_notebook}
\label{ipython_notebook:ipython-notebook}\label{ipython_notebook::doc}\label{ipython_notebook:id1}
The IPython notebook is fairly new and changing rapidly.  The version
originally installed in the class VM is version 0.10.  To get the latest
development version, which has some nicer features, do the following:

\begin{Verbatim}[commandchars=\\\{\}]
\PYGZdl{} cd
\PYGZdl{} git clone https://github.com/ipython/ipython.git
\PYGZdl{} cd ipython
\PYGZdl{} sudo python setup.py install
\end{Verbatim}

Then start the notebook via:

\begin{Verbatim}[commandchars=\\\{\}]
\PYGZdl{} ipython notebook \PYGZhy{}\PYGZhy{}pylab inline
\end{Verbatim}

in order to have the plots appear inline.  If you leave off this argument, a
new window will be opened for each plot.

Read more about the notebook in the \href{http://ipython.org/ipython-doc/dev/interactive/htmlnotebook.html}{documentation}

See some cool examples in the \href{http://nbviewer.ipython.org/}{IPython notebook viewer}.

See also {\hyperref[sage:sagemath]{\crossref{\DUrole{std,std-ref}{Sage}}}}.


\subsection{Interactive notebooks}
\label{ipython_notebook:interactive-notebooks}
\href{http://ipython.org/ipython-doc/dev/whatsnew/version2.0.html}{IPython 2.0}
(released April 1, 2014) includes \href{http://nbviewer.ipython.org/github/ipython/ipython/blob/2.x/examples/Interactive\%20Widgets/Index.ipynb}{interactive widgets}

See these \href{http://ipython.org/install.html}{Tips for installing IPython 2.0}
on your own computer.

SageMathCloud does not yet have IPython 2.0 and for various technical
reasons will not have it for a while.

SageMathCloud does now have \href{https://github.com/jakevdp/mpld3}{mpld3}, which
allows zooming in on plots.  For a demo, see {\hyperref[labs/lab13:lab13]{\crossref{\DUrole{std,std-ref}{Lab 13: Tuesday May 13, 2014}}}} or
\href{http://jakevdp.github.io/blog/2013/12/19/a-d3-viewer-for-matplotlib/}{Jake's blog post on mpld3}

In addition to \titleref{mpld3}, Jake Vanderplass has also developed:
\begin{itemize}
\item {} 
\href{https://github.com/jakevdp/ipywidgets}{ipywidgets} similar to the 2.0
widgets in some ways but persistent also if the ``static'' notebook is
viewed via \href{http://nbviewer.ipython.org}{nbviewer}.
See \href{http://jakevdp.github.io/blog/2013/12/05/static-interactive-widgets/}{Jake's blog post on ipywidgets}.

\item {} 
\href{https://github.com/jakevdp/JSAnimation}{JSAnimation} for persistent
animations.  This is used for example for all the animations in the
\href{http://clawpack.github.io/doc/galleries.html}{Clawpack galleries}.
(\href{http://www.clawpack.org}{Clawpack} is an open source software
package developed by Randy LeVeque and many others for solving
hyperbolic partial differential equations.)

\end{itemize}


\section{Sage}
\label{sage:sage}\label{sage:sagemath}\label{sage::doc}
\href{http://sagemath.org}{Sage} is an open source collection of mathematical
software with a common Python interface.  The lead developer is William
Stein, a number theorist in the Mathematics Department at the University of
Washington.

The Sage notebook was an original model for the {\hyperref[ipython_notebook:ipython\string-notebook]{\crossref{\DUrole{std,std-ref}{IPython\_notebook}}}}.

You can try Sage online by typing into a cell at
\url{https://sagecell.sagemath.org/}.  You can type straight
Python as well as using the enhanced capabilities of Sage.

See:
\begin{itemize}
\item {} 
\url{http://interact.sagemath.org/top-rated-posts}

\item {} 
\url{http://trac.sagemath.org/sage\_trac/}

\end{itemize}

for some online sage examples.


\section{Plotting with Python}
\label{python_plotting:plotting-with-python}\label{python_plotting::doc}\label{python_plotting:python-plotting}

\subsection{matplotlib and pylab}
\label{python_plotting:matplotlib-and-pylab}\label{python_plotting:pylab}
There are nice tools for making plots of 1d and 2d data (curves, contour
plots, etc.) in the module
\href{http://matplotlib.sourceforge.net/}{matplotlib}.
Many of these plot commands are very similar to those in Matlab.

To see some of what's possible (and learn how to do it), visit the
\href{http://matplotlib.sourceforge.net/gallery.html}{matplotlib gallery}.
Clicking on a figure displays the Python commands needed to create it.

The best way to get matplotlib working interactively in a standard Python
shell is to do:

\begin{Verbatim}[commandchars=\\\{\}]
\PYGZdl{} python
\PYGZgt{}\PYGZgt{}\PYGZgt{} import pylab
\PYGZgt{}\PYGZgt{}\PYGZgt{} pylab.interactive(True)
\end{Verbatim}

\emph{pylab} includes not only \emph{matplotlib} but also \emph{numpy}.
Then you should be able to do:

\begin{Verbatim}[commandchars=\\\{\}]
\PYG{g+gp}{\PYGZgt{}\PYGZgt{}\PYGZgt{} }\PYG{n}{x} \PYG{o}{=} \PYG{n}{pylab}\PYG{o}{.}\PYG{n}{linspace}\PYG{p}{(}\PYG{o}{\PYGZhy{}}\PYG{l+m+mi}{1}\PYG{p}{,} \PYG{l+m+mi}{1}\PYG{p}{,} \PYG{l+m+mi}{20}\PYG{p}{)}
\PYG{g+gp}{\PYGZgt{}\PYGZgt{}\PYGZgt{} }\PYG{n}{pylab}\PYG{o}{.}\PYG{n}{plot}\PYG{p}{(}\PYG{n}{x}\PYG{p}{,} \PYG{n}{x}\PYG{o}{*}\PYG{o}{*}\PYG{l+m+mi}{2}\PYG{p}{,} \PYG{l+s}{\PYGZsq{}}\PYG{l+s}{o\PYGZhy{}}\PYG{l+s}{\PYGZsq{}}\PYG{p}{)}
\end{Verbatim}

and see a plot of a parabola appear.  You should also be able to use the
buttons at the bottom of the window, e.g click the
magnifying glass and then use the mouse to select a rectangle in the plot to
zoom in.

Alternatively, you could do:

\begin{Verbatim}[commandchars=\\\{\}]
\PYG{g+gp}{\PYGZgt{}\PYGZgt{}\PYGZgt{} }\PYG{k+kn}{from} \PYG{n+nn}{pylab} \PYG{k}{import} \PYG{o}{*}
\PYG{g+gp}{\PYGZgt{}\PYGZgt{}\PYGZgt{} }\PYG{n}{interactive}\PYG{p}{(}\PYG{k}{True}\PYG{p}{)}
\end{Verbatim}

With this approach you don't need to start every pylab function name with
pylab, e.g.:

\begin{Verbatim}[commandchars=\\\{\}]
\PYG{g+gp}{\PYGZgt{}\PYGZgt{}\PYGZgt{} }\PYG{n}{x} \PYG{o}{=} \PYG{n}{linspace}\PYG{p}{(}\PYG{o}{\PYGZhy{}}\PYG{l+m+mi}{1}\PYG{p}{,} \PYG{l+m+mi}{1}\PYG{p}{,} \PYG{l+m+mi}{20}\PYG{p}{)}
\PYG{g+gp}{\PYGZgt{}\PYGZgt{}\PYGZgt{} }\PYG{n}{plot}\PYG{p}{(}\PYG{n}{x}\PYG{p}{,} \PYG{n}{x}\PYG{o}{*}\PYG{o}{*}\PYG{l+m+mi}{2}\PYG{p}{,} \PYG{l+s}{\PYGZsq{}}\PYG{l+s}{o\PYGZhy{}}\PYG{l+s}{\PYGZsq{}}\PYG{p}{)}
\end{Verbatim}

In these notes we'll generally use module names just so it's clear where
things come from.

If you use the IPython shell, you can do:

\begin{Verbatim}[commandchars=\\\{\}]
\PYGZdl{} ipython \PYGZhy{}\PYGZhy{}pylab

In [1]: x = linspace(\PYGZhy{}1, 1, 20)
In [2]: plot(x, x**2, \PYGZsq{}o\PYGZhy{}\PYGZsq{})
\end{Verbatim}

The --pylab flag causes everything to be imported from pylab and set up for
interactive plotting.


\subsection{Mayavi and mlab}
\label{python_plotting:mayavi}\label{python_plotting:mayavi-and-mlab}
Mayavi is a Python plotting package designed primarily for 3d plots.  See:
\begin{itemize}
\item {} 
\href{http://code.enthought.com/projects/mayavi/docs/development/html/mayavi/index.html}{Documentation}

\item {} 
\href{http://code.enthought.com/projects/mayavi/docs/development/html/mayavi/auto/examples.html}{Gallery}

\end{itemize}

See {\hyperref[software_installation:software\string-installation]{\crossref{\DUrole{std,std-ref}{Downloading and installing software for this class}}}} for some ways to install Mayavi.


\subsection{VisIt}
\label{python_plotting:visit}\label{python_plotting:id1}
VisIt is an open source visualization package being developed at Lawrence Livermore
National Laboratory.  It is designed for industrial-strength visualization problems
and can deal with very large distributed data sets using MPI.

There is a GUI interface and also a Python interface for scripting.

See:
\begin{itemize}
\item {} 
\href{https://wci.llnl.gov/codes/visit/doc.html}{Documentation}

\item {} 
\href{https://wci.llnl.gov/codes/visit/gallery.html}{Gallery}

\item {} 
\href{http://www.visitusers.org/index.php?title=Short\_Tutorial}{Tutorial}

\end{itemize}


\subsection{ParaView}
\label{python_plotting:id4}\label{python_plotting:paraview}
ParaView is another open source package developed originally for work at the
National Labs.

There is a GUI interface and also a Python interface for scripting.

See:
\begin{itemize}
\item {} 
\href{http://www.paraview.org/paraview/help/documentation.html}{Documentation}

\item {} 
\href{http://www.paraview.org/paraview/project/imagegallery.php}{Gallery}

\end{itemize}


\section{Python debugging}
\label{python_debugging:python-debugging}\label{python_debugging::doc}\label{python_debugging:id1}
For some general tips on writing and debugging programs in any language, see
\DUrole{xref,std,std-ref}{debugging}.


\subsection{Python IDEs}
\label{python_debugging:python-ides}
An IDE (Integrated Development Environment) generally provides an editor and
debugger that are linked directly to the language.   See
\begin{itemize}
\item {} 
\url{http://en.wikipedia.org/wiki/Integrated\_development\_environment}.

\end{itemize}

Python has a IDE called IDLE that
provides an editor that has some debugger features.  You might want to
explore this, see
\begin{itemize}
\item {} 
\url{http://docs.python.org/library/idle.html}.

\end{itemize}

Other IDEs also provided Python interfaces, such as Eclipse.
See, e.g.,
\begin{itemize}
\item {} 
\url{http://en.wikipedia.org/wiki/Eclipse\_(software)}

\item {} 
\url{http://www.vogella.de/articles/Python/article.html}

\item {} 
\url{http://heather.cs.ucdavis.edu/~matloff/eclipse.html}

\end{itemize}

These environments generally provide an interface to \titleref{pdb}, the Python
debugger described below.


\subsection{Reloading modules}
\label{python_debugging:reloading-modules}
Note that if you are debugging a Python code by running it repeatedly in an
interactive shell, you need to make sure it is seeing the most recent
version of the code after you make editing changes.  If you run it using
\titleref{execfile} (or \titleref{run} in IPython), it should find the most recent version.

If you import it as a module, then you need to make sure you do a \titleref{reload}
as described at {\hyperref[python_scripts_modules:python\string-reload]{\crossref{\DUrole{std,std-ref}{Reloading modules}}}}.


\subsection{Print statements}
\label{python_debugging:print-statements}
Print statements can be added almost anywhere in a Python code to print
things out to the terminal window as it goes along.

You might want to put some special symbols in debugging statements to flag
them as such, which makes it easier to see what output is your debug output
and also makes it easier to find them again later to remove from the code,
e.g. you might use ``+++'' or ``DEBUG''.

As an example, suppose you are trying to better understand Python namespaces
and the difference between local and global variables.  Then this code might
be useful:

\begin{Verbatim}[commandchars=\\\{\},numbers=left,firstnumber=1,stepnumber=1]
\PYG{l+s+sd}{\PYGZdq{}\PYGZdq{}\PYGZdq{}}
\PYG{l+s+sd}{\PYGZdl{}UWHPSC/codes/python/debugdemo1a.py}

\PYG{l+s+sd}{Debugging demo using print statements}
\PYG{l+s+sd}{\PYGZdq{}\PYGZdq{}\PYGZdq{}}

\PYG{n}{x} \PYG{o}{=} \PYG{l+m+mf}{3.}
\PYG{n}{y} \PYG{o}{=} \PYG{o}{\PYGZhy{}}\PYG{l+m+mf}{22.}

\PYG{k}{def} \PYG{n+nf}{f}\PYG{p}{(}\PYG{n}{z}\PYG{p}{)}\PYG{p}{:}
    \PYG{n}{x} \PYG{o}{=} \PYG{n}{z}\PYG{o}{+}\PYG{l+m+mi}{10}
    \PYG{k}{print} \PYG{l+s}{\PYGZdq{}}\PYG{l+s}{+++ in function f: x = }\PYG{l+s+si}{\PYGZpc{}s}\PYG{l+s}{, y = }\PYG{l+s+si}{\PYGZpc{}s}\PYG{l+s}{, z = }\PYG{l+s+si}{\PYGZpc{}s}\PYG{l+s}{\PYGZdq{}} \PYG{o}{\PYGZpc{}} \PYG{p}{(}\PYG{n}{x}\PYG{p}{,}\PYG{n}{y}\PYG{p}{,}\PYG{n}{z}\PYG{p}{)}
    \PYG{k}{return} \PYG{n}{x}

\PYG{k}{print} \PYG{l+s}{\PYGZdq{}}\PYG{l+s}{+++ before calling f: x = }\PYG{l+s+si}{\PYGZpc{}s}\PYG{l+s}{, y = }\PYG{l+s+si}{\PYGZpc{}s}\PYG{l+s}{\PYGZdq{}} \PYG{o}{\PYGZpc{}} \PYG{p}{(}\PYG{n}{x}\PYG{p}{,}\PYG{n}{y}\PYG{p}{)}
\PYG{n}{y} \PYG{o}{=} \PYG{n}{f}\PYG{p}{(}\PYG{n}{x}\PYG{p}{)}
\PYG{k}{print} \PYG{l+s}{\PYGZdq{}}\PYG{l+s}{+++ after calling f: x = }\PYG{l+s+si}{\PYGZpc{}s}\PYG{l+s}{, y = }\PYG{l+s+si}{\PYGZpc{}s}\PYG{l+s}{\PYGZdq{}} \PYG{o}{\PYGZpc{}} \PYG{p}{(}\PYG{n}{x}\PYG{p}{,}\PYG{n}{y}\PYG{p}{)}

\end{Verbatim}

Here the print function in the definition of \titleref{f(x)} is being used for
debugging purposes.
Executing this code gives:

\begin{Verbatim}[commandchars=\\\{\}]
\PYG{g+gp}{\PYGZgt{}\PYGZgt{}\PYGZgt{} }\PYG{n}{execfile}\PYG{p}{(}\PYG{l+s}{\PYGZdq{}}\PYG{l+s}{debugdemo1a.py}\PYG{l+s}{\PYGZdq{}}\PYG{p}{)}
\PYG{g+go}{+++ before calling f: x = 3.0, y = \PYGZhy{}22.0}
\PYG{g+go}{+++ in function f: x = 13.0, y = \PYGZhy{}22.0, z = 3.0}
\PYG{g+go}{+++ after calling f: x = 3.0, y = 13.0}
\end{Verbatim}

If you are printing large amounts you might want to write to a file rather
than to the terminal, see \DUrole{xref,std,std-ref}{python\_io}.


\subsection{pdb debugger}
\label{python_debugging:pdb}\label{python_debugging:pdb-debugger}
Inserting print statements may work best in some situations, but it is often
better to use a \emph{debugger}.  The Python debugger pdb is very easy to use,
often even easier than inserting print statements and well worth learning.
See the \href{http://docs.python.org/2/library/pdb.html}{pdb documentation}
for more information.

You can insert \emph{breakpoints} in your code where control should pass back to
the user, at which point you can query the value of any variable, or step
through the program line by line from this point on.   For the above example
we might do this as below:

\begin{Verbatim}[commandchars=\\\{\},numbers=left,firstnumber=1,stepnumber=1]
\PYG{l+s+sd}{\PYGZdq{}\PYGZdq{}\PYGZdq{}}
\PYG{l+s+sd}{\PYGZdl{}UWHPSC/codes/python/debugdemo1b.py}

\PYG{l+s+sd}{Debugging demo using pdb.}
\PYG{l+s+sd}{\PYGZdq{}\PYGZdq{}\PYGZdq{}}

\PYG{n}{x} \PYG{o}{=} \PYG{l+m+mf}{3.}
\PYG{n}{y} \PYG{o}{=} \PYG{o}{\PYGZhy{}}\PYG{l+m+mf}{22.}

\PYG{k}{def} \PYG{n+nf}{f}\PYG{p}{(}\PYG{n}{z}\PYG{p}{)}\PYG{p}{:}
    \PYG{n}{x} \PYG{o}{=} \PYG{n}{z}\PYG{o}{+}\PYG{l+m+mi}{10}
    \PYG{k+kn}{import} \PYG{n+nn}{pdb}\PYG{p}{;} \PYG{n}{pdb}\PYG{o}{.}\PYG{n}{set\PYGZus{}trace}\PYG{p}{(}\PYG{p}{)}
    \PYG{k}{return} \PYG{n}{x}

\PYG{n}{y} \PYG{o}{=} \PYG{n}{f}\PYG{p}{(}\PYG{n}{x}\PYG{p}{)}

\PYG{k}{print} \PYG{l+s}{\PYGZdq{}}\PYG{l+s}{x = }\PYG{l+s}{\PYGZdq{}}\PYG{p}{,}\PYG{n}{x}
\PYG{k}{print} \PYG{l+s}{\PYGZdq{}}\PYG{l+s}{y = }\PYG{l+s}{\PYGZdq{}}\PYG{p}{,}\PYG{n}{y}

\end{Verbatim}

Of course one could set multiple breakpoints with other \titleref{pdb.set\_trace()}
commands.

Now we get the prompt for the \titleref{pdb} shell when we hit the breakpoint:

\begin{Verbatim}[commandchars=\\\{\}]
\PYG{g+gp}{\PYGZgt{}\PYGZgt{}\PYGZgt{} }\PYG{n}{execfile}\PYG{p}{(}\PYG{l+s}{\PYGZdq{}}\PYG{l+s}{debugdemo1b.py}\PYG{l+s}{\PYGZdq{}}\PYG{p}{)}

\PYG{g+go}{\PYGZgt{} /Users/rjl/uwhpsc/codes/python/debugdemo1b.py(11)f()}
\PYG{g+go}{\PYGZhy{}\PYGZgt{} return x}

\PYG{g+go}{(Pdb) p x}
\PYG{g+go}{13.0}
\PYG{g+go}{(Pdb) p y}
\PYG{g+go}{\PYGZhy{}22.0}
\PYG{g+go}{(Pdb) p z}
\PYG{g+go}{3.0}
\end{Verbatim}

Note that \titleref{p} is short for \titleref{print}.  You could also type \titleref{print x} but this
would then execute the Python print command instead of the debugger command
(though in this case it would print the same thing).

There are many other \titleref{pdb} commands,
such as \titleref{next} to execute the next line, \titleref{continue} to continue executing
until the next breakpoint, etc.  See the documentation for more details.

For example, lets execute the next two statements and then print \titleref{x} and \titleref{y}:

\begin{Verbatim}[commandchars=\\\{\}]
\PYG{p}{(}\PYG{n}{Pdb}\PYG{p}{)} \PYG{n}{n}
\PYG{o}{\PYGZhy{}}\PYG{o}{\PYGZhy{}}\PYG{n}{Return}\PYG{o}{\PYGZhy{}}\PYG{o}{\PYGZhy{}}
\PYG{o}{\PYGZgt{}} \PYG{o}{/}\PYG{n}{Users}\PYG{o}{/}\PYG{n}{rjl}\PYG{o}{/}\PYG{n}{uwhpsc}\PYG{o}{/}\PYG{n}{codes}\PYG{o}{/}\PYG{n}{python}\PYG{o}{/}\PYG{n}{debugdemo1b}\PYG{o}{.}\PYG{n}{py}\PYG{p}{(}\PYG{l+m+mi}{11}\PYG{p}{)}\PYG{n}{f}\PYG{p}{(}\PYG{p}{)}\PYG{o}{\PYGZhy{}}\PYG{o}{\PYGZgt{}}\PYG{l+m+mf}{13.0}
\PYG{o}{\PYGZhy{}}\PYG{o}{\PYGZgt{}} \PYG{k}{return} \PYG{n}{x}

\PYG{p}{(}\PYG{n}{Pdb}\PYG{p}{)} \PYG{n}{n}
\PYG{o}{\PYGZgt{}} \PYG{o}{/}\PYG{n}{Users}\PYG{o}{/}\PYG{n}{rjl}\PYG{o}{/}\PYG{n}{uwhpsc}\PYG{o}{/}\PYG{n}{codes}\PYG{o}{/}\PYG{n}{python}\PYG{o}{/}\PYG{n}{debugdemo1b}\PYG{o}{.}\PYG{n}{py}\PYG{p}{(}\PYG{l+m+mi}{15}\PYG{p}{)}\PYG{o}{\PYGZlt{}}\PYG{n}{module}\PYG{o}{\PYGZgt{}}\PYG{p}{(}\PYG{p}{)}
\PYG{o}{\PYGZhy{}}\PYG{o}{\PYGZgt{}} \PYG{n+nb}{print} \PYG{l+s}{\PYGZdq{}}\PYG{l+s}{x = }\PYG{l+s}{\PYGZdq{}}\PYG{p}{,}\PYG{n}{x}

\PYG{p}{(}\PYG{n}{Pdb}\PYG{p}{)} \PYG{n}{p} \PYG{n}{x}\PYG{p}{,}\PYG{n}{y}
\PYG{p}{(}\PYG{l+m+mf}{3.0}\PYG{p}{,} \PYG{l+m+mf}{13.0}\PYG{p}{)}

\PYG{p}{(}\PYG{n}{Pdb}\PYG{p}{)} \PYG{n}{p} \PYG{n}{z}
\PYG{o}{*}\PYG{o}{*}\PYG{o}{*} \PYG{n+ne}{NameError}\PYG{p}{:} \PYG{n+ne}{NameError}\PYG{p}{(}\PYG{l+s}{\PYGZdq{}}\PYG{l+s}{name }\PYG{l+s}{\PYGZsq{}}\PYG{l+s}{z}\PYG{l+s}{\PYGZsq{}}\PYG{l+s}{ is not defined}\PYG{l+s}{\PYGZdq{}}\PYG{p}{,}\PYG{p}{)}

\PYG{p}{(}\PYG{n}{Pdb}\PYG{p}{)} \PYG{n}{quit}
\PYG{o}{\PYGZgt{}\PYGZgt{}}\PYG{o}{\PYGZgt{}}
\end{Verbatim}

You can also run the code as a script from the Unix prompt and again you
will be put into the pdb shell when the breakpoint is reached:

\begin{Verbatim}[commandchars=\\\{\}]
\PYGZdl{} python debugdemo1b.py
\PYGZgt{} /Users/rjl/uwhpsc/codes/python/debugdemo1b.py(11)f()
\PYGZhy{}\PYGZgt{} return x
(Pdb) p z
3.0
(Pdb) continue
x =  3.0
y =  13.0
\end{Verbatim}


\subsection{Debugging after an exception occurs}
\label{python_debugging:debugging-after-an-exception-occurs}
Often code has bugs that cause an exception to be raised that causes the
program to halt execution. Consider the following file:

If you change N to 20 it will run fine, but with \titleref{N = 40} we find:

\begin{Verbatim}[commandchars=\\\{\}]
\PYG{g+gp}{\PYGZgt{}\PYGZgt{}\PYGZgt{} }\PYG{n}{execfile}\PYG{p}{(}\PYG{l+s}{\PYGZdq{}}\PYG{l+s}{debugdemo2.py}\PYG{l+s}{\PYGZdq{}}\PYG{p}{)}
\PYG{g+gt}{Traceback (most recent call last):}
  File \PYG{n+nb}{\PYGZdq{}\PYGZlt{}stdin\PYGZgt{}\PYGZdq{}}, line \PYG{l+m}{1}, in \PYG{n}{\PYGZlt{}module\PYGZgt{}}
  File \PYG{n+nb}{\PYGZdq{}debugdemo2.py\PYGZdq{}}, line \PYG{l+m}{14}, in \PYG{n}{\PYGZlt{}module\PYGZgt{}}
    \PYG{n}{w}\PYG{p}{[}\PYG{n}{i}\PYG{p}{]} \PYG{o}{=} \PYG{l+m+mi}{1}\PYG{o}{/}\PYG{n}{eps2}
\PYG{g+gr}{ZeroDivisionError}: \PYG{n}{float division}
\PYG{g+gt}{Traceback (most recent call last):}
  File \PYG{n+nb}{\PYGZdq{}\PYGZlt{}stdin\PYGZgt{}\PYGZdq{}}, line \PYG{l+m}{1}, in \PYG{n}{\PYGZlt{}module\PYGZgt{}}
  File \PYG{n+nb}{\PYGZdq{}debugdemo2.py\PYGZdq{}}, line \PYG{l+m}{14}, in \PYG{n}{\PYGZlt{}module\PYGZgt{}}
    \PYG{n}{w}\PYG{p}{[}\PYG{n}{i}\PYG{p}{]} \PYG{o}{=} \PYG{l+m+mi}{1}\PYG{o}{/}\PYG{n}{eps2}
\PYG{g+gr}{ZeroDivisionError}: \PYG{n}{float division}
\end{Verbatim}

At some point \titleref{eps2} apparently has the value 0.  To figure out when this
happens, we could insert a \titleref{pdb.set\_trace()} command in the loop and step
through it until the error occurs and then look at \titleref{i}, but we can do so
even more easily using a \emph{post-mortem} analysis after it dies, using
\titleref{pdb.pm()}:

\begin{Verbatim}[commandchars=\\\{\}]
\PYG{g+gp}{\PYGZgt{}\PYGZgt{}\PYGZgt{} }\PYG{k+kn}{import} \PYG{n+nn}{pdb}
\PYG{g+gp}{\PYGZgt{}\PYGZgt{}\PYGZgt{} }\PYG{n}{pdb}\PYG{o}{.}\PYG{n}{pm}\PYG{p}{(}\PYG{p}{)}
\PYG{g+go}{\PYGZgt{} /Users/rjl/uwhpsc/codes/python/debugdemo2.py(14)\PYGZlt{}module\PYGZgt{}()\PYGZhy{}\PYGZgt{}None}
\PYG{g+go}{\PYGZhy{}\PYGZgt{} w[i] = 1/eps2}
\PYG{g+go}{(Pdb) p i}
\PYG{g+go}{34}
\PYG{g+go}{(Pdb) p eps2}
\PYG{g+go}{0.0}
\PYG{g+go}{(Pdb) p epsilon}
\PYG{g+go}{5.9962169748381002e\PYGZhy{}17}
\end{Verbatim}

This starts up \titleref{pdb} at exactly the point where the exception is about to
occur.  We see that the divide by zero happens when \titleref{i = 34} (because
\titleref{epsilon} is so small that \titleref{1. + epsilon} is rounded off to \titleref{1.} in the
computer, see \DUrole{xref,std,std-ref}{floats}).


\subsection{Using pdb from IPython}
\label{python_debugging:using-pdb-from-ipython}
In IPython it's even easier to do this post-mortem analysis.  Just type:

\begin{Verbatim}[commandchars=\\\{\}]
\PYG{n}{In} \PYG{p}{[}\PYG{l+m+mi}{50}\PYG{p}{]}\PYG{p}{:} \PYG{n}{pdb}
\PYG{n}{Automatic} \PYG{n}{pdb} \PYG{n}{calling} \PYG{n}{has} \PYG{n}{been} \PYG{n}{turned} \PYG{n}{ON}
\end{Verbatim}

and then \titleref{pdb} will be automatically invoked if an exception occurs:

\begin{Verbatim}[commandchars=\\\{\}]
\PYG{n}{In} \PYG{p}{[}\PYG{l+m+mi}{51}\PYG{p}{]}\PYG{p}{:} \PYG{n}{run} \PYG{n}{debugdemo2}\PYG{o}{.}\PYG{n}{py}
\PYG{o}{\PYGZhy{}}\PYG{o}{\PYGZhy{}}\PYG{o}{\PYGZhy{}}\PYG{o}{\PYGZhy{}}\PYG{o}{\PYGZhy{}}\PYG{o}{\PYGZhy{}}\PYG{o}{\PYGZhy{}}\PYG{o}{\PYGZhy{}}\PYG{o}{\PYGZhy{}}\PYG{o}{\PYGZhy{}}\PYG{o}{\PYGZhy{}}\PYG{o}{\PYGZhy{}}\PYG{o}{\PYGZhy{}}\PYG{o}{\PYGZhy{}}\PYG{o}{\PYGZhy{}}\PYG{o}{\PYGZhy{}}\PYG{o}{\PYGZhy{}}\PYG{o}{\PYGZhy{}}\PYG{o}{\PYGZhy{}}\PYG{o}{\PYGZhy{}}\PYG{o}{\PYGZhy{}}\PYG{o}{\PYGZhy{}}\PYG{o}{\PYGZhy{}}\PYG{o}{\PYGZhy{}}\PYG{o}{\PYGZhy{}}\PYG{o}{\PYGZhy{}}\PYG{o}{\PYGZhy{}}\PYG{o}{\PYGZhy{}}\PYG{o}{\PYGZhy{}}\PYG{o}{\PYGZhy{}}\PYG{o}{\PYGZhy{}}\PYG{o}{\PYGZhy{}}\PYG{o}{\PYGZhy{}}\PYG{o}{\PYGZhy{}}\PYG{o}{\PYGZhy{}}\PYG{o}{\PYGZhy{}}\PYG{o}{\PYGZhy{}}\PYG{o}{\PYGZhy{}}\PYG{o}{\PYGZhy{}}\PYG{o}{\PYGZhy{}}\PYG{o}{\PYGZhy{}}\PYG{o}{\PYGZhy{}}\PYG{o}{\PYGZhy{}}\PYG{o}{\PYGZhy{}}\PYG{o}{\PYGZhy{}}\PYG{o}{\PYGZhy{}}\PYG{o}{\PYGZhy{}}\PYG{o}{\PYGZhy{}}\PYG{o}{\PYGZhy{}}\PYG{o}{\PYGZhy{}}\PYG{o}{\PYGZhy{}}\PYG{o}{\PYGZhy{}}\PYG{o}{\PYGZhy{}}\PYG{o}{\PYGZhy{}}\PYG{o}{\PYGZhy{}}\PYG{o}{\PYGZhy{}}\PYG{o}{\PYGZhy{}}\PYG{o}{\PYGZhy{}}\PYG{o}{\PYGZhy{}}\PYG{o}{\PYGZhy{}}\PYG{o}{\PYGZhy{}}\PYG{o}{\PYGZhy{}}\PYG{o}{\PYGZhy{}}\PYG{o}{\PYGZhy{}}\PYG{o}{\PYGZhy{}}\PYG{o}{\PYGZhy{}}\PYG{o}{\PYGZhy{}}\PYG{o}{\PYGZhy{}}\PYG{o}{\PYGZhy{}}\PYG{o}{\PYGZhy{}}\PYG{o}{\PYGZhy{}}\PYG{o}{\PYGZhy{}}\PYG{o}{\PYGZhy{}}\PYG{o}{\PYGZhy{}}\PYG{o}{\PYGZhy{}}

\PYG{n+ne}{ZeroDivisionError}\PYG{p}{:} \PYG{n+nb}{float} \PYG{n}{division}
\PYG{o}{\PYGZgt{}} \PYG{o}{/}\PYG{n}{Users}\PYG{o}{/}\PYG{n}{rjl}\PYG{o}{/}\PYG{n}{uwhpsc}\PYG{o}{/}\PYG{n}{codes}\PYG{o}{/}\PYG{n}{python}\PYG{o}{/}\PYG{n}{debugdemo2}\PYG{o}{.}\PYG{n}{py}\PYG{p}{(}\PYG{l+m+mi}{14}\PYG{p}{)}\PYG{o}{\PYGZlt{}}\PYG{n}{module}\PYG{o}{\PYGZgt{}}\PYG{p}{(}\PYG{p}{)}
     \PYG{l+m+mi}{13}     \PYG{n}{eps2} \PYG{o}{=} \PYG{n}{z} \PYG{o}{\PYGZhy{}} \PYG{l+m+mf}{1.}         \PYG{c}{\PYGZsh{} expect eps2 == epsilon?}
\PYG{o}{\PYGZhy{}}\PYG{o}{\PYGZhy{}}\PYG{o}{\PYGZhy{}}\PYG{o}{\PYGZgt{}} \PYG{l+m+mi}{14}     \PYG{n}{w}\PYG{p}{[}\PYG{n}{i}\PYG{p}{]} \PYG{o}{=} \PYG{l+m+mi}{1}\PYG{o}{/}\PYG{n}{eps2}
     \PYG{l+m+mi}{15}

\PYG{n}{ipdb}\PYG{o}{\PYGZgt{}} \PYG{n}{p} \PYG{n}{i}
\PYG{l+m+mi}{34}
\PYG{n}{ipdb}\PYG{o}{\PYGZgt{}} \PYG{n}{q}

\PYG{n}{In} \PYG{p}{[}\PYG{l+m+mi}{52}\PYG{p}{]}\PYG{p}{:}
\end{Verbatim}

Type \titleref{pdb} again to turn it off.

Note: \titleref{pdb}, like \titleref{run} is a \titleref{magic function} in IPython, an extension of
the language itself, type \titleref{magic} at the IPython prompt for more info.

If these commands don't work, type \titleref{\%magic} and read this documentation.


\subsection{Other pdb commands}
\label{python_debugging:other-pdb-commands}
There are a number of other commands, see the references above.


\section{Animation in Python}
\label{animation:animation}\label{animation::doc}\label{animation:animation-in-python}
\titleref{matplotlib} has a package \titleref{animation} that can be used directly,
see for example
\url{http://matplotlib.org/examples/animation/dynamic\_image2.html}
or \href{http://jakevdp.github.io/blog/2012/08/18/matplotlib-animation-tutorial/}{this blog post}.

Nicer webpage animations (within IPython notebooks or as stand-alone movies)
can be created using the package \titleref{JSAnimation} created by
Jake Vanderplas.  For an example see this \href{http://nbviewer.ipython.org/github/jakevdp/JSAnimation/blob/master/animation\_example.ipynb}{rendered example}
or {\hyperref[labs/lab15:lab15]{\crossref{\DUrole{std,std-ref}{Lab 15: Tuesday May 20, 2014}}}}.


\section{Installing JSAnimation}
\label{animation:installing-jsanimation}
First clone it from Github:

\begin{Verbatim}[commandchars=\\\{\}]
\PYGZdl{} cd \PYGZdl{}HOME
\PYGZdl{} git clone https://github.com/jakevdp/JSAnimation.git
\PYGZdl{} cd JSAnimation
\end{Verbatim}

On your own laptop or the VM, you can probably install it via:

\begin{Verbatim}[commandchars=\\\{\}]
\PYGZdl{} python setup.py install
\end{Verbatim}

On SageMathCloud you don't have access to the system folder where it
normally installs Python packages, but you can install it for use in
one project only via:

\begin{Verbatim}[commandchars=\\\{\}]
\PYGZdl{} sage \PYGZhy{}python setup.py install \PYGZhy{}\PYGZhy{}user
\end{Verbatim}

Then you should be able to open Python or IPython and \titleref{import JSAnimation}


\subsection{Alternative to installing}
\label{animation:alternative-to-installing}
Rather than installing it as a package,
you can just add the JSAnimation directory to
the Python search path.  For running it from scripts in a bash shell,
you might want to add this line to your
Or you can just add \$HOME/JSAnimation to your PYTHONPATH environment
variable:

\begin{Verbatim}[commandchars=\\\{\}]
\PYGZdl{} export PYTHONPATH=\PYGZdl{}PYTHONPATH:\PYGZdl{}HOME/JSAnimation
\end{Verbatim}

This appends the path to the end of whatever path is already specified in
this environment variable.

As a last resort, you can also modify the path from within a Python
session:

\begin{Verbatim}[commandchars=\\\{\}]
\PYG{g+gp}{\PYGZgt{}\PYGZgt{}\PYGZgt{} }\PYG{k+kn}{import} \PYG{n+nn}{os}\PYG{o}{,} \PYG{n+nn}{sys}
\PYG{g+gp}{\PYGZgt{}\PYGZgt{}\PYGZgt{} }\PYG{n}{HOME} \PYG{o}{=} \PYG{n}{os}\PYG{o}{.}\PYG{n}{environ}\PYG{p}{[}\PYG{l+s}{\PYGZsq{}}\PYG{l+s}{HOME}\PYG{l+s}{\PYGZsq{}}\PYG{p}{]}
\PYG{g+gp}{\PYGZgt{}\PYGZgt{}\PYGZgt{} }\PYG{n}{JSAnimation\PYGZus{}path} \PYG{o}{=} \PYG{n}{os}\PYG{o}{.}\PYG{n}{path}\PYG{o}{.}\PYG{n}{join}\PYG{p}{(}\PYG{n}{HOME}\PYG{p}{,} \PYG{l+s}{\PYGZsq{}}\PYG{l+s}{JSAnimation}\PYG{l+s}{\PYGZsq{}}\PYG{p}{)}
\PYG{g+gp}{\PYGZgt{}\PYGZgt{}\PYGZgt{} }\PYG{n}{sys}\PYG{o}{.}\PYG{n}{path}\PYG{o}{.}\PYG{n}{append}\PYG{p}{(}\PYG{n}{JSAnimation\PYGZus{}path}\PYG{p}{)}
\end{Verbatim}


\subsection{JSAnimation\_frametools}
\label{animation:jsanimation-frametools}
For animations of complex plots, it is sometimes easier to simply
create the plot for each frame as usual using \titleref{matplotlib} and then
save a \titleref{.png} file for each frame.  These can be created and then
combined to create an animation using some tools in
\titleref{JSAnimation\_frametools.py}, currently found in {\hyperref[labs/lab15:lab15]{\crossref{\DUrole{std,std-ref}{Lab 15: Tuesday May 20, 2014}}}} along
with some demos.


\subsection{Matplotlib issues}
\label{animation:matplotlib-issues}
JSAnimation requires a recent version of \titleref{matplotlib}.
(In particular, older Ubuntu versions may not have a recent version.)
If you're having problems with \titleref{matplotlib} in this context, you might
want to try using the \href{https://store.continuum.io/cshop/anaconda}{Anaconda Python distribution}, or switch to \DUrole{xref,std,std-ref}{smc}.


\chapter{Fortran}
\label{index:toc-fortran}\label{index:fortran}

\section{Fortran}
\label{fortran:fortran}\label{fortran::doc}\label{fortran:id1}

\subsection{General References:}
\label{fortran:general-references}\begin{itemize}
\item {} 
See the {\hyperref[biblio:biblio\string-fortran]{\crossref{\DUrole{std,std-ref}{Fortran references}}}} section of the bibliography for links.

\end{itemize}


\subsection{History}
\label{fortran:history}
FORTRAN stands for \emph{FORmula TRANslator} and was the first major \emph{high
level language} to catch on.  The first compiler was written in
1954-57.  Before this, programmers generally had to write programs in
assembly language.

Many version followed: Fortran II, III, IV. Fortran 66
followed a set of standards formulated in 1966.

See
\begin{itemize}
\item {} 
\url{http://www.ibiblio.org/pub/languages/fortran/ch1-1.html}

\item {} 
\url{http://en.wikipedia.org/wiki/Fortran}

\end{itemize}

for brief histories.


\subsection{Fortran 77}
\label{fortran:fortran-77}
The standards established in 1977 lead to Fortran 77, or f77, and
many codes are still in use that follow this standard.

Fortran 77 does not have all the features of newer versions and many
things are done quite differently.

One feature of f77 is that lines of code have a very rigid structure.
This was required in early versions of Fortran due to the fact that
computer programs were written on {\hyperref[punchcard:punchcard]{\crossref{\DUrole{std,std-ref}{Punch cards}}}}.  All statements
must start in column 7 or beyond and no statement may extend beyond
column 72. The first 6 columns are used for things like labels
(numbers associated with particular statements).  In f77 any line
that starts with a `c' in column 1 is a comment.

We will not use f77 in this class but if you need to work with
Fortran in the future you may need to learn more about it because of
all the \emph{legacy codes} that still use it.


\subsection{Fortran 90/95}
\label{fortran:fortran-90-95}
Dramatically new standards were introduced with Fortran 90, and these
were improved in mostly minor ways in Fortran 95.  There are newer
Fortran 2003 and 2008 standards but few compilers implement these fully yet.
See \href{http://gcc.gnu.org/wiki/GFortranStandards}{Wikipedia page on Fortran standards}
for more information.

For this class we will use the Fortran 90/95 standards, which we will
refer to as Fortran 90 for brevity.


\subsection{Compilers}
\label{fortran:fortran-compilers}\label{fortran:compilers}
Unlike Python code, a Fortran program must pass through several
stages before being executed.  There are several different compilers
that can turn Fortran code into an \emph{executable}, as described more
below.

In this class we will use \emph{gfortran}, which is an open source
compiler, part of the \href{http://www.gnu.org/}{GNU Project}.
See \url{http://gcc.gnu.org/fortran/} for more about gfortran.

There is an older compiler in this suite called \emph{g77} which
compiles Fortran 77 code, but \emph{gfortran} can also be used for Fortran
77 code and has replaced g77.

There are several commercial compilers which are better in some ways,
in particular they sometimes do better optimization and produce
faster running executables.  They also may have better debugging
tools built in.  Some popular ones are the Intel and Portland Group
compilers.


\subsection{File extensions}
\label{fortran:file-extensions}
For the gfortran compiler, fixed format code should have the
\emph{.f} while free format code has the extension \emph{.f90} or \emph{.f95}.  We
will use \emph{.f90}.


\subsection{Compiling, linking, and running a Fortran code}
\label{fortran:fortran-compiling}\label{fortran:compiling-linking-and-running-a-fortran-code}
Suppose we have a Fortran file named \titleref{demo1.f90}, for example the
program below.  We can not run this directly the way we did a Python
script.  Instead it must be converted into \emph{object code}, a version
of the code that is in a machine language specific to the type of
computer.  This is done by the \emph{compiler}.

Then a \emph{linker} must be used to convert the object code into an
\emph{executable} that can actually be executed.

This is broken into two steps because often large programs are split
into many different \emph{.f90} files.  Each one can be compiled into a
separate \emph{object file}, which by default has the same name but with a
\emph{.o} extension (for example, from \titleref{demo1.f90} the compiler would
produce \titleref{demo1.o}).  One may also want to call on \emph{library routines} that
have already been compiled and reside in some library.  The linker
combines all of these into a single executable.

For more details on the process, see for example:
\begin{itemize}
\item {} 
\url{http://en.wikipedia.org/wiki/Compiler}

\item {} 
\url{http://en.wikipedia.org/wiki/Linker\_\%28computing\%29}

\end{itemize}

For the simplest case of a self-contained program in one file, we can
combine both stages in a single \titleref{gfortran} command, e.g.

\begin{Verbatim}[commandchars=\\\{\}]
\PYGZdl{} gfortran demo1.f90
\end{Verbatim}

By default this will produce an \emph{executable} named \titleref{a.out} for
obscure historical reasons (it stands for \emph{assembler output},
see \href{http://en.wikipedia.org/wiki/A.out}{wikipedia}).

To run the code you would then type:

\begin{Verbatim}[commandchars=\\\{\}]
\PYGZdl{} ./a.out
\end{Verbatim}

Note we type \titleref{./a.out} to indicate that we are executing \titleref{a.out} from
the current directory.  There is an environment variable \titleref{PATH} that
contains your \emph{search path}, the set of directories that are searched
whenever you type a command name at the Unix prompt.  Often this is
set so that the current directory is the first place searched, in
which case you could just type \titleref{a.out} instead of \titleref{./a.out}.
However, it is generally considered bad practice to include the
current directory in your search path because bad things can happen
if you accidentally execute a file.

If you don't like the name \titleref{a.out} you can specify an output name
using the \titleref{-o} flag with the \titleref{gfortran} command.  For example,
if you like the Windows convention of using the extension \titleref{.exe} for
executable files:

\begin{Verbatim}[commandchars=\\\{\}]
\PYGZdl{} gfortran demo1.f90 \PYGZhy{}o demo1.exe
\PYGZdl{} ./demo1.exe
\end{Verbatim}

will also run the code.

Note that if you try one of the above commands, there will be no file
\titleref{demo1.o} created.  By default \titleref{gfortran} removes this file once the
executed is created.

Later we will see that it is often useful to split up the compile and
link steps, particularly if there are several files that need to be
compiled and linked.  We can do this using the \titleref{-c} flag to compile
without linking:

\begin{Verbatim}[commandchars=\\\{\}]
\PYGZdl{} gfortran \PYGZhy{}c demo1.f90              \PYGZsh{} produces demo1.o
\PYGZdl{} gfortran demo1.o \PYGZhy{}o demo1.exe      \PYGZsh{} produces demo1.exe
\end{Verbatim}

There are many other compiler flags that can be used, see
\href{http://linux.die.net/man/1/gfortran}{linux man page for gfortran} for a list.


\subsection{Sample codes}
\label{fortran:fortran-ex1}\label{fortran:sample-codes}
The first example simply assigns some numbers to variables and then
prints them out.   The comments below the code explain some features.

\begin{Verbatim}[commandchars=\\\{\},numbers=left,firstnumber=1,stepnumber=1]
\PYG{c}{! \PYGZdl{}UWHPSC/codes/fortran/demo1.f90}

\PYG{k}{program }\PYG{n}{demo1}

\PYG{c}{! Fortran 90 program illustrating data types.}

\PYG{k}{implicit }\PYG{k}{none}   \PYG{c}{! to give error if a variable not declared}
\PYG{k+kt}{real} \PYG{k+kd}{::} \PYG{n}{x}
\PYG{k+kt}{real} \PYG{p}{(}\PYG{n+nb}{kind}\PYG{o}{=}\PYG{l+m+mi}{8}\PYG{p}{)} \PYG{k+kd}{::} \PYG{n}{y}\PYG{p}{,} \PYG{n}{z}
\PYG{k+kt}{integer} \PYG{k+kd}{::} \PYG{n}{m}

\PYG{n}{m} \PYG{o}{=} \PYG{l+m+mi}{3}
\PYG{k}{print} \PYG{o}{*}\PYG{p}{,} \PYG{l+s+s2}{\PYGZdq{} \PYGZdq{}}
\PYG{k}{print} \PYG{o}{*}\PYG{p}{,} \PYG{l+s+s2}{\PYGZdq{}M = \PYGZdq{}}\PYG{p}{,}\PYG{n}{M}   \PYG{c}{! note that M==m  (case insensitive)}


\PYG{k}{print} \PYG{o}{*}\PYG{p}{,} \PYG{l+s+s2}{\PYGZdq{} \PYGZdq{}}
\PYG{k}{print} \PYG{o}{*}\PYG{p}{,} \PYG{l+s+s2}{\PYGZdq{}x is real (kind=4)\PYGZdq{}}
\PYG{n}{x} \PYG{o}{=} \PYG{l+m+mf}{1.e0} \PYG{o}{+} \PYG{l+m+mf}{1.23456789e\PYGZhy{}6}
\PYG{k}{print} \PYG{o}{*}\PYG{p}{,} \PYG{l+s+s2}{\PYGZdq{}x = \PYGZdq{}}\PYG{p}{,} \PYG{n}{x}


\PYG{k}{print} \PYG{o}{*}\PYG{p}{,} \PYG{l+s+s2}{\PYGZdq{} \PYGZdq{}}
\PYG{k}{print} \PYG{o}{*}\PYG{p}{,} \PYG{l+s+s2}{\PYGZdq{}y is real (kind=8)\PYGZdq{}}
\PYG{k}{print} \PYG{o}{*}\PYG{p}{,} \PYG{l+s+s2}{\PYGZdq{}  but 1.e0 is real (kind=4):\PYGZdq{}}
\PYG{n}{y} \PYG{o}{=} \PYG{l+m+mf}{1.e0} \PYG{o}{+} \PYG{l+m+mf}{1.23456789e\PYGZhy{}6}
\PYG{k}{print} \PYG{o}{*}\PYG{p}{,} \PYG{l+s+s2}{\PYGZdq{}y = \PYGZdq{}}\PYG{p}{,} \PYG{n}{y}


\PYG{k}{print} \PYG{o}{*}\PYG{p}{,} \PYG{l+s+s2}{\PYGZdq{} \PYGZdq{}}
\PYG{k}{print} \PYG{o}{*}\PYG{p}{,} \PYG{l+s+s2}{\PYGZdq{}z is real (kind=8)\PYGZdq{}}
\PYG{n}{z} \PYG{o}{=} \PYG{l+m+mf}{1.} \PYG{o}{+} \PYG{l+m+mf}{1.23456789}\PYG{n}{d}\PYG{o}{\PYGZhy{}}\PYG{l+m+mi}{6}
\PYG{k}{print} \PYG{o}{*}\PYG{p}{,} \PYG{l+s+s2}{\PYGZdq{}z = \PYGZdq{}}\PYG{p}{,} \PYG{n}{z}

\PYG{k}{end }\PYG{k}{program }\PYG{n}{demo1}
\end{Verbatim}

\emph{Comments:}
\begin{itemize}
\item {} 
Exclamation points are used for comments

\item {} 
The \titleref{implicit none} statement in line 7 means that any variable
to be used must be explicitly declared.  See
\DUrole{xref,std,std-ref}{fortran\_implicit} for more about this.

\item {} 
Lines 8-10 declare four variables \titleref{x, y, z, n}.   Note that \titleref{x} is declared to
have type \titleref{real} which is a floating point number stored in 4
bytes, also known as \emph{single precision}.  This could have
equivalently been written as:

\begin{Verbatim}[commandchars=\\\{\}]
\PYG{n}{real} \PYG{p}{(}\PYG{n}{kind}\PYG{o}{=}\PYG{l+m+mi}{4}\PYG{p}{)} \PYG{p}{:}\PYG{p}{:} \PYG{n}{x}
\end{Verbatim}

\titleref{y} and \titleref{z} are floating point numbers stored in 8 bytes
(corresponding to \emph{double precision} in older versions of
Fortran).  This is generally what you want to use.

\item {} 
Fortran is not case-sensitive, so \titleref{M} and \titleref{m} refer to the same
variable!!

\item {} 
\titleref{1.23456789e-10} specifies a 4-byte real number.  The 8-byte
equivalent is \titleref{1.23456789d-10}, with a \titleref{d} instead of \titleref{e}.
This is apparent from the output below.

\end{itemize}

Compiling and running this program produces:

\begin{Verbatim}[commandchars=\\\{\}]
\PYGZdl{} gfortran demo1.f90 \PYGZhy{}o demo1.exe
\PYGZdl{} ./demo1.exe

 M =            3

 x is real (kind=4)
 x =    1.000001

 y is real (kind=8)
   but 1.e0 is real (kind=4):
 y =    1.00000119209290

 z is real (kind=8)
 z =    1.00000123456789
\end{Verbatim}

For most of what we'll do in this class, we will use real numbers
with \titleref{(kind=8)}.  Be careful to specify constants using the \titleref{d}
rather than \titleref{e} notation if you need to use scientific notation.

(But see {\hyperref[fortran:fortran\string-default8]{\crossref{\DUrole{std,std-ref}{Default 8-byte real numbers}}}} below for another approach.)


\subsection{Intrinsic functions}
\label{fortran:fortran-intrinsic}\label{fortran:intrinsic-functions}
There are a number of built-in functions that you can use in Fortran,
for example the trig functions:

\begin{Verbatim}[commandchars=\\\{\},numbers=left,firstnumber=1,stepnumber=1]
\PYG{c}{! \PYGZdl{}UWHPSC/codes/fortran/builtinfcns.f90}

\PYG{k}{program }\PYG{n}{builtinfcns}

    \PYG{k}{implicit }\PYG{k}{none}
\PYG{k}{    }\PYG{k+kt}{real} \PYG{p}{(}\PYG{n+nb}{kind}\PYG{o}{=}\PYG{l+m+mi}{8}\PYG{p}{)} \PYG{k+kd}{::} \PYG{n}{pi}\PYG{p}{,} \PYG{n}{x}\PYG{p}{,} \PYG{n}{y}

    \PYG{c}{! compute pi as arc\PYGZhy{}cosine of \PYGZhy{}1:}
    \PYG{n}{pi} \PYG{o}{=} \PYG{n+nb}{acos}\PYG{p}{(}\PYG{o}{\PYGZhy{}}\PYG{l+m+mf}{1.}\PYG{n}{d0}\PYG{p}{)}  \PYG{c}{! need \PYGZhy{}1.d0 for full precision!}

    \PYG{n}{x} \PYG{o}{=} \PYG{n+nb}{cos}\PYG{p}{(}\PYG{n}{pi}\PYG{p}{)}
    \PYG{n}{y} \PYG{o}{=} \PYG{n+nb}{sqrt}\PYG{p}{(}\PYG{n+nb}{exp}\PYG{p}{(}\PYG{n+nb}{log}\PYG{p}{(}\PYG{n}{pi}\PYG{p}{)}\PYG{p}{)}\PYG{p}{)}\PYG{o}{**}\PYG{l+m+mi}{2}

    \PYG{k}{print} \PYG{o}{*}\PYG{p}{,} \PYG{l+s+s2}{\PYGZdq{}pi = \PYGZdq{}}\PYG{p}{,} \PYG{n}{pi}
    \PYG{k}{print} \PYG{o}{*}\PYG{p}{,} \PYG{l+s+s2}{\PYGZdq{}x = \PYGZdq{}}\PYG{p}{,} \PYG{n}{x}
    \PYG{k}{print} \PYG{o}{*}\PYG{p}{,} \PYG{l+s+s2}{\PYGZdq{}y = \PYGZdq{}}\PYG{p}{,} \PYG{n}{y}

\PYG{k}{end }\PYG{k}{program }\PYG{n}{builtinfcns}
\end{Verbatim}

This produces:

\begin{Verbatim}[commandchars=\\\{\}]
\PYGZdl{} gfortran builtinfcns.f90
\PYGZdl{} ./a.out
 pi =    3.14159265358979
 x =   \PYGZhy{}1.00000000000000
 y =    3.14159265358979
\end{Verbatim}

See \url{http://www.nsc.liu.se/~boein/f77to90/a5.html} for a good list
of other intrinsic functions.


\subsection{Default 8-byte real numbers}
\label{fortran:default-8-byte-real-numbers}\label{fortran:fortran-default8}
Note that you can declare variables to be real without appending \titleref{(kind=8)}
if you compile programs with the gfortran flag \titleref{-fdefault-real-8}, e.g.
if we modify the program above to:

\begin{Verbatim}[commandchars=\\\{\},numbers=left,firstnumber=1,stepnumber=1]
\PYG{c}{! \PYGZdl{}UWHPSC/codes/fortran/builtinfcns2.f90}

\PYG{k}{program }\PYG{n}{builtinfcns}

    \PYG{k}{implicit }\PYG{k}{none}
\PYG{k}{    }\PYG{k+kt}{real} \PYG{k+kd}{::} \PYG{n}{pi}\PYG{p}{,} \PYG{n}{x}\PYG{p}{,} \PYG{n}{y}    \PYG{c}{! note kind is not specified}

    \PYG{c}{! compute pi as arc\PYGZhy{}cosine of \PYGZhy{}1:}
    \PYG{n}{pi} \PYG{o}{=} \PYG{n+nb}{acos}\PYG{p}{(}\PYG{o}{\PYGZhy{}}\PYG{l+m+mf}{1.0}\PYG{p}{)}

    \PYG{n}{x} \PYG{o}{=} \PYG{n+nb}{cos}\PYG{p}{(}\PYG{n}{pi}\PYG{p}{)}
    \PYG{n}{y} \PYG{o}{=} \PYG{n+nb}{sqrt}\PYG{p}{(}\PYG{n+nb}{exp}\PYG{p}{(}\PYG{n+nb}{log}\PYG{p}{(}\PYG{n}{pi}\PYG{p}{)}\PYG{p}{)}\PYG{p}{)}\PYG{o}{**}\PYG{l+m+mi}{2}

    \PYG{k}{print} \PYG{o}{*}\PYG{p}{,} \PYG{l+s+s2}{\PYGZdq{}pi = \PYGZdq{}}\PYG{p}{,} \PYG{n}{pi}
    \PYG{k}{print} \PYG{o}{*}\PYG{p}{,} \PYG{l+s+s2}{\PYGZdq{}x = \PYGZdq{}}\PYG{p}{,} \PYG{n}{x}
    \PYG{k}{print} \PYG{o}{*}\PYG{p}{,} \PYG{l+s+s2}{\PYGZdq{}y = \PYGZdq{}}\PYG{p}{,} \PYG{n}{y}

\PYG{k}{end }\PYG{k}{program }\PYG{n}{builtinfcns}
\end{Verbatim}

Then:

\begin{Verbatim}[commandchars=\\\{\}]
\PYGZdl{} gfortran builtinfcns2.f90
\PYGZdl{} ./a.out
 pi =    3.141593
 x =   \PYGZhy{}1.000000
 y =    3.141593
\end{Verbatim}

gives single precision results, but we can obtain double precisions with:

\begin{Verbatim}[commandchars=\\\{\}]
\PYGZdl{} gfortran \PYGZhy{}fdefault\PYGZhy{}real\PYGZhy{}8 builtinfcns2.f90
\PYGZdl{} ./a.out
 pi =    3.14159265358979
 x =   \PYGZhy{}1.00000000000000
 y =    3.14159265358979
\end{Verbatim}

Note that if you plan to do this you might want to define a Unix alias, e.g.

\begin{Verbatim}[commandchars=\\\{\}]
\PYGZdl{} alias gfort=\PYGZdq{}gfortran \PYGZhy{}fdefault\PYGZhy{}real\PYGZhy{}8\PYGZdq{}
\end{Verbatim}

so you can just type:

\begin{Verbatim}[commandchars=\\\{\}]
\PYGZdl{} gfort builtinfcns2.f90
\PYGZdl{} ./a.out
 pi =    3.14159265358979
 x =   \PYGZhy{}1.00000000000000
 y =    3.14159265358979
\end{Verbatim}

Such an alias could be put in your {\hyperref[unix:bashrc]{\crossref{\DUrole{std,std-ref}{.bashrc file}}}}.

We'll also see how to specify compiler flags easily in a \DUrole{xref,std,std-ref}{makefile}.


\subsection{Fortran Arrays}
\label{fortran:fortran-arrays}\label{fortran:id2}
Note that arrays are indexed starting at 1 by default, rather than 0 as in
Python.  Also note that components of an array are accessed using
parentheses, not square brackets!

Arrays can be dimensioned and used as in the following example:

\begin{Verbatim}[commandchars=\\\{\},numbers=left,firstnumber=1,stepnumber=1]
\PYG{c}{! \PYGZdl{}UWHPSC/codes/fortran/array1}

\PYG{k}{program }\PYG{n}{array1}

    \PYG{c}{! demonstrate declaring and using arrays}

    \PYG{k}{implicit }\PYG{k}{none}
\PYG{k}{    }\PYG{k+kt}{integer}\PYG{p}{,} \PYG{k}{parameter} \PYG{k+kd}{::} \PYG{n}{m} \PYG{o}{=} \PYG{l+m+mi}{3}\PYG{p}{,} \PYG{n}{n}\PYG{o}{=}\PYG{l+m+mi}{2}
    \PYG{k+kt}{real} \PYG{p}{(}\PYG{n+nb}{kind}\PYG{o}{=}\PYG{l+m+mi}{8}\PYG{p}{)}\PYG{p}{,} \PYG{k}{dimension}\PYG{p}{(}\PYG{n}{m}\PYG{p}{,}\PYG{n}{n}\PYG{p}{)} \PYG{k+kd}{::} \PYG{n}{A} 
    \PYG{k+kt}{real} \PYG{p}{(}\PYG{n+nb}{kind}\PYG{o}{=}\PYG{l+m+mi}{8}\PYG{p}{)}\PYG{p}{,} \PYG{k}{dimension}\PYG{p}{(}\PYG{n}{m}\PYG{p}{)} \PYG{k+kd}{::} \PYG{n}{b} 
    \PYG{k+kt}{real} \PYG{p}{(}\PYG{n+nb}{kind}\PYG{o}{=}\PYG{l+m+mi}{8}\PYG{p}{)}\PYG{p}{,} \PYG{k}{dimension}\PYG{p}{(}\PYG{n}{n}\PYG{p}{)} \PYG{k+kd}{::} \PYG{n}{x} 
    \PYG{k+kt}{integer} \PYG{k+kd}{::} \PYG{n}{i}\PYG{p}{,}\PYG{n}{j}

    \PYG{c}{! initialize matrix A and vector x:}
    \PYG{k}{do }\PYG{n}{j}\PYG{o}{=}\PYG{l+m+mi}{1}\PYG{p}{,}\PYG{n}{n}
        \PYG{k}{do }\PYG{n}{i}\PYG{o}{=}\PYG{l+m+mi}{1}\PYG{p}{,}\PYG{n}{m}
            \PYG{n}{A}\PYG{p}{(}\PYG{n}{i}\PYG{p}{,}\PYG{n}{j}\PYG{p}{)} \PYG{o}{=} \PYG{n}{i}\PYG{o}{+}\PYG{n}{j}
            \PYG{n}{enddo}
        \PYG{n}{x}\PYG{p}{(}\PYG{n}{j}\PYG{p}{)} \PYG{o}{=} \PYG{l+m+mf}{1.}
        \PYG{n}{enddo}

    \PYG{c}{! multiply A*x to get b:}
    \PYG{k}{do }\PYG{n}{i}\PYG{o}{=}\PYG{l+m+mi}{1}\PYG{p}{,}\PYG{n}{m}
        \PYG{n}{b}\PYG{p}{(}\PYG{n}{i}\PYG{p}{)} \PYG{o}{=} \PYG{l+m+mf}{0.}
        \PYG{k}{do }\PYG{n}{j}\PYG{o}{=}\PYG{l+m+mi}{1}\PYG{p}{,}\PYG{n}{n}
            \PYG{n}{b}\PYG{p}{(}\PYG{n}{i}\PYG{p}{)} \PYG{o}{=} \PYG{n}{b}\PYG{p}{(}\PYG{n}{i}\PYG{p}{)} \PYG{o}{+} \PYG{n}{A}\PYG{p}{(}\PYG{n}{i}\PYG{p}{,}\PYG{n}{j}\PYG{p}{)}\PYG{o}{*}\PYG{n}{x}\PYG{p}{(}\PYG{n}{j}\PYG{p}{)}
            \PYG{n}{enddo}
        \PYG{n}{enddo}

    \PYG{k}{print} \PYG{o}{*}\PYG{p}{,} \PYG{l+s+s2}{\PYGZdq{}A = \PYGZdq{}}
    \PYG{k}{do }\PYG{n}{i}\PYG{o}{=}\PYG{l+m+mi}{1}\PYG{p}{,}\PYG{n}{m}
        \PYG{k}{print} \PYG{o}{*}\PYG{p}{,} \PYG{n}{A}\PYG{p}{(}\PYG{n}{i}\PYG{p}{,}\PYG{p}{:}\PYG{p}{)}   \PYG{c}{! i\PYGZsq{}th row of A}
        \PYG{n}{enddo}
    \PYG{k}{print} \PYG{l+s+s2}{\PYGZdq{}(2d16.6)\PYGZdq{}}\PYG{p}{,} \PYG{p}{(}\PYG{p}{(}\PYG{n}{A}\PYG{p}{(}\PYG{n}{i}\PYG{p}{,}\PYG{n}{j}\PYG{p}{)}\PYG{p}{,} \PYG{n}{j}\PYG{o}{=}\PYG{l+m+mi}{1}\PYG{p}{,}\PYG{l+m+mi}{2}\PYG{p}{)}\PYG{p}{,} \PYG{n}{i}\PYG{o}{=}\PYG{l+m+mi}{1}\PYG{p}{,}\PYG{l+m+mi}{3}\PYG{p}{)}
    \PYG{k}{print} \PYG{o}{*}\PYG{p}{,} \PYG{l+s+s2}{\PYGZdq{}x = \PYGZdq{}}
    \PYG{k}{print} \PYG{l+s+s2}{\PYGZdq{}(d16.6)\PYGZdq{}}\PYG{p}{,} \PYG{n}{x}
    \PYG{k}{print} \PYG{o}{*}\PYG{p}{,} \PYG{l+s+s2}{\PYGZdq{}b = \PYGZdq{}}
    \PYG{k}{print} \PYG{l+s+s2}{\PYGZdq{}(d16.6)\PYGZdq{}}\PYG{p}{,} \PYG{n}{b}

\PYG{k}{end }\PYG{k}{program }\PYG{n}{array1}
\end{Verbatim}

Compiling and running this code gives the output:

\begin{Verbatim}[commandchars=\\\{\}]
\PYG{n}{A} \PYG{o}{=}
  \PYG{l+m+mf}{2.00000000000000}        \PYG{l+m+mf}{3.00000000000000}
  \PYG{l+m+mf}{3.00000000000000}        \PYG{l+m+mf}{4.00000000000000}
  \PYG{l+m+mf}{4.00000000000000}        \PYG{l+m+mf}{5.00000000000000}
\PYG{n}{x} \PYG{o}{=}
  \PYG{l+m+mf}{1.00000000000000}        \PYG{l+m+mf}{1.00000000000000}
\PYG{n}{b} \PYG{o}{=}
  \PYG{l+m+mf}{5.00000000000000}        \PYG{l+m+mf}{7.00000000000000}        \PYG{l+m+mf}{9.00000000000000}
\end{Verbatim}

\emph{Comments:}
\begin{itemize}
\item {} 
In printing \titleref{A} we have used a \emph{slice} operation: \titleref{A(i,:)}
refers to the i'th row of \titleref{A}.  In Fortran 90 there are many other
array operations that can be done more easily than we have done in
the loops above.  We will investigate this further later.

\item {} 
Here we set the values of \titleref{m,n} as integer parameters before
declaring the arrays \titleref{A,x,b}.  Being parameters means we can not
change their values later in the program.

\item {} 
It is possible to declare arrays and determine their size later,
using \titleref{allocatable} arrays, which we will also see later.

\end{itemize}

There are many array operations you can do, for example:

\begin{Verbatim}[commandchars=\\\{\},numbers=left,firstnumber=1,stepnumber=1]
\PYG{c}{! \PYGZdl{}UWHPSC/codes/fortran/vectorops.f90}

\PYG{k}{program }\PYG{n}{vectorops}

    \PYG{k}{implicit }\PYG{k}{none}
\PYG{k}{    }\PYG{k+kt}{real}\PYG{p}{(}\PYG{n+nb}{kind}\PYG{o}{=}\PYG{l+m+mi}{8}\PYG{p}{)}\PYG{p}{,} \PYG{k}{dimension}\PYG{p}{(}\PYG{l+m+mi}{3}\PYG{p}{)} \PYG{k+kd}{::} \PYG{n}{x}\PYG{p}{,} \PYG{n}{y}

    \PYG{n}{x} \PYG{o}{=} \PYG{p}{(}\PYG{o}{/}\PYG{l+m+mi}{1}\PYG{l+m+mf}{0.}\PYG{p}{,}\PYG{l+m+mi}{2}\PYG{l+m+mf}{0.}\PYG{p}{,}\PYG{l+m+mi}{3}\PYG{l+m+mf}{0.}\PYG{o}{/}\PYG{p}{)}
    \PYG{n}{y} \PYG{o}{=} \PYG{p}{(}\PYG{o}{/}\PYG{l+m+mi}{10}\PYG{l+m+mf}{0.}\PYG{p}{,}\PYG{l+m+mi}{40}\PYG{l+m+mf}{0.}\PYG{p}{,}\PYG{l+m+mi}{90}\PYG{l+m+mf}{0.}\PYG{o}{/}\PYG{p}{)}

    \PYG{k}{print} \PYG{o}{*}\PYG{p}{,} \PYG{l+s+s2}{\PYGZdq{}x = \PYGZdq{}}
    \PYG{k}{print} \PYG{o}{*}\PYG{p}{,} \PYG{n}{x}

    \PYG{k}{print} \PYG{o}{*}\PYG{p}{,} \PYG{l+s+s2}{\PYGZdq{}x**2 + y = \PYGZdq{}}
    \PYG{k}{print} \PYG{o}{*}\PYG{p}{,} \PYG{n}{x}\PYG{o}{**}\PYG{l+m+mi}{2} \PYG{o}{+} \PYG{n}{y}

    \PYG{k}{print} \PYG{o}{*}\PYG{p}{,} \PYG{l+s+s2}{\PYGZdq{}x*y = \PYGZdq{}}
    \PYG{k}{print} \PYG{o}{*}\PYG{p}{,} \PYG{n}{x}\PYG{o}{*}\PYG{n}{y}

    \PYG{k}{print} \PYG{o}{*}\PYG{p}{,} \PYG{l+s+s2}{\PYGZdq{}sqrt(y) = \PYGZdq{}}
    \PYG{k}{print} \PYG{o}{*}\PYG{p}{,} \PYG{n+nb}{sqrt}\PYG{p}{(}\PYG{n}{y}\PYG{p}{)}

    \PYG{k}{print} \PYG{o}{*}\PYG{p}{,} \PYG{l+s+s2}{\PYGZdq{}dot\PYGZus{}product(x,y) = \PYGZdq{}}
    \PYG{k}{print} \PYG{o}{*}\PYG{p}{,} \PYG{n+nb}{dot\PYGZus{}product}\PYG{p}{(}\PYG{n}{x}\PYG{p}{,}\PYG{n}{y}\PYG{p}{)}


\PYG{k}{end }\PYG{k}{program }\PYG{n}{vectorops}
\end{Verbatim}

produces:

\begin{Verbatim}[commandchars=\\\{\}]
\PYG{n}{x} \PYG{o}{=}
  \PYG{l+m+mf}{10.0000000000000}        \PYG{l+m+mf}{20.0000000000000}        \PYG{l+m+mf}{30.0000000000000}
\PYG{n}{x}\PYG{o}{*}\PYG{o}{*}\PYG{l+m+mi}{2} \PYG{o}{+} \PYG{n}{y} \PYG{o}{=}
  \PYG{l+m+mf}{200.000000000000}        \PYG{l+m+mf}{800.000000000000}        \PYG{l+m+mf}{1800.00000000000}
\PYG{n}{x}\PYG{o}{*}\PYG{n}{y} \PYG{o}{=}
  \PYG{l+m+mf}{1000.00000000000}        \PYG{l+m+mf}{8000.00000000000}        \PYG{l+m+mf}{27000.0000000000}
\PYG{n}{sqrt}\PYG{p}{(}\PYG{n}{y}\PYG{p}{)} \PYG{o}{=}
  \PYG{l+m+mf}{10.0000000000000}        \PYG{l+m+mf}{20.0000000000000}        \PYG{l+m+mf}{30.0000000000000}
\PYG{n}{dot\PYGZus{}product}\PYG{p}{(}\PYG{n}{x}\PYG{p}{,}\PYG{n}{y}\PYG{p}{)} \PYG{o}{=}
  \PYG{l+m+mf}{36000.0000000000}
\end{Verbatim}

Note that addition, multiplication, exponentiation, and intrinsic functions
such as \titleref{sqrt} all apply component-wise.

Multidimensional arrays can be manipulated in similar manner.
The produce to two arrays when computed with \titleref{*} is always component-wise.
For matrix multiplication, use \titleref{matmul}.   There is also a \titleref{transpose}
function:

\begin{Verbatim}[commandchars=\\\{\},numbers=left,firstnumber=1,stepnumber=1]
\PYG{c}{! \PYGZdl{}UWHPSC/codes/fortran/arrayops.f90}

\PYG{k}{program }\PYG{n}{arrayops}

    \PYG{k}{implicit }\PYG{k}{none}
\PYG{k}{    }\PYG{k+kt}{real}\PYG{p}{(}\PYG{n+nb}{kind}\PYG{o}{=}\PYG{l+m+mi}{8}\PYG{p}{)}\PYG{p}{,} \PYG{k}{dimension}\PYG{p}{(}\PYG{l+m+mi}{3}\PYG{p}{,}\PYG{l+m+mi}{2}\PYG{p}{)} \PYG{k+kd}{::} \PYG{n}{a}
    \PYG{k+kt}{real}\PYG{p}{(}\PYG{n+nb}{kind}\PYG{o}{=}\PYG{l+m+mi}{8}\PYG{p}{)}\PYG{p}{,} \PYG{k}{dimension}\PYG{p}{(}\PYG{l+m+mi}{2}\PYG{p}{,}\PYG{l+m+mi}{3}\PYG{p}{)} \PYG{k+kd}{::} \PYG{n}{b}
    \PYG{k+kt}{real}\PYG{p}{(}\PYG{n+nb}{kind}\PYG{o}{=}\PYG{l+m+mi}{8}\PYG{p}{)}\PYG{p}{,} \PYG{k}{dimension}\PYG{p}{(}\PYG{l+m+mi}{3}\PYG{p}{,}\PYG{l+m+mi}{3}\PYG{p}{)} \PYG{k+kd}{::} \PYG{n}{c}
    \PYG{k+kt}{real}\PYG{p}{(}\PYG{n+nb}{kind}\PYG{o}{=}\PYG{l+m+mi}{8}\PYG{p}{)}\PYG{p}{,} \PYG{k}{dimension}\PYG{p}{(}\PYG{l+m+mi}{2}\PYG{p}{)} \PYG{k+kd}{::} \PYG{n}{x}
    \PYG{k+kt}{real}\PYG{p}{(}\PYG{n+nb}{kind}\PYG{o}{=}\PYG{l+m+mi}{8}\PYG{p}{)}\PYG{p}{,} \PYG{k}{dimension}\PYG{p}{(}\PYG{l+m+mi}{3}\PYG{p}{)} \PYG{k+kd}{::} \PYG{n}{y}
    \PYG{k+kt}{integer }\PYG{n}{i}

    \PYG{n}{a} \PYG{o}{=} \PYG{n+nb}{reshape}\PYG{p}{(}\PYG{p}{(}\PYG{o}{/}\PYG{l+m+mi}{1}\PYG{p}{,}\PYG{l+m+mi}{2}\PYG{p}{,}\PYG{l+m+mi}{3}\PYG{p}{,}\PYG{l+m+mi}{4}\PYG{p}{,}\PYG{l+m+mi}{5}\PYG{p}{,}\PYG{l+m+mi}{6}\PYG{o}{/}\PYG{p}{)}\PYG{p}{,} \PYG{p}{(}\PYG{o}{/}\PYG{l+m+mi}{3}\PYG{p}{,}\PYG{l+m+mi}{2}\PYG{o}{/}\PYG{p}{)}\PYG{p}{)}

    \PYG{k}{print} \PYG{o}{*}\PYG{p}{,} \PYG{l+s+s2}{\PYGZdq{}a = \PYGZdq{}}
    \PYG{k}{do }\PYG{n}{i}\PYG{o}{=}\PYG{l+m+mi}{1}\PYG{p}{,}\PYG{l+m+mi}{3}
        \PYG{k}{print} \PYG{o}{*}\PYG{p}{,} \PYG{n}{a}\PYG{p}{(}\PYG{n}{i}\PYG{p}{,}\PYG{p}{:}\PYG{p}{)}   \PYG{c}{! i\PYGZsq{}th row}
        \PYG{n}{enddo}

    \PYG{n}{b} \PYG{o}{=} \PYG{n+nb}{transpose}\PYG{p}{(}\PYG{n}{a}\PYG{p}{)}

    \PYG{k}{print} \PYG{o}{*}\PYG{p}{,} \PYG{l+s+s2}{\PYGZdq{}b = \PYGZdq{}}
    \PYG{k}{do }\PYG{n}{i}\PYG{o}{=}\PYG{l+m+mi}{1}\PYG{p}{,}\PYG{l+m+mi}{2}
        \PYG{k}{print} \PYG{o}{*}\PYG{p}{,} \PYG{n}{b}\PYG{p}{(}\PYG{n}{i}\PYG{p}{,}\PYG{p}{:}\PYG{p}{)}   \PYG{c}{! i\PYGZsq{}th row}
        \PYG{n}{enddo}

    \PYG{n}{c} \PYG{o}{=} \PYG{n+nb}{matmul}\PYG{p}{(}\PYG{n}{a}\PYG{p}{,}\PYG{n}{b}\PYG{p}{)}
    \PYG{k}{print} \PYG{o}{*}\PYG{p}{,} \PYG{l+s+s2}{\PYGZdq{}c = \PYGZdq{}}
    \PYG{k}{do }\PYG{n}{i}\PYG{o}{=}\PYG{l+m+mi}{1}\PYG{p}{,}\PYG{l+m+mi}{3}
        \PYG{k}{print} \PYG{o}{*}\PYG{p}{,} \PYG{n}{c}\PYG{p}{(}\PYG{n}{i}\PYG{p}{,}\PYG{p}{:}\PYG{p}{)}   \PYG{c}{! i\PYGZsq{}th row}
        \PYG{n}{enddo}

    \PYG{n}{x} \PYG{o}{=} \PYG{p}{(}\PYG{o}{/}\PYG{l+m+mi}{5}\PYG{p}{,}\PYG{l+m+mi}{6}\PYG{o}{/}\PYG{p}{)}
    \PYG{n}{y} \PYG{o}{=} \PYG{n+nb}{matmul}\PYG{p}{(}\PYG{n}{a}\PYG{p}{,}\PYG{n}{x}\PYG{p}{)}
    \PYG{k}{print} \PYG{o}{*}\PYG{p}{,} \PYG{l+s+s2}{\PYGZdq{}x = \PYGZdq{}}\PYG{p}{,}\PYG{n}{x}
    \PYG{k}{print} \PYG{o}{*}\PYG{p}{,} \PYG{l+s+s2}{\PYGZdq{}y = \PYGZdq{}}\PYG{p}{,}\PYG{n}{y}

\PYG{k}{end }\PYG{k}{program }\PYG{n}{arrayops}
\end{Verbatim}

produces:

\begin{Verbatim}[commandchars=\\\{\}]
\PYG{n}{a} \PYG{o}{=}
  \PYG{l+m+mf}{1.00000000000000}        \PYG{l+m+mf}{4.00000000000000}
  \PYG{l+m+mf}{2.00000000000000}        \PYG{l+m+mf}{5.00000000000000}
  \PYG{l+m+mf}{3.00000000000000}        \PYG{l+m+mf}{6.00000000000000}

\PYG{n}{b} \PYG{o}{=}
  \PYG{l+m+mf}{1.00000000000000}        \PYG{l+m+mf}{2.00000000000000}        \PYG{l+m+mf}{3.00000000000000}
  \PYG{l+m+mf}{4.00000000000000}        \PYG{l+m+mf}{5.00000000000000}        \PYG{l+m+mf}{6.00000000000000}

\PYG{n}{c} \PYG{o}{=}
  \PYG{l+m+mf}{17.0000000000000}        \PYG{l+m+mf}{22.0000000000000}        \PYG{l+m+mf}{27.0000000000000}
  \PYG{l+m+mf}{22.0000000000000}        \PYG{l+m+mf}{29.0000000000000}        \PYG{l+m+mf}{36.0000000000000}
  \PYG{l+m+mf}{27.0000000000000}        \PYG{l+m+mf}{36.0000000000000}        \PYG{l+m+mf}{45.0000000000000}

\PYG{n}{x} \PYG{o}{=}    \PYG{l+m+mf}{5.00000000000000}        \PYG{l+m+mf}{6.00000000000000}
\PYG{n}{y} \PYG{o}{=}    \PYG{l+m+mf}{29.0000000000000}        \PYG{l+m+mf}{40.0000000000000}        \PYG{l+m+mf}{51.0000000000000}
\end{Verbatim}


\subsection{Loops}
\label{fortran:loops}\label{fortran:fortran-loops}
\begin{Verbatim}[commandchars=\\\{\},numbers=left,firstnumber=1,stepnumber=1]
\PYG{c}{! \PYGZdl{}UWHPSC/codes/fortran/loops1.f90}

\PYG{k}{program }\PYG{n}{loops1}

   \PYG{k}{implicit }\PYG{k}{none}
\PYG{k}{   }\PYG{k+kt}{integer} \PYG{k+kd}{::} \PYG{n}{i}

   \PYG{k}{do }\PYG{n}{i}\PYG{o}{=}\PYG{l+m+mi}{1}\PYG{p}{,}\PYG{l+m+mi}{3}           \PYG{c}{! prints 1,2,3}
      \PYG{k}{print} \PYG{o}{*}\PYG{p}{,} \PYG{n}{i}
      \PYG{n}{enddo}

   \PYG{k}{do }\PYG{n}{i}\PYG{o}{=}\PYG{l+m+mi}{5}\PYG{p}{,}\PYG{l+m+mi}{11}\PYG{p}{,}\PYG{l+m+mi}{2}        \PYG{c}{! prints 5,7,9,11}
      \PYG{k}{print} \PYG{o}{*}\PYG{p}{,} \PYG{n}{i}
      \PYG{n}{enddo}

   \PYG{k}{do }\PYG{n}{i}\PYG{o}{=}\PYG{l+m+mi}{6}\PYG{p}{,}\PYG{l+m+mi}{2}\PYG{p}{,}\PYG{o}{\PYGZhy{}}\PYG{l+m+mi}{1}        \PYG{c}{! prints 6,5,4,3,2}
      \PYG{k}{print} \PYG{o}{*}\PYG{p}{,} \PYG{n}{i}
      \PYG{n}{enddo}

   \PYG{n}{i} \PYG{o}{=} \PYG{l+m+mi}{0}
   \PYG{k}{do }\PYG{k}{while} \PYG{p}{(}\PYG{n}{i} \PYG{o}{\PYGZlt{}} \PYG{l+m+mi}{5}\PYG{p}{)}   \PYG{c}{! prints 0,1,2,3,4}
      \PYG{k}{print} \PYG{o}{*}\PYG{p}{,} \PYG{n}{i}
      \PYG{n}{i} \PYG{o}{=} \PYG{n}{i}\PYG{o}{+}\PYG{l+m+mi}{1}
      \PYG{n}{enddo}

\PYG{k}{end }\PYG{k}{program }\PYG{n}{loops1}
\end{Verbatim}

The \titleref{while} statement used in the last example is considered obsolete.  It
is better to use a \titleref{do} loop with an \titleref{exit} statement if a condition is
satisfied.  The last loop could be rewritten as:

\begin{Verbatim}[commandchars=\\\{\}]
i = 0
do                 ! prints 0,1,2,3,4
   if (i\PYGZgt{}=5) exit
   print *, i
   i = i+1
   enddo
\end{Verbatim}

This form of the \titleref{do} is valid but is generally not a good idea.
Like the while loop, this has the danger that a bug in the code may
cause it to loop forever (e.g. if you typed \titleref{i = i-1} instead of \titleref{i = i+1}).

A better approach for loops of this form is to limit the number of iterations
to some maximum value (chosen to be ample for your application), and then
print a warning message, or take more drastic action, if this is exceeded, e.g.:

\begin{Verbatim}[commandchars=\\\{\},numbers=left,firstnumber=1,stepnumber=1]
\PYG{c}{! \PYGZdl{}UWHPSC/codes/fortran/loops2.f90}

\PYG{k}{program }\PYG{n}{loops2}

   \PYG{k}{implicit }\PYG{k}{none}
\PYG{k}{   }\PYG{k+kt}{integer} \PYG{k+kd}{::} \PYG{n}{i}\PYG{p}{,}\PYG{n}{j}\PYG{p}{,}\PYG{n}{jmax}

   \PYG{n}{i} \PYG{o}{=} \PYG{l+m+mi}{0}
   \PYG{n}{jmax} \PYG{o}{=} \PYG{l+m+mi}{100}
   \PYG{k}{do }\PYG{n}{j}\PYG{o}{=}\PYG{l+m+mi}{1}\PYG{p}{,}\PYG{n}{jmax}        \PYG{c}{! prints 0,1,2,3,4}
      \PYG{k}{if} \PYG{p}{(}\PYG{n}{i}\PYG{o}{\PYGZgt{}}\PYG{o}{=}\PYG{l+m+mi}{5}\PYG{p}{)} \PYG{k}{exit}
\PYG{k}{      }\PYG{k}{print} \PYG{o}{*}\PYG{p}{,} \PYG{n}{i}
      \PYG{n}{i} \PYG{o}{=} \PYG{n}{i}\PYG{o}{+}\PYG{l+m+mi}{1}
      \PYG{n}{enddo}

   \PYG{k}{if} \PYG{p}{(}\PYG{n}{j}\PYG{o}{==}\PYG{n}{jmax}\PYG{o}{+}\PYG{l+m+mi}{1}\PYG{p}{)} \PYG{k}{then}
\PYG{k}{      }\PYG{k}{print} \PYG{o}{*}\PYG{p}{,} \PYG{l+s+s2}{\PYGZdq{}Warning: jmax iterations reached.\PYGZdq{}}
      \PYG{n}{endif}

\PYG{k}{end }\PYG{k}{program }\PYG{n}{loops2}
\end{Verbatim}

Note: \titleref{j} is incremented \emph{before} comparing to \titleref{jmax}.


\subsection{if-then-else}
\label{fortran:fortran-if}\label{fortran:if-then-else}
\begin{Verbatim}[commandchars=\\\{\},numbers=left,firstnumber=1,stepnumber=1]
\PYG{c}{! \PYGZdl{}UWHPSC/codes/fortran/ifelse1.f90}

\PYG{k}{program }\PYG{n}{ifelse1}

    \PYG{k}{implicit }\PYG{k}{none}
\PYG{k}{    }\PYG{k+kt}{real}\PYG{p}{(}\PYG{n+nb}{kind}\PYG{o}{=}\PYG{l+m+mi}{8}\PYG{p}{)} \PYG{k+kd}{::} \PYG{n}{x}
    \PYG{k+kt}{integer} \PYG{k+kd}{::} \PYG{n}{i}

    \PYG{n}{i} \PYG{o}{=} \PYG{l+m+mi}{3}
    \PYG{k}{if} \PYG{p}{(}\PYG{n}{i}\PYG{o}{\PYGZlt{}}\PYG{l+m+mi}{2}\PYG{p}{)} \PYG{k}{then}
\PYG{k}{        }\PYG{k}{print} \PYG{o}{*}\PYG{p}{,} \PYG{l+s+s2}{\PYGZdq{}i is less than 2\PYGZdq{}}
    \PYG{k}{else}
\PYG{k}{        }\PYG{k}{print} \PYG{o}{*}\PYG{p}{,} \PYG{l+s+s2}{\PYGZdq{}i is not less than 2\PYGZdq{}}
    \PYG{n}{endif}

    \PYG{k}{if} \PYG{p}{(}\PYG{n}{i}\PYG{o}{\PYGZlt{}}\PYG{o}{=}\PYG{l+m+mi}{2}\PYG{p}{)} \PYG{k}{then}
\PYG{k}{        }\PYG{k}{print} \PYG{o}{*}\PYG{p}{,} \PYG{l+s+s2}{\PYGZdq{}i is less or equal to 2\PYGZdq{}}
    \PYG{k}{else }\PYG{k}{if} \PYG{p}{(}\PYG{n}{i}\PYG{o}{/}\PYG{o}{=}\PYG{l+m+mi}{5}\PYG{p}{)} \PYG{k}{then}
\PYG{k}{        }\PYG{k}{print} \PYG{o}{*}\PYG{p}{,} \PYG{l+s+s2}{\PYGZdq{}i is greater than 2 but not equal to 5\PYGZdq{}}
    \PYG{k}{else }
\PYG{k}{        }\PYG{k}{print} \PYG{o}{*}\PYG{p}{,} \PYG{l+s+s2}{\PYGZdq{}i is equal to 5\PYGZdq{}}
    \PYG{n}{endif}

\PYG{k}{end }\PYG{k}{program }\PYG{n}{ifelse1}
\end{Verbatim}

Comments:
\begin{itemize}
\item {} 
The \titleref{else} clause is optional

\item {} 
You can have optional \titleref{else if} clauses

\end{itemize}

There is also a one-line form of an \titleref{if} statement that was seen in a
previous example on this page:

\begin{Verbatim}[commandchars=\\\{\}]
\PYG{k}{if} \PYG{p}{(}\PYG{n}{i}\PYG{o}{\PYGZgt{}}\PYG{o}{=}\PYG{l+m+mi}{5}\PYG{p}{)} \PYG{n}{exit}
\end{Verbatim}

This is equivalent to:

\begin{Verbatim}[commandchars=\\\{\}]
\PYG{k}{if} \PYG{p}{(}\PYG{n}{i}\PYG{o}{\PYGZgt{}}\PYG{o}{=}\PYG{l+m+mi}{5}\PYG{p}{)} \PYG{n}{then}
    \PYG{n}{exit}
    \PYG{n}{endif}
\end{Verbatim}


\subsection{Booleans}
\label{fortran:booleans}\begin{itemize}
\item {} 
Compare with \titleref{\textless{}, \textgreater{}, \textless{}=, \textgreater{}=, ==, /=}. You can also use the older Fortran 77
style: \titleref{.lt., .gt., .le., .ge., .eq., .neq.}.

\item {} 
Combine with \titleref{.and.} and \titleref{.or.}

\end{itemize}

For example:

\begin{Verbatim}[commandchars=\\\{\}]
\PYG{p}{(}\PYG{p}{(}\PYG{n}{x}\PYG{o}{\PYGZgt{}}\PYG{o}{=}\PYG{l+m+mf}{1.0}\PYG{p}{)} \PYG{o}{.}\PYG{o+ow}{and}\PYG{o}{.} \PYG{p}{(}\PYG{n}{x}\PYG{o}{\PYGZlt{}}\PYG{o}{=}\PYG{l+m+mf}{2.0}\PYG{p}{)}\PYG{p}{)} \PYG{o}{.}\PYG{o+ow}{or}\PYG{o}{.} \PYG{p}{(}\PYG{n}{x}\PYG{o}{\PYGZgt{}}\PYG{l+m+mi}{5}\PYG{p}{)}
\end{Verbatim}

A boolean variable is declared with type \titleref{logical} in Fortran, as for
example in the following code:

\begin{Verbatim}[commandchars=\\\{\},numbers=left,firstnumber=1,stepnumber=1]
\PYG{c}{! \PYGZdl{}UWHPSC/codes/fortran/boolean1.f90}

\PYG{k}{program }\PYG{n}{boolean1}

    \PYG{k}{implicit }\PYG{k}{none}
\PYG{k}{    }\PYG{k+kt}{integer} \PYG{k+kd}{::} \PYG{n}{i}\PYG{p}{,}\PYG{n}{k}
    \PYG{k+kt}{logical} \PYG{k+kd}{::} \PYG{n}{ever\PYGZus{}zero}

    \PYG{n}{ever\PYGZus{}zero} \PYG{o}{=} \PYG{p}{.}\PYG{n}{false}\PYG{p}{.}
    \PYG{k}{do }\PYG{n}{i}\PYG{o}{=}\PYG{l+m+mi}{1}\PYG{p}{,}\PYG{l+m+mi}{10}
        \PYG{n}{k} \PYG{o}{=} \PYG{l+m+mi}{3}\PYG{o}{*}\PYG{n}{i} \PYG{o}{\PYGZhy{}} \PYG{l+m+mi}{1}
        \PYG{n}{ever\PYGZus{}zero} \PYG{o}{=} \PYG{p}{(}\PYG{n}{ever\PYGZus{}zero} \PYG{p}{.}\PYG{n+nb}{or}\PYG{p}{.} \PYG{p}{(}\PYG{n}{k} \PYG{o}{==} \PYG{l+m+mi}{0}\PYG{p}{)}\PYG{p}{)}
        \PYG{n}{enddo}

    \PYG{k}{if} \PYG{p}{(}\PYG{n}{ever\PYGZus{}zero}\PYG{p}{)} \PYG{k}{then}
\PYG{k}{        }\PYG{k}{print} \PYG{o}{*}\PYG{p}{,} \PYG{l+s+s2}{\PYGZdq{}3*i \PYGZhy{} 1 takes the value 0 for some i\PYGZdq{}}
    \PYG{k}{else}
\PYG{k}{        }\PYG{k}{print} \PYG{o}{*}\PYG{p}{,} \PYG{l+s+s2}{\PYGZdq{}3*i \PYGZhy{} 1 is never 0 for i tested\PYGZdq{}}
    \PYG{n}{endif}

\PYG{k}{end }\PYG{k}{program }\PYG{n}{boolean1}
\end{Verbatim}


\subsection{Further reading}
\label{fortran:further-reading}\begin{itemize}
\item {} 
{\hyperref[fortran_sub:fortran\string-sub]{\crossref{\DUrole{std,std-ref}{Fortran subroutines and functions}}}}

\item {} 
{\hyperref[fortran_taylor:fortran\string-taylor]{\crossref{\DUrole{std,std-ref}{Fortran examples: Taylor series}}}}

\item {} 
\href{http://www.personal.psu.edu/hdk/fortran.html}{Fortran Resources page}

\end{itemize}


\section{Useful gfortran flags}
\label{gfortran_flags:useful-gfortran-flags}\label{gfortran_flags::doc}\label{gfortran_flags:gfortran-flags}
gfortran has many different command line options (also known as
\emph{flags}) that control what the compiler does and how it does it.  To
use these flags, simply include them on the command line when you run
gfortran, e.g.:

\begin{Verbatim}[commandchars=\\\{\}]
\PYGZdl{} gfortran \PYGZhy{}Wall \PYGZhy{}Wextra \PYGZhy{}c mysubroutine.f90 \PYGZhy{}o mysubroutine.o
\end{Verbatim}

If you find you use certain flags often, you can add them to an alias
in your \code{.bashrc} file, such as:

\begin{Verbatim}[commandchars=\\\{\}]
\PYG{n}{alias} \PYG{n}{gf}\PYG{o}{=}\PYG{l+s}{\PYGZdq{}}\PYG{l+s}{gfortran \PYGZhy{}Wall \PYGZhy{}Wextra \PYGZhy{}Wconversion \PYGZhy{}pedantic}\PYG{l+s}{\PYGZdq{}}
\end{Verbatim}

See the \href{http://linux.die.net/man/1/gfortran}{gfortran man page}
for more information.
Note a ``man page'' is the Unix help manual documentation that is available
for many Unix commands by typing, e.g.:

\begin{Verbatim}[commandchars=\\\{\}]
\PYGZdl{} man gfortran
\end{Verbatim}

\begin{notice}{warning}{Warning:}
Different fortran compilers use different names for similar flags!
\end{notice}


\subsection{Output flags}
\label{gfortran_flags:output-flags}
These flags control what kind of output gfortran generates, and where
that output goes.
\begin{itemize}
\item {} 
\code{-c}: Compile to an object file, rather than producing a
standalone program.  This flag is useful if your program source
code is split into multiple files.  The object files produced by
this command can later be linked together into a complete program.

\item {} 
\code{-o FILENAME}: Specifies the name of the output file.  Without
this flag, the default output file is \code{a.out} if compiling a
complete program, or \code{SOURCEFILE.o} if compiling to an object
file, where \code{SOURCEFILE.f90} is the name of the Fortran source
file being compiled.

\end{itemize}


\subsection{Warning flags}
\label{gfortran_flags:warning-flags}
Warning flags tell gfortran to warn you about legal but potentially
questionable sections of code.  These sections of code may be correct,
but warnings will often identify bugs before you even run your
program.
\begin{itemize}
\item {} 
\code{-Wall}: Short for ``warn about all,'' this flag tells gfortran to
generate warnings about many common sources of bugs, such as having
a subroutine or function with the same name as a built-in one, or
passing the same variable as an \code{intent(in)} and an
\code{intent(out)} argument of the same subroutine.
In spite of its name, this does not turn all possible \titleref{-W} options on.

\item {} 
\code{-Wextra}: In conjunction with \code{-Wall}, gives warnings about
even more potential problems.  In particular, \code{-Wextra} warns
about subroutine arguments that are never used, which is almost
always a bug.

\item {} 
\code{-Wconversion}: Warns about implicit conversion. For example, if
you want a double precision variable \code{sqrt2} to hold an accurate
value for the square root of 2, you might write by accident \code{sqrt2
= sqrt(2.)}.  Since \code{2.} is a single-precision value, the
single-precision \code{sqrt} function will be used, and the value of
\code{sqrt2} will not be as accurate as it could be.  \code{-Wconversion}
will generate a warning here, because the single-precision result
of \code{sqrt} is implicitly converted into a double-precision value.

\item {} 
\code{-pedantic}: Generate warnings about language features that are
supported by gfortran but are not part of the official Fortran 95
standard.  Useful if you want be sure your code will work with any
Fortran 95 compiler.

\end{itemize}


\subsection{Fortran dialect flags}
\label{gfortran_flags:fortran-dialect-flags}
Fortran dialect flags control how gfortran interprets your program,
and whether various language extensions such as OpenMP are enabled.
\begin{itemize}
\item {} 
\code{-fopenmp}: Tells gfortran to compile using OpenMP.  Without this
flag, OpenMP directives in your code will be ignored.

\item {} 
\code{-std=f95}: Enforces strict compliance with the Fortran 95
standard.  This is like \code{-pedantic}, except it generates errors
instead of warnings.

\end{itemize}


\subsection{Debugging flags}
\label{gfortran_flags:debugging-flags}
Debugging flags tell the compiler to include information inside the
compiled program that is useful in debugging, or alter the behavior of
the program to help find bugs.
\begin{itemize}
\item {} 
\code{-g}: Generates extra debugging information usable by GDB.
\code{-g3} includes even more debugging information.

\item {} 
\code{-fbacktrace}: Specifies that if the program crashes, a backtrace
should be produced if possible, showing what functions or
subroutines were being called at the time of the error.

\item {} 
\code{-fbounds-check}: Add a check that the array index is within the
bounds of the array every time an array element is accessed.  This
substantially slows down a program using it, but is a very useful
way to find bugs related to arrays; without this flag, an illegal
array access will produce either a subtle error that might not
become apparent until much later in the program, or will cause an
immediate segmentation fault with very little information about
cause of the error.

\item {} 
\code{-ffpe-trap=zero,overflow,underflow} tells Fortran to \emph{trap} the listed
floating point errors (fpe).  Having \titleref{zero} on the list means that
if you divide by zero the code will die rather than setting the result to
\titleref{+INFINITY} and continuing.  Similarly, if \titleref{overflow} is on the list it
will halt if you try to store a number larger than can be stored for the
type of real number you are using because the exponent is too large.

Trapping \titleref{underflow} will halt if you compute a number that is too small
because the exponent is a very large negative number.  For 8-byte
floating point numbers, this happens if the number is smaller than
approximate \titleref{1E-324}.   If you don't trap underflows, such numbers will
just be set to 0, which is generally the correct thing to do.  But
computing such small numbers may indicate a bug of some sort in the
program, so you might want to trap them.

\end{itemize}


\subsection{Optimization flags}
\label{gfortran_flags:optimization-flags}
Optimization options control how the compiler optimizes your code.
Optimization usually makes a program faster, but this is not always
true.
\begin{itemize}
\item {} 
\code{-O} \emph{level}: Use optimizations up to and including the specified
level.  Higher levels usually produce faster code but take longer
to compile.  Levels range from \code{-O0} (no optimization, the
default) to \code{-O3} (the most optimization available).

\end{itemize}


\subsection{Further reading}
\label{gfortran_flags:further-reading}
This list is by no means exhaustive.  A more complete list of
gfortran-specific specific flags is at
\url{http://gcc.gnu.org/onlinedocs/gfortran/Invoking-GNU-Fortran.html}
or on the \href{http://linux.die.net/man/1/gfortran}{gfortran man page}.

gfortran is part of the GCC family of compilers; more general
information on GCC command line options is available at
\url{http://gcc.gnu.org/onlinedocs/gcc/Invoking-GCC.html}, although
some of this information is specific to compiling C programs rather
than Fortran.

See also \url{http://linux.die.net/man/1/gfortran}.

A list of debug flags can also be found at
\url{http://www.fortran-2000.com/ArnaudRecipes/CompilerTricks.html\#CompTable\_fortran}


\section{Fortran subroutines and functions}
\label{fortran_sub:fortran-sub}\label{fortran_sub::doc}\label{fortran_sub:fortran-subroutines-and-functions}

\subsection{Functions}
\label{fortran_sub:functions}
Here's an example of the use of a function:

\begin{Verbatim}[commandchars=\\\{\},numbers=left,firstnumber=1,stepnumber=1]
\PYG{c}{! \PYGZdl{}UWHPSC/codes/fortran/fcn1.f90}

\PYG{k}{program }\PYG{n}{fcn1}
    \PYG{k}{implicit }\PYG{k}{none}
\PYG{k}{    }\PYG{k+kt}{real}\PYG{p}{(}\PYG{n+nb}{kind}\PYG{o}{=}\PYG{l+m+mi}{8}\PYG{p}{)} \PYG{k+kd}{::} \PYG{n}{y}\PYG{p}{,}\PYG{n}{z}
    \PYG{k+kt}{real}\PYG{p}{(}\PYG{n+nb}{kind}\PYG{o}{=}\PYG{l+m+mi}{8}\PYG{p}{)}\PYG{p}{,} \PYG{k}{external} \PYG{k+kd}{::} \PYG{n}{f}

    \PYG{n}{y} \PYG{o}{=} \PYG{l+m+mf}{2.}
    \PYG{n}{z} \PYG{o}{=} \PYG{n}{f}\PYG{p}{(}\PYG{n}{y}\PYG{p}{)}
    \PYG{k}{print} \PYG{o}{*}\PYG{p}{,} \PYG{l+s+s2}{\PYGZdq{}z = \PYGZdq{}}\PYG{p}{,}\PYG{n}{z}
\PYG{k}{end }\PYG{k}{program }\PYG{n}{fcn1}

\PYG{k+kt}{real}\PYG{p}{(}\PYG{n+nb}{kind}\PYG{o}{=}\PYG{l+m+mi}{8}\PYG{p}{)} \PYG{k}{function }\PYG{n}{f}\PYG{p}{(}\PYG{n}{x}\PYG{p}{)}
    \PYG{k}{implicit }\PYG{k}{none}
\PYG{k}{    }\PYG{k+kt}{real}\PYG{p}{(}\PYG{n+nb}{kind}\PYG{o}{=}\PYG{l+m+mi}{8}\PYG{p}{)}\PYG{p}{,} \PYG{k}{intent}\PYG{p}{(}\PYG{n}{in}\PYG{p}{)} \PYG{k+kd}{::} \PYG{n}{x}
    \PYG{n}{f} \PYG{o}{=} \PYG{n}{x}\PYG{o}{**}\PYG{l+m+mi}{2}
\PYG{k}{end }\PYG{k}{function }\PYG{n}{f}
\end{Verbatim}

It prints out:

\begin{Verbatim}[commandchars=\\\{\}]
\PYG{n}{z} \PYG{o}{=}    \PYG{l+m+mf}{4.00000000000000}
\end{Verbatim}

Comments:
\begin{itemize}
\item {} 
A function returns a single value.  Since this function is named \titleref{f},
the value of \titleref{f} must be set in the function somewhere.  You cannot use
\titleref{f} on the right hand side of any expression, e.g. you cannot set
\titleref{g = f} in the function.

\item {} 
f is declared \titleref{external} in the main program to let the compiler
know it is a function defined elsewhere rather than a variable.

\item {} 
The \titleref{intent(in)} statement tells the compiler that \titleref{x} is a variable
passed into the function that will not be modified in the function.

\item {} 
Here the program and function are in the same file.  Later we will see
how to break things up so each function or subroutine is in a separate
file.

\end{itemize}


\subsection{Modifying arguments}
\label{fortran_sub:modifying-arguments}
The input argument(s) to a function might also be modified, though this is
not so common as using a subroutine as described below.  But here's an
example:

\begin{Verbatim}[commandchars=\\\{\},numbers=left,firstnumber=1,stepnumber=1]
\PYG{c}{! \PYGZdl{}UWHPSC/codes/fortran/fcn2.f90}

\PYG{k}{program }\PYG{n}{fcn2}
    \PYG{k}{implicit }\PYG{k}{none}
\PYG{k}{    }\PYG{k+kt}{real}\PYG{p}{(}\PYG{n+nb}{kind}\PYG{o}{=}\PYG{l+m+mi}{8}\PYG{p}{)} \PYG{k+kd}{::} \PYG{n}{y}\PYG{p}{,}\PYG{n}{z}
    \PYG{k+kt}{real}\PYG{p}{(}\PYG{n+nb}{kind}\PYG{o}{=}\PYG{l+m+mi}{8}\PYG{p}{)}\PYG{p}{,} \PYG{k}{external} \PYG{k+kd}{::} \PYG{n}{f}

    \PYG{n}{y} \PYG{o}{=} \PYG{l+m+mf}{2.}
    \PYG{k}{print} \PYG{o}{*}\PYG{p}{,} \PYG{l+s+s2}{\PYGZdq{}Before calling f: y = \PYGZdq{}}\PYG{p}{,}\PYG{n}{y}
    \PYG{n}{z} \PYG{o}{=} \PYG{n}{f}\PYG{p}{(}\PYG{n}{y}\PYG{p}{)}
    \PYG{k}{print} \PYG{o}{*}\PYG{p}{,} \PYG{l+s+s2}{\PYGZdq{}After calling f:  y = \PYGZdq{}}\PYG{p}{,}\PYG{n}{y}
    \PYG{k}{print} \PYG{o}{*}\PYG{p}{,} \PYG{l+s+s2}{\PYGZdq{}z = \PYGZdq{}}\PYG{p}{,}\PYG{n}{z}
\PYG{k}{end }\PYG{k}{program }\PYG{n}{fcn2}

\PYG{k+kt}{real}\PYG{p}{(}\PYG{n+nb}{kind}\PYG{o}{=}\PYG{l+m+mi}{8}\PYG{p}{)} \PYG{k}{function }\PYG{n}{f}\PYG{p}{(}\PYG{n}{x}\PYG{p}{)}
    \PYG{k}{implicit }\PYG{k}{none}
\PYG{k}{    }\PYG{k+kt}{real}\PYG{p}{(}\PYG{n+nb}{kind}\PYG{o}{=}\PYG{l+m+mi}{8}\PYG{p}{)}\PYG{p}{,} \PYG{k}{intent}\PYG{p}{(}\PYG{n}{inout}\PYG{p}{)} \PYG{k+kd}{::} \PYG{n}{x}
    \PYG{n}{f} \PYG{o}{=} \PYG{n}{x}\PYG{o}{**}\PYG{l+m+mi}{2}
    \PYG{n}{x} \PYG{o}{=} \PYG{l+m+mf}{5.}
\PYG{k}{end }\PYG{k}{function }\PYG{n}{f}
\end{Verbatim}

This produces:

\begin{Verbatim}[commandchars=\\\{\}]
\PYG{n}{Before} \PYG{n}{calling} \PYG{n}{f}\PYG{p}{:} \PYG{n}{y} \PYG{o}{=}    \PYG{l+m+mf}{2.00000000000000}
\PYG{n}{After} \PYG{n}{calling} \PYG{n}{f}\PYG{p}{:}  \PYG{n}{y} \PYG{o}{=}    \PYG{l+m+mf}{5.00000000000000}
\PYG{n}{z} \PYG{o}{=}    \PYG{l+m+mf}{4.00000000000000}
\end{Verbatim}


\subsection{The use of \emph{intent}}
\label{fortran_sub:the-use-of-intent}
You are not required to specify the intent of each argument, but there are
several good reasons for doing so:
\begin{itemize}
\item {} 
It helps catch bugs.  If you specify \titleref{intent(in)} and then this variable
is changed in the function or subroutine, the compiler will give an
error.

\item {} 
It can help the compiler produce machine code that runs faster.  For
example, if it
is known to the compiler that some variables will never change during
execution, this fact can be used.

\end{itemize}


\subsection{Subroutines}
\label{fortran_sub:subroutines}
In Fortran, subroutines are generally used much more frequently than
functions.  Functions are expected to produce a single output variable and
examples like the one just given where an argument is modified are considered
bad programming style.

Subroutines are more flexible since they can have any number of inputs and
outputs.  In particular they may have not output variable.  For example a
subroutine might take an array as an argument and print the array to some
file on disk but not return a value to the calling program.

Here is a version of the same program  as \titleref{fcn1} above,
that instead uses a subroutine:

\begin{Verbatim}[commandchars=\\\{\},numbers=left,firstnumber=1,stepnumber=1]
\PYG{c}{! \PYGZdl{}UWHPSC/codes/fortran/sub1.f90}

\PYG{k}{program }\PYG{n}{sub1}
    \PYG{k}{implicit }\PYG{k}{none}
\PYG{k}{    }\PYG{k+kt}{real}\PYG{p}{(}\PYG{n+nb}{kind}\PYG{o}{=}\PYG{l+m+mi}{8}\PYG{p}{)} \PYG{k+kd}{::} \PYG{n}{y}\PYG{p}{,}\PYG{n}{z}

    \PYG{n}{y} \PYG{o}{=} \PYG{l+m+mf}{2.}
    \PYG{k}{call }\PYG{n}{fsub}\PYG{p}{(}\PYG{n}{y}\PYG{p}{,}\PYG{n}{z}\PYG{p}{)}
    \PYG{k}{print} \PYG{o}{*}\PYG{p}{,} \PYG{l+s+s2}{\PYGZdq{}z = \PYGZdq{}}\PYG{p}{,}\PYG{n}{z}
\PYG{k}{end }\PYG{k}{program }\PYG{n}{sub1}

\PYG{k}{subroutine }\PYG{n}{fsub}\PYG{p}{(}\PYG{n}{x}\PYG{p}{,}\PYG{n}{f}\PYG{p}{)}
    \PYG{k}{implicit }\PYG{k}{none}
\PYG{k}{    }\PYG{k+kt}{real}\PYG{p}{(}\PYG{n+nb}{kind}\PYG{o}{=}\PYG{l+m+mi}{8}\PYG{p}{)}\PYG{p}{,} \PYG{k}{intent}\PYG{p}{(}\PYG{n}{in}\PYG{p}{)} \PYG{k+kd}{::} \PYG{n}{x}
    \PYG{k+kt}{real}\PYG{p}{(}\PYG{n+nb}{kind}\PYG{o}{=}\PYG{l+m+mi}{8}\PYG{p}{)}\PYG{p}{,} \PYG{k}{intent}\PYG{p}{(}\PYG{n}{out}\PYG{p}{)} \PYG{k+kd}{::} \PYG{n}{f}
    \PYG{n}{f} \PYG{o}{=} \PYG{n}{x}\PYG{o}{**}\PYG{l+m+mi}{2}
\PYG{k}{end }\PYG{k}{subroutine }\PYG{n}{fsub}
\end{Verbatim}

Comments:
\begin{itemize}
\item {} 
Now we tell the compiler that \titleref{x} is an input variable and \titleref{y} is an
output variable for the subroutine.  The \titleref{intent} declarations are
optional but sometimes help the compiler optimize code.

\end{itemize}

Here is a version that works on an array \titleref{x} instead of a single value:

\begin{Verbatim}[commandchars=\\\{\},numbers=left,firstnumber=1,stepnumber=1]
\PYG{c}{! \PYGZdl{}UWHPSC/codes/fortran/sub2.f90}

\PYG{k}{program }\PYG{n}{sub2}
    \PYG{k}{implicit }\PYG{k}{none}
\PYG{k}{    }\PYG{k+kt}{real}\PYG{p}{(}\PYG{n+nb}{kind}\PYG{o}{=}\PYG{l+m+mi}{8}\PYG{p}{)}\PYG{p}{,} \PYG{k}{dimension}\PYG{p}{(}\PYG{l+m+mi}{3}\PYG{p}{)} \PYG{k+kd}{::} \PYG{n}{y}\PYG{p}{,}\PYG{n}{z}
    \PYG{k+kt}{integer }\PYG{n}{n}

    \PYG{n}{y} \PYG{o}{=} \PYG{p}{(}\PYG{o}{/}\PYG{l+m+mf}{2.}\PYG{p}{,} \PYG{l+m+mf}{3.}\PYG{p}{,} \PYG{l+m+mf}{4.}\PYG{o}{/}\PYG{p}{)}
    \PYG{n}{n} \PYG{o}{=} \PYG{n}{size}\PYG{p}{(}\PYG{n}{y}\PYG{p}{)}
    \PYG{k}{call }\PYG{n}{fsub}\PYG{p}{(}\PYG{n}{y}\PYG{p}{,}\PYG{n}{n}\PYG{p}{,}\PYG{n}{z}\PYG{p}{)}
    \PYG{k}{print} \PYG{o}{*}\PYG{p}{,} \PYG{l+s+s2}{\PYGZdq{}z = \PYGZdq{}}\PYG{p}{,}\PYG{n}{z}
\PYG{k}{end }\PYG{k}{program }\PYG{n}{sub2}

\PYG{k}{subroutine }\PYG{n}{fsub}\PYG{p}{(}\PYG{n}{x}\PYG{p}{,}\PYG{n}{n}\PYG{p}{,}\PYG{n}{f}\PYG{p}{)}
  \PYG{c}{! compute f(x) = x**2 for all elements of the array x }
  \PYG{c}{! of length n.}
  \PYG{k}{implicit }\PYG{k}{none}
\PYG{k}{  }\PYG{k+kt}{integer}\PYG{p}{,} \PYG{k}{intent}\PYG{p}{(}\PYG{n}{in}\PYG{p}{)} \PYG{k+kd}{::} \PYG{n}{n}
  \PYG{k+kt}{real}\PYG{p}{(}\PYG{n+nb}{kind}\PYG{o}{=}\PYG{l+m+mi}{8}\PYG{p}{)}\PYG{p}{,} \PYG{k}{dimension}\PYG{p}{(}\PYG{n}{n}\PYG{p}{)}\PYG{p}{,} \PYG{k}{intent}\PYG{p}{(}\PYG{n}{in}\PYG{p}{)} \PYG{k+kd}{::} \PYG{n}{x}
  \PYG{k+kt}{real}\PYG{p}{(}\PYG{n+nb}{kind}\PYG{o}{=}\PYG{l+m+mi}{8}\PYG{p}{)}\PYG{p}{,} \PYG{k}{dimension}\PYG{p}{(}\PYG{n}{n}\PYG{p}{)}\PYG{p}{,} \PYG{k}{intent}\PYG{p}{(}\PYG{n}{out}\PYG{p}{)} \PYG{k+kd}{::} \PYG{n}{f}
  \PYG{n}{f} \PYG{o}{=} \PYG{n}{x}\PYG{o}{**}\PYG{l+m+mi}{2}
\PYG{k}{end }\PYG{k}{subroutine }\PYG{n}{fsub}
\end{Verbatim}

This produces:

\begin{Verbatim}[commandchars=\\\{\}]
\PYG{l+m+mf}{4.00000000000000}        \PYG{l+m+mf}{9.00000000000000}        \PYG{l+m+mf}{16.0000000000000}
\end{Verbatim}

Comments:
\begin{itemize}
\item {} 
The length of the array is also passed into the subroutine.  You can
avoid this in Fortran 90 (see the next example below), but it
was unavoidable in Fortran 77 and subroutines working on arrays in
Fortran are often written so that the dimensions are passed in as
arguments.

\end{itemize}


\subsection{Subroutine in a module}
\label{fortran_sub:subroutine-in-a-module}
Here is a version that avoids passing the length of the array into the
subroutine.  In order for this to work some additional \emph{interface}
information must be specified.  This is most easily done by including the
subroutine in a \emph{module}.

\begin{Verbatim}[commandchars=\\\{\},numbers=left,firstnumber=1,stepnumber=1]
\PYG{c}{! \PYGZdl{}UWHPSC/codes/fortran/sub3.f90}

\PYG{k}{module }\PYG{n}{sub3module}

\PYG{k}{contains }

\PYG{k}{subroutine }\PYG{n}{fsub}\PYG{p}{(}\PYG{n}{x}\PYG{p}{,}\PYG{n}{f}\PYG{p}{)}
  \PYG{c}{! compute f(x) = x**2 for all elements of the array x. }
  \PYG{k}{implicit }\PYG{k}{none}
\PYG{k}{  }\PYG{k+kt}{real}\PYG{p}{(}\PYG{n+nb}{kind}\PYG{o}{=}\PYG{l+m+mi}{8}\PYG{p}{)}\PYG{p}{,} \PYG{k}{dimension}\PYG{p}{(}\PYG{p}{:}\PYG{p}{)}\PYG{p}{,} \PYG{k}{intent}\PYG{p}{(}\PYG{n}{in}\PYG{p}{)} \PYG{k+kd}{::} \PYG{n}{x}
  \PYG{k+kt}{real}\PYG{p}{(}\PYG{n+nb}{kind}\PYG{o}{=}\PYG{l+m+mi}{8}\PYG{p}{)}\PYG{p}{,} \PYG{k}{dimension}\PYG{p}{(}\PYG{n}{size}\PYG{p}{(}\PYG{n}{x}\PYG{p}{)}\PYG{p}{)}\PYG{p}{,} \PYG{k}{intent}\PYG{p}{(}\PYG{n}{out}\PYG{p}{)} \PYG{k+kd}{::} \PYG{n}{f}
  \PYG{n}{f} \PYG{o}{=} \PYG{n}{x}\PYG{o}{**}\PYG{l+m+mi}{2}
\PYG{k}{end }\PYG{k}{subroutine }\PYG{n}{fsub}

\PYG{k}{end }\PYG{k}{module }\PYG{n}{sub3module}

\PYG{c}{!\PYGZhy{}\PYGZhy{}\PYGZhy{}\PYGZhy{}\PYGZhy{}\PYGZhy{}\PYGZhy{}\PYGZhy{}\PYGZhy{}\PYGZhy{}\PYGZhy{}\PYGZhy{}\PYGZhy{}\PYGZhy{}\PYGZhy{}\PYGZhy{}\PYGZhy{}\PYGZhy{}\PYGZhy{}\PYGZhy{}\PYGZhy{}\PYGZhy{}\PYGZhy{}\PYGZhy{}\PYGZhy{}\PYGZhy{}\PYGZhy{}\PYGZhy{}\PYGZhy{}\PYGZhy{}\PYGZhy{}\PYGZhy{}\PYGZhy{}\PYGZhy{}\PYGZhy{}\PYGZhy{}\PYGZhy{}\PYGZhy{}\PYGZhy{}\PYGZhy{}\PYGZhy{}\PYGZhy{}\PYGZhy{}\PYGZhy{}\PYGZhy{}\PYGZhy{}}

\PYG{k}{program }\PYG{n}{sub3}
    \PYG{k}{use }\PYG{n}{sub3module}
    \PYG{k}{implicit }\PYG{k}{none}
\PYG{k}{    }\PYG{k+kt}{real}\PYG{p}{(}\PYG{n+nb}{kind}\PYG{o}{=}\PYG{l+m+mi}{8}\PYG{p}{)}\PYG{p}{,} \PYG{k}{dimension}\PYG{p}{(}\PYG{l+m+mi}{3}\PYG{p}{)} \PYG{k+kd}{::} \PYG{n}{y}\PYG{p}{,}\PYG{n}{z}

    \PYG{n}{y} \PYG{o}{=} \PYG{p}{(}\PYG{o}{/}\PYG{l+m+mf}{2.}\PYG{p}{,} \PYG{l+m+mf}{3.}\PYG{p}{,} \PYG{l+m+mf}{4.}\PYG{o}{/}\PYG{p}{)}
    \PYG{k}{call }\PYG{n}{fsub}\PYG{p}{(}\PYG{n}{y}\PYG{p}{,}\PYG{n}{z}\PYG{p}{)}
    \PYG{k}{print} \PYG{o}{*}\PYG{p}{,} \PYG{l+s+s2}{\PYGZdq{}z = \PYGZdq{}}\PYG{p}{,}\PYG{n}{z}
\PYG{k}{end }\PYG{k}{program }\PYG{n}{sub3}
\end{Verbatim}

Comments:
\begin{itemize}
\item {} 
See the section {\hyperref[fortran_modules:fortran\string-modules]{\crossref{\DUrole{std,std-ref}{Fortran modules}}}} for more information about modules.
Note in particular that the module must be defined first and then used
in the program via the \titleref{use} statement.

\item {} 
The declaration of \titleref{x} in the subroutine uses \titleref{dimension(:)} to indicate
that it is an array with a single index (having \titleref{rank 1}),
without specifying how long the
array is.  If \titleref{x} was a rank 2 array indexed by \titleref{x(i,j)}
then the dimension statement would be \titleref{dimension(:,:)}.

\item {} 
The declaration of \titleref{f} in the subroutine uses \titleref{dimension(size(x))} to
indicate that it is an array with one index and the length of the array
is the same as that of \titleref{x}.  In fact it would be sufficient to tell the
compiler that it is an array of rank 1:

\begin{Verbatim}[commandchars=\\\{\}]
\PYG{n}{real}\PYG{p}{(}\PYG{n}{kind}\PYG{o}{=}\PYG{l+m+mi}{8}\PYG{p}{)}\PYG{p}{,} \PYG{n}{dimension}\PYG{p}{(}\PYG{p}{:}\PYG{p}{)}\PYG{p}{,} \PYG{n}{intent}\PYG{p}{(}\PYG{n}{out}\PYG{p}{)} \PYG{p}{:}\PYG{p}{:} \PYG{n}{f}
\end{Verbatim}

but indicating that it has the same size ax \titleref{x} is useful for someone
trying to understand the code.

\end{itemize}


\subsection{Further reading}
\label{fortran_sub:further-reading}\begin{itemize}
\item {} 
{\hyperref[fortran:fortran]{\crossref{\DUrole{std,std-ref}{Fortran}}}}

\item {} 
{\hyperref[fortran_taylor:fortran\string-taylor]{\crossref{\DUrole{std,std-ref}{Fortran examples: Taylor series}}}}

\end{itemize}


\section{Fortran examples: Taylor series}
\label{fortran_taylor:fortran-taylor}\label{fortran_taylor:fortran-examples-taylor-series}\label{fortran_taylor::doc}
Here is an example code that uses the Taylor series for \(exp(x)\) to
estimate the value of this function at \(x=1\):

\begin{Verbatim}[commandchars=\\\{\},numbers=left,firstnumber=1,stepnumber=1]
\PYG{c}{! \PYGZdl{}UWHPSC/codes/fortran/taylor.f90}

\PYG{k}{program }\PYG{n}{taylor}

    \PYG{k}{implicit }\PYG{k}{none                  }
\PYG{k}{    }\PYG{k+kt}{real} \PYG{p}{(}\PYG{n+nb}{kind}\PYG{o}{=}\PYG{l+m+mi}{8}\PYG{p}{)} \PYG{k+kd}{::} \PYG{n}{x}\PYG{p}{,} \PYG{n}{exp\PYGZus{}true}\PYG{p}{,} \PYG{n}{y}
    \PYG{k+kt}{real} \PYG{p}{(}\PYG{n+nb}{kind}\PYG{o}{=}\PYG{l+m+mi}{8}\PYG{p}{)}\PYG{p}{,} \PYG{k}{external} \PYG{k+kd}{::} \PYG{n}{exptaylor}
    \PYG{k+kt}{integer} \PYG{k+kd}{::} \PYG{n}{n}

    \PYG{n}{n} \PYG{o}{=} \PYG{l+m+mi}{20}               \PYG{c}{! number of terms to use}
    \PYG{n}{x} \PYG{o}{=} \PYG{l+m+mf}{1.0}
    \PYG{n}{exp\PYGZus{}true} \PYG{o}{=} \PYG{n+nb}{exp}\PYG{p}{(}\PYG{n}{x}\PYG{p}{)}
    \PYG{n}{y} \PYG{o}{=} \PYG{n}{exptaylor}\PYG{p}{(}\PYG{n}{x}\PYG{p}{,}\PYG{n}{n}\PYG{p}{)}   \PYG{c}{! uses function below}
    \PYG{k}{print} \PYG{o}{*}\PYG{p}{,} \PYG{l+s+s2}{\PYGZdq{}x = \PYGZdq{}}\PYG{p}{,}\PYG{n}{x}
    \PYG{k}{print} \PYG{o}{*}\PYG{p}{,} \PYG{l+s+s2}{\PYGZdq{}exp\PYGZus{}true  = \PYGZdq{}}\PYG{p}{,}\PYG{n}{exp\PYGZus{}true}
    \PYG{k}{print} \PYG{o}{*}\PYG{p}{,} \PYG{l+s+s2}{\PYGZdq{}exptaylor = \PYGZdq{}}\PYG{p}{,}\PYG{n}{y}
    \PYG{k}{print} \PYG{o}{*}\PYG{p}{,} \PYG{l+s+s2}{\PYGZdq{}error     = \PYGZdq{}}\PYG{p}{,}\PYG{n}{y} \PYG{o}{\PYGZhy{}} \PYG{n}{exp\PYGZus{}true}

\PYG{k}{end }\PYG{k}{program }\PYG{n}{taylor}

\PYG{c}{!==========================}
\PYG{k}{function }\PYG{n}{exptaylor}\PYG{p}{(}\PYG{n}{x}\PYG{p}{,}\PYG{n}{n}\PYG{p}{)}
\PYG{c}{!==========================}
    \PYG{k}{implicit }\PYG{k}{none}

    \PYG{c}{! function arguments:}
    \PYG{k+kt}{real} \PYG{p}{(}\PYG{n+nb}{kind}\PYG{o}{=}\PYG{l+m+mi}{8}\PYG{p}{)}\PYG{p}{,} \PYG{k}{intent}\PYG{p}{(}\PYG{n}{in}\PYG{p}{)} \PYG{k+kd}{::} \PYG{n}{x}
    \PYG{k+kt}{integer}\PYG{p}{,} \PYG{k}{intent}\PYG{p}{(}\PYG{n}{in}\PYG{p}{)} \PYG{k+kd}{::} \PYG{n}{n}
    \PYG{k+kt}{real} \PYG{p}{(}\PYG{n+nb}{kind}\PYG{o}{=}\PYG{l+m+mi}{8}\PYG{p}{)} \PYG{k+kd}{::} \PYG{n}{exptaylor}

    \PYG{c}{! local variables:}
    \PYG{k+kt}{real} \PYG{p}{(}\PYG{n+nb}{kind}\PYG{o}{=}\PYG{l+m+mi}{8}\PYG{p}{)} \PYG{k+kd}{::} \PYG{n}{term}\PYG{p}{,} \PYG{n}{partial\PYGZus{}sum}
    \PYG{k+kt}{integer} \PYG{k+kd}{::} \PYG{n}{j}

    \PYG{n}{term} \PYG{o}{=} \PYG{l+m+mf}{1.}
    \PYG{n}{partial\PYGZus{}sum} \PYG{o}{=} \PYG{n}{term}

    \PYG{k}{do }\PYG{n}{j}\PYG{o}{=}\PYG{l+m+mi}{1}\PYG{p}{,}\PYG{n}{n}
        \PYG{c}{! j\PYGZsq{}th term is  x**j / j!  which is the previous term times x/j:}
        \PYG{n}{term} \PYG{o}{=} \PYG{n}{term}\PYG{o}{*}\PYG{n}{x}\PYG{o}{/}\PYG{n}{j}   
        \PYG{c}{! add this term to the partial sum:}
        \PYG{n}{partial\PYGZus{}sum} \PYG{o}{=} \PYG{n}{partial\PYGZus{}sum} \PYG{o}{+} \PYG{n}{term}   
        \PYG{n}{enddo}
     \PYG{n}{exptaylor} \PYG{o}{=} \PYG{n}{partial\PYGZus{}sum}  \PYG{c}{! this is the value returned}
\PYG{k}{end }\PYG{k}{function }\PYG{n}{exptaylor}
\end{Verbatim}

Running this code gives:

\begin{Verbatim}[commandchars=\\\{\}]
\PYG{n}{x} \PYG{o}{=}    \PYG{l+m+mf}{1.00000000000000}
\PYG{n}{exp\PYGZus{}true}  \PYG{o}{=}    \PYG{l+m+mf}{2.71828182845905}
\PYG{n}{exptaylor} \PYG{o}{=}    \PYG{l+m+mf}{2.71828174591064}
\PYG{n}{error}     \PYG{o}{=}  \PYG{o}{\PYGZhy{}}\PYG{l+m+mf}{8.254840055954560E\PYGZhy{}008}
\end{Verbatim}

Here's a modification where the number of terms to use is computed based on
the size of the next term in the series.  The function has been rewritten as
a subroutine so the number of terms can be returned as well.

\begin{Verbatim}[commandchars=\\\{\},numbers=left,firstnumber=1,stepnumber=1]
\PYG{c}{! \PYGZdl{}UWHPSC/codes/fortran/taylor\PYGZus{}converge.f90}

\PYG{k}{program }\PYG{n}{taylor\PYGZus{}converge}

    \PYG{k}{implicit }\PYG{k}{none                  }
\PYG{k}{    }\PYG{k+kt}{real} \PYG{p}{(}\PYG{n+nb}{kind}\PYG{o}{=}\PYG{l+m+mi}{8}\PYG{p}{)} \PYG{k+kd}{::} \PYG{n}{x}\PYG{p}{,} \PYG{n}{exp\PYGZus{}true}\PYG{p}{,} \PYG{n}{y}\PYG{p}{,} \PYG{n}{relative\PYGZus{}error}
    \PYG{k+kt}{integer} \PYG{k+kd}{::} \PYG{n}{nmax}\PYG{p}{,} \PYG{n}{nterms}\PYG{p}{,} \PYG{n}{j}

    \PYG{n}{nmax} \PYG{o}{=} \PYG{l+m+mi}{100}

    \PYG{k}{print} \PYG{o}{*}\PYG{p}{,} \PYG{l+s+s2}{\PYGZdq{}     x         true              approximate          error         nterms\PYGZdq{}}
    \PYG{k}{do }\PYG{n}{j} \PYG{o}{=} \PYG{o}{\PYGZhy{}}\PYG{l+m+mi}{20}\PYG{p}{,}\PYG{l+m+mi}{20}\PYG{p}{,}\PYG{l+m+mi}{4}
       \PYG{n}{x} \PYG{o}{=} \PYG{n+nb}{float}\PYG{p}{(}\PYG{n}{j}\PYG{p}{)}                      \PYG{c}{! convert to a real}
       \PYG{k}{call }\PYG{n}{exptaylor}\PYG{p}{(}\PYG{n}{x}\PYG{p}{,}\PYG{n}{nmax}\PYG{p}{,}\PYG{n}{y}\PYG{p}{,}\PYG{n}{nterms}\PYG{p}{)}   \PYG{c}{! defined below}
       \PYG{n}{exp\PYGZus{}true} \PYG{o}{=} \PYG{n+nb}{exp}\PYG{p}{(}\PYG{n}{x}\PYG{p}{)}
       \PYG{n}{relative\PYGZus{}error} \PYG{o}{=} \PYG{n+nb}{abs}\PYG{p}{(}\PYG{n}{y}\PYG{o}{\PYGZhy{}}\PYG{n}{exp\PYGZus{}true}\PYG{p}{)} \PYG{o}{/} \PYG{n}{exp\PYGZus{}true}
       \PYG{k}{print} \PYG{l+s+s1}{\PYGZsq{}(f10.3,3d19.10,i6)\PYGZsq{}}\PYG{p}{,} \PYG{n}{x}\PYG{p}{,} \PYG{n}{exp\PYGZus{}true}\PYG{p}{,} \PYG{n}{y}\PYG{p}{,} \PYG{n}{relative\PYGZus{}error}\PYG{p}{,} \PYG{n}{nterms}
       \PYG{n}{enddo}

\PYG{k}{end }\PYG{k}{program }\PYG{n}{taylor\PYGZus{}converge}

\PYG{c}{!====================================}
\PYG{k}{subroutine }\PYG{n}{exptaylor}\PYG{p}{(}\PYG{n}{x}\PYG{p}{,}\PYG{n}{nmax}\PYG{p}{,}\PYG{n}{y}\PYG{p}{,}\PYG{n}{nterms}\PYG{p}{)}
\PYG{c}{!====================================}
    \PYG{k}{implicit }\PYG{k}{none}

    \PYG{c}{! subroutine arguments:}
    \PYG{k+kt}{real} \PYG{p}{(}\PYG{n+nb}{kind}\PYG{o}{=}\PYG{l+m+mi}{8}\PYG{p}{)}\PYG{p}{,} \PYG{k}{intent}\PYG{p}{(}\PYG{n}{in}\PYG{p}{)} \PYG{k+kd}{::} \PYG{n}{x}
    \PYG{k+kt}{integer}\PYG{p}{,} \PYG{k}{intent}\PYG{p}{(}\PYG{n}{in}\PYG{p}{)} \PYG{k+kd}{::} \PYG{n}{nmax}
    \PYG{k+kt}{real} \PYG{p}{(}\PYG{n+nb}{kind}\PYG{o}{=}\PYG{l+m+mi}{8}\PYG{p}{)}\PYG{p}{,} \PYG{k}{intent}\PYG{p}{(}\PYG{n}{out}\PYG{p}{)} \PYG{k+kd}{::} \PYG{n}{y}
    \PYG{k+kt}{integer}\PYG{p}{,} \PYG{k}{intent}\PYG{p}{(}\PYG{n}{out}\PYG{p}{)} \PYG{k+kd}{::} \PYG{n}{nterms}

    \PYG{c}{! local variables:}
    \PYG{k+kt}{real} \PYG{p}{(}\PYG{n+nb}{kind}\PYG{o}{=}\PYG{l+m+mi}{8}\PYG{p}{)} \PYG{k+kd}{::} \PYG{n}{term}\PYG{p}{,} \PYG{n}{partial\PYGZus{}sum}
    \PYG{k+kt}{integer} \PYG{k+kd}{::} \PYG{n}{j}

    \PYG{n}{term} \PYG{o}{=} \PYG{l+m+mf}{1.}
    \PYG{n}{partial\PYGZus{}sum} \PYG{o}{=} \PYG{n}{term}

    \PYG{k}{do }\PYG{n}{j}\PYG{o}{=}\PYG{l+m+mi}{1}\PYG{p}{,}\PYG{n}{nmax}
        \PYG{c}{! j\PYGZsq{}th term is  x**j / j!  which is the previous term times x/j:}
        \PYG{n}{term} \PYG{o}{=} \PYG{n}{term}\PYG{o}{*}\PYG{n}{x}\PYG{o}{/}\PYG{n}{j}   
        \PYG{c}{! add this term to the partial sum:}
        \PYG{n}{partial\PYGZus{}sum} \PYG{o}{=} \PYG{n}{partial\PYGZus{}sum} \PYG{o}{+} \PYG{n}{term}   
        \PYG{k}{if} \PYG{p}{(}\PYG{n+nb}{abs}\PYG{p}{(}\PYG{n}{term}\PYG{p}{)} \PYG{o}{\PYGZlt{}} \PYG{l+m+mf}{1.}\PYG{n}{d}\PYG{o}{\PYGZhy{}}\PYG{l+m+mi}{16}\PYG{o}{*}\PYG{n}{partial\PYGZus{}sum}\PYG{p}{)} \PYG{k}{exit}
\PYG{k}{        }\PYG{n}{enddo}
     \PYG{n}{nterms} \PYG{o}{=} \PYG{n}{j}       \PYG{c}{! number of terms used}
     \PYG{n}{y} \PYG{o}{=} \PYG{n}{partial\PYGZus{}sum}  \PYG{c}{! this is the value returned}
\PYG{k}{end }\PYG{k}{subroutine }\PYG{n}{exptaylor}
\end{Verbatim}

This produces:

\begin{Verbatim}[commandchars=\\\{\}]
   \PYG{n}{x}         \PYG{n}{true}              \PYG{n}{approximate}          \PYG{n}{error}         \PYG{n}{nterms}
\PYG{o}{\PYGZhy{}}\PYG{l+m+mf}{20.000}   \PYG{l+m+mf}{0.2061153622}\PYG{n}{D}\PYG{o}{\PYGZhy{}}\PYG{l+m+mi}{08}   \PYG{l+m+mf}{0.5621884472}\PYG{n}{D}\PYG{o}{\PYGZhy{}}\PYG{l+m+mi}{08}   \PYG{l+m+mf}{0.1727542678}\PYG{n}{D}\PYG{o}{+}\PYG{l+m+mi}{01}    \PYG{l+m+mi}{95}
\PYG{o}{\PYGZhy{}}\PYG{l+m+mf}{16.000}   \PYG{l+m+mf}{0.1125351747}\PYG{n}{D}\PYG{o}{\PYGZhy{}}\PYG{l+m+mi}{06}   \PYG{l+m+mf}{0.1125418051}\PYG{n}{D}\PYG{o}{\PYGZhy{}}\PYG{l+m+mi}{06}   \PYG{l+m+mf}{0.5891819580}\PYG{n}{D}\PYG{o}{\PYGZhy{}}\PYG{l+m+mi}{04}    \PYG{l+m+mi}{81}
\PYG{o}{\PYGZhy{}}\PYG{l+m+mf}{12.000}   \PYG{l+m+mf}{0.6144212353}\PYG{n}{D}\PYG{o}{\PYGZhy{}}\PYG{l+m+mi}{05}   \PYG{l+m+mf}{0.6144213318}\PYG{n}{D}\PYG{o}{\PYGZhy{}}\PYG{l+m+mi}{05}   \PYG{l+m+mf}{0.1569943213}\PYG{n}{D}\PYG{o}{\PYGZhy{}}\PYG{l+m+mi}{06}    \PYG{l+m+mi}{66}
 \PYG{o}{\PYGZhy{}}\PYG{l+m+mf}{8.000}   \PYG{l+m+mf}{0.3354626279}\PYG{n}{D}\PYG{o}{\PYGZhy{}}\PYG{l+m+mi}{03}   \PYG{l+m+mf}{0.3354626279}\PYG{n}{D}\PYG{o}{\PYGZhy{}}\PYG{l+m+mi}{03}   \PYG{l+m+mf}{0.6586251980}\PYG{n}{D}\PYG{o}{\PYGZhy{}}\PYG{l+m+mi}{11}    \PYG{l+m+mi}{51}
 \PYG{o}{\PYGZhy{}}\PYG{l+m+mf}{4.000}   \PYG{l+m+mf}{0.1831563889}\PYG{n}{D}\PYG{o}{\PYGZhy{}}\PYG{l+m+mi}{01}   \PYG{l+m+mf}{0.1831563889}\PYG{n}{D}\PYG{o}{\PYGZhy{}}\PYG{l+m+mi}{01}   \PYG{l+m+mf}{0.1723771005}\PYG{n}{D}\PYG{o}{\PYGZhy{}}\PYG{l+m+mi}{13}    \PYG{l+m+mi}{34}
  \PYG{l+m+mf}{0.000}   \PYG{l+m+mf}{0.1000000000}\PYG{n}{D}\PYG{o}{+}\PYG{l+m+mi}{01}   \PYG{l+m+mf}{0.1000000000}\PYG{n}{D}\PYG{o}{+}\PYG{l+m+mi}{01}   \PYG{l+m+mf}{0.0000000000}\PYG{n}{D}\PYG{o}{+}\PYG{l+m+mi}{00}     \PYG{l+m+mi}{1}
  \PYG{l+m+mf}{4.000}   \PYG{l+m+mf}{0.5459815003}\PYG{n}{D}\PYG{o}{+}\PYG{l+m+mi}{02}   \PYG{l+m+mf}{0.5459815003}\PYG{n}{D}\PYG{o}{+}\PYG{l+m+mi}{02}   \PYG{l+m+mf}{0.5205617665}\PYG{n}{D}\PYG{o}{\PYGZhy{}}\PYG{l+m+mi}{15}    \PYG{l+m+mi}{30}
  \PYG{l+m+mf}{8.000}   \PYG{l+m+mf}{0.2980957987}\PYG{n}{D}\PYG{o}{+}\PYG{l+m+mi}{04}   \PYG{l+m+mf}{0.2980957987}\PYG{n}{D}\PYG{o}{+}\PYG{l+m+mi}{04}   \PYG{l+m+mf}{0.1525507414}\PYG{n}{D}\PYG{o}{\PYGZhy{}}\PYG{l+m+mi}{15}    \PYG{l+m+mi}{42}
 \PYG{l+m+mf}{12.000}   \PYG{l+m+mf}{0.1627547914}\PYG{n}{D}\PYG{o}{+}\PYG{l+m+mi}{06}   \PYG{l+m+mf}{0.1627547914}\PYG{n}{D}\PYG{o}{+}\PYG{l+m+mi}{06}   \PYG{l+m+mf}{0.3576402292}\PYG{n}{D}\PYG{o}{\PYGZhy{}}\PYG{l+m+mi}{15}    \PYG{l+m+mi}{51}
 \PYG{l+m+mf}{16.000}   \PYG{l+m+mf}{0.8886110521}\PYG{n}{D}\PYG{o}{+}\PYG{l+m+mi}{07}   \PYG{l+m+mf}{0.8886110521}\PYG{n}{D}\PYG{o}{+}\PYG{l+m+mi}{07}   \PYG{l+m+mf}{0.0000000000}\PYG{n}{D}\PYG{o}{+}\PYG{l+m+mi}{00}    \PYG{l+m+mi}{59}
 \PYG{l+m+mf}{20.000}   \PYG{l+m+mf}{0.4851651954}\PYG{n}{D}\PYG{o}{+}\PYG{l+m+mi}{09}   \PYG{l+m+mf}{0.4851651954}\PYG{n}{D}\PYG{o}{+}\PYG{l+m+mi}{09}   \PYG{l+m+mf}{0.1228543295}\PYG{n}{D}\PYG{o}{\PYGZhy{}}\PYG{l+m+mi}{15}    \PYG{l+m+mi}{67}
\end{Verbatim}

Comments:
\begin{itemize}
\item {} 
Note the use of \titleref{exit} to break out of the loop.

\item {} 
Note that it is getting full machine precision for positive values of \titleref{x}
but, as expected, the accuracy suffers for negative \titleref{x} due to cancellation.

\end{itemize}


\section{Array storage in Fortran}
\label{fortran_arrays:array-storage-in-fortran}\label{fortran_arrays:fortran-arrays}\label{fortran_arrays::doc}
When an array is declared in Fortran, a set of storage locations in memory
are set aside for the storage of all the values in the array.  How many
bytes of memory this requires depends on how large the array is and what
data type each element has. If the array is declared \emph{allocatable} then the
declaration only determines the \emph{rank} of the array (the number of
indices it will have), and memory is not actually allocated until the
\titleref{allocate} statement is encountered.

By default, arrays in Fortran are indexed starting at 1. So if you declare:

\begin{Verbatim}[commandchars=\\\{\}]
\PYG{n}{real}\PYG{p}{(}\PYG{n}{kind}\PYG{o}{=}\PYG{l+m+mi}{8}\PYG{p}{)} \PYG{p}{:}\PYG{p}{:} \PYG{n}{x}\PYG{p}{(}\PYG{l+m+mi}{3}\PYG{p}{)}
\end{Verbatim}

or equivalently:

\begin{Verbatim}[commandchars=\\\{\}]
\PYG{n}{real}\PYG{p}{(}\PYG{n}{kind}\PYG{o}{=}\PYG{l+m+mi}{8}\PYG{p}{)}\PYG{p}{,} \PYG{n}{dimension}\PYG{p}{(}\PYG{l+m+mi}{3}\PYG{p}{)} \PYG{p}{:}\PYG{p}{:} \PYG{n}{x}
\end{Verbatim}

for example, then the elements should be referred to as \titleref{x(1), x(2),} and
\titleref{x(3)}.

You can also specify a different starting index, for example here are
several arrays of length 3 with different starting indices:

\begin{Verbatim}[commandchars=\\\{\}]
\PYG{n}{real}\PYG{p}{(}\PYG{n}{kind}\PYG{o}{=}\PYG{l+m+mi}{8}\PYG{p}{)} \PYG{p}{:}\PYG{p}{:} \PYG{n}{x}\PYG{p}{(}\PYG{l+m+mi}{0}\PYG{p}{:}\PYG{l+m+mi}{2}\PYG{p}{)}\PYG{p}{,} \PYG{n}{y}\PYG{p}{(}\PYG{l+m+mi}{4}\PYG{p}{:}\PYG{l+m+mi}{6}\PYG{p}{)}\PYG{p}{,} \PYG{n}{z}\PYG{p}{(}\PYG{o}{\PYGZhy{}}\PYG{l+m+mi}{2}\PYG{p}{:}\PYG{l+m+mi}{0}\PYG{p}{)}
\end{Verbatim}

A statement like
\begin{quote}

x(0) = z(-1)
\end{quote}

would then be valid.

Arrays in Fortran occupy successive memory locations starting at some
address in memory, say \titleref{istart}, and increasing by some constant number for
each element of the array.  For example, for an array of \titleref{real (kind=8)} values
the byte-address would increase by 8 for each successive element.  As
programmers we don't need to worry much about the actual addresses, but it
is important to understand how arrays are laid out in memory, particularly
if the rank of the array (number of indices) is larger than 1, as discussed
below in Section {\hyperref[fortran_arrays:fortran\string-arrays\string-hirank]{\crossref{\DUrole{std,std-ref}{Fortran arrays of higher rank}}}}.


\subsection{Passing rank 1 arrays to subroutines, Fortran 77 style}
\label{fortran_arrays:passing-rank-1-arrays-to-subroutines-fortran-77-style}
Even for arrays of rank 1, it is sometimes useful to remember that to a
Fortran compiler an array is often specified simply the the memory address
of the first component.  This helps explain the strange behavior of the
following program:

\begin{Verbatim}[commandchars=\\\{\},numbers=left,firstnumber=1,stepnumber=1]
\PYG{c}{! \PYGZdl{}UWHPSC/codes/fortran/arraypassing1.f90}

\PYG{k}{program }\PYG{n}{arraypassing1}

    \PYG{k}{implicit }\PYG{k}{none}
\PYG{k}{    }\PYG{k+kt}{real}\PYG{p}{(}\PYG{n+nb}{kind}\PYG{o}{=}\PYG{l+m+mi}{8}\PYG{p}{)} \PYG{k+kd}{::} \PYG{n}{x}\PYG{p}{,}\PYG{n}{y}
    \PYG{k+kt}{integer} \PYG{k+kd}{::} \PYG{n}{i}\PYG{p}{,}\PYG{n}{j}

    \PYG{n}{x} \PYG{o}{=} \PYG{l+m+mf}{1.}
    \PYG{n}{y} \PYG{o}{=} \PYG{l+m+mf}{2.}
    \PYG{n}{i} \PYG{o}{=} \PYG{l+m+mi}{3}
    \PYG{n}{j} \PYG{o}{=} \PYG{l+m+mi}{4}
    \PYG{k}{call }\PYG{n}{setvals}\PYG{p}{(}\PYG{n}{x}\PYG{p}{)}
    \PYG{k}{print} \PYG{o}{*}\PYG{p}{,} \PYG{l+s+s2}{\PYGZdq{}x = \PYGZdq{}}\PYG{p}{,}\PYG{n}{x}
    \PYG{k}{print} \PYG{o}{*}\PYG{p}{,} \PYG{l+s+s2}{\PYGZdq{}y = \PYGZdq{}}\PYG{p}{,}\PYG{n}{y}
    \PYG{k}{print} \PYG{o}{*}\PYG{p}{,} \PYG{l+s+s2}{\PYGZdq{}i = \PYGZdq{}}\PYG{p}{,}\PYG{n}{i}
    \PYG{k}{print} \PYG{o}{*}\PYG{p}{,} \PYG{l+s+s2}{\PYGZdq{}j = \PYGZdq{}}\PYG{p}{,}\PYG{n}{j}

\PYG{k}{end }\PYG{k}{program }\PYG{n}{arraypassing1}

\PYG{k}{subroutine }\PYG{n}{setvals}\PYG{p}{(}\PYG{n}{a}\PYG{p}{)}
    \PYG{c}{! subroutine that sets values in an array a of length 3.}
    \PYG{k}{implicit }\PYG{k}{none}
\PYG{k}{    }\PYG{k+kt}{real}\PYG{p}{(}\PYG{n+nb}{kind}\PYG{o}{=}\PYG{l+m+mi}{8}\PYG{p}{)}\PYG{p}{,} \PYG{k}{intent}\PYG{p}{(}\PYG{n}{inout}\PYG{p}{)} \PYG{k+kd}{::} \PYG{n}{a}\PYG{p}{(}\PYG{l+m+mi}{3}\PYG{p}{)}
    \PYG{k+kt}{integer }\PYG{n}{i}
    \PYG{k}{do }\PYG{n}{i} \PYG{o}{=} \PYG{l+m+mi}{1}\PYG{p}{,}\PYG{l+m+mi}{3}
        \PYG{n}{a}\PYG{p}{(}\PYG{n}{i}\PYG{p}{)} \PYG{o}{=} \PYG{l+m+mf}{5.}
        \PYG{n}{enddo}
\PYG{k}{end }\PYG{k}{subroutine }\PYG{n}{setvals}
\end{Verbatim}

which produces the output:

\begin{Verbatim}[commandchars=\\\{\}]
\PYG{n}{x} \PYG{o}{=}    \PYG{l+m+mf}{5.00000000000000}
\PYG{n}{y} \PYG{o}{=}    \PYG{l+m+mf}{5.00000000000000}
\PYG{n}{i} \PYG{o}{=}   \PYG{l+m+mi}{1075052544}
\PYG{n}{j} \PYG{o}{=}            \PYG{l+m+mi}{0}
\end{Verbatim}

The address of \titleref{x} is passed to the subroutine, which interprets it as the
address of the starting point of an array of length 3.  The subroutine sets
the value of \titleref{x} to 5 and also sets the values of the next 2 memory
locations (based on 8-byte real numbers) to 5.  Because \titleref{y}, \titleref{i}, and \titleref{j}
were declared after \titleref{x} and hence happen to occupy memory after \titleref{x},
these values are corrupted by the subroutine.

Note that integers are stored in 4 bytes so both \titleref{i} and \titleref{j} are covered by
the single 8-bytes interpreted as \titleref{a(3)}.  Storing a value as
an 8-byte float and then interpreting  the two halfs as
integers (when \titleref{i} and \titleref{j} are
printed) is likely to give nonsense.

It is also possible to make the code crash by improperly accessing memory.
(This is actually be better than just producing nonsense with no
warning, but figuring out \emph{why} the code crashed may be difficult.)

If you change the dimension of \titleref{a} from 3 to 1000 in the subroutine above:

\begin{Verbatim}[commandchars=\\\{\},numbers=left,firstnumber=1,stepnumber=1]
\PYG{c}{! \PYGZdl{}UWHPSC/codes/fortran/arraypassing2.f90}

\PYG{k}{program }\PYG{n}{arraypassing2}

    \PYG{k}{implicit }\PYG{k}{none}
\PYG{k}{    }\PYG{k+kt}{real}\PYG{p}{(}\PYG{n+nb}{kind}\PYG{o}{=}\PYG{l+m+mi}{8}\PYG{p}{)} \PYG{k+kd}{::} \PYG{n}{x}\PYG{p}{,}\PYG{n}{y}
    \PYG{k+kt}{integer} \PYG{k+kd}{::} \PYG{n}{i}\PYG{p}{,}\PYG{n}{j}

    \PYG{n}{x} \PYG{o}{=} \PYG{l+m+mf}{1.}
    \PYG{n}{y} \PYG{o}{=} \PYG{l+m+mf}{2.}
    \PYG{n}{i} \PYG{o}{=} \PYG{l+m+mi}{3}
    \PYG{n}{j} \PYG{o}{=} \PYG{l+m+mi}{4}
    \PYG{k}{call }\PYG{n}{setvals}\PYG{p}{(}\PYG{n}{x}\PYG{p}{)}
    \PYG{k}{print} \PYG{o}{*}\PYG{p}{,} \PYG{l+s+s2}{\PYGZdq{}x = \PYGZdq{}}\PYG{p}{,}\PYG{n}{x}
    \PYG{k}{print} \PYG{o}{*}\PYG{p}{,} \PYG{l+s+s2}{\PYGZdq{}y = \PYGZdq{}}\PYG{p}{,}\PYG{n}{y}
    \PYG{k}{print} \PYG{o}{*}\PYG{p}{,} \PYG{l+s+s2}{\PYGZdq{}i = \PYGZdq{}}\PYG{p}{,}\PYG{n}{i}
    \PYG{k}{print} \PYG{o}{*}\PYG{p}{,} \PYG{l+s+s2}{\PYGZdq{}j = \PYGZdq{}}\PYG{p}{,}\PYG{n}{j}

\PYG{k}{end }\PYG{k}{program }\PYG{n}{arraypassing2}

\PYG{k}{subroutine }\PYG{n}{setvals}\PYG{p}{(}\PYG{n}{a}\PYG{p}{)}
    \PYG{c}{! subroutine that sets values in an array a of length 1000.}
    \PYG{k}{implicit }\PYG{k}{none}
\PYG{k}{    }\PYG{k+kt}{real}\PYG{p}{(}\PYG{n+nb}{kind}\PYG{o}{=}\PYG{l+m+mi}{8}\PYG{p}{)}\PYG{p}{,} \PYG{k}{intent}\PYG{p}{(}\PYG{n}{inout}\PYG{p}{)} \PYG{k+kd}{::} \PYG{n}{a}\PYG{p}{(}\PYG{l+m+mi}{1000}\PYG{p}{)}
    \PYG{k+kt}{integer }\PYG{n}{i}
    \PYG{k}{do }\PYG{n}{i} \PYG{o}{=} \PYG{l+m+mi}{1}\PYG{p}{,}\PYG{l+m+mi}{1000}
        \PYG{n}{a}\PYG{p}{(}\PYG{n}{i}\PYG{p}{)} \PYG{o}{=} \PYG{l+m+mf}{5.}
        \PYG{n}{enddo}
\PYG{k}{end }\PYG{k}{subroutine }\PYG{n}{setvals}
\end{Verbatim}

then the code produces:

\begin{Verbatim}[commandchars=\\\{\}]
\PYG{n}{Segmentation} \PYG{n}{fault}
\end{Verbatim}

This means that the subroutine attempted to write into a memory location
that it should not have access to.  A small number of memory locations were
reserved for data when the variables \titleref{x,y,i,j} were declared.  No new memory
is allocated in the subroutine -- the statement

\begin{Verbatim}[commandchars=\\\{\}]
\PYG{n}{real}\PYG{p}{(}\PYG{n}{kind}\PYG{o}{=}\PYG{l+m+mi}{8}\PYG{p}{)}\PYG{p}{,} \PYG{n}{intent}\PYG{p}{(}\PYG{n}{inout}\PYG{p}{)} \PYG{p}{:}\PYG{p}{:} \PYG{n}{a}\PYG{p}{(}\PYG{l+m+mi}{1000}\PYG{p}{)}
\end{Verbatim}

simply declares a \titleref{dummy argument} of rank 1.  This statement could be
replaced by

\begin{Verbatim}[commandchars=\\\{\}]
\PYG{n}{real}\PYG{p}{(}\PYG{n}{kind}\PYG{o}{=}\PYG{l+m+mi}{8}\PYG{p}{)}\PYG{p}{,} \PYG{n}{intent}\PYG{p}{(}\PYG{n}{inout}\PYG{p}{)} \PYG{p}{:}\PYG{p}{:} \PYG{n}{a}\PYG{p}{(}\PYG{p}{:}\PYG{p}{)}
\end{Verbatim}

and the code would still compile and give the same results.  The starting
address of a set of storage locations is passed into the subroutine and the
location of any element in the array is computed from this, whether or not
it actually lies in the array as it was declared in the calling program!

The programs above are written in Fortran 77 style.
In Fortran 77, every subroutine or function is compiled independently of
others with no way to check that the arguments passed into a subroutine
actually agree in type with the dummy arguments used in the subroutine.
This is a limitation that often leads to debugging nightmares.

Luckily Fortran 90 can help reduce these problems, since it is possible to
create an \titleref{interface} that provides more information about the arguments
expected.  Here's one simple way to do this for the code above:

\begin{Verbatim}[commandchars=\\\{\},numbers=left,firstnumber=1,stepnumber=1]
\PYG{c}{! \PYGZdl{}UWHPSC/codes/fortran/arraypassing3.f90}

\PYG{k}{program }\PYG{n}{arraypassing3}

    \PYG{k}{implicit }\PYG{k}{none}
\PYG{k}{    }\PYG{k+kt}{real}\PYG{p}{(}\PYG{n+nb}{kind}\PYG{o}{=}\PYG{l+m+mi}{8}\PYG{p}{)} \PYG{k+kd}{::} \PYG{n}{x}\PYG{p}{,}\PYG{n}{y}
    \PYG{k+kt}{integer} \PYG{k+kd}{::} \PYG{n}{i}\PYG{p}{,}\PYG{n}{j}

    \PYG{n}{x} \PYG{o}{=} \PYG{l+m+mf}{1.}
    \PYG{n}{y} \PYG{o}{=} \PYG{l+m+mf}{2.}
    \PYG{n}{i} \PYG{o}{=} \PYG{l+m+mi}{3}
    \PYG{n}{j} \PYG{o}{=} \PYG{l+m+mi}{4}
    \PYG{k}{call }\PYG{n}{setvals}\PYG{p}{(}\PYG{n}{x}\PYG{p}{)}
    \PYG{k}{print} \PYG{o}{*}\PYG{p}{,} \PYG{l+s+s2}{\PYGZdq{}x = \PYGZdq{}}\PYG{p}{,}\PYG{n}{x}
    \PYG{k}{print} \PYG{o}{*}\PYG{p}{,} \PYG{l+s+s2}{\PYGZdq{}y = \PYGZdq{}}\PYG{p}{,}\PYG{n}{y}
    \PYG{k}{print} \PYG{o}{*}\PYG{p}{,} \PYG{l+s+s2}{\PYGZdq{}i = \PYGZdq{}}\PYG{p}{,}\PYG{n}{i}
    \PYG{k}{print} \PYG{o}{*}\PYG{p}{,} \PYG{l+s+s2}{\PYGZdq{}j = \PYGZdq{}}\PYG{p}{,}\PYG{n}{j}

\PYG{k}{contains}

\PYG{k}{subroutine }\PYG{n}{setvals}\PYG{p}{(}\PYG{n}{a}\PYG{p}{)}
    \PYG{c}{! subroutine that sets values in an array a of length 3.}
    \PYG{k}{implicit }\PYG{k}{none}
\PYG{k}{    }\PYG{k+kt}{real}\PYG{p}{(}\PYG{n+nb}{kind}\PYG{o}{=}\PYG{l+m+mi}{8}\PYG{p}{)}\PYG{p}{,} \PYG{k}{intent}\PYG{p}{(}\PYG{n}{inout}\PYG{p}{)} \PYG{k+kd}{::} \PYG{n}{a}\PYG{p}{(}\PYG{l+m+mi}{3}\PYG{p}{)}
    \PYG{k+kt}{integer }\PYG{n}{i}
    \PYG{k}{do }\PYG{n}{i} \PYG{o}{=} \PYG{l+m+mi}{1}\PYG{p}{,}\PYG{l+m+mi}{3}
        \PYG{n}{a}\PYG{p}{(}\PYG{n}{i}\PYG{p}{)} \PYG{o}{=} \PYG{l+m+mf}{5.}
        \PYG{n}{enddo}
\PYG{k}{end }\PYG{k}{subroutine }\PYG{n}{setvals}

\PYG{k}{end }\PYG{k}{program }\PYG{n}{arraypassing3}
\end{Verbatim}

Trying to compile this code produces an error message:

\begin{Verbatim}[commandchars=\\\{\}]
\PYGZdl{} gfortran arraypassing3.f90
arraypassing3.f90:14.17:

    call setvals(x)
                1
Error: Type/rank mismatch in argument \PYGZsq{}a\PYGZsq{} at (1)
\end{Verbatim}

Now at least the compiler recognizes that an array is expected
rather than a single
value.  But it is still possible to write a code that compiles and produces
nonsense by declaring \titleref{x} and \titleref{y} to be rank 1 arrays of length 1:

\begin{Verbatim}[commandchars=\\\{\},numbers=left,firstnumber=1,stepnumber=1]
\PYG{c}{! \PYGZdl{}UWHPSC/codes/fortran/arraypassing4.f90}

\PYG{k}{program }\PYG{n}{arraypassing4}

    \PYG{k}{implicit }\PYG{k}{none}
\PYG{k}{    }\PYG{k+kt}{real}\PYG{p}{(}\PYG{n+nb}{kind}\PYG{o}{=}\PYG{l+m+mi}{8}\PYG{p}{)}\PYG{p}{,} \PYG{k}{dimension}\PYG{p}{(}\PYG{l+m+mi}{1}\PYG{p}{)} \PYG{k+kd}{::} \PYG{n}{x}\PYG{p}{,}\PYG{n}{y}
    \PYG{k+kt}{integer} \PYG{k+kd}{::} \PYG{n}{i}\PYG{p}{,}\PYG{n}{j}

    \PYG{n}{x}\PYG{p}{(}\PYG{l+m+mi}{1}\PYG{p}{)} \PYG{o}{=} \PYG{l+m+mf}{1.}
    \PYG{n}{y}\PYG{p}{(}\PYG{l+m+mi}{1}\PYG{p}{)} \PYG{o}{=} \PYG{l+m+mf}{2.}
    \PYG{n}{i} \PYG{o}{=} \PYG{l+m+mi}{3}
    \PYG{n}{j} \PYG{o}{=} \PYG{l+m+mi}{4}
    \PYG{k}{call }\PYG{n}{setvals}\PYG{p}{(}\PYG{n}{x}\PYG{p}{)}
    \PYG{k}{print} \PYG{o}{*}\PYG{p}{,} \PYG{l+s+s2}{\PYGZdq{}x = \PYGZdq{}}\PYG{p}{,}\PYG{n}{x}\PYG{p}{(}\PYG{l+m+mi}{1}\PYG{p}{)}
    \PYG{k}{print} \PYG{o}{*}\PYG{p}{,} \PYG{l+s+s2}{\PYGZdq{}y = \PYGZdq{}}\PYG{p}{,}\PYG{n}{y}\PYG{p}{(}\PYG{l+m+mi}{1}\PYG{p}{)}
    \PYG{k}{print} \PYG{o}{*}\PYG{p}{,} \PYG{l+s+s2}{\PYGZdq{}i = \PYGZdq{}}\PYG{p}{,}\PYG{n}{i}
    \PYG{k}{print} \PYG{o}{*}\PYG{p}{,} \PYG{l+s+s2}{\PYGZdq{}j = \PYGZdq{}}\PYG{p}{,}\PYG{n}{j}

\PYG{k}{contains}

\PYG{k}{subroutine }\PYG{n}{setvals}\PYG{p}{(}\PYG{n}{a}\PYG{p}{)}
    \PYG{c}{! subroutine that sets values in an array a of length 3.}
    \PYG{k}{implicit }\PYG{k}{none}
\PYG{k}{    }\PYG{k+kt}{real}\PYG{p}{(}\PYG{n+nb}{kind}\PYG{o}{=}\PYG{l+m+mi}{8}\PYG{p}{)}\PYG{p}{,} \PYG{k}{intent}\PYG{p}{(}\PYG{n}{inout}\PYG{p}{)} \PYG{k+kd}{::} \PYG{n}{a}\PYG{p}{(}\PYG{l+m+mi}{3}\PYG{p}{)}
    \PYG{k+kt}{integer }\PYG{n}{i}
    \PYG{k}{do }\PYG{n}{i} \PYG{o}{=} \PYG{l+m+mi}{1}\PYG{p}{,}\PYG{l+m+mi}{3}
        \PYG{n}{a}\PYG{p}{(}\PYG{n}{i}\PYG{p}{)} \PYG{o}{=} \PYG{l+m+mf}{5.}
        \PYG{n}{enddo}
\PYG{k}{end }\PYG{k}{subroutine }\PYG{n}{setvals}

\PYG{k}{end }\PYG{k}{program }\PYG{n}{arraypassing4}
\end{Verbatim}

The compiler sees that an object of the correct type (a rank 1 array) is
being passed and does not give an error.  Running the code gives

\begin{Verbatim}[commandchars=\\\{\}]
\PYG{n}{x} \PYG{o}{=}    \PYG{l+m+mf}{5.00000000000000}
\PYG{n}{y} \PYG{o}{=}    \PYG{l+m+mf}{5.00000000000000}
\PYG{n}{i} \PYG{o}{=}   \PYG{l+m+mi}{1075052544}
\PYG{n}{j} \PYG{o}{=}            \PYG{l+m+mi}{0}
\end{Verbatim}

You might hope that using the gfortran flag \titleref{-fbounds-check} (see
{\hyperref[gfortran_flags:gfortran\string-flags]{\crossref{\DUrole{std,std-ref}{Useful gfortran flags}}}}) would catch such bugs.  Unfortunately it does not.  It
will catch it if you refer to \titleref{x(2)} in the main program of the code just
given, where it knows the length of \titleref{x} that was declared, but not in the
subroutine.


\subsection{Fortran arrays of higher rank}
\label{fortran_arrays:fortran-arrays-of-higher-rank}\label{fortran_arrays:fortran-arrays-hirank}
Suppose we declare \titleref{A} to be a rank 2 array with 3 rows and 4 columns,
which we might want to do to store a 3 by 4 matrix.
\begin{quote}

real(kind=8) :: A(3,4)
\end{quote}

Since memory is laid out linearly (each location has a single address, not a
set of indices), the compiler must decide how to map the 12
array elements to memory locations.

Unfortunately different languages use different conventions.  In Fortran
arrays are stored \emph{by column} in memory, so the 12 consecutive memory
locations would correspond to:

\begin{Verbatim}[commandchars=\\\{\}]
\PYG{n}{A}\PYG{p}{(}\PYG{l+m+mi}{1}\PYG{p}{,}\PYG{l+m+mi}{1}\PYG{p}{)}
\PYG{n}{A}\PYG{p}{(}\PYG{l+m+mi}{2}\PYG{p}{,}\PYG{l+m+mi}{1}\PYG{p}{)}
\PYG{n}{A}\PYG{p}{(}\PYG{l+m+mi}{3}\PYG{p}{,}\PYG{l+m+mi}{1}\PYG{p}{)}
\PYG{n}{A}\PYG{p}{(}\PYG{l+m+mi}{1}\PYG{p}{,}\PYG{l+m+mi}{2}\PYG{p}{)}
\PYG{n}{A}\PYG{p}{(}\PYG{l+m+mi}{2}\PYG{p}{,}\PYG{l+m+mi}{2}\PYG{p}{)}
\PYG{n}{A}\PYG{p}{(}\PYG{l+m+mi}{3}\PYG{p}{,}\PYG{l+m+mi}{2}\PYG{p}{)}
\PYG{n}{A}\PYG{p}{(}\PYG{l+m+mi}{1}\PYG{p}{,}\PYG{l+m+mi}{3}\PYG{p}{)}
\PYG{n}{A}\PYG{p}{(}\PYG{l+m+mi}{2}\PYG{p}{,}\PYG{l+m+mi}{3}\PYG{p}{)}
\PYG{n}{A}\PYG{p}{(}\PYG{l+m+mi}{3}\PYG{p}{,}\PYG{l+m+mi}{3}\PYG{p}{)}
\PYG{n}{A}\PYG{p}{(}\PYG{l+m+mi}{1}\PYG{p}{,}\PYG{l+m+mi}{4}\PYG{p}{)}
\PYG{n}{A}\PYG{p}{(}\PYG{l+m+mi}{2}\PYG{p}{,}\PYG{l+m+mi}{4}\PYG{p}{)}
\PYG{n}{A}\PYG{p}{(}\PYG{l+m+mi}{3}\PYG{p}{,}\PYG{l+m+mi}{4}\PYG{p}{)}
\end{Verbatim}

To illustrate this, consider the following program.  It declares an array
\titleref{A} of shape (3,4) and a rank-1 array \titleref{B} of length 12.  It also uses the
Fortran \titleref{equivalence} statement to tell the compiler that these two arrays
should point to the \emph{same} locations in memory.  This program shows that
\titleref{A(i,j)} is the same as \titleref{B(3*(j-1) + i)}:

\begin{Verbatim}[commandchars=\\\{\},numbers=left,firstnumber=1,stepnumber=1]
\PYG{c}{! \PYGZdl{}UWHPSC/codes/fortran/rank2.f90}

\PYG{k}{program }\PYG{n}{rank2}

    \PYG{k}{implicit }\PYG{k}{none}
\PYG{k}{    }\PYG{k+kt}{real}\PYG{p}{(}\PYG{n+nb}{kind}\PYG{o}{=}\PYG{l+m+mi}{8}\PYG{p}{)} \PYG{k+kd}{::} \PYG{n}{A}\PYG{p}{(}\PYG{l+m+mi}{3}\PYG{p}{,}\PYG{l+m+mi}{4}\PYG{p}{)}\PYG{p}{,} \PYG{n}{B}\PYG{p}{(}\PYG{l+m+mi}{12}\PYG{p}{)}
    \PYG{k}{equivalence} \PYG{p}{(}\PYG{n}{A}\PYG{p}{,}\PYG{n}{B}\PYG{p}{)}
    \PYG{k+kt}{integer} \PYG{k+kd}{::} \PYG{n}{i}\PYG{p}{,}\PYG{n}{j}

    \PYG{n}{A} \PYG{o}{=} \PYG{n+nb}{reshape}\PYG{p}{(}\PYG{p}{(}\PYG{o}{/}\PYG{p}{(}\PYG{l+m+mi}{10}\PYG{o}{*}\PYG{n}{i}\PYG{p}{,} \PYG{n}{i}\PYG{o}{=}\PYG{l+m+mi}{1}\PYG{p}{,}\PYG{l+m+mi}{12}\PYG{p}{)}\PYG{o}{/}\PYG{p}{)}\PYG{p}{,} \PYG{p}{(}\PYG{o}{/}\PYG{l+m+mi}{3}\PYG{p}{,}\PYG{l+m+mi}{4}\PYG{o}{/}\PYG{p}{)}\PYG{p}{)}

    \PYG{k}{do }\PYG{n}{i}\PYG{o}{=}\PYG{l+m+mi}{1}\PYG{p}{,}\PYG{l+m+mi}{3}
        \PYG{k}{print }\PYG{l+m+mi}{20}\PYG{p}{,} \PYG{n}{i}\PYG{p}{,} \PYG{p}{(}\PYG{n}{A}\PYG{p}{(}\PYG{n}{i}\PYG{p}{,}\PYG{n}{j}\PYG{p}{)}\PYG{p}{,} \PYG{n}{j}\PYG{o}{=}\PYG{l+m+mi}{1}\PYG{p}{,}\PYG{l+m+mi}{4}\PYG{p}{)}
 \PYG{l+m+mi}{20}     \PYG{k}{format}\PYG{p}{(}\PYG{l+s+s2}{\PYGZdq{}Row \PYGZdq{}}\PYG{p}{,}\PYG{n}{i1}\PYG{p}{,}\PYG{l+s+s2}{\PYGZdq{} of A contains: \PYGZdq{}}\PYG{p}{,} \PYG{l+m+mi}{11}\PYG{n}{x}\PYG{p}{,} \PYG{l+m+mi}{4}\PYG{n}{f6}\PYG{p}{.}\PYG{l+m+mi}{1}\PYG{p}{)}
        \PYG{k}{print }\PYG{l+m+mi}{21}\PYG{p}{,} \PYG{n}{i}\PYG{p}{,} \PYG{p}{(}\PYG{l+m+mi}{3}\PYG{o}{*}\PYG{p}{(}\PYG{n}{j}\PYG{o}{\PYGZhy{}}\PYG{l+m+mi}{1}\PYG{p}{)}\PYG{o}{+}\PYG{n}{i}\PYG{p}{,} \PYG{n}{j}\PYG{o}{=}\PYG{l+m+mi}{1}\PYG{p}{,}\PYG{l+m+mi}{4}\PYG{p}{)}
 \PYG{l+m+mi}{21}     \PYG{k}{format}\PYG{p}{(}\PYG{l+s+s2}{\PYGZdq{}Row \PYGZdq{}}\PYG{p}{,}\PYG{n}{i1}\PYG{p}{,}\PYG{l+s+s2}{\PYGZdq{} is in locations\PYGZdq{}}\PYG{p}{,}\PYG{l+m+mi}{4}\PYG{n}{i3}\PYG{p}{)}
        \PYG{k}{print }\PYG{l+m+mi}{22}\PYG{p}{,} \PYG{p}{(}\PYG{n}{B}\PYG{p}{(}\PYG{l+m+mi}{3}\PYG{o}{*}\PYG{p}{(}\PYG{n}{j}\PYG{o}{\PYGZhy{}}\PYG{l+m+mi}{1}\PYG{p}{)}\PYG{o}{+}\PYG{n}{i}\PYG{p}{)}\PYG{p}{,} \PYG{n}{j}\PYG{o}{=}\PYG{l+m+mi}{1}\PYG{p}{,}\PYG{l+m+mi}{4}\PYG{p}{)}
 \PYG{l+m+mi}{22}     \PYG{k}{format}\PYG{p}{(}\PYG{l+s+s2}{\PYGZdq{}These elements of B contain:\PYGZdq{}}\PYG{p}{,} \PYG{l+m+mi}{4}\PYG{n}{x}\PYG{p}{,} \PYG{l+m+mi}{4}\PYG{n}{f6}\PYG{p}{.}\PYG{l+m+mi}{1}\PYG{p}{,} \PYG{o}{/}\PYG{p}{)}
        \PYG{n}{enddo}

\PYG{k}{end }\PYG{k}{program }\PYG{n}{rank2}
\end{Verbatim}

This program produces:

\begin{Verbatim}[commandchars=\\\{\}]
\PYG{n}{Row} \PYG{l+m+mi}{1} \PYG{n}{of} \PYG{n}{A} \PYG{n}{contains}\PYG{p}{:}              \PYG{l+m+mf}{10.0}  \PYG{l+m+mf}{40.0}  \PYG{l+m+mf}{70.0} \PYG{l+m+mf}{100.0}
\PYG{n}{Row} \PYG{l+m+mi}{1} \PYG{o+ow}{is} \PYG{o+ow}{in} \PYG{n}{locations}  \PYG{l+m+mi}{1}  \PYG{l+m+mi}{4}  \PYG{l+m+mi}{7} \PYG{l+m+mi}{10}
\PYG{n}{These} \PYG{n}{elements} \PYG{n}{of} \PYG{n}{B} \PYG{n}{contain}\PYG{p}{:}      \PYG{l+m+mf}{10.0}  \PYG{l+m+mf}{40.0}  \PYG{l+m+mf}{70.0} \PYG{l+m+mf}{100.0}

\PYG{n}{Row} \PYG{l+m+mi}{2} \PYG{n}{of} \PYG{n}{A} \PYG{n}{contains}\PYG{p}{:}              \PYG{l+m+mf}{20.0}  \PYG{l+m+mf}{50.0}  \PYG{l+m+mf}{80.0} \PYG{l+m+mf}{110.0}
\PYG{n}{Row} \PYG{l+m+mi}{2} \PYG{o+ow}{is} \PYG{o+ow}{in} \PYG{n}{locations}  \PYG{l+m+mi}{2}  \PYG{l+m+mi}{5}  \PYG{l+m+mi}{8} \PYG{l+m+mi}{11}
\PYG{n}{These} \PYG{n}{elements} \PYG{n}{of} \PYG{n}{B} \PYG{n}{contain}\PYG{p}{:}      \PYG{l+m+mf}{20.0}  \PYG{l+m+mf}{50.0}  \PYG{l+m+mf}{80.0} \PYG{l+m+mf}{110.0}

\PYG{n}{Row} \PYG{l+m+mi}{3} \PYG{n}{of} \PYG{n}{A} \PYG{n}{contains}\PYG{p}{:}              \PYG{l+m+mf}{30.0}  \PYG{l+m+mf}{60.0}  \PYG{l+m+mf}{90.0} \PYG{l+m+mf}{120.0}
\PYG{n}{Row} \PYG{l+m+mi}{3} \PYG{o+ow}{is} \PYG{o+ow}{in} \PYG{n}{locations}  \PYG{l+m+mi}{3}  \PYG{l+m+mi}{6}  \PYG{l+m+mi}{9} \PYG{l+m+mi}{12}
\PYG{n}{These} \PYG{n}{elements} \PYG{n}{of} \PYG{n}{B} \PYG{n}{contain}\PYG{p}{:}      \PYG{l+m+mf}{30.0}  \PYG{l+m+mf}{60.0}  \PYG{l+m+mf}{90.0} \PYG{l+m+mf}{120.0}
\end{Verbatim}

Note also that the \titleref{reshape} command used in Line 10 of this program takes
the set of values and assigns them to elements of the array \emph{by columns}.
Actually it just puts these values into the 12 memory elements used for the
matrix one after another, but because of the way arrays are interpreted,
this corresponds to filling the array by columns.

Note some other things about this program:
\begin{itemize}
\item {} 
Lines 10, 13, 15, and 17 use an \titleref{implied do} construct, in which \titleref{i} or \titleref{j}
loops over the values specified.

\item {} 
Lines 14, 16, and 18 are \emph{format statements} and these formats are used
in the lines preceding them instead of the default format \titleref{*}.
For more about formatted I/O see {\hyperref[fortran_io:fortran\string-io]{\crossref{\DUrole{std,std-ref}{Fortran Input / Output}}}}.

\end{itemize}

In C or Python, storage is \emph{by rows} instead, so the 12 consecutive
memorylocations would correspond to:

\begin{Verbatim}[commandchars=\\\{\}]
\PYG{n}{A}\PYG{p}{(}\PYG{l+m+mi}{1}\PYG{p}{,}\PYG{l+m+mi}{1}\PYG{p}{)}
\PYG{n}{A}\PYG{p}{(}\PYG{l+m+mi}{1}\PYG{p}{,}\PYG{l+m+mi}{2}\PYG{p}{)}
\PYG{n}{A}\PYG{p}{(}\PYG{l+m+mi}{1}\PYG{p}{,}\PYG{l+m+mi}{3}\PYG{p}{)}
\PYG{n}{A}\PYG{p}{(}\PYG{l+m+mi}{2}\PYG{p}{,}\PYG{l+m+mi}{1}\PYG{p}{)}
\PYG{n}{etc}\PYG{o}{.}
\end{Verbatim}


\subsection{Why do we care how arrays are stored?}
\label{fortran_arrays:why-do-we-care-how-arrays-are-stored}
The layout of arrays in memory
is often important to know when doing high-performance computing, as we
will see when we discuss cache properties.

It is also important to know this in order to understand older Fortran
programs.  When an array of rank 2 is passed to a subroutine, the subroutine
needs to know not only that the array has rank 2, but also what the \emph{leading
dimension} of the array is, the number of rows.  In older codes this value
is often passed in to a subroutine along with the array.  In Fortran 90 this
can be avoided if there is a suitable interface, for example if the
subroutine is in a module.

As we saw above, the \titleref{(i,j)} element of the 3 by 4 array \titleref{A} is in location
\titleref{(3*(j-1) + i)}.  The same would be true for a 3 by 5 array or a 3 by 1000
array.  More generally, if the array is \titleref{nrows} by \titleref{ncols}, then the \titleref{(i,j)}
element is in location \titleref{nrows*(j-1) + i} and so the value of \titleref{nrows} must be
known by the compiler in order to translate the indices \titleref{(i,j)} into the
propoer storage location.


\section{Fortran modules}
\label{fortran_modules:fortran-modules}\label{fortran_modules::doc}\label{fortran_modules:id1}
The general structure of a Fortran module:

\begin{Verbatim}[commandchars=\\\{\}]
module \PYGZlt{}MODULE\PYGZhy{}NAME\PYGZgt{}
    ! Declare variables
contains
    ! Define subroutines or functions
end module \PYGZlt{}MODULE\PYGZhy{}NAME\PYGZgt{}
\end{Verbatim}

A program or subroutine can \emph{use} this module:

\begin{Verbatim}[commandchars=\\\{\}]
program \PYGZlt{}NAME\PYGZgt{}
    use \PYGZlt{}MODULE\PYGZhy{}NAME\PYGZgt{}
    ! Declare variables
    ! Executable statements
end program \PYGZlt{}NAME\PYGZgt{}
\end{Verbatim}

The line:

\begin{Verbatim}[commandchars=\\\{\}]
\PYG{n}{use} \PYG{o}{\PYGZlt{}}\PYG{n}{MODULE}\PYG{o}{\PYGZhy{}}\PYG{n}{NAME}\PYG{o}{\PYGZgt{}}
\end{Verbatim}

can be replaced by:

\begin{Verbatim}[commandchars=\\\{\}]
\PYG{n}{use} \PYG{o}{\PYGZlt{}}\PYG{n}{MODULE}\PYG{o}{\PYGZhy{}}\PYG{n}{NAME}\PYG{o}{\PYGZgt{}}\PYG{p}{,} \PYG{n}{only}\PYG{p}{:} \PYG{o}{\PYGZlt{}}\PYG{n}{LIST} \PYG{n}{OF} \PYG{n}{SYMBOLS}\PYG{o}{\PYGZgt{}}
\end{Verbatim}

to specify that only certain variables/subroutines/functions from the module
should be used.  Doing it this way also makes it clear exactly what symbols
are coming from which module in the case where you \emph{use} several modules.

A very simple module is:

\begin{Verbatim}[commandchars=\\\{\},numbers=left,firstnumber=1,stepnumber=1]
\PYG{c}{! \PYGZdl{}UWHPSC/codes/fortran/multifile2/sub1m.f90}

\PYG{k}{module }\PYG{n}{sub1m}

\PYG{k}{contains}

\PYG{k}{subroutine }\PYG{n}{sub1}\PYG{p}{(}\PYG{p}{)}
    \PYG{k}{print} \PYG{o}{*}\PYG{p}{,} \PYG{l+s+s2}{\PYGZdq{}In sub1\PYGZdq{}}
\PYG{k}{end }\PYG{k}{subroutine }\PYG{n}{sub1}

\PYG{k}{end }\PYG{k}{module }\PYG{n}{sub1m}
\end{Verbatim}

and a program that uses this:

\begin{Verbatim}[commandchars=\\\{\},numbers=left,firstnumber=1,stepnumber=1]
\PYG{c}{! \PYGZdl{}UWHPSC/codes/fortran/multifile2/main.f90}

\PYG{k}{program }\PYG{n}{demo}
    \PYG{k}{use }\PYG{n}{sub1m}\PYG{p}{,} \PYG{n}{only}\PYG{p}{:} \PYG{n}{sub1}
    \PYG{k}{print} \PYG{o}{*}\PYG{p}{,} \PYG{l+s+s2}{\PYGZdq{}In main program\PYGZdq{}}
    \PYG{k}{call }\PYG{n}{sub1}\PYG{p}{(}\PYG{p}{)}
\PYG{k}{end }\PYG{k}{program }\PYG{n}{demo}
\end{Verbatim}


\subsection{Some reasons to use modules}
\label{fortran_modules:some-reasons-to-use-modules}\begin{itemize}
\item {} 
Can define global variables in modules to be used in several different
routines.

In Fortran 77 this had to be done with common blocks — much less elegant.

\item {} 
Subroutine/function interface information is generated to aid in checking
that proper arguments are passed.

It’s often best to put all subroutines and functions in modules for this
reason.

\item {} 
Can define new data types to be used in several routines.

\end{itemize}


\subsection{Compiling modules}
\label{fortran_modules:compiling-modules}
Modules must be compiled \emph{before} any program units that \emph{use} the module.
When a module is compiled, a \titleref{.o} file is created, but also a \titleref{.mod} file is
created that must be present in order to compile a unit that \emph{uses} the
module.


\subsection{Circles module example}
\label{fortran_modules:circles-module-example}
Here is an example of a module that defines one parameter \titleref{pi} and
two functions:

\begin{Verbatim}[commandchars=\\\{\},numbers=left,firstnumber=1,stepnumber=1]
\PYG{c}{! \PYGZdl{}UWHPSC/codes/fortran/circles/circle\PYGZus{}mod.f90}

\PYG{k}{module }\PYG{n}{circle\PYGZus{}mod}

    \PYG{k}{implicit }\PYG{k}{none}
\PYG{k}{    }\PYG{k+kt}{real}\PYG{p}{(}\PYG{n+nb}{kind}\PYG{o}{=}\PYG{l+m+mi}{8}\PYG{p}{)}\PYG{p}{,} \PYG{k}{parameter} \PYG{k+kd}{::} \PYG{n}{pi} \PYG{o}{=} \PYG{l+m+mf}{3.141592653589793}\PYG{n}{d0}

\PYG{k}{contains}

\PYG{k}{    }\PYG{k+kt}{real}\PYG{p}{(}\PYG{n+nb}{kind}\PYG{o}{=}\PYG{l+m+mi}{8}\PYG{p}{)} \PYG{k}{function }\PYG{n}{area}\PYG{p}{(}\PYG{n}{r}\PYG{p}{)}
        \PYG{k+kt}{real}\PYG{p}{(}\PYG{n+nb}{kind}\PYG{o}{=}\PYG{l+m+mi}{8}\PYG{p}{)}\PYG{p}{,} \PYG{k}{intent}\PYG{p}{(}\PYG{n}{in}\PYG{p}{)} \PYG{k+kd}{::} \PYG{n}{r}
        \PYG{n}{area} \PYG{o}{=} \PYG{n}{pi} \PYG{o}{*} \PYG{n}{r}\PYG{o}{**}\PYG{l+m+mi}{2}
    \PYG{k}{end }\PYG{k}{function }\PYG{n}{area}

    \PYG{k+kt}{real}\PYG{p}{(}\PYG{n+nb}{kind}\PYG{o}{=}\PYG{l+m+mi}{8}\PYG{p}{)} \PYG{k}{function }\PYG{n}{circumference}\PYG{p}{(}\PYG{n}{r}\PYG{p}{)}
        \PYG{k+kt}{real}\PYG{p}{(}\PYG{n+nb}{kind}\PYG{o}{=}\PYG{l+m+mi}{8}\PYG{p}{)}\PYG{p}{,} \PYG{k}{intent}\PYG{p}{(}\PYG{n}{in}\PYG{p}{)} \PYG{k+kd}{::} \PYG{n}{r}
        \PYG{n}{circumference} \PYG{o}{=} \PYG{l+m+mf}{2.}\PYG{n}{d0} \PYG{o}{*} \PYG{n}{pi} \PYG{o}{*} \PYG{n}{r}
    \PYG{k}{end }\PYG{k}{function }\PYG{n}{circumference}

\PYG{k}{end }\PYG{k}{module }\PYG{n}{circle\PYGZus{}mod}
\end{Verbatim}

This might be used as follows:

\begin{Verbatim}[commandchars=\\\{\},numbers=left,firstnumber=1,stepnumber=1]
\PYG{c}{! \PYGZdl{}UWHPSC/codes/fortran/circles/main.f90}

\PYG{k}{program }\PYG{n}{main}

    \PYG{k}{use }\PYG{n}{circle\PYGZus{}mod}\PYG{p}{,} \PYG{n}{only}\PYG{p}{:} \PYG{n}{pi}\PYG{p}{,} \PYG{n}{area}
    \PYG{k}{implicit }\PYG{k}{none}
\PYG{k}{    }\PYG{k+kt}{real}\PYG{p}{(}\PYG{n+nb}{kind}\PYG{o}{=}\PYG{l+m+mi}{8}\PYG{p}{)} \PYG{k+kd}{::} \PYG{n}{a}

    \PYG{c}{! print parameter pi defined in module:}
    \PYG{k}{print} \PYG{o}{*}\PYG{p}{,} \PYG{l+s+s1}{\PYGZsq{}pi = \PYGZsq{}}\PYG{p}{,} \PYG{n}{pi}

    \PYG{c}{! test the area function from module:}
    \PYG{n}{a} \PYG{o}{=} \PYG{n}{area}\PYG{p}{(}\PYG{l+m+mf}{2.}\PYG{n}{d0}\PYG{p}{)}
    \PYG{k}{print} \PYG{o}{*}\PYG{p}{,} \PYG{l+s+s1}{\PYGZsq{}area for a circle of radius 2: \PYGZsq{}}\PYG{p}{,} \PYG{n}{a}

\PYG{k}{end }\PYG{k}{program }\PYG{n}{main}
\end{Verbatim}

This gives the following output:

\begin{Verbatim}[commandchars=\\\{\}]
\PYG{n}{pi} \PYG{o}{=}    \PYG{l+m+mf}{3.14159265358979}
\PYG{n}{area} \PYG{k}{for} \PYG{n}{a} \PYG{n}{circle} \PYG{n}{of} \PYG{n}{radius} \PYG{l+m+mi}{2}\PYG{p}{:}    \PYG{l+m+mf}{12.5663706143592}
\end{Verbatim}

Note: that a parameter can be defined with a specific value that will then be
available to all program units using the module.


\subsection{Module variables}
\label{fortran_modules:module-variables}\label{fortran_modules:id2}
It is also possible to declare \emph{variables} that can be shared between all
program units using the module.  This is a way to define ``global variables''
that might be set in one program unit and used in another, without the need
to pass the variable as a subroutine or function argument.
Module variables can be defined in a module and the Fortran statement

\begin{Verbatim}[commandchars=\\\{\}]
\PYG{n}{save}
\end{Verbatim}

is used to indicate that variables defined in the module should have values
saved between one use of the module to another.  You should generally
specify this if you use any module variables.

Here is another version of the circles code that stores \emph{pi} as a module
variable rather than a parameter:

\begin{Verbatim}[commandchars=\\\{\},numbers=left,firstnumber=1,stepnumber=1]
\PYG{c}{! \PYGZdl{}UWHPSC/codes/fortran/circles/circle\PYGZus{}mod.f90}
\PYG{c}{! Version where pi is a module variable.}

\PYG{k}{module }\PYG{n}{circle\PYGZus{}mod}

    \PYG{k}{implicit }\PYG{k}{none}
\PYG{k}{    }\PYG{k+kt}{real}\PYG{p}{(}\PYG{n+nb}{kind}\PYG{o}{=}\PYG{l+m+mi}{8}\PYG{p}{)} \PYG{k+kd}{::} \PYG{n}{pi} 
    \PYG{k}{save  }

\PYG{k}{contains}

\PYG{k}{    }\PYG{k+kt}{real}\PYG{p}{(}\PYG{n+nb}{kind}\PYG{o}{=}\PYG{l+m+mi}{8}\PYG{p}{)} \PYG{k}{function }\PYG{n}{area}\PYG{p}{(}\PYG{n}{r}\PYG{p}{)}
        \PYG{k+kt}{real}\PYG{p}{(}\PYG{n+nb}{kind}\PYG{o}{=}\PYG{l+m+mi}{8}\PYG{p}{)}\PYG{p}{,} \PYG{k}{intent}\PYG{p}{(}\PYG{n}{in}\PYG{p}{)} \PYG{k+kd}{::} \PYG{n}{r}
        \PYG{n}{area} \PYG{o}{=} \PYG{n}{pi} \PYG{o}{*} \PYG{n}{r}\PYG{o}{**}\PYG{l+m+mi}{2}
    \PYG{k}{end }\PYG{k}{function }\PYG{n}{area}

    \PYG{k+kt}{real}\PYG{p}{(}\PYG{n+nb}{kind}\PYG{o}{=}\PYG{l+m+mi}{8}\PYG{p}{)} \PYG{k}{function }\PYG{n}{circumference}\PYG{p}{(}\PYG{n}{r}\PYG{p}{)}
        \PYG{k+kt}{real}\PYG{p}{(}\PYG{n+nb}{kind}\PYG{o}{=}\PYG{l+m+mi}{8}\PYG{p}{)}\PYG{p}{,} \PYG{k}{intent}\PYG{p}{(}\PYG{n}{in}\PYG{p}{)} \PYG{k+kd}{::} \PYG{n}{r}
        \PYG{n}{circumference} \PYG{o}{=} \PYG{l+m+mf}{2.}\PYG{n}{d0} \PYG{o}{*} \PYG{n}{pi} \PYG{o}{*} \PYG{n}{r}
    \PYG{k}{end }\PYG{k}{function }\PYG{n}{circumference}

\PYG{k}{end }\PYG{k}{module }\PYG{n}{circle\PYGZus{}mod}
\end{Verbatim}

In this case we also need to initialize the variable \emph{pi} by means of a
subroutine such as:

\begin{Verbatim}[commandchars=\\\{\},numbers=left,firstnumber=1,stepnumber=1]
\PYG{c}{! \PYGZdl{}UWHPSC/codes/fortran/circles/initialize.f90}

\PYG{k}{subroutine }\PYG{n}{initialize}\PYG{p}{(}\PYG{p}{)}

    \PYG{c}{! Set the value of pi used elsewhere.}
    \PYG{k}{use }\PYG{n}{circle\PYGZus{}mod}\PYG{p}{,} \PYG{n}{only}\PYG{p}{:} \PYG{n}{pi}
    \PYG{n}{pi} \PYG{o}{=} \PYG{n+nb}{acos}\PYG{p}{(}\PYG{o}{\PYGZhy{}}\PYG{l+m+mf}{1.}\PYG{n}{d0}\PYG{p}{)}

\PYG{k}{end }\PYG{k}{subroutine }\PYG{n}{initialize}
\end{Verbatim}

These might be used as follows in a main program:

\begin{Verbatim}[commandchars=\\\{\},numbers=left,firstnumber=1,stepnumber=1]
\PYG{c}{! \PYGZdl{}UWHPSC/codes/fortran/circles/main.f90}

\PYG{k}{program }\PYG{n}{main}

    \PYG{k}{use }\PYG{n}{circle\PYGZus{}mod}\PYG{p}{,} \PYG{n}{only}\PYG{p}{:} \PYG{n}{pi}\PYG{p}{,} \PYG{n}{area}
    \PYG{k}{implicit }\PYG{k}{none}
\PYG{k}{    }\PYG{k+kt}{real}\PYG{p}{(}\PYG{n+nb}{kind}\PYG{o}{=}\PYG{l+m+mi}{8}\PYG{p}{)} \PYG{k+kd}{::} \PYG{n}{a}

    \PYG{k}{call }\PYG{n}{initialize}\PYG{p}{(}\PYG{p}{)}   \PYG{c}{! sets pi}

    \PYG{c}{! print module variable pi:}
    \PYG{k}{print} \PYG{o}{*}\PYG{p}{,} \PYG{l+s+s1}{\PYGZsq{}pi = \PYGZsq{}}\PYG{p}{,} \PYG{n}{pi}

    \PYG{c}{! test the area function from module:}
    \PYG{n}{a} \PYG{o}{=} \PYG{n}{area}\PYG{p}{(}\PYG{l+m+mf}{2.}\PYG{n}{d0}\PYG{p}{)}
    \PYG{k}{print} \PYG{o}{*}\PYG{p}{,} \PYG{l+s+s1}{\PYGZsq{}area for a circle of radius 2: \PYGZsq{}}\PYG{p}{,} \PYG{n}{a}

\PYG{k}{end }\PYG{k}{program }\PYG{n}{main}
\end{Verbatim}

This example can be compiled and executed by going into the directory
\titleref{\$UWHPSC/fortran/circles2/} and typing:

\begin{Verbatim}[commandchars=\\\{\}]
\PYGZdl{} gfortran circle\PYGZus{}mod.f90 initialize.f90 main.f90 \PYGZhy{}o main.exe
\PYGZdl{} ./main.exe
\end{Verbatim}

Or by using the Makefile in this directory:

\begin{Verbatim}[commandchars=\\\{\}]
\PYGZdl{} make main.exe
\PYGZdl{} ./main.exe
\end{Verbatim}

Here is the Makefile:

\begin{Verbatim}[commandchars=\\\{\},numbers=left,firstnumber=1,stepnumber=1]
\PYG{c}{\PYGZsh{} \PYGZdl{}UWHPSC/codes/fortran/circles2/Makefile}

\PYG{n+nv}{OBJECTS} \PYG{o}{=} circle\PYGZus{}mod.o \PYG{l+s+se}{\PYGZbs{}}
          main.o \PYG{l+s+se}{\PYGZbs{}}
          initialize.o

\PYG{n+nv}{MODULES} \PYG{o}{=} circle\PYGZus{}mod.mod

\PYG{n+nf}{.PHONY}\PYG{o}{:} \PYG{n}{clean}

\PYG{n+nf}{output.txt}\PYG{o}{:} \PYG{n}{main}.\PYG{n}{exe}
	./main.exe \PYGZgt{} output.txt

\PYG{n+nf}{main.exe}\PYG{o}{:} \PYG{k}{\PYGZdl{}(}\PYG{n+nv}{MODULES}\PYG{k}{)} \PYG{k}{\PYGZdl{}(}\PYG{n+nv}{OBJECTS}\PYG{k}{)}
	gfortran \PYG{k}{\PYGZdl{}(}OBJECTS\PYG{k}{)} \PYGZhy{}o main.exe

\PYG{n+nf}{\PYGZpc{}.o}\PYG{o}{:} \PYGZpc{}.\PYG{n}{f}90
	gfortran \PYGZhy{}c \PYG{n+nv}{\PYGZdl{}\PYGZlt{}}

\PYG{n+nf}{\PYGZpc{}.mod}\PYG{o}{:} \PYGZpc{}.\PYG{n}{f}90
	gfortran \PYGZhy{}c \PYG{n+nv}{\PYGZdl{}\PYGZlt{}}

\PYG{n+nf}{clean}\PYG{o}{:}
	rm \PYGZhy{}f \PYG{k}{\PYGZdl{}(}OBJECTS\PYG{k}{)} \PYG{k}{\PYGZdl{}(}MODULES\PYG{k}{)} main.exe
\end{Verbatim}

For more about Makefiles, see {\hyperref[makefiles:makefiles]{\crossref{\DUrole{std,std-ref}{Makefiles}}}} and {\hyperref[biblio:biblio\string-make]{\crossref{\DUrole{std,std-ref}{Makefile references}}}}.


\section{Fortran Input / Output}
\label{fortran_io:fortran-io}\label{fortran_io::doc}\label{fortran_io:fortran-input-output}

\subsection{Formats vs. unformatted}
\label{fortran_io:formats-vs-unformatted}
\titleref{print} or \titleref{write} statements for output and \titleref{read} statements for input can
specify a \emph{format} or can be \emph{unformatted}.

For example,

\begin{Verbatim}[commandchars=\\\{\}]
\PYG{n+nb}{print} \PYG{o}{*}\PYG{p}{,} \PYG{l+s}{\PYGZsq{}}\PYG{l+s}{x = }\PYG{l+s}{\PYGZsq{}}\PYG{p}{,} \PYG{n}{x}
\end{Verbatim}

is an \emph{unformatted} print statement that prints a character string followed
by the value of a variable \titleref{x}.  The format used to print \titleref{x} (e.g. the
number of digits shown, the number of spaces in front) will be chosen
by the compiler based on what type of variable \titleref{x} is.

The statements:

\begin{Verbatim}[commandchars=\\\{\}]
\PYG{n}{i} \PYG{o}{=} \PYG{l+m+mi}{4}
\PYG{n}{x} \PYG{o}{=} \PYG{l+m+mf}{2.}\PYG{n}{d0} \PYG{o}{/} \PYG{l+m+mf}{3.}\PYG{n}{d0}
\PYG{n+nb}{print} \PYG{o}{*}\PYG{p}{,} \PYG{l+s}{\PYGZsq{}}\PYG{l+s}{i = }\PYG{l+s}{\PYGZsq{}}\PYG{p}{,} \PYG{n}{i}\PYG{p}{,} \PYG{l+s}{\PYGZsq{}}\PYG{l+s}{ and x = }\PYG{l+s}{\PYGZsq{}}\PYG{p}{,} \PYG{n}{x}
\end{Verbatim}

yield:

\begin{Verbatim}[commandchars=\\\{\}]
\PYG{n}{i} \PYG{o}{=}            \PYG{l+m+mi}{4}  \PYG{o+ow}{and} \PYG{n}{x} \PYG{o}{=}   \PYG{l+m+mf}{0.666666666666667}
\end{Verbatim}

The * in the print statement tells the compiler to choose the format.

To have more control over the format, a \emph{formatted print} statement can be
used.  A format can be placed directly in the statement in place of the * ,
or can be
written separately with a label, and the label number used in the \titleref{print}
statement.

For example, if we wish to display the integer \titleref{i} in a \emph{field} of
3 spaces and
print \titleref{x} in scientific notation with 12 digits of the mantissa displayed,
in a \emph{field} that is 18 digits wide, we could do

\begin{Verbatim}[commandchars=\\\{\}]
    \PYG{n+nb}{print} \PYG{l+m+mi}{600}\PYG{p}{,} \PYG{n}{i}\PYG{p}{,} \PYG{n}{x}
\PYG{l+m+mi}{600} \PYG{n+nb}{format}\PYG{p}{(}\PYG{l+s}{\PYGZsq{}}\PYG{l+s}{i = }\PYG{l+s}{\PYGZsq{}}\PYG{p}{,}\PYG{n}{i3}\PYG{p}{,}\PYG{l+s}{\PYGZsq{}}\PYG{l+s}{ and x = }\PYG{l+s}{\PYGZsq{}}\PYG{p}{,} \PYG{n}{e17}\PYG{o}{.}\PYG{l+m+mi}{10}\PYG{p}{)}
\end{Verbatim}

This yields:

\begin{Verbatim}[commandchars=\\\{\}]
\PYG{n}{i} \PYG{o}{=}   \PYG{l+m+mi}{4} \PYG{o+ow}{and} \PYG{n}{x} \PYG{o}{=}  \PYG{l+m+mf}{0.6666666667E+00}
\end{Verbatim}

The 4 is right-justified in a field of 3 characters after the `i = `
string.

Note that if the number doesn't fit in the field, asterisks will be printed
instead!

\begin{Verbatim}[commandchars=\\\{\}]
\PYG{n}{i} \PYG{o}{=} \PYG{l+m+mi}{4000}
\PYG{n+nb}{print} \PYG{l+m+mi}{600}\PYG{p}{,} \PYG{n}{i}\PYG{p}{,} \PYG{n}{x}
\end{Verbatim}

gives:

\begin{Verbatim}[commandchars=\\\{\}]
\PYG{n}{i} \PYG{o}{=} \PYG{o}{*}\PYG{o}{*}\PYG{o}{*} \PYG{o+ow}{and} \PYG{n}{x} \PYG{o}{=}  \PYG{l+m+mf}{0.6666666667E+00}
\end{Verbatim}

Instead of using a label and writing the format on a separate line, it can
be put directly in the print statement, though this is often hard to read.
The above print statement can be written as:

\begin{Verbatim}[commandchars=\\\{\}]
\PYG{n+nb}{print} \PYG{l+s}{\PYGZdq{}}\PYG{l+s}{(}\PYG{l+s}{\PYGZsq{}}\PYG{l+s}{i = }\PYG{l+s}{\PYGZsq{}}\PYG{l+s}{,i3,}\PYG{l+s}{\PYGZsq{}}\PYG{l+s}{ and x = }\PYG{l+s}{\PYGZsq{}}\PYG{l+s}{, e17.10)}\PYG{l+s}{\PYGZdq{}}\PYG{p}{,} \PYG{n}{i}\PYG{p}{,} \PYG{n}{x}
\end{Verbatim}


\subsection{Writing to a file}
\label{fortran_io:writing-to-a-file}
Instead of printing directly to the terminal, we often want to write results
out to a file.  This can be done using the \titleref{open} statement to open a file
and attach it to a particular unit number, and then use the \titleref{write}
statement to write to this unit:

\begin{Verbatim}[commandchars=\\\{\}]
\PYG{n+nb}{open}\PYG{p}{(}\PYG{n}{unit}\PYG{o}{=}\PYG{l+m+mi}{20}\PYG{p}{,} \PYG{n}{file}\PYG{o}{=}\PYG{l+s}{\PYGZsq{}}\PYG{l+s}{output.txt}\PYG{l+s}{\PYGZsq{}}\PYG{p}{)}
\PYG{n}{write}\PYG{p}{(}\PYG{l+m+mi}{20}\PYG{p}{,}\PYG{o}{*}\PYG{p}{)} \PYG{n}{i}\PYG{p}{,} \PYG{n}{x}
\PYG{n}{close}\PYG{p}{(}\PYG{l+m+mi}{20}\PYG{p}{)}
\end{Verbatim}

This would do an \emph{unformatted} write to the file `output.txt' instead of
writing to the terminal.  The * in the write statement can be replaced by a
format, or a format label, as in the \titleref{print} statement.

There are many other optional arguments to the \titleref{open} command.

Unit numbers should generally be larger than 6.  By default, unit 6 refers
to the terminal for output, so

\begin{Verbatim}[commandchars=\\\{\}]
\PYG{n}{write}\PYG{p}{(}\PYG{l+m+mi}{6}\PYG{p}{,}\PYG{o}{*}\PYG{p}{)} \PYG{n}{i}\PYG{p}{,} \PYG{n}{x}
\end{Verbatim}

is the same as

\begin{Verbatim}[commandchars=\\\{\}]
\PYG{n+nb}{print} \PYG{o}{*}\PYG{p}{,} \PYG{n}{i}\PYG{p}{,} \PYG{n}{x}
\end{Verbatim}


\subsection{Reading input}
\label{fortran_io:reading-input}
Unformatted read:

\begin{Verbatim}[commandchars=\\\{\}]
\PYG{n+nb}{print} \PYG{o}{*}\PYG{p}{,} \PYG{l+s}{\PYGZdq{}}\PYG{l+s}{Please input n... }\PYG{l+s}{\PYGZdq{}}
\PYG{n}{read} \PYG{o}{*}\PYG{p}{,} \PYG{n}{n}
\end{Verbatim}

Reading from a file:

\begin{Verbatim}[commandchars=\\\{\}]
\PYG{n+nb}{open}\PYG{p}{(}\PYG{n}{unit}\PYG{o}{=}\PYG{l+m+mi}{21}\PYG{p}{,} \PYG{n}{file}\PYG{o}{=}\PYG{l+s}{\PYGZdq{}}\PYG{l+s}{infile.txt}\PYG{l+s}{\PYGZdq{}}\PYG{p}{)}
\PYG{n}{read}\PYG{p}{(}\PYG{l+m+mi}{21}\PYG{p}{,}\PYG{o}{*}\PYG{p}{)} \PYG{n}{n}
\PYG{n}{close}\PYG{p}{(}\PYG{l+m+mi}{21}\PYG{p}{)}
\end{Verbatim}


\section{Fortran debugging}
\label{fortran_debugging:fortran-debugging}\label{fortran_debugging::doc}\label{fortran_debugging:id1}

\subsection{Print statements}
\label{fortran_debugging:print-statements}
Adding print statements to a program is a tried and true method of
debugging, and the only method that many programmers use.
Not because it's the best method, but it's sometimes the simplest way to
examine what's going on at a particular point in a program.

Print statements can be added almost anywhere in a Fortran code to print
things out to the terminal window as it goes along.

You might want to put some special symbols in debugging statements to flag
them as such, which makes it easier to see what output is your debug output,
and also makes it easier to find them again later to remove from the code,
e.g. you might use ``+++'' or ``DEBUG''.


\subsection{Compiling with various gfortran flags}
\label{fortran_debugging:compiling-with-various-gfortran-flags}
There are a number of flags you can use when compiling your code that will
make it easier to debug.

Here's a generic set of options you might try:

\begin{Verbatim}[commandchars=\\\{\}]
\PYGZdl{} gfortran \PYGZhy{}g \PYGZhy{}W \PYGZhy{}Wall \PYGZhy{}fbounds\PYGZhy{}check \PYGZhy{}pedantic\PYGZhy{}errors \PYGZbs{}
      \PYGZhy{}ffpe\PYGZhy{}trap=zero,invalid,overflow,underflow  program.f90
\end{Verbatim}

See {\hyperref[gfortran_flags:gfortran\string-flags]{\crossref{\DUrole{std,std-ref}{Useful gfortran flags}}}} or the
\titleref{gfortran man page \textless{}http://linux.die.net/man/1/gfortran\textgreater{}}
for more information.  Most of these options
indicate that the program should give warnings or die if certain bad things
happen.

Compiling with the \titleref{-g} flag indicates that information should be
generated and saved during compilation that can be used to help debug the
code using a debugger such as \titleref{gdb} or \titleref{totalview}.  You generally have to
compile with this option to use a debugger.


\subsection{The \titleref{gdb} debugger}
\label{fortran_debugging:the-gdb-debugger}
This is the Gnu open source debugger for Gnu compilers such as gfortran.
Unfortunately it often works very poorly for Fortran.

You can install it on the VM using:

\begin{Verbatim}[commandchars=\\\{\}]
\PYGZdl{} sudo apt\PYGZhy{}get install gdb
\end{Verbatim}

And then use via, e.g.:
\begin{quote}

\$ cd \$UWHPSC/codes/fortran
\$ gfortran -g segfault1.f90
\$ gdb a.out
(gdb) run
\begin{quote}
\begin{description}
\item[{.... runs for a while and then prints}] \leavevmode
Program received signal EXC\_BAD\_ACCESS, Could not access memory.
Tells what line it died in.

\end{description}
\end{quote}

(gdb) p i
\$1 = 241
(gdb) q
\end{quote}

This at least reveals where the error happened and allows printing the value
of \titleref{i} when it died.


\subsection{Totalview}
\label{fortran_debugging:totalview}
Totalview is a commercial debugger that works quite well on Fortran codes
together with various compilers, including gfortran.  It also works with
other languages, and for parallel computing.

See \href{http://www.roguewave.com/products/totalview-family.aspx}{Rogue Wave Softare -- totalview family}.


\section{Fortran example for Newton's method}
\label{fortran_newton:fortran-newton}\label{fortran_newton:fortran-example-for-newton-s-method}\label{fortran_newton::doc}
This example shows one way to implement Newton's method for solving an
equation \(f(x)=0\), i.e. for a zero or root of the function \titleref{f(x)}.

See {\hyperref[special_functions:special\string-newton]{\crossref{\DUrole{std,std-ref}{Newton's method for the square root}}}} for a description of how Newton's method works.

These codes can be found in \titleref{\$UWHPSC/codes/fortran/newton}.

Here is the module that implements Newton's method in the subroutine
\titleref{solve}:

\begin{Verbatim}[commandchars=\\\{\},numbers=left,firstnumber=1,stepnumber=1]
\PYG{c}{! \PYGZdl{}UWHPSC/codes/fortran/newton/newton.f90}

\PYG{k}{module }\PYG{n}{newton}

    \PYG{c}{! module parameters:}
    \PYG{k}{implicit }\PYG{k}{none}
\PYG{k}{    }\PYG{k+kt}{integer}\PYG{p}{,} \PYG{k}{parameter} \PYG{k+kd}{::} \PYG{n}{maxiter} \PYG{o}{=} \PYG{l+m+mi}{20}
    \PYG{k+kt}{real}\PYG{p}{(}\PYG{n+nb}{kind}\PYG{o}{=}\PYG{l+m+mi}{8}\PYG{p}{)}\PYG{p}{,} \PYG{k}{parameter} \PYG{k+kd}{::} \PYG{n}{tol} \PYG{o}{=} \PYG{l+m+mf}{1.}\PYG{n}{d}\PYG{o}{\PYGZhy{}}\PYG{l+m+mi}{14}

\PYG{k}{contains}

\PYG{k}{subroutine }\PYG{n}{solve}\PYG{p}{(}\PYG{n}{f}\PYG{p}{,} \PYG{n}{fp}\PYG{p}{,} \PYG{n}{x0}\PYG{p}{,} \PYG{n}{x}\PYG{p}{,} \PYG{n}{iters}\PYG{p}{,} \PYG{n}{debug}\PYG{p}{)}

    \PYG{c}{! Estimate the zero of f(x) using Newton\PYGZsq{}s method. }
    \PYG{c}{! Input:}
    \PYG{c}{!   f:  the function to find a root of}
    \PYG{c}{!   fp: function returning the derivative f\PYGZsq{}}
    \PYG{c}{!   x0: the initial guess}
    \PYG{c}{!   debug: logical, prints iterations if debug=.true.}
    \PYG{c}{! Returns:}
    \PYG{c}{!   the estimate x satisfying f(x)=0 (assumes Newton converged!) }
    \PYG{c}{!   the number of iterations iters}
     
    \PYG{k}{implicit }\PYG{k}{none}
\PYG{k}{    }\PYG{k+kt}{real}\PYG{p}{(}\PYG{n+nb}{kind}\PYG{o}{=}\PYG{l+m+mi}{8}\PYG{p}{)}\PYG{p}{,} \PYG{k}{intent}\PYG{p}{(}\PYG{n}{in}\PYG{p}{)} \PYG{k+kd}{::} \PYG{n}{x0}
    \PYG{k+kt}{real}\PYG{p}{(}\PYG{n+nb}{kind}\PYG{o}{=}\PYG{l+m+mi}{8}\PYG{p}{)}\PYG{p}{,} \PYG{k}{external} \PYG{k+kd}{::} \PYG{n}{f}\PYG{p}{,} \PYG{n}{fp}
    \PYG{k+kt}{logical}\PYG{p}{,} \PYG{k}{intent}\PYG{p}{(}\PYG{n}{in}\PYG{p}{)} \PYG{k+kd}{::} \PYG{n}{debug}
    \PYG{k+kt}{real}\PYG{p}{(}\PYG{n+nb}{kind}\PYG{o}{=}\PYG{l+m+mi}{8}\PYG{p}{)}\PYG{p}{,} \PYG{k}{intent}\PYG{p}{(}\PYG{n}{out}\PYG{p}{)} \PYG{k+kd}{::} \PYG{n}{x}
    \PYG{k+kt}{integer}\PYG{p}{,} \PYG{k}{intent}\PYG{p}{(}\PYG{n}{out}\PYG{p}{)} \PYG{k+kd}{::} \PYG{n}{iters}

    \PYG{c}{! Declare any local variables:}
    \PYG{k+kt}{real}\PYG{p}{(}\PYG{n+nb}{kind}\PYG{o}{=}\PYG{l+m+mi}{8}\PYG{p}{)} \PYG{k+kd}{::} \PYG{n}{deltax}\PYG{p}{,} \PYG{n}{fx}\PYG{p}{,} \PYG{n}{fxprime}
    \PYG{k+kt}{integer} \PYG{k+kd}{::} \PYG{n}{k}


    \PYG{c}{! initial guess}
    \PYG{n}{x} \PYG{o}{=} \PYG{n}{x0}

    \PYG{k}{if} \PYG{p}{(}\PYG{n}{debug}\PYG{p}{)} \PYG{k}{then}
\PYG{k}{        }\PYG{k}{print }\PYG{l+m+mi}{11}\PYG{p}{,} \PYG{n}{x}
 \PYG{l+m+mi}{11}     \PYG{k}{format}\PYG{p}{(}\PYG{l+s+s1}{\PYGZsq{}Initial guess: x = \PYGZsq{}}\PYG{p}{,} \PYG{n}{e22}\PYG{p}{.}\PYG{l+m+mi}{15}\PYG{p}{)}
        \PYG{n}{endif}

    \PYG{c}{! Newton iteration to find a zero of f(x) }

    \PYG{k}{do }\PYG{n}{k}\PYG{o}{=}\PYG{l+m+mi}{1}\PYG{p}{,}\PYG{n}{maxiter}

        \PYG{c}{! evaluate function and its derivative:}
        \PYG{n}{fx} \PYG{o}{=} \PYG{n}{f}\PYG{p}{(}\PYG{n}{x}\PYG{p}{)}
        \PYG{n}{fxprime} \PYG{o}{=} \PYG{n}{fp}\PYG{p}{(}\PYG{n}{x}\PYG{p}{)}

        \PYG{k}{if} \PYG{p}{(}\PYG{n+nb}{abs}\PYG{p}{(}\PYG{n}{fx}\PYG{p}{)} \PYG{o}{\PYGZlt{}} \PYG{n}{tol}\PYG{p}{)} \PYG{k}{then}
\PYG{k}{            }\PYG{k}{exit}  \PYG{c}{! jump out of do loop}
            \PYG{n}{endif}

        \PYG{c}{! compute Newton increment x:}
        \PYG{n}{deltax} \PYG{o}{=} \PYG{n}{fx}\PYG{o}{/}\PYG{n}{fxprime}

        \PYG{c}{! update x:}
        \PYG{n}{x} \PYG{o}{=} \PYG{n}{x} \PYG{o}{\PYGZhy{}} \PYG{n}{deltax}

        \PYG{k}{if} \PYG{p}{(}\PYG{n}{debug}\PYG{p}{)} \PYG{k}{then}
\PYG{k}{            }\PYG{k}{print }\PYG{l+m+mi}{12}\PYG{p}{,} \PYG{n}{k}\PYG{p}{,}\PYG{n}{x}
 \PYG{l+m+mi}{12}         \PYG{k}{format}\PYG{p}{(}\PYG{l+s+s1}{\PYGZsq{}After\PYGZsq{}}\PYG{p}{,} \PYG{n}{i3}\PYG{p}{,} \PYG{l+s+s1}{\PYGZsq{} iterations, x = \PYGZsq{}}\PYG{p}{,} \PYG{n}{e22}\PYG{p}{.}\PYG{l+m+mi}{15}\PYG{p}{)}
            \PYG{n}{endif}

        \PYG{n}{enddo}


    \PYG{k}{if} \PYG{p}{(}\PYG{n}{k} \PYG{o}{\PYGZgt{}} \PYG{n}{maxiter}\PYG{p}{)} \PYG{k}{then}
        \PYG{c}{! might not have converged}

        \PYG{n}{fx} \PYG{o}{=} \PYG{n}{f}\PYG{p}{(}\PYG{n}{x}\PYG{p}{)}
        \PYG{k}{if} \PYG{p}{(}\PYG{n+nb}{abs}\PYG{p}{(}\PYG{n}{fx}\PYG{p}{)} \PYG{o}{\PYGZgt{}} \PYG{n}{tol}\PYG{p}{)} \PYG{k}{then}
\PYG{k}{            }\PYG{k}{print} \PYG{o}{*}\PYG{p}{,} \PYG{l+s+s1}{\PYGZsq{}*** Warning: has not yet converged\PYGZsq{}}
            \PYG{n}{endif}
        \PYG{n}{endif} 

    \PYG{c}{! number of iterations taken:}
    \PYG{n}{iters} \PYG{o}{=} \PYG{n}{k}\PYG{o}{\PYGZhy{}}\PYG{l+m+mi}{1}


\PYG{k}{end }\PYG{k}{subroutine }\PYG{n}{solve}

\PYG{k}{end }\PYG{k}{module }\PYG{n}{newton}
\end{Verbatim}

The \titleref{solve} routine takes two functions \titleref{f} and \titleref{fp} as arguments.  These
functions must return the value \(f(x)\) and \(f'(x)\) respectively
for any input \titleref{x}.

A sample test code shows how \titleref{solve} is called:

\begin{Verbatim}[commandchars=\\\{\},numbers=left,firstnumber=1,stepnumber=1]
\PYG{c}{! \PYGZdl{}UWHPSC/codes/fortran/newton/test1.f90}

\PYG{k}{program }\PYG{n}{test1}

    \PYG{k}{use }\PYG{n}{newton}\PYG{p}{,} \PYG{n}{only}\PYG{p}{:} \PYG{n}{solve}
    \PYG{k}{use }\PYG{n}{functions}\PYG{p}{,} \PYG{n}{only}\PYG{p}{:} \PYG{n}{f\PYGZus{}sqrt}\PYG{p}{,} \PYG{n}{fprime\PYGZus{}sqrt}

    \PYG{k}{implicit }\PYG{k}{none}
\PYG{k}{    }\PYG{k+kt}{real}\PYG{p}{(}\PYG{n+nb}{kind}\PYG{o}{=}\PYG{l+m+mi}{8}\PYG{p}{)} \PYG{k+kd}{::} \PYG{n}{x}\PYG{p}{,} \PYG{n}{x0}\PYG{p}{,} \PYG{n}{fx}
    \PYG{k+kt}{real}\PYG{p}{(}\PYG{n+nb}{kind}\PYG{o}{=}\PYG{l+m+mi}{8}\PYG{p}{)} \PYG{k+kd}{::} \PYG{n}{x0vals}\PYG{p}{(}\PYG{l+m+mi}{3}\PYG{p}{)}
    \PYG{k+kt}{integer} \PYG{k+kd}{::} \PYG{n}{iters}\PYG{p}{,} \PYG{n}{itest}
	\PYG{k+kt}{logical} \PYG{k+kd}{::} \PYG{n}{debug}         \PYG{c}{! set to .true. or .false.}

    \PYG{k}{print} \PYG{o}{*}\PYG{p}{,} \PYG{l+s+s2}{\PYGZdq{}Test routine for computing zero of f\PYGZdq{}}
    \PYG{n}{debug} \PYG{o}{=} \PYG{p}{.}\PYG{n}{true}\PYG{p}{.}

    \PYG{c}{! values to test as x0:}
    \PYG{n}{x0vals} \PYG{o}{=} \PYG{p}{(}\PYG{o}{/}\PYG{l+m+mf}{1.}\PYG{n}{d0}\PYG{p}{,} \PYG{l+m+mf}{2.}\PYG{n}{d0}\PYG{p}{,} \PYG{l+m+mi}{10}\PYG{l+m+mf}{0.}\PYG{n}{d0} \PYG{o}{/}\PYG{p}{)}

    \PYG{k}{do }\PYG{n}{itest}\PYG{o}{=}\PYG{l+m+mi}{1}\PYG{p}{,}\PYG{l+m+mi}{3}
        \PYG{n}{x0} \PYG{o}{=} \PYG{n}{x0vals}\PYG{p}{(}\PYG{n}{itest}\PYG{p}{)}
		\PYG{k}{print} \PYG{o}{*}\PYG{p}{,} \PYG{l+s+s1}{\PYGZsq{} \PYGZsq{}}  \PYG{c}{! blank line}
        \PYG{k}{call }\PYG{n}{solve}\PYG{p}{(}\PYG{n}{f\PYGZus{}sqrt}\PYG{p}{,} \PYG{n}{fprime\PYGZus{}sqrt}\PYG{p}{,} \PYG{n}{x0}\PYG{p}{,} \PYG{n}{x}\PYG{p}{,} \PYG{n}{iters}\PYG{p}{,} \PYG{n}{debug}\PYG{p}{)}

        \PYG{k}{print }\PYG{l+m+mi}{11}\PYG{p}{,} \PYG{n}{x}\PYG{p}{,} \PYG{n}{iters}
\PYG{l+m+mi}{11}      \PYG{k}{format}\PYG{p}{(}\PYG{l+s+s1}{\PYGZsq{}solver returns x = \PYGZsq{}}\PYG{p}{,} \PYG{n}{e22}\PYG{p}{.}\PYG{l+m+mi}{15}\PYG{p}{,} \PYG{l+s+s1}{\PYGZsq{} after\PYGZsq{}}\PYG{p}{,} \PYG{n}{i3}\PYG{p}{,} \PYG{l+s+s1}{\PYGZsq{} iterations\PYGZsq{}}\PYG{p}{)}

        \PYG{n}{fx} \PYG{o}{=} \PYG{n}{f\PYGZus{}sqrt}\PYG{p}{(}\PYG{n}{x}\PYG{p}{)}
        \PYG{k}{print }\PYG{l+m+mi}{12}\PYG{p}{,} \PYG{n}{fx}
\PYG{l+m+mi}{12}      \PYG{k}{format}\PYG{p}{(}\PYG{l+s+s1}{\PYGZsq{}the value of f(x) is \PYGZsq{}}\PYG{p}{,} \PYG{n}{e22}\PYG{p}{.}\PYG{l+m+mi}{15}\PYG{p}{)}

        \PYG{k}{if} \PYG{p}{(}\PYG{n+nb}{abs}\PYG{p}{(}\PYG{n}{x}\PYG{o}{\PYGZhy{}}\PYG{l+m+mf}{2.}\PYG{n}{d0}\PYG{p}{)} \PYG{o}{\PYGZgt{}} \PYG{l+m+mi}{1}\PYG{n}{d}\PYG{o}{\PYGZhy{}}\PYG{l+m+mi}{14}\PYG{p}{)} \PYG{k}{then}
\PYG{k}{            }\PYG{k}{print }\PYG{l+m+mi}{13}\PYG{p}{,} \PYG{n}{x}
\PYG{l+m+mi}{13}          \PYG{k}{format}\PYG{p}{(}\PYG{l+s+s1}{\PYGZsq{}*** Unexpected result: x = \PYGZsq{}}\PYG{p}{,} \PYG{n}{e22}\PYG{p}{.}\PYG{l+m+mi}{15}\PYG{p}{)}
            \PYG{n}{endif}
        \PYG{n}{enddo}

\PYG{k}{end }\PYG{k}{program }\PYG{n}{test1}
\end{Verbatim}

This makes use of a module \titleref{functions.f90} that includes the definition of
the functions used here.  Since \(f(x) = x^2 - 4\), we are attempting to
compute the square root of 4.

\begin{Verbatim}[commandchars=\\\{\},numbers=left,firstnumber=1,stepnumber=1]
\PYG{c}{! \PYGZdl{}UWHPSC/codes/fortran/newton/functions.f90}

\PYG{k}{module }\PYG{n}{functions}

\PYG{k}{contains}

\PYG{k+kt}{real}\PYG{p}{(}\PYG{n+nb}{kind}\PYG{o}{=}\PYG{l+m+mi}{8}\PYG{p}{)} \PYG{k}{function }\PYG{n}{f\PYGZus{}sqrt}\PYG{p}{(}\PYG{n}{x}\PYG{p}{)}
    \PYG{k}{implicit }\PYG{k}{none}
\PYG{k}{    }\PYG{k+kt}{real}\PYG{p}{(}\PYG{n+nb}{kind}\PYG{o}{=}\PYG{l+m+mi}{8}\PYG{p}{)}\PYG{p}{,} \PYG{k}{intent}\PYG{p}{(}\PYG{n}{in}\PYG{p}{)} \PYG{k+kd}{::} \PYG{n}{x}

    \PYG{n}{f\PYGZus{}sqrt} \PYG{o}{=} \PYG{n}{x}\PYG{o}{**}\PYG{l+m+mi}{2} \PYG{o}{\PYGZhy{}} \PYG{l+m+mf}{4.}\PYG{n}{d0}

\PYG{k}{end }\PYG{k}{function }\PYG{n}{f\PYGZus{}sqrt}


\PYG{k+kt}{real}\PYG{p}{(}\PYG{n+nb}{kind}\PYG{o}{=}\PYG{l+m+mi}{8}\PYG{p}{)} \PYG{k}{function }\PYG{n}{fprime\PYGZus{}sqrt}\PYG{p}{(}\PYG{n}{x}\PYG{p}{)}
    \PYG{k}{implicit }\PYG{k}{none}
\PYG{k}{    }\PYG{k+kt}{real}\PYG{p}{(}\PYG{n+nb}{kind}\PYG{o}{=}\PYG{l+m+mi}{8}\PYG{p}{)}\PYG{p}{,} \PYG{k}{intent}\PYG{p}{(}\PYG{n}{in}\PYG{p}{)} \PYG{k+kd}{::} \PYG{n}{x}
    
    \PYG{n}{fprime\PYGZus{}sqrt} \PYG{o}{=} \PYG{l+m+mf}{2.}\PYG{n}{d0} \PYG{o}{*} \PYG{n}{x}

\PYG{k}{end }\PYG{k}{function }\PYG{n}{fprime\PYGZus{}sqrt}

\PYG{k}{end }\PYG{k}{module }\PYG{n}{functions}
\end{Verbatim}

This test can be run via:

\begin{Verbatim}[commandchars=\\\{\}]
\PYGZdl{} make test1
\end{Verbatim}

which uses the Makefile in the same directory:

\begin{Verbatim}[commandchars=\\\{\},numbers=left,firstnumber=1,stepnumber=1]
\PYG{c}{\PYGZsh{} \PYGZdl{}UWHPSC/codes/fortran/newton/Makefile}

\PYG{n+nv}{OBJECTS} \PYG{o}{=} functions.o newton.o test1.o
\PYG{n+nv}{MODULES} \PYG{o}{=} functions.mod newton.mod

\PYG{n+nv}{FFLAGS} \PYG{o}{=} \PYGZhy{}g

\PYG{n+nf}{.PHONY}\PYG{o}{:} \PYG{n}{test}1 \PYG{n}{clean} 

\PYG{n+nf}{test1}\PYG{o}{:} \PYG{n}{test}1.\PYG{n}{exe}
	./test1.exe

\PYG{n+nf}{test1.exe}\PYG{o}{:} \PYG{k}{\PYGZdl{}(}\PYG{n+nv}{MODULES}\PYG{k}{)} \PYG{k}{\PYGZdl{}(}\PYG{n+nv}{OBJECTS}\PYG{k}{)}
	gfortran \PYG{k}{\PYGZdl{}(}FFLAGS\PYG{k}{)} \PYG{k}{\PYGZdl{}(}OBJECTS\PYG{k}{)} \PYGZhy{}o test1.exe

\PYG{n+nf}{\PYGZpc{}.o }\PYG{o}{:} \PYGZpc{}.\PYG{n}{f}90
	gfortran \PYG{k}{\PYGZdl{}(}FFLAGS\PYG{k}{)} \PYGZhy{}c  \PYG{n+nv}{\PYGZdl{}\PYGZlt{}} 

\PYG{n+nf}{\PYGZpc{}.mod}\PYG{o}{:} \PYGZpc{}.\PYG{n}{f}90
	gfortran \PYG{k}{\PYGZdl{}(}FFLAGS\PYG{k}{)} \PYGZhy{}c \PYG{n+nv}{\PYGZdl{}\PYGZlt{}}

\PYG{n+nf}{clean}\PYG{o}{:}
	rm \PYGZhy{}f *.o *.exe *.mod
\end{Verbatim}


\chapter{Parallel computing}
\label{index:parallel-computing}\label{index:toc-parallel}

\section{OpenMP}
\label{openmp:openmp}\label{openmp::doc}\label{openmp:id1}
OpenMP is discussed in \DUrole{xref,std,std-ref}{slides} starting with Lecture 13.


\subsection{Sample codes}
\label{openmp:sample-codes}
There are a few sample codes in the \titleref{\$UWHPSC/codes/openmp} directory.
See the \titleref{README.txt} file for instructions on compiling and executing.

Here is a very simple code, that simply evaluates a costly function at many
points:

\begin{Verbatim}[commandchars=\\\{\},numbers=left,firstnumber=1,stepnumber=1]
\PYG{c}{! \PYGZdl{}UWHPSC/codes/openmp/yeval.f90}

\PYG{c}{! If this code gives a Segmentation Fault when compiled and run with \PYGZhy{}fopenmp }
\PYG{c}{! Then you could try:}
\PYG{c}{!   \PYGZdl{} ulimit \PYGZhy{}s unlimited}
\PYG{c}{! to increase the allowed stack size.}
\PYG{c}{! This may not work on all computers.  On a Mac the best you can do is}
\PYG{c}{!   \PYGZdl{} ulimit \PYGZhy{}s hard}
\PYG{c}{! Correction... you can do the following on a Mac:}
\PYG{c}{!   \PYGZdl{} gfortran \PYGZhy{}fopenmp \PYGZhy{}Wl,\PYGZhy{}stack\PYGZus{}size,2faf2000 yeval.f90 }
\PYG{c}{! here 2faf2000 is a hexadecimal number slightly larger than 8e8, the }
\PYG{c}{! number of bytes needed for the value of n used in this program.}


\PYG{k}{program }\PYG{n}{yeval}
   
   \PYG{k}{use }\PYG{n}{omp\PYGZus{}lib}
   \PYG{k}{implicit }\PYG{k}{none}
\PYG{k}{   }\PYG{k+kt}{integer}\PYG{p}{,} \PYG{k}{parameter} \PYG{k+kd}{::} \PYG{n}{n} \PYG{o}{=} \PYG{l+m+mi}{100000000}
   \PYG{k+kt}{integer} \PYG{k+kd}{::} \PYG{n}{i}\PYG{p}{,} \PYG{n}{nthreads}
   \PYG{k+kt}{real}\PYG{p}{(}\PYG{n+nb}{kind}\PYG{o}{=}\PYG{l+m+mi}{8}\PYG{p}{)}\PYG{p}{,} \PYG{k}{dimension}\PYG{p}{(}\PYG{n}{n}\PYG{p}{)} \PYG{k+kd}{::} \PYG{n}{y}
   \PYG{k+kt}{real}\PYG{p}{(}\PYG{n+nb}{kind}\PYG{o}{=}\PYG{l+m+mi}{8}\PYG{p}{)} \PYG{k+kd}{::} \PYG{n}{dx}\PYG{p}{,} \PYG{n}{x}

   \PYG{c}{! Specify number of threads to use:}
   \PYG{c}{!\PYGZdl{} print *, \PYGZdq{}How many threads to use? \PYGZdq{}}
   \PYG{c}{!\PYGZdl{} read *, nthreads}
   \PYG{c}{!\PYGZdl{} call omp\PYGZus{}set\PYGZus{}num\PYGZus{}threads(nthreads)}
   \PYG{c}{!\PYGZdl{} print \PYGZdq{}(\PYGZsq{}Using OpenMP with \PYGZsq{},i3,\PYGZsq{} threads\PYGZsq{})\PYGZdq{}, nthreads}

   \PYG{n}{dx} \PYG{o}{=} \PYG{l+m+mf}{1.}\PYG{n}{d0} \PYG{o}{/} \PYG{p}{(}\PYG{n}{n}\PYG{o}{+}\PYG{l+m+mf}{1.}\PYG{n}{d0}\PYG{p}{)}

   \PYG{c}{!\PYGZdl{}omp parallel do private(x) }
   \PYG{k}{do }\PYG{n}{i}\PYG{o}{=}\PYG{l+m+mi}{1}\PYG{p}{,}\PYG{n}{n}
      \PYG{n}{x} \PYG{o}{=} \PYG{n}{i}\PYG{o}{*}\PYG{n}{dx}
      \PYG{n}{y}\PYG{p}{(}\PYG{n}{i}\PYG{p}{)} \PYG{o}{=} \PYG{n+nb}{exp}\PYG{p}{(}\PYG{n}{x}\PYG{p}{)}\PYG{o}{*}\PYG{n+nb}{cos}\PYG{p}{(}\PYG{n}{x}\PYG{p}{)}\PYG{o}{*}\PYG{n+nb}{sin}\PYG{p}{(}\PYG{n}{x}\PYG{p}{)}\PYG{o}{*}\PYG{n+nb}{sqrt}\PYG{p}{(}\PYG{l+m+mi}{5}\PYG{o}{*}\PYG{n}{x}\PYG{o}{+}\PYG{l+m+mf}{6.}\PYG{n}{d0}\PYG{p}{)}
   \PYG{n}{enddo}

   \PYG{k}{print} \PYG{o}{*}\PYG{p}{,} \PYG{l+s+s2}{\PYGZdq{}Filled vector y of length\PYGZdq{}}\PYG{p}{,} \PYG{n}{n}

\PYG{k}{end }\PYG{k}{program }\PYG{n}{yeval}
\end{Verbatim}

Note the following:
\begin{itemize}
\item {} 
Lines starting with \titleref{!\$} are only executed if the code is compiled and run
with the flag \titleref{-fopenmp}, otherwise they are comments.

\item {} 
\titleref{x} must be declared a \titleref{private} variable in the \titleref{omp parallel do} loop,
so that each thread has its own version.  Otherwise another thread might
reset \titleref{x} between the time its assigned a value and the time this value is
used to set \titleref{y(i)}.

\item {} 
The loop iterator \titleref{i} is private by default, but all other varaibles
are shared by default.

\item {} 
If you try to run this and get a ``segmentation fault'', it is probably
because the stack size limit is too small.  You can see the limit with:

\begin{Verbatim}[commandchars=\\\{\}]
\PYGZdl{} ulimit \PYGZhy{}s
\end{Verbatim}

On linux you can do:

\begin{Verbatim}[commandchars=\\\{\}]
\PYGZdl{} ulimit \PYGZhy{}s unlimited
\end{Verbatim}

But on a Mac there is a hard limit and the best you can do is:

\begin{Verbatim}[commandchars=\\\{\}]
\PYGZdl{} ulimit \PYGZhy{}s hard
\end{Verbatim}

If you still get a segmentation fault you will have to decrease \titleref{n} for
this example.

\end{itemize}


\subsection{Other directives}
\label{openmp:other-directives}
This example illustrates some directives beyond the \emph{parallel do}:

\begin{Verbatim}[commandchars=\\\{\},numbers=left,firstnumber=1,stepnumber=1]
\PYG{c}{! \PYGZdl{}UWHPSC/codes/openmp/demo2}

\PYG{k}{program }\PYG{n}{demo2}
   
   \PYG{k}{use }\PYG{n}{omp\PYGZus{}lib}
   \PYG{k}{implicit }\PYG{k}{none}
\PYG{k}{   }\PYG{k+kt}{integer} \PYG{k+kd}{::} \PYG{n}{i}
   \PYG{k+kt}{integer}\PYG{p}{,} \PYG{k}{parameter} \PYG{k+kd}{::} \PYG{n}{n} \PYG{o}{=} \PYG{l+m+mi}{100000}
   \PYG{k+kt}{real}\PYG{p}{(}\PYG{n+nb}{kind}\PYG{o}{=}\PYG{l+m+mi}{8}\PYG{p}{)}\PYG{p}{,} \PYG{k}{dimension}\PYG{p}{(}\PYG{n}{n}\PYG{p}{)} \PYG{k+kd}{::} \PYG{n}{x}\PYG{p}{,}\PYG{n}{y}\PYG{p}{,}\PYG{n}{z}
   
   \PYG{c}{! Specify number of threads to use:}
   \PYG{c}{!\PYGZdl{} call omp\PYGZus{}set\PYGZus{}num\PYGZus{}threads(2)}

   \PYG{c}{!\PYGZdl{}omp parallel  ! spawn two threads}
   \PYG{c}{!\PYGZdl{}omp sections  ! split up work between them}

     \PYG{c}{!\PYGZdl{}omp section}
     \PYG{n}{x} \PYG{o}{=} \PYG{l+m+mf}{1.}\PYG{n}{d0}   \PYG{c}{! one thread initializes x array}

     \PYG{c}{!\PYGZdl{}omp section}
     \PYG{n}{y} \PYG{o}{=} \PYG{l+m+mf}{1.}\PYG{n}{d0}   \PYG{c}{! another thread initializes y array}

   \PYG{c}{!\PYGZdl{}omp end sections}
   \PYG{c}{!\PYGZdl{}omp barrier   ! not needed, implied at end of sections}

   \PYG{c}{!\PYGZdl{}omp single    ! only want to print once:}
   \PYG{k}{print} \PYG{o}{*}\PYG{p}{,} \PYG{l+s+s2}{\PYGZdq{}Done initializing x and y\PYGZdq{}}
   \PYG{c}{!\PYGZdl{}omp end single nowait  ! ok for other thread to continue}

   \PYG{c}{!\PYGZdl{}omp do   ! split work between threads:}
   \PYG{k}{do }\PYG{n}{i}\PYG{o}{=}\PYG{l+m+mi}{1}\PYG{p}{,}\PYG{n}{n}
       \PYG{n}{z}\PYG{p}{(}\PYG{n}{i}\PYG{p}{)} \PYG{o}{=} \PYG{n}{x}\PYG{p}{(}\PYG{n}{i}\PYG{p}{)} \PYG{o}{+} \PYG{n}{y}\PYG{p}{(}\PYG{n}{i}\PYG{p}{)}
       \PYG{n}{enddo}

   \PYG{c}{!\PYGZdl{}omp end parallel}
   \PYG{k}{print} \PYG{o}{*}\PYG{p}{,} \PYG{l+s+s2}{\PYGZdq{}max value of z is \PYGZdq{}}\PYG{p}{,}\PYG{n+nb}{maxval}\PYG{p}{(}\PYG{n}{z}\PYG{p}{)}
    

\PYG{k}{end }\PYG{k}{program }\PYG{n}{demo2}
\end{Verbatim}

Notes:
\begin{itemize}
\item {} 
\titleref{!\$omp sections} is used to split up work between threads

\item {} 
There is an implicit barrier after \titleref{!\$omp end sections}, so the
explicit barrier here is optional.

\item {} 
The print statement is only done once since it is in \titleref{!\$omp single}.
The \titleref{nowait} clause indicates that the other thread can proceed without
waiting for this print to be executed.

\end{itemize}


\subsection{Fine-grain vs. coarse-grain paralellism}
\label{openmp:fine-grain-vs-coarse-grain-paralellism}
Consider the problem of normalizing a vector by dividing each element by the
1-norm of the vector, defined by \(\|x\|_1 = \sum_{i=1}^n |x_i|\).

We must first loop over all points to compute the norm.  Then we must loop
over all points and set \(y_i = x_i / \|x\|_1\).  Note that we cannot
combine these two loops into a single loop!

Here is an example with \emph{fine-grain paralellism}, where we use the OpenMP
\titleref{omp parallel do} directive or the \titleref{omp do} directive within a \titleref{omp
parallel} block.

\begin{Verbatim}[commandchars=\\\{\},numbers=left,firstnumber=1,stepnumber=1]
\PYG{c}{! \PYGZdl{}UWHPSC/codes/openmp/normalize1.f90}

\PYG{c}{! Example of normalizing a vector using fine\PYGZhy{}grain parallelism.}

\PYG{k}{program }\PYG{n}{main}
   
    \PYG{k}{use }\PYG{n}{omp\PYGZus{}lib}
    \PYG{k}{implicit }\PYG{k}{none}
\PYG{k}{    }\PYG{k+kt}{integer} \PYG{k+kd}{::} \PYG{n}{i}\PYG{p}{,} \PYG{n}{thread\PYGZus{}num}
    \PYG{k+kt}{integer}\PYG{p}{,} \PYG{k}{parameter} \PYG{k+kd}{::} \PYG{n}{n} \PYG{o}{=} \PYG{l+m+mi}{1000}
 
    \PYG{k+kt}{real}\PYG{p}{(}\PYG{n+nb}{kind}\PYG{o}{=}\PYG{l+m+mi}{8}\PYG{p}{)}\PYG{p}{,} \PYG{k}{dimension}\PYG{p}{(}\PYG{n}{n}\PYG{p}{)} \PYG{k+kd}{::} \PYG{n}{x}\PYG{p}{,} \PYG{n}{y}
    \PYG{k+kt}{real}\PYG{p}{(}\PYG{n+nb}{kind}\PYG{o}{=}\PYG{l+m+mi}{8}\PYG{p}{)} \PYG{k+kd}{::} \PYG{n}{norm}\PYG{p}{,}\PYG{n}{ynorm}
 
    \PYG{k+kt}{integer} \PYG{k+kd}{::} \PYG{n}{nthreads}
    
    \PYG{c}{! Specify number of threads to use:}
    \PYG{n}{nthreads} \PYG{o}{=} \PYG{l+m+mi}{1}       \PYG{c}{! need this value in serial mode}
    \PYG{c}{!\PYGZdl{} nthreads = 4    }
    \PYG{c}{!\PYGZdl{} call omp\PYGZus{}set\PYGZus{}num\PYGZus{}threads(nthreads)}
    \PYG{c}{!\PYGZdl{} print \PYGZdq{}(\PYGZsq{}Using OpenMP with \PYGZsq{},i3,\PYGZsq{} threads\PYGZsq{})\PYGZdq{}, nthreads}

    \PYG{c}{! Specify number of threads to use:}
    \PYG{c}{!\PYGZdl{} call omp\PYGZus{}set\PYGZus{}num\PYGZus{}threads(4)}
 
    \PYG{c}{! initialize x:}
    \PYG{c}{!\PYGZdl{}omp parallel do }
    \PYG{k}{do }\PYG{n}{i}\PYG{o}{=}\PYG{l+m+mi}{1}\PYG{p}{,}\PYG{n}{n}
        \PYG{n}{x}\PYG{p}{(}\PYG{n}{i}\PYG{p}{)} \PYG{o}{=} \PYG{n+nb}{dble}\PYG{p}{(}\PYG{n}{i}\PYG{p}{)}  \PYG{c}{! convert to double float}
    \PYG{n}{enddo}

    \PYG{n}{norm} \PYG{o}{=} \PYG{l+m+mf}{0.}\PYG{n}{d0}
    \PYG{n}{ynorm} \PYG{o}{=} \PYG{l+m+mf}{0.}\PYG{n}{d0}

    \PYG{c}{!\PYGZdl{}omp parallel private(i)}

    \PYG{c}{!\PYGZdl{}omp do reduction(+ : norm)}
    \PYG{k}{do }\PYG{n}{i}\PYG{o}{=}\PYG{l+m+mi}{1}\PYG{p}{,}\PYG{n}{n}
        \PYG{n}{norm} \PYG{o}{=} \PYG{n}{norm} \PYG{o}{+} \PYG{n+nb}{abs}\PYG{p}{(}\PYG{n}{x}\PYG{p}{(}\PYG{n}{i}\PYG{p}{)}\PYG{p}{)}
        \PYG{n}{enddo}

     \PYG{c}{!\PYGZdl{}omp barrier   ! not needed (implicit)}

    \PYG{c}{!\PYGZdl{}omp do reduction(+ : ynorm)}
    \PYG{k}{do }\PYG{n}{i}\PYG{o}{=}\PYG{l+m+mi}{1}\PYG{p}{,}\PYG{n}{n}
        \PYG{n}{y}\PYG{p}{(}\PYG{n}{i}\PYG{p}{)} \PYG{o}{=} \PYG{n}{x}\PYG{p}{(}\PYG{n}{i}\PYG{p}{)} \PYG{o}{/} \PYG{n}{norm}
        \PYG{n}{ynorm} \PYG{o}{=} \PYG{n}{ynorm} \PYG{o}{+} \PYG{n+nb}{abs}\PYG{p}{(}\PYG{n}{y}\PYG{p}{(}\PYG{n}{i}\PYG{p}{)}\PYG{p}{)}
        \PYG{n}{enddo}

    \PYG{c}{!\PYGZdl{}omp end parallel}

    \PYG{k}{print} \PYG{o}{*}\PYG{p}{,} \PYG{l+s+s2}{\PYGZdq{}norm of x = \PYGZdq{}}\PYG{p}{,}\PYG{n}{norm}\PYG{p}{,} \PYG{l+s+s2}{\PYGZdq{}  n(n+1)/2 = \PYGZdq{}}\PYG{p}{,}\PYG{n}{n}\PYG{o}{*}\PYG{p}{(}\PYG{n}{n}\PYG{o}{+}\PYG{l+m+mi}{1}\PYG{p}{)}\PYG{o}{/}\PYG{l+m+mi}{2}
    \PYG{k}{print} \PYG{o}{*}\PYG{p}{,} \PYG{l+s+s1}{\PYGZsq{}ynorm should be 1.0:   ynorm = \PYGZsq{}}\PYG{p}{,} \PYG{n}{ynorm}

\PYG{k}{end }\PYG{k}{program }\PYG{n}{main}
\end{Verbatim}

Note the following:
\begin{itemize}
\item {} 
We initialize \(x_i=i\) as a test, so \(\|x\|_1 = n(n+1)/2\).

\item {} 
The compiler decides how to split the loop between threads.
The loop starting on line 38 might be split differently than the
loop starting on line 45.

\item {} 
Because of this, all threads must have access to all of memory.

\end{itemize}

Next is a version with \emph{coarse-grain parallelism}, were we decide how to
split up the array between threads and then execute the same code on each
thread, but each thread will compute its own version of \titleref{istart} and \titleref{iend}
for its portion of the array.  With this code we are guaranteed that
thread 0 always handles \titleref{x(1)}, for example, so in principle the data could
be distributed.  When using OpenMP on a shared memory computer this doesn't
matter, but this version is more easily generalized to MPI.

\begin{Verbatim}[commandchars=\\\{\},numbers=left,firstnumber=1,stepnumber=1]
\PYG{c}{! \PYGZdl{}UWHPSC/codes/openmp/normalize2.f90}

\PYG{c}{! Example of normalizing a vector using coarse\PYGZhy{}grain parallelism.}

\PYG{k}{program }\PYG{n}{main}
    
    \PYG{k}{use }\PYG{n}{omp\PYGZus{}lib}
    \PYG{k}{implicit }\PYG{k}{none}
\PYG{k}{    }\PYG{k+kt}{integer}\PYG{p}{,} \PYG{k}{parameter} \PYG{k+kd}{::} \PYG{n}{n} \PYG{o}{=} \PYG{l+m+mi}{1000}
    \PYG{k+kt}{real}\PYG{p}{(}\PYG{n+nb}{kind}\PYG{o}{=}\PYG{l+m+mi}{8}\PYG{p}{)}\PYG{p}{,} \PYG{k}{dimension}\PYG{p}{(}\PYG{n}{n}\PYG{p}{)} \PYG{k+kd}{::} \PYG{n}{x}\PYG{p}{,}\PYG{n}{y}
    \PYG{k+kt}{real}\PYG{p}{(}\PYG{n+nb}{kind}\PYG{o}{=}\PYG{l+m+mi}{8}\PYG{p}{)} \PYG{k+kd}{::} \PYG{n}{norm}\PYG{p}{,}\PYG{n}{norm\PYGZus{}thread}\PYG{p}{,}\PYG{n}{ynorm}\PYG{p}{,}\PYG{n}{ynorm\PYGZus{}thread}
    \PYG{k+kt}{integer} \PYG{k+kd}{::} \PYG{n}{nthreads}\PYG{p}{,} \PYG{n}{points\PYGZus{}per\PYGZus{}thread}\PYG{p}{,}\PYG{n}{thread\PYGZus{}num}
    \PYG{k+kt}{integer} \PYG{k+kd}{::} \PYG{n}{i}\PYG{p}{,}\PYG{n}{istart}\PYG{p}{,}\PYG{n}{iend}

    \PYG{c}{! Specify number of threads to use:}
    \PYG{n}{nthreads} \PYG{o}{=} \PYG{l+m+mi}{1}       \PYG{c}{! need this value in serial mode}
    \PYG{c}{!\PYGZdl{} nthreads = 4    }
    \PYG{c}{!\PYGZdl{} call omp\PYGZus{}set\PYGZus{}num\PYGZus{}threads(nthreads)}
    \PYG{c}{!\PYGZdl{} print \PYGZdq{}(\PYGZsq{}Using OpenMP with \PYGZsq{},i3,\PYGZsq{} threads\PYGZsq{})\PYGZdq{}, nthreads}

    \PYG{c}{! Determine how many points to handle with each thread.}
    \PYG{c}{! Note that dividing two integers and assigning to an integer will}
    \PYG{c}{! round down if the result is not an integer.  }
    \PYG{c}{! This, together with the min(...) in the definition of iend below,}
    \PYG{c}{! insures that all points will get distributed to some thread.}
    \PYG{n}{points\PYGZus{}per\PYGZus{}thread} \PYG{o}{=} \PYG{p}{(}\PYG{n}{n} \PYG{o}{+} \PYG{n}{nthreads} \PYG{o}{\PYGZhy{}} \PYG{l+m+mi}{1}\PYG{p}{)} \PYG{o}{/} \PYG{n}{nthreads}
    \PYG{k}{print} \PYG{o}{*}\PYG{p}{,} \PYG{l+s+s2}{\PYGZdq{}points\PYGZus{}per\PYGZus{}thread = \PYGZdq{}}\PYG{p}{,}\PYG{n}{points\PYGZus{}per\PYGZus{}thread}

    \PYG{c}{! initialize x:}
    \PYG{k}{do }\PYG{n}{i}\PYG{o}{=}\PYG{l+m+mi}{1}\PYG{p}{,}\PYG{n}{n}
        \PYG{n}{x}\PYG{p}{(}\PYG{n}{i}\PYG{p}{)} \PYG{o}{=} \PYG{n+nb}{dble}\PYG{p}{(}\PYG{n}{i}\PYG{p}{)}  \PYG{c}{! convert to double float}
        \PYG{n}{enddo}

    \PYG{n}{norm} \PYG{o}{=} \PYG{l+m+mf}{0.}\PYG{n}{d0}
    \PYG{n}{ynorm} \PYG{o}{=} \PYG{l+m+mf}{0.}\PYG{n}{d0}

    \PYG{c}{!\PYGZdl{}omp parallel private(i,norm\PYGZus{}thread, \PYGZam{}}
    \PYG{c}{!\PYGZdl{}omp                  istart,iend,thread\PYGZus{}num,ynorm\PYGZus{}thread) }

    \PYG{n}{thread\PYGZus{}num} \PYG{o}{=} \PYG{l+m+mi}{0}     \PYG{c}{! needed in serial mode}
    \PYG{c}{!\PYGZdl{} thread\PYGZus{}num = omp\PYGZus{}get\PYGZus{}thread\PYGZus{}num()    ! unique for each thread}

    \PYG{c}{! Determine start and end index for the set of points to be }
    \PYG{c}{! handled by this thread:}
    \PYG{n}{istart} \PYG{o}{=} \PYG{n}{thread\PYGZus{}num} \PYG{o}{*} \PYG{n}{points\PYGZus{}per\PYGZus{}thread} \PYG{o}{+} \PYG{l+m+mi}{1}
    \PYG{n}{iend} \PYG{o}{=} \PYG{n+nb}{min}\PYG{p}{(}\PYG{p}{(}\PYG{n}{thread\PYGZus{}num}\PYG{o}{+}\PYG{l+m+mi}{1}\PYG{p}{)} \PYG{o}{*} \PYG{n}{points\PYGZus{}per\PYGZus{}thread}\PYG{p}{,} \PYG{n}{n}\PYG{p}{)}

    \PYG{c}{!\PYGZdl{}omp critical}
    \PYG{k}{print }\PYG{l+m+mi}{201}\PYG{p}{,} \PYG{n}{thread\PYGZus{}num}\PYG{p}{,} \PYG{n}{istart}\PYG{p}{,} \PYG{n}{iend}
    \PYG{c}{!\PYGZdl{}omp end critical}
\PYG{l+m+mi}{201} \PYG{k}{format}\PYG{p}{(}\PYG{l+s+s2}{\PYGZdq{}Thread \PYGZdq{}}\PYG{p}{,}\PYG{n}{i2}\PYG{p}{,}\PYG{l+s+s2}{\PYGZdq{} will take i = \PYGZdq{}}\PYG{p}{,}\PYG{n}{i6}\PYG{p}{,}\PYG{l+s+s2}{\PYGZdq{} through i = \PYGZdq{}}\PYG{p}{,}\PYG{n}{i6}\PYG{p}{)}

    \PYG{n}{norm\PYGZus{}thread} \PYG{o}{=} \PYG{l+m+mf}{0.}\PYG{n}{d0}
    \PYG{k}{do }\PYG{n}{i}\PYG{o}{=}\PYG{n}{istart}\PYG{p}{,}\PYG{n}{iend}
        \PYG{n}{norm\PYGZus{}thread} \PYG{o}{=} \PYG{n}{norm\PYGZus{}thread} \PYG{o}{+} \PYG{n+nb}{abs}\PYG{p}{(}\PYG{n}{x}\PYG{p}{(}\PYG{n}{i}\PYG{p}{)}\PYG{p}{)}
        \PYG{n}{enddo}

    \PYG{c}{! update global norm with value from each thread:}
    \PYG{c}{!\PYGZdl{}omp critical}
      \PYG{n}{norm} \PYG{o}{=} \PYG{n}{norm} \PYG{o}{+} \PYG{n}{norm\PYGZus{}thread}
      \PYG{k}{print} \PYG{o}{*}\PYG{p}{,} \PYG{l+s+s2}{\PYGZdq{}norm updated to: \PYGZdq{}}\PYG{p}{,}\PYG{n}{norm}
    \PYG{c}{!\PYGZdl{}omp end critical}

    \PYG{c}{! make sure all have updated norm before proceeding:}
    \PYG{c}{!\PYGZdl{}omp barrier}

    \PYG{n}{ynorm\PYGZus{}thread} \PYG{o}{=} \PYG{l+m+mf}{0.}\PYG{n}{d0}
    \PYG{k}{do }\PYG{n}{i}\PYG{o}{=}\PYG{n}{istart}\PYG{p}{,}\PYG{n}{iend}
        \PYG{n}{y}\PYG{p}{(}\PYG{n}{i}\PYG{p}{)} \PYG{o}{=} \PYG{n}{x}\PYG{p}{(}\PYG{n}{i}\PYG{p}{)} \PYG{o}{/} \PYG{n}{norm}
        \PYG{n}{ynorm\PYGZus{}thread} \PYG{o}{=} \PYG{n}{ynorm\PYGZus{}thread} \PYG{o}{+} \PYG{n+nb}{abs}\PYG{p}{(}\PYG{n}{y}\PYG{p}{(}\PYG{n}{i}\PYG{p}{)}\PYG{p}{)}
        \PYG{n}{enddo}

    \PYG{c}{! update global ynorm with value from each thread:}
    \PYG{c}{!\PYGZdl{}omp critical}
      \PYG{n}{ynorm} \PYG{o}{=} \PYG{n}{ynorm} \PYG{o}{+} \PYG{n}{ynorm\PYGZus{}thread}
      \PYG{k}{print} \PYG{o}{*}\PYG{p}{,} \PYG{l+s+s2}{\PYGZdq{}ynorm updated to: \PYGZdq{}}\PYG{p}{,}\PYG{n}{ynorm}
    \PYG{c}{!\PYGZdl{}omp end critical}
    \PYG{c}{!\PYGZdl{}omp barrier}

    \PYG{c}{!\PYGZdl{}omp end parallel }

    \PYG{k}{print} \PYG{o}{*}\PYG{p}{,} \PYG{l+s+s2}{\PYGZdq{}norm of x = \PYGZdq{}}\PYG{p}{,}\PYG{n}{norm}\PYG{p}{,} \PYG{l+s+s2}{\PYGZdq{}  n(n+1)/2 = \PYGZdq{}}\PYG{p}{,}\PYG{n}{n}\PYG{o}{*}\PYG{p}{(}\PYG{n}{n}\PYG{o}{+}\PYG{l+m+mi}{1}\PYG{p}{)}\PYG{o}{/}\PYG{l+m+mi}{2}
    \PYG{k}{print} \PYG{o}{*}\PYG{p}{,} \PYG{l+s+s1}{\PYGZsq{}ynorm should be 1.0:   ynorm = \PYGZsq{}}\PYG{p}{,} \PYG{n}{ynorm}

\PYG{k}{end }\PYG{k}{program }\PYG{n}{main}
\end{Verbatim}

Note the following:
\begin{itemize}
\item {} 
\titleref{istart} and \titleref{iend}, the starting and ending values of \titleref{i}
taken by each thread, are explicitly computed in terms of the thread
number.  We must be careful to handle the case when the number of
threads does not evenly divide \titleref{n}.

\item {} 
Various variables must be declared \titleref{private} in lines 37-38.

\item {} 
\titleref{norm} must be initialized to 0 before the \titleref{omp parallel} block.
Otherwise some thread might set it to 0 after another thread has already
updated it by its \titleref{norm\_thread}.

\item {} 
The update to \titleref{norm}  on line 60
must be in a \titleref{omp critical} block, so two threads
don't try to update it simultaneously (data race).

\item {} 
There must be an \titleref{omp barrier} on line 65
between updating \titleref{norm} by each thread and
using \titleref{norm} to compute each \titleref{y(i)}.   We must make sure all threads have
updated \titleref{norm} or it won't have the correct value when we use it.

\end{itemize}

For comparison of fine-grain and
coarse-grain parallelism on Jacobi iteration, see
\begin{itemize}
\item {} 
{\hyperref[jacobi1d_omp1:jacobi1d\string-omp1]{\crossref{\DUrole{std,std-ref}{Jacobi iteration using OpenMP with parallel do constructs}}}}

\item {} 
{\hyperref[jacobi1d_omp2:jacobi1d\string-omp2]{\crossref{\DUrole{std,std-ref}{Jacobi iteration using OpenMP with coarse-grain parallel block}}}}

\end{itemize}


\subsection{Further reading}
\label{openmp:further-reading}\begin{itemize}
\item {} 
{\hyperref[biblio:biblio\string-openmp]{\crossref{\DUrole{std,std-ref}{OpenMP references}}}} in bibliography

\end{itemize}


\section{MPI}
\label{mpi:id1}\label{mpi::doc}\label{mpi:mpi}
MPI stands for \emph{Message Passing Interface} and is a standard approach for
programming distributed memory machines such as clusters, supercomputers, or
heterogeneous networks of computers.  It can also be used on a single
shared memory computer, although it is often more cumbersome to program in
MPI than in OpenMP.


\subsection{MPI implementations}
\label{mpi:mpi-implementations}
A number of different implementations are available (open source and vendor
supplied for specific machines).   See
\href{http://www.mcs.anl.gov/research/projects/mpi/implementations.html}{this list}, for
example.


\subsection{MPI on the class VM}
\label{mpi:mpi-on-the-class-vm}
The VM has \href{http://www.open-mpi.org/}{open-mpi}  partially installed.

You will need to do the following:

\begin{Verbatim}[commandchars=\\\{\}]
\PYGZdl{} sudo apt\PYGZhy{}get update
\PYGZdl{} sudo apt\PYGZhy{}get install openmpi\PYGZhy{}dev
\end{Verbatim}

On other Ubuntu installations you will also have to do:

\begin{Verbatim}[commandchars=\\\{\}]
\PYGZdl{} sudo apt\PYGZhy{}get install openmpi\PYGZhy{}bin          \PYGZsh{} Already on the VM
\end{Verbatim}

You should then be able to do the following:

\begin{Verbatim}[commandchars=\\\{\}]
\PYGZdl{} cd \PYGZdl{}UWHPSC/codes/mpi
\PYGZdl{} mpif90 test1.f90
\PYGZdl{} mpiexec \PYGZhy{}n 4 a.out
\end{Verbatim}

and see output like:

\begin{Verbatim}[commandchars=\\\{\}]
\PYG{n}{Hello} \PYG{k+kn}{from} \PYG{n+nn}{Process} \PYG{n}{number}           \PYG{l+m+mi}{1}  \PYG{n}{of}            \PYG{l+m+mi}{4}  \PYG{n}{processes}
\PYG{n}{Hello} \PYG{k+kn}{from} \PYG{n+nn}{Process} \PYG{n}{number}           \PYG{l+m+mi}{3}  \PYG{n}{of}            \PYG{l+m+mi}{4}  \PYG{n}{processes}
\PYG{n}{Hello} \PYG{k+kn}{from} \PYG{n+nn}{Process} \PYG{n}{number}           \PYG{l+m+mi}{0}  \PYG{n}{of}            \PYG{l+m+mi}{4}  \PYG{n}{processes}
\PYG{n}{Hello} \PYG{k+kn}{from} \PYG{n+nn}{Process} \PYG{n}{number}           \PYG{l+m+mi}{2}  \PYG{n}{of}            \PYG{l+m+mi}{4}  \PYG{n}{processes}
\end{Verbatim}


\subsection{Test code}
\label{mpi:test-code}
The simple test code used above illustrates use of some of the basic MPI
subroutines.

\begin{Verbatim}[commandchars=\\\{\},numbers=left,firstnumber=1,stepnumber=1]
\PYG{c}{! \PYGZdl{}UWHPSC/codes/mpi/test1.f90}

\PYG{k}{program }\PYG{n}{test1}
    \PYG{k}{use }\PYG{n}{mpi}
    \PYG{k}{implicit }\PYG{k}{none}
\PYG{k}{    }\PYG{k+kt}{integer} \PYG{k+kd}{::} \PYG{n}{ierr}\PYG{p}{,} \PYG{n}{numprocs}\PYG{p}{,} \PYG{n}{proc\PYGZus{}num}

    \PYG{k}{call }\PYG{n}{mpi\PYGZus{}init}\PYG{p}{(}\PYG{n}{ierr}\PYG{p}{)}
    \PYG{k}{call }\PYG{n}{mpi\PYGZus{}comm\PYGZus{}size}\PYG{p}{(}\PYG{n}{MPI\PYGZus{}COMM\PYGZus{}WORLD}\PYG{p}{,} \PYG{n}{numprocs}\PYG{p}{,} \PYG{n}{ierr}\PYG{p}{)}
    \PYG{k}{call }\PYG{n}{mpi\PYGZus{}comm\PYGZus{}rank}\PYG{p}{(}\PYG{n}{MPI\PYGZus{}COMM\PYGZus{}WORLD}\PYG{p}{,} \PYG{n}{proc\PYGZus{}num}\PYG{p}{,} \PYG{n}{ierr}\PYG{p}{)}

    \PYG{k}{print} \PYG{o}{*}\PYG{p}{,} \PYG{l+s+s1}{\PYGZsq{}Hello from Process number\PYGZsq{}}\PYG{p}{,} \PYG{n}{proc\PYGZus{}num}\PYG{p}{,} \PYG{p}{\PYGZam{}}
             \PYG{l+s+s1}{\PYGZsq{} of \PYGZsq{}}\PYG{p}{,} \PYG{n}{numprocs}\PYG{p}{,} \PYG{l+s+s1}{\PYGZsq{} processes\PYGZsq{}}

    \PYG{k}{call }\PYG{n}{mpi\PYGZus{}finalize}\PYG{p}{(}\PYG{n}{ierr}\PYG{p}{)}

\PYG{k}{end }\PYG{k}{program }\PYG{n}{test1}
\end{Verbatim}


\subsection{Reduction example}
\label{mpi:reduction-example}
The next example uses \titleref{MPI\_REDUCE} to add up partial sums computed by
independent processes.

\begin{Verbatim}[commandchars=\\\{\},numbers=left,firstnumber=1,stepnumber=1]
\PYG{c}{! \PYGZdl{}UWHPSC/codes/mpi/pisum1.f90}

\PYG{c}{! Computes pi using MPI.  }
\PYG{c}{! Compare to \PYGZdl{}UWHPSC/codes/openmp/pisum2.f90 }

\PYG{k}{program }\PYG{n}{pisum1}
    \PYG{k}{use }\PYG{n}{mpi}
    \PYG{k}{implicit }\PYG{k}{none}
\PYG{k}{    }\PYG{k+kt}{integer} \PYG{k+kd}{::} \PYG{n}{ierr}\PYG{p}{,} \PYG{n}{numprocs}\PYG{p}{,} \PYG{n}{proc\PYGZus{}num}\PYG{p}{,} \PYG{n}{points\PYGZus{}per\PYGZus{}proc}\PYG{p}{,} \PYG{n}{n}\PYG{p}{,} \PYG{p}{\PYGZam{}}
               \PYG{n}{i}\PYG{p}{,} \PYG{n}{istart}\PYG{p}{,} \PYG{n}{iend}
    \PYG{k+kt}{real} \PYG{p}{(}\PYG{n+nb}{kind}\PYG{o}{=}\PYG{l+m+mi}{8}\PYG{p}{)} \PYG{k+kd}{::} \PYG{n}{x}\PYG{p}{,} \PYG{n}{dx}\PYG{p}{,} \PYG{n}{pisum}\PYG{p}{,} \PYG{n}{pisum\PYGZus{}proc}\PYG{p}{,} \PYG{n}{pi}

    \PYG{k}{call }\PYG{n}{mpi\PYGZus{}init}\PYG{p}{(}\PYG{n}{ierr}\PYG{p}{)}
    \PYG{k}{call }\PYG{n}{mpi\PYGZus{}comm\PYGZus{}size}\PYG{p}{(}\PYG{n}{MPI\PYGZus{}COMM\PYGZus{}WORLD}\PYG{p}{,} \PYG{n}{numprocs}\PYG{p}{,} \PYG{n}{ierr}\PYG{p}{)}
    \PYG{k}{call }\PYG{n}{mpi\PYGZus{}comm\PYGZus{}rank}\PYG{p}{(}\PYG{n}{MPI\PYGZus{}COMM\PYGZus{}WORLD}\PYG{p}{,} \PYG{n}{proc\PYGZus{}num}\PYG{p}{,} \PYG{n}{ierr}\PYG{p}{)}

    \PYG{c}{! Ask the user for the number of points}
    \PYG{k}{if} \PYG{p}{(}\PYG{n}{proc\PYGZus{}num} \PYG{o}{==} \PYG{l+m+mi}{0}\PYG{p}{)} \PYG{k}{then}
\PYG{k}{        }\PYG{k}{print} \PYG{o}{*}\PYG{p}{,} \PYG{l+s+s2}{\PYGZdq{}Using \PYGZdq{}}\PYG{p}{,}\PYG{n}{numprocs}\PYG{p}{,}\PYG{l+s+s2}{\PYGZdq{} processors\PYGZdq{}}
        \PYG{k}{print} \PYG{o}{*}\PYG{p}{,} \PYG{l+s+s2}{\PYGZdq{}Input n ... \PYGZdq{}}
        \PYG{k}{read} \PYG{o}{*}\PYG{p}{,} \PYG{n}{n}
    \PYG{k}{end }\PYG{k}{if}

    \PYG{c}{! Broadcast to all procs; everybody gets the value of n from proc 0}
    \PYG{k}{call }\PYG{n}{mpi\PYGZus{}bcast}\PYG{p}{(}\PYG{n}{n}\PYG{p}{,} \PYG{l+m+mi}{1}\PYG{p}{,} \PYG{n}{MPI\PYGZus{}INTEGER}\PYG{p}{,} \PYG{l+m+mi}{0}\PYG{p}{,} \PYG{n}{MPI\PYGZus{}COMM\PYGZus{}WORLD}\PYG{p}{,} \PYG{n}{ierr}\PYG{p}{)}

    \PYG{n}{dx} \PYG{o}{=} \PYG{l+m+mf}{1.}\PYG{n}{d0}\PYG{o}{/}\PYG{n}{n}

    \PYG{c}{! Determine how many points to handle with each proc}
    \PYG{n}{points\PYGZus{}per\PYGZus{}proc} \PYG{o}{=} \PYG{p}{(}\PYG{n}{n} \PYG{o}{+} \PYG{n}{numprocs} \PYG{o}{\PYGZhy{}} \PYG{l+m+mi}{1}\PYG{p}{)}\PYG{o}{/}\PYG{n}{numprocs}
    \PYG{k}{if} \PYG{p}{(}\PYG{n}{proc\PYGZus{}num} \PYG{o}{==} \PYG{l+m+mi}{0}\PYG{p}{)} \PYG{k}{then}   \PYG{c}{! Only one proc should print to avoid clutter}
        \PYG{k}{print} \PYG{o}{*}\PYG{p}{,} \PYG{l+s+s2}{\PYGZdq{}points\PYGZus{}per\PYGZus{}proc = \PYGZdq{}}\PYG{p}{,} \PYG{n}{points\PYGZus{}per\PYGZus{}proc}
    \PYG{k}{end }\PYG{k}{if}

    \PYG{c}{! Determine start and end index for this proc\PYGZsq{}s points}
    \PYG{n}{istart} \PYG{o}{=} \PYG{n}{proc\PYGZus{}num} \PYG{o}{*} \PYG{n}{points\PYGZus{}per\PYGZus{}proc} \PYG{o}{+} \PYG{l+m+mi}{1}
    \PYG{n}{iend} \PYG{o}{=} \PYG{n+nb}{min}\PYG{p}{(}\PYG{p}{(}\PYG{n}{proc\PYGZus{}num} \PYG{o}{+} \PYG{l+m+mi}{1}\PYG{p}{)}\PYG{o}{*}\PYG{n}{points\PYGZus{}per\PYGZus{}proc}\PYG{p}{,} \PYG{n}{n}\PYG{p}{)}

    \PYG{c}{! Diagnostic: tell the user which points will be handled by which proc}
    \PYG{k}{print} \PYG{l+s+s1}{\PYGZsq{}(\PYGZdq{}Process \PYGZdq{},i2,\PYGZdq{} will take i = \PYGZdq{},i6,\PYGZdq{} through i = \PYGZdq{},i6)\PYGZsq{}}\PYG{p}{,} \PYG{p}{\PYGZam{}}
          \PYG{n}{proc\PYGZus{}num}\PYG{p}{,} \PYG{n}{istart}\PYG{p}{,} \PYG{n}{iend}

    \PYG{n}{pisum\PYGZus{}proc} \PYG{o}{=} \PYG{l+m+mf}{0.}\PYG{n}{d0}
    \PYG{k}{do }\PYG{n}{i}\PYG{o}{=}\PYG{n}{istart}\PYG{p}{,}\PYG{n}{iend}
        \PYG{n}{x} \PYG{o}{=} \PYG{p}{(}\PYG{n}{i}\PYG{o}{\PYGZhy{}}\PYG{l+m+mf}{0.5}\PYG{n}{d0}\PYG{p}{)}\PYG{o}{*}\PYG{n}{dx}
        \PYG{n}{pisum\PYGZus{}proc} \PYG{o}{=} \PYG{n}{pisum\PYGZus{}proc} \PYG{o}{+} \PYG{l+m+mf}{1.}\PYG{n}{d0} \PYG{o}{/} \PYG{p}{(}\PYG{l+m+mf}{1.}\PYG{n}{d0} \PYG{o}{+} \PYG{n}{x}\PYG{o}{**}\PYG{l+m+mi}{2}\PYG{p}{)}
        \PYG{n}{enddo}

    \PYG{k}{call }\PYG{n}{MPI\PYGZus{}REDUCE}\PYG{p}{(}\PYG{n}{pisum\PYGZus{}proc}\PYG{p}{,}\PYG{n}{pisum}\PYG{p}{,}\PYG{l+m+mi}{1}\PYG{p}{,}\PYG{n}{MPI\PYGZus{}DOUBLE\PYGZus{}PRECISION}\PYG{p}{,}\PYG{n}{MPI\PYGZus{}SUM}\PYG{p}{,}\PYG{l+m+mi}{0}\PYG{p}{,} \PYG{p}{\PYGZam{}}
                        \PYG{n}{MPI\PYGZus{}COMM\PYGZus{}WORLD}\PYG{p}{,}\PYG{n}{ierr}\PYG{p}{)}

    \PYG{k}{if} \PYG{p}{(}\PYG{n}{proc\PYGZus{}num} \PYG{o}{==} \PYG{l+m+mi}{0}\PYG{p}{)} \PYG{k}{then}
\PYG{k}{        }\PYG{n}{pi} \PYG{o}{=} \PYG{l+m+mf}{4.}\PYG{n}{d0} \PYG{o}{*} \PYG{n}{dx} \PYG{o}{*} \PYG{n}{pisum} 
        \PYG{k}{print} \PYG{o}{*}\PYG{p}{,} \PYG{l+s+s2}{\PYGZdq{}The approximation to pi is \PYGZdq{}}\PYG{p}{,}\PYG{n}{pi}
        \PYG{n}{endif}

    \PYG{k}{call }\PYG{n}{mpi\PYGZus{}finalize}\PYG{p}{(}\PYG{n}{ierr}\PYG{p}{)}

\PYG{k}{end }\PYG{k}{program }\PYG{n}{pisum1}
\end{Verbatim}


\subsection{Send-Receive example}
\label{mpi:send-receive-example}
In this example, a value is set in Process 0 and then passed to Process 1
and on to Process 2, etc. until it reaches the last process, where it is
printed out.

\begin{Verbatim}[commandchars=\\\{\},numbers=left,firstnumber=1,stepnumber=1]
\PYG{c}{! \PYGZdl{}UWHPSC/codes/mpi/copyvalue.f90}
\PYG{c}{!}
\PYG{c}{! Set value in Process 0 and copy this through a chain of processes}
\PYG{c}{! and finally print result from Process numprocs\PYGZhy{}1.}
\PYG{c}{!}

\PYG{k}{program }\PYG{n}{copyvalue}

    \PYG{k}{use }\PYG{n}{mpi}

    \PYG{k}{implicit }\PYG{k}{none}

\PYG{k}{    }\PYG{k+kt}{integer} \PYG{k+kd}{::} \PYG{n}{i}\PYG{p}{,} \PYG{n}{proc\PYGZus{}num}\PYG{p}{,} \PYG{n}{num\PYGZus{}procs}\PYG{p}{,}\PYG{n}{ierr}
    \PYG{k+kt}{integer}\PYG{p}{,} \PYG{k}{dimension}\PYG{p}{(}\PYG{n}{MPI\PYGZus{}STATUS\PYGZus{}SIZE}\PYG{p}{)} \PYG{k+kd}{::} \PYG{n}{status}

    \PYG{k}{call }\PYG{n}{MPI\PYGZus{}INIT}\PYG{p}{(}\PYG{n}{ierr}\PYG{p}{)}
    \PYG{k}{call }\PYG{n}{MPI\PYGZus{}COMM\PYGZus{}SIZE}\PYG{p}{(}\PYG{n}{MPI\PYGZus{}COMM\PYGZus{}WORLD}\PYG{p}{,} \PYG{n}{num\PYGZus{}procs}\PYG{p}{,} \PYG{n}{ierr}\PYG{p}{)}
    \PYG{k}{call }\PYG{n}{MPI\PYGZus{}COMM\PYGZus{}RANK}\PYG{p}{(}\PYG{n}{MPI\PYGZus{}COMM\PYGZus{}WORLD}\PYG{p}{,} \PYG{n}{proc\PYGZus{}num}\PYG{p}{,} \PYG{n}{ierr}\PYG{p}{)}

    \PYG{k}{if} \PYG{p}{(}\PYG{n}{num\PYGZus{}procs}\PYG{o}{==}\PYG{l+m+mi}{1}\PYG{p}{)} \PYG{k}{then}
\PYG{k}{        }\PYG{k}{print} \PYG{o}{*}\PYG{p}{,} \PYG{l+s+s2}{\PYGZdq{}Only one process, cannot do anything\PYGZdq{}}
        \PYG{k}{call }\PYG{n}{MPI\PYGZus{}FINALIZE}\PYG{p}{(}\PYG{n}{ierr}\PYG{p}{)}
        \PYG{k}{stop}
\PYG{k}{        }\PYG{n}{endif}


    \PYG{k}{if} \PYG{p}{(}\PYG{n}{proc\PYGZus{}num}\PYG{o}{==}\PYG{l+m+mi}{0}\PYG{p}{)} \PYG{k}{then}
\PYG{k}{        }\PYG{n}{i} \PYG{o}{=} \PYG{l+m+mi}{55}
        \PYG{k}{print} \PYG{l+s+s1}{\PYGZsq{}(\PYGZdq{}Process \PYGZdq{},i3,\PYGZdq{} setting      i = \PYGZdq{},i3)\PYGZsq{}}\PYG{p}{,} \PYG{n}{proc\PYGZus{}num}\PYG{p}{,} \PYG{n}{i}

        \PYG{k}{call }\PYG{n}{MPI\PYGZus{}SEND}\PYG{p}{(}\PYG{n}{i}\PYG{p}{,} \PYG{l+m+mi}{1}\PYG{p}{,} \PYG{n}{MPI\PYGZus{}INTEGER}\PYG{p}{,} \PYG{l+m+mi}{1}\PYG{p}{,} \PYG{l+m+mi}{21}\PYG{p}{,} \PYG{p}{\PYGZam{}}
                      \PYG{n}{MPI\PYGZus{}COMM\PYGZus{}WORLD}\PYG{p}{,} \PYG{n}{ierr}\PYG{p}{)}

      \PYG{k}{else }\PYG{k}{if} \PYG{p}{(}\PYG{n}{proc\PYGZus{}num} \PYG{o}{\PYGZlt{}} \PYG{n}{num\PYGZus{}procs} \PYG{o}{\PYGZhy{}} \PYG{l+m+mi}{1}\PYG{p}{)} \PYG{k}{then}

\PYG{k}{        }\PYG{k}{call }\PYG{n}{MPI\PYGZus{}RECV}\PYG{p}{(}\PYG{n}{i}\PYG{p}{,} \PYG{l+m+mi}{1}\PYG{p}{,} \PYG{n}{MPI\PYGZus{}INTEGER}\PYG{p}{,} \PYG{n}{proc\PYGZus{}num}\PYG{o}{\PYGZhy{}}\PYG{l+m+mi}{1}\PYG{p}{,} \PYG{l+m+mi}{21}\PYG{p}{,} \PYG{p}{\PYGZam{}}
                      \PYG{n}{MPI\PYGZus{}COMM\PYGZus{}WORLD}\PYG{p}{,} \PYG{n}{status}\PYG{p}{,} \PYG{n}{ierr}\PYG{p}{)}

        \PYG{k}{print} \PYG{l+s+s1}{\PYGZsq{}(\PYGZdq{}Process \PYGZdq{},i3,\PYGZdq{} receives     i = \PYGZdq{},i3)\PYGZsq{}}\PYG{p}{,} \PYG{n}{proc\PYGZus{}num}\PYG{p}{,} \PYG{n}{i}
        \PYG{k}{print} \PYG{l+s+s1}{\PYGZsq{}(\PYGZdq{}Process \PYGZdq{},i3,\PYGZdq{} sends        i = \PYGZdq{},i3)\PYGZsq{}}\PYG{p}{,} \PYG{n}{proc\PYGZus{}num}\PYG{p}{,} \PYG{n}{i}

        \PYG{k}{call }\PYG{n}{MPI\PYGZus{}SEND}\PYG{p}{(}\PYG{n}{i}\PYG{p}{,} \PYG{l+m+mi}{1}\PYG{p}{,} \PYG{n}{MPI\PYGZus{}INTEGER}\PYG{p}{,} \PYG{n}{proc\PYGZus{}num}\PYG{o}{+}\PYG{l+m+mi}{1}\PYG{p}{,} \PYG{l+m+mi}{21}\PYG{p}{,} \PYG{p}{\PYGZam{}}
                      \PYG{n}{MPI\PYGZus{}COMM\PYGZus{}WORLD}\PYG{p}{,} \PYG{n}{ierr}\PYG{p}{)}


      \PYG{k}{else }\PYG{k}{if} \PYG{p}{(}\PYG{n}{proc\PYGZus{}num} \PYG{o}{==} \PYG{n}{num\PYGZus{}procs} \PYG{o}{\PYGZhy{}} \PYG{l+m+mi}{1}\PYG{p}{)} \PYG{k}{then}

\PYG{k}{        }\PYG{k}{call }\PYG{n}{MPI\PYGZus{}RECV}\PYG{p}{(}\PYG{n}{i}\PYG{p}{,} \PYG{l+m+mi}{1}\PYG{p}{,} \PYG{n}{MPI\PYGZus{}INTEGER}\PYG{p}{,} \PYG{n}{proc\PYGZus{}num}\PYG{o}{\PYGZhy{}}\PYG{l+m+mi}{1}\PYG{p}{,} \PYG{l+m+mi}{21}\PYG{p}{,} \PYG{p}{\PYGZam{}}
                      \PYG{n}{MPI\PYGZus{}COMM\PYGZus{}WORLD}\PYG{p}{,} \PYG{n}{status}\PYG{p}{,} \PYG{n}{ierr}\PYG{p}{)}

        \PYG{k}{print} \PYG{l+s+s1}{\PYGZsq{}(\PYGZdq{}Process \PYGZdq{},i3,\PYGZdq{} ends up with i = \PYGZdq{},i3)\PYGZsq{}}\PYG{p}{,} \PYG{n}{proc\PYGZus{}num}\PYG{p}{,} \PYG{n}{i}
      \PYG{n}{endif}

    \PYG{k}{call }\PYG{n}{MPI\PYGZus{}FINALIZE}\PYG{p}{(}\PYG{n}{ierr}\PYG{p}{)}

\PYG{k}{end }\PYG{k}{program }\PYG{n}{copyvalue}
\end{Verbatim}


\subsection{Master-worker examples}
\label{mpi:master-worker-examples}
The next two examples illustrate using Process 0 as a \emph{master} process to
farm work out to the other processes.  In both cases the 1-norm of a matrix
is computed, which is the maximum over \titleref{j} of the 1-norm of the {\color{red}\bfseries{}{}`}j{}`th column
of the matrix.

In the first case we assume there are the same number of worker processes as
columns in the matrix:

\begin{Verbatim}[commandchars=\\\{\},numbers=left,firstnumber=1,stepnumber=1]
\PYG{c}{! \PYGZdl{}UWHPSC/codes/mpi/matrix1norm1.f90}
\PYG{c}{!}
\PYG{c}{! Compute 1\PYGZhy{}norm of a matrix using mpi.}
\PYG{c}{! Process 0 is the master that sets things up and then sends a column}
\PYG{c}{! to each worker (Processes 1, 2, ..., num\PYGZus{}procs \PYGZhy{} 1).}
\PYG{c}{!}
\PYG{c}{! This version assumes there are at least as many workers as columns.}

\PYG{k}{program }\PYG{n}{matrix1norm1}

    \PYG{k}{use }\PYG{n}{mpi}

    \PYG{k}{implicit }\PYG{k}{none}

\PYG{k}{    }\PYG{k+kt}{integer} \PYG{k+kd}{::} \PYG{n}{i}\PYG{p}{,}\PYG{n}{j}\PYG{p}{,}\PYG{n}{jj}\PYG{p}{,}\PYG{n}{nrows}\PYG{p}{,}\PYG{n}{ncols}\PYG{p}{,}\PYG{n}{proc\PYGZus{}num}\PYG{p}{,} \PYG{n}{num\PYGZus{}procs}\PYG{p}{,}\PYG{n}{ierr}\PYG{p}{,}\PYG{n}{nerr}
    \PYG{k+kt}{integer}\PYG{p}{,} \PYG{k}{dimension}\PYG{p}{(}\PYG{n}{MPI\PYGZus{}STATUS\PYGZus{}SIZE}\PYG{p}{)} \PYG{k+kd}{::} \PYG{n}{status}
    \PYG{k+kt}{real}\PYG{p}{(}\PYG{n+nb}{kind}\PYG{o}{=}\PYG{l+m+mi}{8}\PYG{p}{)} \PYG{k+kd}{::} \PYG{n}{colnorm}
    \PYG{k+kt}{real}\PYG{p}{(}\PYG{n+nb}{kind}\PYG{o}{=}\PYG{l+m+mi}{8}\PYG{p}{)}\PYG{p}{,} \PYG{k}{allocatable}\PYG{p}{,} \PYG{k}{dimension}\PYG{p}{(}\PYG{p}{:}\PYG{p}{,}\PYG{p}{:}\PYG{p}{)} \PYG{k+kd}{::} \PYG{n}{a}
    \PYG{k+kt}{real}\PYG{p}{(}\PYG{n+nb}{kind}\PYG{o}{=}\PYG{l+m+mi}{8}\PYG{p}{)}\PYG{p}{,} \PYG{k}{allocatable}\PYG{p}{,} \PYG{k}{dimension}\PYG{p}{(}\PYG{p}{:}\PYG{p}{)} \PYG{k+kd}{::} \PYG{n}{anorm}\PYG{p}{,} \PYG{n}{colvect}

    \PYG{k}{call }\PYG{n}{MPI\PYGZus{}INIT}\PYG{p}{(}\PYG{n}{ierr}\PYG{p}{)}
    \PYG{k}{call }\PYG{n}{MPI\PYGZus{}COMM\PYGZus{}SIZE}\PYG{p}{(}\PYG{n}{MPI\PYGZus{}COMM\PYGZus{}WORLD}\PYG{p}{,} \PYG{n}{num\PYGZus{}procs}\PYG{p}{,} \PYG{n}{ierr}\PYG{p}{)}
    \PYG{k}{call }\PYG{n}{MPI\PYGZus{}COMM\PYGZus{}RANK}\PYG{p}{(}\PYG{n}{MPI\PYGZus{}COMM\PYGZus{}WORLD}\PYG{p}{,} \PYG{n}{proc\PYGZus{}num}\PYG{p}{,} \PYG{n}{ierr}\PYG{p}{)}

    \PYG{n}{nerr} \PYG{o}{=} \PYG{l+m+mi}{0}
    \PYG{k}{if} \PYG{p}{(}\PYG{n}{proc\PYGZus{}num}\PYG{o}{==}\PYG{l+m+mi}{0}\PYG{p}{)} \PYG{k}{then}
\PYG{k}{        }\PYG{k}{print} \PYG{o}{*}\PYG{p}{,} \PYG{l+s+s2}{\PYGZdq{}Input nrows, ncols\PYGZdq{}}
        \PYG{k}{read} \PYG{o}{*}\PYG{p}{,} \PYG{n}{nrows}\PYG{p}{,} \PYG{n}{ncols}
        \PYG{k}{if} \PYG{p}{(}\PYG{n}{ncols} \PYG{o}{\PYGZgt{}} \PYG{n}{num\PYGZus{}procs}\PYG{o}{\PYGZhy{}}\PYG{l+m+mi}{1}\PYG{p}{)} \PYG{k}{then}
\PYG{k}{            }\PYG{k}{print} \PYG{o}{*}\PYG{p}{,} \PYG{l+s+s2}{\PYGZdq{}*** Error, this version requires ncols \PYGZlt{} num\PYGZus{}procs = \PYGZdq{}}\PYG{p}{,}\PYG{p}{\PYGZam{}}
                  \PYG{n}{num\PYGZus{}procs}
            \PYG{n}{nerr} \PYG{o}{=} \PYG{l+m+mi}{1}
            \PYG{n}{endif}
        \PYG{k}{allocate}\PYG{p}{(}\PYG{n}{a}\PYG{p}{(}\PYG{n}{nrows}\PYG{p}{,}\PYG{n}{ncols}\PYG{p}{)}\PYG{p}{)}  \PYG{c}{! only master process 0 needs the matrix}
        \PYG{n}{a} \PYG{o}{=} \PYG{l+m+mf}{1.}\PYG{n}{d0}  \PYG{c}{! initialize to all 1\PYGZsq{}s for this test}
        \PYG{k}{allocate}\PYG{p}{(}\PYG{n}{anorm}\PYG{p}{(}\PYG{n}{ncols}\PYG{p}{)}\PYG{p}{)}    \PYG{c}{! to hold norm of each column in MPI\PYGZus{}RECV}
        \PYG{n}{endif}

    \PYG{c}{! if nerr == 1 then all processes must stop:}
    \PYG{k}{call }\PYG{n}{MPI\PYGZus{}BCAST}\PYG{p}{(}\PYG{n}{nerr}\PYG{p}{,} \PYG{l+m+mi}{1}\PYG{p}{,} \PYG{n}{MPI\PYGZus{}DOUBLE\PYGZus{}PRECISION}\PYG{p}{,} \PYG{l+m+mi}{0}\PYG{p}{,} \PYG{n}{MPI\PYGZus{}COMM\PYGZus{}WORLD}\PYG{p}{,} \PYG{n}{ierr}\PYG{p}{)}

    \PYG{k}{if} \PYG{p}{(}\PYG{n}{nerr} \PYG{o}{==} \PYG{l+m+mi}{1}\PYG{p}{)} \PYG{k}{then}
        \PYG{c}{! Note that error message already printed by Process 0}
        \PYG{c}{! All processes must execute the MPI\PYGZus{}FINALIZE }
        \PYG{c}{! (Could also just have \PYGZdq{}go to 99\PYGZdq{} here.)}
        \PYG{k}{call }\PYG{n}{MPI\PYGZus{}FINALIZE}\PYG{p}{(}\PYG{n}{ierr}\PYG{p}{)}
        \PYG{k}{stop}
\PYG{k}{        }\PYG{n}{endif}
        
    \PYG{k}{call }\PYG{n}{MPI\PYGZus{}BCAST}\PYG{p}{(}\PYG{n}{nrows}\PYG{p}{,} \PYG{l+m+mi}{1}\PYG{p}{,} \PYG{n}{MPI\PYGZus{}DOUBLE\PYGZus{}PRECISION}\PYG{p}{,} \PYG{l+m+mi}{0}\PYG{p}{,} \PYG{n}{MPI\PYGZus{}COMM\PYGZus{}WORLD}\PYG{p}{,} \PYG{n}{ierr}\PYG{p}{)}
    \PYG{k}{call }\PYG{n}{MPI\PYGZus{}BCAST}\PYG{p}{(}\PYG{n}{ncols}\PYG{p}{,} \PYG{l+m+mi}{1}\PYG{p}{,} \PYG{n}{MPI\PYGZus{}DOUBLE\PYGZus{}PRECISION}\PYG{p}{,} \PYG{l+m+mi}{0}\PYG{p}{,} \PYG{n}{MPI\PYGZus{}COMM\PYGZus{}WORLD}\PYG{p}{,} \PYG{n}{ierr}\PYG{p}{)}

    \PYG{k}{if} \PYG{p}{(}\PYG{n}{proc\PYGZus{}num} \PYG{o}{\PYGZgt{}} \PYG{l+m+mi}{0}\PYG{p}{)} \PYG{k}{then}
\PYG{k}{        }\PYG{k}{allocate}\PYG{p}{(}\PYG{n}{colvect}\PYG{p}{(}\PYG{n}{nrows}\PYG{p}{)}\PYG{p}{)}   \PYG{c}{! to hold a column vector sent from master}
        \PYG{n}{endif} 


    
    \PYG{c}{! \PYGZhy{}\PYGZhy{}\PYGZhy{}\PYGZhy{}\PYGZhy{}\PYGZhy{}\PYGZhy{}\PYGZhy{}\PYGZhy{}\PYGZhy{}\PYGZhy{}\PYGZhy{}\PYGZhy{}\PYGZhy{}\PYGZhy{}\PYGZhy{}\PYGZhy{}\PYGZhy{}\PYGZhy{}\PYGZhy{}\PYGZhy{}\PYGZhy{}\PYGZhy{}\PYGZhy{}\PYGZhy{}\PYGZhy{}\PYGZhy{}\PYGZhy{}\PYGZhy{}\PYGZhy{}\PYGZhy{}\PYGZhy{}\PYGZhy{}\PYGZhy{}\PYGZhy{}\PYGZhy{}\PYGZhy{}\PYGZhy{}\PYGZhy{}\PYGZhy{}\PYGZhy{}}
    \PYG{c}{! code for Master (Processor 0):}
    \PYG{c}{! \PYGZhy{}\PYGZhy{}\PYGZhy{}\PYGZhy{}\PYGZhy{}\PYGZhy{}\PYGZhy{}\PYGZhy{}\PYGZhy{}\PYGZhy{}\PYGZhy{}\PYGZhy{}\PYGZhy{}\PYGZhy{}\PYGZhy{}\PYGZhy{}\PYGZhy{}\PYGZhy{}\PYGZhy{}\PYGZhy{}\PYGZhy{}\PYGZhy{}\PYGZhy{}\PYGZhy{}\PYGZhy{}\PYGZhy{}\PYGZhy{}\PYGZhy{}\PYGZhy{}\PYGZhy{}\PYGZhy{}\PYGZhy{}\PYGZhy{}\PYGZhy{}\PYGZhy{}\PYGZhy{}\PYGZhy{}\PYGZhy{}\PYGZhy{}\PYGZhy{}\PYGZhy{}}

    \PYG{k}{if} \PYG{p}{(}\PYG{n}{proc\PYGZus{}num} \PYG{o}{==} \PYG{l+m+mi}{0}\PYG{p}{)} \PYG{k}{then}

\PYG{k}{      }\PYG{k}{do }\PYG{n}{j}\PYG{o}{=}\PYG{l+m+mi}{1}\PYG{p}{,}\PYG{n}{ncols}
        \PYG{k}{call }\PYG{n}{MPI\PYGZus{}SEND}\PYG{p}{(}\PYG{n}{a}\PYG{p}{(}\PYG{l+m+mi}{1}\PYG{p}{,}\PYG{n}{j}\PYG{p}{)}\PYG{p}{,} \PYG{n}{nrows}\PYG{p}{,} \PYG{n}{MPI\PYGZus{}DOUBLE\PYGZus{}PRECISION}\PYG{p}{,}\PYG{p}{\PYGZam{}}
                        \PYG{n}{j}\PYG{p}{,} \PYG{n}{j}\PYG{p}{,} \PYG{n}{MPI\PYGZus{}COMM\PYGZus{}WORLD}\PYG{p}{,} \PYG{n}{ierr}\PYG{p}{)}
        \PYG{n}{enddo}

      \PYG{k}{do }\PYG{n}{j}\PYG{o}{=}\PYG{l+m+mi}{1}\PYG{p}{,}\PYG{n}{ncols}
        \PYG{k}{call }\PYG{n}{MPI\PYGZus{}RECV}\PYG{p}{(}\PYG{n}{colnorm}\PYG{p}{,} \PYG{l+m+mi}{1}\PYG{p}{,} \PYG{n}{MPI\PYGZus{}DOUBLE\PYGZus{}PRECISION}\PYG{p}{,} \PYG{p}{\PYGZam{}}
                        \PYG{n}{MPI\PYGZus{}ANY\PYGZus{}SOURCE}\PYG{p}{,} \PYG{n}{MPI\PYGZus{}ANY\PYGZus{}TAG}\PYG{p}{,} \PYG{p}{\PYGZam{}}
                        \PYG{n}{MPI\PYGZus{}COMM\PYGZus{}WORLD}\PYG{p}{,} \PYG{n}{status}\PYG{p}{,} \PYG{n}{ierr}\PYG{p}{)}
        \PYG{n}{jj} \PYG{o}{=} \PYG{n}{status}\PYG{p}{(}\PYG{n}{MPI\PYGZus{}TAG}\PYG{p}{)}
        \PYG{n}{anorm}\PYG{p}{(}\PYG{n}{jj}\PYG{p}{)} \PYG{o}{=} \PYG{n}{colnorm}
        \PYG{n}{enddo}

      \PYG{k}{print} \PYG{o}{*}\PYG{p}{,} \PYG{l+s+s2}{\PYGZdq{}Finished filling anorm with values... \PYGZdq{}}
      \PYG{k}{print} \PYG{o}{*}\PYG{p}{,} \PYG{n}{anorm}
      \PYG{k}{print} \PYG{o}{*}\PYG{p}{,} \PYG{l+s+s2}{\PYGZdq{}1\PYGZhy{}norm of matrix a = \PYGZdq{}}\PYG{p}{,} \PYG{n+nb}{maxval}\PYG{p}{(}\PYG{n}{anorm}\PYG{p}{)}
      \PYG{n}{endif}


    \PYG{c}{! \PYGZhy{}\PYGZhy{}\PYGZhy{}\PYGZhy{}\PYGZhy{}\PYGZhy{}\PYGZhy{}\PYGZhy{}\PYGZhy{}\PYGZhy{}\PYGZhy{}\PYGZhy{}\PYGZhy{}\PYGZhy{}\PYGZhy{}\PYGZhy{}\PYGZhy{}\PYGZhy{}\PYGZhy{}\PYGZhy{}\PYGZhy{}\PYGZhy{}\PYGZhy{}\PYGZhy{}\PYGZhy{}\PYGZhy{}\PYGZhy{}\PYGZhy{}\PYGZhy{}\PYGZhy{}\PYGZhy{}\PYGZhy{}\PYGZhy{}\PYGZhy{}\PYGZhy{}\PYGZhy{}\PYGZhy{}\PYGZhy{}\PYGZhy{}\PYGZhy{}\PYGZhy{}}
    \PYG{c}{! code for Workers (Processors 1, 2, ...):}
    \PYG{c}{! \PYGZhy{}\PYGZhy{}\PYGZhy{}\PYGZhy{}\PYGZhy{}\PYGZhy{}\PYGZhy{}\PYGZhy{}\PYGZhy{}\PYGZhy{}\PYGZhy{}\PYGZhy{}\PYGZhy{}\PYGZhy{}\PYGZhy{}\PYGZhy{}\PYGZhy{}\PYGZhy{}\PYGZhy{}\PYGZhy{}\PYGZhy{}\PYGZhy{}\PYGZhy{}\PYGZhy{}\PYGZhy{}\PYGZhy{}\PYGZhy{}\PYGZhy{}\PYGZhy{}\PYGZhy{}\PYGZhy{}\PYGZhy{}\PYGZhy{}\PYGZhy{}\PYGZhy{}\PYGZhy{}\PYGZhy{}\PYGZhy{}\PYGZhy{}\PYGZhy{}\PYGZhy{}}
    \PYG{k}{if} \PYG{p}{(}\PYG{n}{proc\PYGZus{}num} \PYG{o}{/}\PYG{o}{=} \PYG{l+m+mi}{0}\PYG{p}{)} \PYG{k}{then}

\PYG{k}{        }\PYG{k}{if} \PYG{p}{(}\PYG{n}{proc\PYGZus{}num} \PYG{o}{\PYGZgt{}} \PYG{n}{ncols}\PYG{p}{)} \PYG{n}{go} \PYG{n}{to} \PYG{l+m+mi}{99}   \PYG{c}{! no work expected}

        \PYG{k}{call }\PYG{n}{MPI\PYGZus{}RECV}\PYG{p}{(}\PYG{n}{colvect}\PYG{p}{,} \PYG{n}{nrows}\PYG{p}{,} \PYG{n}{MPI\PYGZus{}DOUBLE\PYGZus{}PRECISION}\PYG{p}{,}\PYG{p}{\PYGZam{}}
                      \PYG{l+m+mi}{0}\PYG{p}{,} \PYG{n}{MPI\PYGZus{}ANY\PYGZus{}TAG}\PYG{p}{,} \PYG{p}{\PYGZam{}}
                      \PYG{n}{MPI\PYGZus{}COMM\PYGZus{}WORLD}\PYG{p}{,} \PYG{n}{status}\PYG{p}{,} \PYG{n}{ierr}\PYG{p}{)}

        \PYG{n}{j} \PYG{o}{=} \PYG{n}{status}\PYG{p}{(}\PYG{n}{MPI\PYGZus{}TAG}\PYG{p}{)}   \PYG{c}{! this is the column number}
                              \PYG{c}{! (should agree with proc\PYGZus{}num)}

        \PYG{n}{colnorm} \PYG{o}{=} \PYG{n+nb}{sum}\PYG{p}{(}\PYG{n+nb}{abs}\PYG{p}{(}\PYG{n}{colvect}\PYG{p}{)}\PYG{p}{)}

        \PYG{k}{call }\PYG{n}{MPI\PYGZus{}SEND}\PYG{p}{(}\PYG{n}{colnorm}\PYG{p}{,} \PYG{l+m+mi}{1}\PYG{p}{,} \PYG{n}{MPI\PYGZus{}DOUBLE\PYGZus{}PRECISION}\PYG{p}{,} \PYG{p}{\PYGZam{}}
                    \PYG{l+m+mi}{0}\PYG{p}{,} \PYG{n}{j}\PYG{p}{,} \PYG{n}{MPI\PYGZus{}COMM\PYGZus{}WORLD}\PYG{p}{,} \PYG{n}{ierr}\PYG{p}{)}

        \PYG{n}{endif}

\PYG{l+m+mi}{99}  \PYG{k}{continue}   \PYG{c}{! might jump to here if finished early}
    \PYG{k}{call }\PYG{n}{MPI\PYGZus{}FINALIZE}\PYG{p}{(}\PYG{n}{ierr}\PYG{p}{)}

\PYG{k}{end }\PYG{k}{program }\PYG{n}{matrix1norm1}


            
\end{Verbatim}

In the next case we consider the more realistic situation where there may be
many more columns in the matrix than worker processes.  In this case the
\emph{master} process must do more work to keep track of how which columns have
already been handled and farm out work as worker processes become free.

\begin{Verbatim}[commandchars=\\\{\},numbers=left,firstnumber=1,stepnumber=1]
\PYG{c}{! \PYGZdl{}UWHPSC/codes/mpi/matrix1norm2.f90}
\PYG{c}{!}
\PYG{c}{! Compute 1\PYGZhy{}norm of a matrix using mpi.}
\PYG{c}{! Process 0 is the master that sets things up and then sends a column}
\PYG{c}{! to each worker (Processes 1, 2, ..., num\PYGZus{}procs \PYGZhy{} 1).}
\PYG{c}{!}
\PYG{c}{! This version allows more columns than workers.}

\PYG{k}{program }\PYG{n}{matrix1norm2}

    \PYG{k}{use }\PYG{n}{mpi}

    \PYG{k}{implicit }\PYG{k}{none}

\PYG{k}{    }\PYG{k+kt}{integer} \PYG{k+kd}{::} \PYG{n}{i}\PYG{p}{,}\PYG{n}{j}\PYG{p}{,}\PYG{n}{jj}\PYG{p}{,}\PYG{n}{nrows}\PYG{p}{,}\PYG{n}{ncols}\PYG{p}{,}\PYG{n}{proc\PYGZus{}num}\PYG{p}{,} \PYG{n}{num\PYGZus{}procs}\PYG{p}{,}\PYG{n}{ierr}\PYG{p}{,}\PYG{n}{nerr}
    \PYG{k+kt}{integer} \PYG{k+kd}{::} \PYG{n}{numsent}\PYG{p}{,} \PYG{n}{sender}\PYG{p}{,} \PYG{n}{nextcol}
    \PYG{k+kt}{integer}\PYG{p}{,} \PYG{k}{dimension}\PYG{p}{(}\PYG{n}{MPI\PYGZus{}STATUS\PYGZus{}SIZE}\PYG{p}{)} \PYG{k+kd}{::} \PYG{n}{status}
    \PYG{k+kt}{real}\PYG{p}{(}\PYG{n+nb}{kind}\PYG{o}{=}\PYG{l+m+mi}{8}\PYG{p}{)} \PYG{k+kd}{::} \PYG{n}{colnorm}
    \PYG{k+kt}{real}\PYG{p}{(}\PYG{n+nb}{kind}\PYG{o}{=}\PYG{l+m+mi}{8}\PYG{p}{)}\PYG{p}{,} \PYG{k}{allocatable}\PYG{p}{,} \PYG{k}{dimension}\PYG{p}{(}\PYG{p}{:}\PYG{p}{,}\PYG{p}{:}\PYG{p}{)} \PYG{k+kd}{::} \PYG{n}{a}
    \PYG{k+kt}{real}\PYG{p}{(}\PYG{n+nb}{kind}\PYG{o}{=}\PYG{l+m+mi}{8}\PYG{p}{)}\PYG{p}{,} \PYG{k}{allocatable}\PYG{p}{,} \PYG{k}{dimension}\PYG{p}{(}\PYG{p}{:}\PYG{p}{)} \PYG{k+kd}{::} \PYG{n}{anorm}\PYG{p}{,} \PYG{n}{colvect}

    \PYG{k+kt}{logical} \PYG{k+kd}{::} \PYG{n}{debug}

    \PYG{n}{debug} \PYG{o}{=} \PYG{p}{.}\PYG{n}{true}\PYG{p}{.}

    \PYG{k}{call }\PYG{n}{MPI\PYGZus{}INIT}\PYG{p}{(}\PYG{n}{ierr}\PYG{p}{)}
    \PYG{k}{call }\PYG{n}{MPI\PYGZus{}COMM\PYGZus{}SIZE}\PYG{p}{(}\PYG{n}{MPI\PYGZus{}COMM\PYGZus{}WORLD}\PYG{p}{,} \PYG{n}{num\PYGZus{}procs}\PYG{p}{,} \PYG{n}{ierr}\PYG{p}{)}
    \PYG{k}{call }\PYG{n}{MPI\PYGZus{}COMM\PYGZus{}RANK}\PYG{p}{(}\PYG{n}{MPI\PYGZus{}COMM\PYGZus{}WORLD}\PYG{p}{,} \PYG{n}{proc\PYGZus{}num}\PYG{p}{,} \PYG{n}{ierr}\PYG{p}{)}

    \PYG{k}{if} \PYG{p}{(}\PYG{n}{proc\PYGZus{}num}\PYG{o}{==}\PYG{l+m+mi}{0}\PYG{p}{)} \PYG{k}{then}
\PYG{k}{        }\PYG{k}{print} \PYG{o}{*}\PYG{p}{,} \PYG{l+s+s2}{\PYGZdq{}Input nrows, ncols\PYGZdq{}}
        \PYG{k}{read} \PYG{o}{*}\PYG{p}{,} \PYG{n}{nrows}\PYG{p}{,} \PYG{n}{ncols}
        \PYG{k}{allocate}\PYG{p}{(}\PYG{n}{a}\PYG{p}{(}\PYG{n}{nrows}\PYG{p}{,}\PYG{n}{ncols}\PYG{p}{)}\PYG{p}{)}  \PYG{c}{! only master process 0 needs the matrix}
        \PYG{n}{a} \PYG{o}{=} \PYG{l+m+mf}{1.}\PYG{n}{d0}  \PYG{c}{! initialize to all 1\PYGZsq{}s for this test}
        \PYG{k}{allocate}\PYG{p}{(}\PYG{n}{anorm}\PYG{p}{(}\PYG{n}{ncols}\PYG{p}{)}\PYG{p}{)}    \PYG{c}{! to hold norm of each column in MPI\PYGZus{}RECV}
        \PYG{n}{endif}

        
    \PYG{k}{call }\PYG{n}{MPI\PYGZus{}BCAST}\PYG{p}{(}\PYG{n}{nrows}\PYG{p}{,} \PYG{l+m+mi}{1}\PYG{p}{,} \PYG{n}{MPI\PYGZus{}DOUBLE\PYGZus{}PRECISION}\PYG{p}{,} \PYG{l+m+mi}{0}\PYG{p}{,} \PYG{n}{MPI\PYGZus{}COMM\PYGZus{}WORLD}\PYG{p}{,} \PYG{n}{ierr}\PYG{p}{)}
    \PYG{k}{call }\PYG{n}{MPI\PYGZus{}BCAST}\PYG{p}{(}\PYG{n}{ncols}\PYG{p}{,} \PYG{l+m+mi}{1}\PYG{p}{,} \PYG{n}{MPI\PYGZus{}DOUBLE\PYGZus{}PRECISION}\PYG{p}{,} \PYG{l+m+mi}{0}\PYG{p}{,} \PYG{n}{MPI\PYGZus{}COMM\PYGZus{}WORLD}\PYG{p}{,} \PYG{n}{ierr}\PYG{p}{)}

    \PYG{k}{if} \PYG{p}{(}\PYG{n}{proc\PYGZus{}num} \PYG{o}{\PYGZgt{}} \PYG{l+m+mi}{0}\PYG{p}{)} \PYG{k}{then}
\PYG{k}{        }\PYG{k}{allocate}\PYG{p}{(}\PYG{n}{colvect}\PYG{p}{(}\PYG{n}{nrows}\PYG{p}{)}\PYG{p}{)}   \PYG{c}{! to hold a column vector sent from master}
        \PYG{n}{endif} 


    
    \PYG{c}{! \PYGZhy{}\PYGZhy{}\PYGZhy{}\PYGZhy{}\PYGZhy{}\PYGZhy{}\PYGZhy{}\PYGZhy{}\PYGZhy{}\PYGZhy{}\PYGZhy{}\PYGZhy{}\PYGZhy{}\PYGZhy{}\PYGZhy{}\PYGZhy{}\PYGZhy{}\PYGZhy{}\PYGZhy{}\PYGZhy{}\PYGZhy{}\PYGZhy{}\PYGZhy{}\PYGZhy{}\PYGZhy{}\PYGZhy{}\PYGZhy{}\PYGZhy{}\PYGZhy{}\PYGZhy{}\PYGZhy{}\PYGZhy{}\PYGZhy{}\PYGZhy{}\PYGZhy{}\PYGZhy{}\PYGZhy{}\PYGZhy{}\PYGZhy{}\PYGZhy{}\PYGZhy{}}
    \PYG{c}{! code for Master (Processor 0):}
    \PYG{c}{! \PYGZhy{}\PYGZhy{}\PYGZhy{}\PYGZhy{}\PYGZhy{}\PYGZhy{}\PYGZhy{}\PYGZhy{}\PYGZhy{}\PYGZhy{}\PYGZhy{}\PYGZhy{}\PYGZhy{}\PYGZhy{}\PYGZhy{}\PYGZhy{}\PYGZhy{}\PYGZhy{}\PYGZhy{}\PYGZhy{}\PYGZhy{}\PYGZhy{}\PYGZhy{}\PYGZhy{}\PYGZhy{}\PYGZhy{}\PYGZhy{}\PYGZhy{}\PYGZhy{}\PYGZhy{}\PYGZhy{}\PYGZhy{}\PYGZhy{}\PYGZhy{}\PYGZhy{}\PYGZhy{}\PYGZhy{}\PYGZhy{}\PYGZhy{}\PYGZhy{}\PYGZhy{}}

    \PYG{k}{if} \PYG{p}{(}\PYG{n}{proc\PYGZus{}num} \PYG{o}{==} \PYG{l+m+mi}{0}\PYG{p}{)} \PYG{k}{then}

\PYG{k}{      }\PYG{n}{numsent} \PYG{o}{=} \PYG{l+m+mi}{0} \PYG{c}{! keep track of how many columns sent}

      \PYG{c}{! send the first batch to get all workers working:}
      \PYG{k}{do }\PYG{n}{j}\PYG{o}{=}\PYG{l+m+mi}{1}\PYG{p}{,}\PYG{n+nb}{min}\PYG{p}{(}\PYG{n}{num\PYGZus{}procs}\PYG{o}{\PYGZhy{}}\PYG{l+m+mi}{1}\PYG{p}{,} \PYG{n}{ncols}\PYG{p}{)}
        \PYG{k}{call }\PYG{n}{MPI\PYGZus{}SEND}\PYG{p}{(}\PYG{n}{a}\PYG{p}{(}\PYG{l+m+mi}{1}\PYG{p}{,}\PYG{n}{j}\PYG{p}{)}\PYG{p}{,} \PYG{n}{nrows}\PYG{p}{,} \PYG{n}{MPI\PYGZus{}DOUBLE\PYGZus{}PRECISION}\PYG{p}{,}\PYG{p}{\PYGZam{}}
                        \PYG{n}{j}\PYG{p}{,} \PYG{n}{j}\PYG{p}{,} \PYG{n}{MPI\PYGZus{}COMM\PYGZus{}WORLD}\PYG{p}{,} \PYG{n}{ierr}\PYG{p}{)}
        \PYG{n}{numsent} \PYG{o}{=} \PYG{n}{numsent} \PYG{o}{+} \PYG{l+m+mi}{1}
        \PYG{n}{enddo}

      \PYG{c}{! as results come back, send out more work...}
      \PYG{c}{! the variable sender tells who sent back a result and ready for more}
      \PYG{k}{do }\PYG{n}{j}\PYG{o}{=}\PYG{l+m+mi}{1}\PYG{p}{,}\PYG{n}{ncols}
        \PYG{k}{call }\PYG{n}{MPI\PYGZus{}RECV}\PYG{p}{(}\PYG{n}{colnorm}\PYG{p}{,} \PYG{l+m+mi}{1}\PYG{p}{,} \PYG{n}{MPI\PYGZus{}DOUBLE\PYGZus{}PRECISION}\PYG{p}{,} \PYG{p}{\PYGZam{}}
                        \PYG{n}{MPI\PYGZus{}ANY\PYGZus{}SOURCE}\PYG{p}{,} \PYG{n}{MPI\PYGZus{}ANY\PYGZus{}TAG}\PYG{p}{,} \PYG{p}{\PYGZam{}}
                        \PYG{n}{MPI\PYGZus{}COMM\PYGZus{}WORLD}\PYG{p}{,} \PYG{n}{status}\PYG{p}{,} \PYG{n}{ierr}\PYG{p}{)}
        \PYG{n}{sender} \PYG{o}{=} \PYG{n}{status}\PYG{p}{(}\PYG{n}{MPI\PYGZus{}SOURCE}\PYG{p}{)}
        \PYG{n}{jj} \PYG{o}{=} \PYG{n}{status}\PYG{p}{(}\PYG{n}{MPI\PYGZus{}TAG}\PYG{p}{)}
        \PYG{n}{anorm}\PYG{p}{(}\PYG{n}{jj}\PYG{p}{)} \PYG{o}{=} \PYG{n}{colnorm}

        \PYG{k}{if} \PYG{p}{(}\PYG{n}{numsent} \PYG{o}{\PYGZlt{}} \PYG{n}{ncols}\PYG{p}{)} \PYG{k}{then}
            \PYG{c}{! still more work to do, the next column will be sent and}
            \PYG{c}{! this index also used as the tag:}
            \PYG{n}{nextcol} \PYG{o}{=} \PYG{n}{numsent} \PYG{o}{+} \PYG{l+m+mi}{1} 
            \PYG{k}{call }\PYG{n}{MPI\PYGZus{}SEND}\PYG{p}{(}\PYG{n}{a}\PYG{p}{(}\PYG{l+m+mi}{1}\PYG{p}{,}\PYG{n}{nextcol}\PYG{p}{)}\PYG{p}{,} \PYG{n}{nrows}\PYG{p}{,} \PYG{n}{MPI\PYGZus{}DOUBLE\PYGZus{}PRECISION}\PYG{p}{,}\PYG{p}{\PYGZam{}}
                            \PYG{n}{sender}\PYG{p}{,} \PYG{n}{nextcol}\PYG{p}{,} \PYG{n}{MPI\PYGZus{}COMM\PYGZus{}WORLD}\PYG{p}{,} \PYG{n}{ierr}\PYG{p}{)}
            \PYG{n}{numsent} \PYG{o}{=} \PYG{n}{numsent} \PYG{o}{+} \PYG{l+m+mi}{1}
          \PYG{k}{else}
            \PYG{c}{! send an empty message with tag=0 to indicate this worker}
            \PYG{c}{! is done:}
            \PYG{k}{call }\PYG{n}{MPI\PYGZus{}SEND}\PYG{p}{(}\PYG{n}{MPI\PYGZus{}BOTTOM}\PYG{p}{,} \PYG{l+m+mi}{0}\PYG{p}{,} \PYG{n}{MPI\PYGZus{}DOUBLE\PYGZus{}PRECISION}\PYG{p}{,}\PYG{p}{\PYGZam{}}
                            \PYG{n}{sender}\PYG{p}{,} \PYG{l+m+mi}{0}\PYG{p}{,} \PYG{n}{MPI\PYGZus{}COMM\PYGZus{}WORLD}\PYG{p}{,} \PYG{n}{ierr}\PYG{p}{)}
          \PYG{n}{endif}
            
        \PYG{n}{enddo}

      \PYG{k}{print} \PYG{o}{*}\PYG{p}{,} \PYG{l+s+s2}{\PYGZdq{}Finished filling anorm with values... \PYGZdq{}}
      \PYG{k}{print} \PYG{o}{*}\PYG{p}{,} \PYG{n}{anorm}
      \PYG{k}{print} \PYG{o}{*}\PYG{p}{,} \PYG{l+s+s2}{\PYGZdq{}1\PYGZhy{}norm of matrix a = \PYGZdq{}}\PYG{p}{,} \PYG{n+nb}{maxval}\PYG{p}{(}\PYG{n}{anorm}\PYG{p}{)}
      \PYG{n}{endif}


    \PYG{c}{! \PYGZhy{}\PYGZhy{}\PYGZhy{}\PYGZhy{}\PYGZhy{}\PYGZhy{}\PYGZhy{}\PYGZhy{}\PYGZhy{}\PYGZhy{}\PYGZhy{}\PYGZhy{}\PYGZhy{}\PYGZhy{}\PYGZhy{}\PYGZhy{}\PYGZhy{}\PYGZhy{}\PYGZhy{}\PYGZhy{}\PYGZhy{}\PYGZhy{}\PYGZhy{}\PYGZhy{}\PYGZhy{}\PYGZhy{}\PYGZhy{}\PYGZhy{}\PYGZhy{}\PYGZhy{}\PYGZhy{}\PYGZhy{}\PYGZhy{}\PYGZhy{}\PYGZhy{}\PYGZhy{}\PYGZhy{}\PYGZhy{}\PYGZhy{}\PYGZhy{}\PYGZhy{}}
    \PYG{c}{! code for Workers (Processors 1, 2, ...):}
    \PYG{c}{! \PYGZhy{}\PYGZhy{}\PYGZhy{}\PYGZhy{}\PYGZhy{}\PYGZhy{}\PYGZhy{}\PYGZhy{}\PYGZhy{}\PYGZhy{}\PYGZhy{}\PYGZhy{}\PYGZhy{}\PYGZhy{}\PYGZhy{}\PYGZhy{}\PYGZhy{}\PYGZhy{}\PYGZhy{}\PYGZhy{}\PYGZhy{}\PYGZhy{}\PYGZhy{}\PYGZhy{}\PYGZhy{}\PYGZhy{}\PYGZhy{}\PYGZhy{}\PYGZhy{}\PYGZhy{}\PYGZhy{}\PYGZhy{}\PYGZhy{}\PYGZhy{}\PYGZhy{}\PYGZhy{}\PYGZhy{}\PYGZhy{}\PYGZhy{}\PYGZhy{}\PYGZhy{}}
    \PYG{k}{if} \PYG{p}{(}\PYG{n}{proc\PYGZus{}num} \PYG{o}{/}\PYG{o}{=} \PYG{l+m+mi}{0}\PYG{p}{)} \PYG{k}{then}

\PYG{k}{        }\PYG{k}{if} \PYG{p}{(}\PYG{n}{proc\PYGZus{}num} \PYG{o}{\PYGZgt{}} \PYG{n}{ncols}\PYG{p}{)} \PYG{n}{go} \PYG{n}{to} \PYG{l+m+mi}{99}   \PYG{c}{! no work expected}

        \PYG{k}{do }\PYG{k}{while} \PYG{p}{(}\PYG{p}{.}\PYG{n}{true}\PYG{p}{.}\PYG{p}{)}
            \PYG{c}{! repeat until message with tag==0 received...}

            \PYG{k}{call }\PYG{n}{MPI\PYGZus{}RECV}\PYG{p}{(}\PYG{n}{colvect}\PYG{p}{,} \PYG{n}{nrows}\PYG{p}{,} \PYG{n}{MPI\PYGZus{}DOUBLE\PYGZus{}PRECISION}\PYG{p}{,}\PYG{p}{\PYGZam{}}
                          \PYG{l+m+mi}{0}\PYG{p}{,} \PYG{n}{MPI\PYGZus{}ANY\PYGZus{}TAG}\PYG{p}{,} \PYG{p}{\PYGZam{}}
                          \PYG{n}{MPI\PYGZus{}COMM\PYGZus{}WORLD}\PYG{p}{,} \PYG{n}{status}\PYG{p}{,} \PYG{n}{ierr}\PYG{p}{)}

            \PYG{n}{j} \PYG{o}{=} \PYG{n}{status}\PYG{p}{(}\PYG{n}{MPI\PYGZus{}TAG}\PYG{p}{)}   \PYG{c}{! this is the column number}
                                  \PYG{c}{! may not be proc\PYGZus{}num in general}

            \PYG{k}{if} \PYG{p}{(}\PYG{n}{debug}\PYG{p}{)} \PYG{k}{then}
\PYG{k}{                }\PYG{k}{print} \PYG{l+s+s1}{\PYGZsq{}(\PYGZdq{}+++ Process \PYGZdq{},i4,\PYGZdq{}  received message with tag \PYGZdq{},i6)\PYGZsq{}}\PYG{p}{,} \PYG{p}{\PYGZam{}}
                    \PYG{n}{proc\PYGZus{}num}\PYG{p}{,} \PYG{n}{j}       
                \PYG{n}{endif}

            \PYG{k}{if} \PYG{p}{(}\PYG{n}{j}\PYG{o}{==}\PYG{l+m+mi}{0}\PYG{p}{)} \PYG{n}{go} \PYG{n}{to} \PYG{l+m+mi}{99}    \PYG{c}{! received \PYGZdq{}done\PYGZdq{} message}

            \PYG{n}{colnorm} \PYG{o}{=} \PYG{n+nb}{sum}\PYG{p}{(}\PYG{n+nb}{abs}\PYG{p}{(}\PYG{n}{colvect}\PYG{p}{)}\PYG{p}{)}

            \PYG{k}{call }\PYG{n}{MPI\PYGZus{}SEND}\PYG{p}{(}\PYG{n}{colnorm}\PYG{p}{,} \PYG{l+m+mi}{1}\PYG{p}{,} \PYG{n}{MPI\PYGZus{}DOUBLE\PYGZus{}PRECISION}\PYG{p}{,} \PYG{p}{\PYGZam{}}
                        \PYG{l+m+mi}{0}\PYG{p}{,} \PYG{n}{j}\PYG{p}{,} \PYG{n}{MPI\PYGZus{}COMM\PYGZus{}WORLD}\PYG{p}{,} \PYG{n}{ierr}\PYG{p}{)}

            \PYG{n}{enddo}
        \PYG{n}{endif}

\PYG{l+m+mi}{99}  \PYG{k}{continue}   \PYG{c}{! might jump to here if finished early}
    \PYG{k}{call }\PYG{n}{MPI\PYGZus{}FINALIZE}\PYG{p}{(}\PYG{n}{ierr}\PYG{p}{)}

\PYG{k}{end }\PYG{k}{program }\PYG{n}{matrix1norm2}


            
\end{Verbatim}


\subsection{Sample codes}
\label{mpi:sample-codes}
Some other sample codes can also be found in the \titleref{\$UWHPSC/codes/mpi} directory.
\begin{itemize}
\item {} 
{\hyperref[jacobi1d_mpi:jacobi1d\string-mpi]{\crossref{\DUrole{std,std-ref}{Jacobi iteration using MPI}}}}

\end{itemize}

See also the samples in the list below.


\subsection{Further reading}
\label{mpi:further-reading}\begin{itemize}
\item {} 
{\hyperref[biblio:biblio\string-mpi]{\crossref{\DUrole{std,std-ref}{MPI references}}}} section of the bibliography lists some books.

\item {} 
\href{http://www.mcs.anl.gov/research/projects/mpi/tutorial/}{Argonne tutorials}

\item {} 
\href{http://www.mcs.anl.gov/research/projects/mpi/tutorial/gropp/talk.html}{Tutorial slides by Bill Gropp}

\item {} 
\href{https://computing.llnl.gov/tutorials/mpi/}{Livermore tutorials}

\item {} 
\href{http://www.open-mpi.org/}{Open MPI}

\item {} 
\href{http://www.mcs.anl.gov/research/projects/mpi/}{The MPI Standard}

\item {} 
\href{http://www.mcs.anl.gov/research/projects/mpi/usingmpi/examples/simplempi/main.htm}{Some sample codes}

\item {} 
\href{http://www.lam-mpi.org/tutorials/}{LAM MPI tutorials}

\item {} 
Google ``MPI tutorial'' to find more.

\item {} 
\href{http://www.mcs.anl.gov/research/projects/mpi/www/www3/}{Documentation on MPI subroutines}

\end{itemize}


\chapter{Miscellaneous}
\label{index:miscellaneous}\label{index:toc-misc}

\section{Makefiles}
\label{makefiles::doc}\label{makefiles:makefiles}\label{makefiles:id1}
The directory \titleref{\$UWHPSC/codes/fortran/multifile1} contains a Fortran code
\titleref{fullcode.f90} that consists of a main program and two subroutines:

\begin{Verbatim}[commandchars=\\\{\},numbers=left,firstnumber=1,stepnumber=1]
\PYG{c}{! \PYGZdl{}UWHPSC/codes/fortran/multifile1/fullcode.f90}

\PYG{k}{program }\PYG{n}{demo}
    \PYG{k}{print} \PYG{o}{*}\PYG{p}{,} \PYG{l+s+s2}{\PYGZdq{}In main program\PYGZdq{}}
    \PYG{k}{call }\PYG{n}{sub1}\PYG{p}{(}\PYG{p}{)}
    \PYG{k}{call }\PYG{n}{sub2}\PYG{p}{(}\PYG{p}{)}
\PYG{k}{end }\PYG{k}{program }\PYG{n}{demo}

\PYG{k}{subroutine }\PYG{n}{sub1}\PYG{p}{(}\PYG{p}{)}
    \PYG{k}{print} \PYG{o}{*}\PYG{p}{,} \PYG{l+s+s2}{\PYGZdq{}In sub1\PYGZdq{}}
\PYG{k}{end }\PYG{k}{subroutine }\PYG{n}{sub1}

\PYG{k}{subroutine }\PYG{n}{sub2}\PYG{p}{(}\PYG{p}{)}
    \PYG{k}{print} \PYG{o}{*}\PYG{p}{,} \PYG{l+s+s2}{\PYGZdq{}In sub2\PYGZdq{}}
\PYG{k}{end }\PYG{k}{subroutine }\PYG{n}{sub2}
\end{Verbatim}

To illustrate the construction of a Makefile, we first break this up into
three separate files:

\begin{Verbatim}[commandchars=\\\{\},numbers=left,firstnumber=1,stepnumber=1]
\PYG{c}{! \PYGZdl{}UWHPSC/codes/fortran/multifile1/main.f90}

\PYG{k}{program }\PYG{n}{demo}
    \PYG{k}{print} \PYG{o}{*}\PYG{p}{,} \PYG{l+s+s2}{\PYGZdq{}In main program\PYGZdq{}}
    \PYG{k}{call }\PYG{n}{sub1}\PYG{p}{(}\PYG{p}{)}
    \PYG{k}{call }\PYG{n}{sub2}\PYG{p}{(}\PYG{p}{)}
\PYG{k}{end }\PYG{k}{program }\PYG{n}{demo}
\end{Verbatim}

\begin{Verbatim}[commandchars=\\\{\},numbers=left,firstnumber=1,stepnumber=1]
\PYG{c}{! \PYGZdl{}UWHPSC/codes/fortran/multifile1/sub1.f90}

\PYG{k}{subroutine }\PYG{n}{sub1}\PYG{p}{(}\PYG{p}{)}
    \PYG{k}{print} \PYG{o}{*}\PYG{p}{,} \PYG{l+s+s2}{\PYGZdq{}In sub1\PYGZdq{}}
\PYG{k}{end }\PYG{k}{subroutine }\PYG{n}{sub1}
\end{Verbatim}

\begin{Verbatim}[commandchars=\\\{\},numbers=left,firstnumber=1,stepnumber=1]
\PYG{c}{! \PYGZdl{}UWHPSC/codes/fortran/multifile1/sub2.f90}

\PYG{k}{subroutine }\PYG{n}{sub2}\PYG{p}{(}\PYG{p}{)}
    \PYG{k}{print} \PYG{o}{*}\PYG{p}{,} \PYG{l+s+s2}{\PYGZdq{}In sub2\PYGZdq{}}
\PYG{k}{end }\PYG{k}{subroutine }\PYG{n}{sub2}
\end{Verbatim}

The directory \titleref{\$UWHPSC/codes/fortran/multifile1} contains several Makefiles
that get successively more sophisticated to compile the codes in this
directory.

In the first version we write out explicitly what to do for each file:

\begin{Verbatim}[commandchars=\\\{\},numbers=left,firstnumber=1,stepnumber=1]
\PYG{c}{\PYGZsh{} \PYGZdl{}UWHPSC/codes/fortran/multifile1/Makefile}

\PYG{n+nf}{output.txt}\PYG{o}{:} \PYG{n}{main}.\PYG{n}{exe}
	./main.exe \PYGZgt{} output.txt

\PYG{n+nf}{main.exe}\PYG{o}{:} \PYG{n}{main}.\PYG{n}{o} \PYG{n}{sub}1.\PYG{n}{o} \PYG{n}{sub}2.\PYG{n}{o}
	gfortran main.o sub1.o sub2.o \PYGZhy{}o main.exe

\PYG{n+nf}{main.o}\PYG{o}{:} \PYG{n}{main}.\PYG{n}{f}90
	gfortran \PYGZhy{}c main.f90
\PYG{n+nf}{sub1.o}\PYG{o}{:} \PYG{n}{sub}1.\PYG{n}{f}90
	gfortran \PYGZhy{}c sub1.f90
\PYG{n+nf}{sub2.o}\PYG{o}{:} \PYG{n}{sub}2.\PYG{n}{f}90
	gfortran \PYGZhy{}c sub2.f90
\end{Verbatim}

In the second version there is a general rule for creating \titleref{.o} files from
\titleref{.f90} files:

\begin{Verbatim}[commandchars=\\\{\},numbers=left,firstnumber=1,stepnumber=1]
\PYG{c}{\PYGZsh{} \PYGZdl{}UWHPSC/codes/fortran/multifile1/Makefile2}

\PYG{n+nf}{output.txt}\PYG{o}{:} \PYG{n}{main}.\PYG{n}{exe}
	./main.exe \PYGZgt{} output.txt

\PYG{n+nf}{main.exe}\PYG{o}{:} \PYG{n}{main}.\PYG{n}{o} \PYG{n}{sub}1.\PYG{n}{o} \PYG{n}{sub}2.\PYG{n}{o}
	gfortran main.o sub1.o sub2.o \PYGZhy{}o main.exe

\PYG{n+nf}{\PYGZpc{}.o }\PYG{o}{:} \PYGZpc{}.\PYG{n}{f}90
	gfortran \PYGZhy{}c \PYG{n+nv}{\PYGZdl{}\PYGZlt{}} 
\end{Verbatim}

In the third version we define a macro \titleref{OBJECTS} so we only have to write
out this list once, which minimizes the chance of introducing errors:

\begin{Verbatim}[commandchars=\\\{\},numbers=left,firstnumber=1,stepnumber=1]
\PYG{c}{\PYGZsh{} \PYGZdl{}UWHPSC/codes/fortran/multifile1/Makefile3}

\PYG{n+nv}{OBJECTS} \PYG{o}{=} main.o sub1.o sub2.o

\PYG{n+nf}{output.txt}\PYG{o}{:} \PYG{n}{main}.\PYG{n}{exe}
	./main.exe \PYGZgt{} output.txt

\PYG{n+nf}{main.exe}\PYG{o}{:} \PYG{k}{\PYGZdl{}(}\PYG{n+nv}{OBJECTS}\PYG{k}{)}
	gfortran \PYG{k}{\PYGZdl{}(}OBJECTS\PYG{k}{)} \PYGZhy{}o main.exe

\PYG{n+nf}{\PYGZpc{}.o }\PYG{o}{:} \PYGZpc{}.\PYG{n}{f}90
	gfortran \PYGZhy{}c \PYG{n+nv}{\PYGZdl{}\PYGZlt{}} 
\end{Verbatim}

In the fourth version, we add a Fortran compile flag (for level 3
optimization) and an linker flag (blank in this example):

\begin{Verbatim}[commandchars=\\\{\},numbers=left,firstnumber=1,stepnumber=1]
\PYG{c}{\PYGZsh{} \PYGZdl{}UWHPSC/codes/fortran/multifile1/Makefile4}

\PYG{n+nv}{FC} \PYG{o}{=} gfortran    
\PYG{n+nv}{FFLAGS} \PYG{o}{=} \PYGZhy{}O3
\PYG{n+nv}{LFLAGS} \PYG{o}{=}
\PYG{n+nv}{OBJECTS} \PYG{o}{=} main.o sub1.o sub2.o

\PYG{n+nf}{output.txt}\PYG{o}{:} \PYG{n}{main}.\PYG{n}{exe}
	./main.exe \PYGZgt{} output.txt

\PYG{n+nf}{main.exe}\PYG{o}{:} \PYG{k}{\PYGZdl{}(}\PYG{n+nv}{OBJECTS}\PYG{k}{)}
	\PYG{k}{\PYGZdl{}(}FC\PYG{k}{)} \PYG{k}{\PYGZdl{}(}LFLAGS\PYG{k}{)} \PYG{k}{\PYGZdl{}(}OBJECTS\PYG{k}{)} \PYGZhy{}o main.exe

\PYG{n+nf}{\PYGZpc{}.o }\PYG{o}{:} \PYGZpc{}.\PYG{n}{f}90
	\PYG{k}{\PYGZdl{}(}FC\PYG{k}{)} \PYG{k}{\PYGZdl{}(}FFLAGS\PYG{k}{)} \PYGZhy{}c \PYG{n+nv}{\PYGZdl{}\PYGZlt{}} 
\end{Verbatim}

Next we add a \titleref{phony} target \titleref{clean}
that removes the files created when compiling
the code in order to facilitate cleanup.  It is \emph{phony} because it does not
create a file named \titleref{clean}.

\begin{Verbatim}[commandchars=\\\{\},numbers=left,firstnumber=1,stepnumber=1]
\PYG{c}{\PYGZsh{} \PYGZdl{}UWHPSC/codes/fortran/multifile1/Makefile5}

\PYG{n+nv}{OBJECTS} \PYG{o}{=} main.o sub1.o sub2.o
\PYG{n+nf}{.PHONY}\PYG{o}{:} \PYG{n}{clean}

\PYG{n+nf}{output.txt}\PYG{o}{:} \PYG{n}{main}.\PYG{n}{exe}
	./main.exe \PYGZgt{} output.txt

\PYG{n+nf}{main.exe}\PYG{o}{:} \PYG{k}{\PYGZdl{}(}\PYG{n+nv}{OBJECTS}\PYG{k}{)}
	gfortran \PYG{k}{\PYGZdl{}(}OBJECTS\PYG{k}{)} \PYGZhy{}o main.exe

\PYG{n+nf}{\PYGZpc{}.o }\PYG{o}{:} \PYGZpc{}.\PYG{n}{f}90
	gfortran \PYGZhy{}c \PYG{n+nv}{\PYGZdl{}\PYGZlt{}} 

\PYG{n+nf}{clean}\PYG{o}{:}
	rm \PYGZhy{}f \PYG{k}{\PYGZdl{}(}OBJECTS\PYG{k}{)} main.exe
\end{Verbatim}

Finally we add a help message so that \titleref{make help} says something useful:

\begin{Verbatim}[commandchars=\\\{\},numbers=left,firstnumber=1,stepnumber=1]
\PYG{c}{\PYGZsh{} \PYGZdl{}UWHPSC/codes/fortran/multifile1/Makefile6}

\PYG{n+nv}{OBJECTS} \PYG{o}{=} main.o sub1.o sub2.o
\PYG{n+nf}{.PHONY}\PYG{o}{:} \PYG{n}{clean} \PYG{n}{help}

\PYG{n+nf}{output.txt}\PYG{o}{:} \PYG{n}{main}.\PYG{n}{exe}
	./main.exe \PYGZgt{} output.txt

\PYG{n+nf}{main.exe}\PYG{o}{:} \PYG{k}{\PYGZdl{}(}\PYG{n+nv}{OBJECTS}\PYG{k}{)}
	gfortran \PYG{k}{\PYGZdl{}(}OBJECTS\PYG{k}{)} \PYGZhy{}o main.exe

\PYG{n+nf}{\PYGZpc{}.o }\PYG{o}{:} \PYGZpc{}.\PYG{n}{f}90
	gfortran \PYGZhy{}c \PYG{n+nv}{\PYGZdl{}\PYGZlt{}} 

\PYG{n+nf}{clean}\PYG{o}{:}
	rm \PYGZhy{}f \PYG{k}{\PYGZdl{}(}OBJECTS\PYG{k}{)} main.exe

\PYG{n+nf}{help}\PYG{o}{:}
	@echo \PYG{l+s+s2}{\PYGZdq{}Valid targets:\PYGZdq{}}
	@echo \PYG{l+s+s2}{\PYGZdq{}  main.exe\PYGZdq{}}
	@echo \PYG{l+s+s2}{\PYGZdq{}  main.o\PYGZdq{}}
	@echo \PYG{l+s+s2}{\PYGZdq{}  sub1.o\PYGZdq{}}
	@echo \PYG{l+s+s2}{\PYGZdq{}  sub2.o\PYGZdq{}}
	@echo \PYG{l+s+s2}{\PYGZdq{}  clean:  removes .o and .exe files\PYGZdq{}}
\end{Verbatim}

Fancier things are also possible, for example automatically detecting all
the \titleref{.f90} files in the directory to construct the list of \titleref{SOURCES}
and \titleref{OBJECTS}:

\begin{Verbatim}[commandchars=\\\{\},numbers=left,firstnumber=1,stepnumber=1]
\PYG{c}{\PYGZsh{} \PYGZdl{}UWHPSC/codes/fortran/multifile1/Makefile7}

\PYG{n+nv}{SOURCES} \PYG{o}{=} \PYG{k}{\PYGZdl{}(}wildcard *.f90\PYG{k}{)}  
\PYG{n+nv}{OBJECTS} \PYG{o}{=} \PYG{k}{\PYGZdl{}(}subst .f90,.o,\PYG{k}{\PYGZdl{}(}SOURCES\PYG{k}{)}\PYG{k}{)}

\PYG{n+nf}{.PHONY}\PYG{o}{:} \PYG{n}{test}

\PYG{n+nf}{test}\PYG{o}{:}
	@echo \PYG{l+s+s2}{\PYGZdq{}Sources are: \PYGZdq{}} \PYG{k}{\PYGZdl{}(}SOURCES\PYG{k}{)}
	@echo \PYG{l+s+s2}{\PYGZdq{}Objects are: \PYGZdq{}} \PYG{k}{\PYGZdl{}(}OBJECTS\PYG{k}{)}
\end{Verbatim}


\subsection{Further reading}
\label{makefiles:further-reading}\begin{itemize}
\item {} 
\url{http://software-carpentry.org/4\_0/make/}

\item {} 
{\hyperref[biblio:biblio\string-make]{\crossref{\DUrole{std,std-ref}{Makefile references}}}} in bibliography.

\item {} 
\href{http://bashdb.sourceforge.net/remake/}{remake}, a make debugger

\end{itemize}


\section{Special functions}
\label{special_functions:special-functions}\label{special_functions::doc}\label{special_functions:id1}
There are many functions that are used so frequently that they are given
special names. Familiar examples are sin, cos, sqrt, exp, log, etc.

Most programming languages have build-in (intrinsic) functions with these
same names that can be called to compute the value of the function for
arbitrary arguments.

But ``under the hood'' these functions must be evaluated somehow.  The basic
arithmetic units of a computer are only capable of doing very basic
arithmetic in hardware: addition, subtraction, multiplication, and division.
Every other function must be \emph{approximated} by applying some sequence of
arithmetic operations.

There are several ways this might be done....


\subsection{Taylor series expansions}
\label{special_functions:taylor-series-expansions}
A finite number of terms in the Taylor series expansion of a function gives
a polynomial that approximates the desired function.  Polynomials can be
evaluated at any point using basic arithmetic.

For example, the section on {\hyperref[fortran_taylor:fortran\string-taylor]{\crossref{\DUrole{std,std-ref}{Fortran examples: Taylor series}}}} discusses an
implementation of the Taylor series approximation to the exponential
function,
\begin{quote}

\(exp(x) = 1 + x + \frac 1 2 x ^2 + \frac 1 6 x^3 + \cdots\)
\end{quote}

If this is truncated after the 4 terms shown, for example, we obtain a
polynomial of degree 3 that is easily evaluated at any point.

Some other Taylor series that can be used to approximate functions:
\begin{quote}

\(sin(x) = x - \frac{1}{3!} x^3 + \frac 1 {5!} x^5 - \frac{1}{7!} x^7 + \cdots\)

\(cos(x) = 1 - \frac 1 2 x^2 + \frac{1}{4!} x^4 - \frac{1}{6!} x^6 + \cdots\)
\end{quote}


\subsection{Newton's method for the square root}
\label{special_functions:newton-s-method-for-the-square-root}\label{special_functions:special-newton}
One way to evaluate the square root function is to use \emph{Newton's method}, a
general procedure for estimating a \emph{zero} of a function \(f(x)\), i.e. a
value \(x^*\) for which \(f(x^*) = 0\).

The square root of any value \(a>0\) is a zero of the function \(f(x)
= x^2 - a\).  (This function has two zeros, at \(\pm\sqrt{a}\).

Newton's method is based on taking a current estimate \(x_k\) to
\(x^*\) and (hopefully) improving it by setting
\begin{quote}

\(x_{k+1} = x_k - \delta_k\)
\end{quote}

The increment \(\delta_k\) is determined by \emph{linearizing} the function
\(f(x)\) about the current esimate \(x_k\) using only the linear
term in the Taylor series expansion.  This linear function \(G_k(x)\) is
\begin{quote}

\(G_k(x) = f(x_k) + f'(x_k)(x-x_k)\).
\end{quote}

The next point \(x_{k+1}\) is set to be the zero of this linear
function, which is trivial to find, and so
\begin{quote}

\(\delta_k = f(x_k) / f'(x_k)\).
\end{quote}

Geometrically, this means that we move along the tangent line to
\(f(x)\) from the point \((x_k,f(x_k))\) to the point where this
line hits the x-axis, at \((x_{k+1}, 0)\).  See the figures below.

There are several potential difficulties with this approach, e.g.
\begin{itemize}
\item {} 
To get started we require an \emph{initial guess} \(x_0\).

\item {} 
The method may converge to a zero from some starting points but not
converge at all from others, or may converge to different zeros depending
on where we start.

\end{itemize}

For example, applying Newton's method to find a zero of \(f(x) = x^2-2\)
coverges to \(\sqrt{2} \approx 1.414\) if we start at \(x_0=0.5\),
but convergest o \(-\sqrt{2} \approx -1.414\) if we start at
\(x_0=-0.5\) as illustrated in the figures below.

\includegraphics[width=10cm]{{newtonsqrt2m}.png}

\includegraphics[width=10cm]{{newtonsqrt2}.png}

An advantage of Newton's method is that it is guaranteed to converge to a
root provided the function \(f(x)\) is smooth enough and the starting
guess \(x_0\) is sufficiently close to the zero.  Moreover, in general
one usually observes \emph{quadratic convergence}.  This means that once we get
close to the zero, the error roughly satisfies
\begin{quote}

\(|x_{k+1} - x^*| \approx C|x_k-x^*|^2\)
\end{quote}

The error is roughly squared in each step.  In practice this means that once
you have 2 correct digits in the solution, the next few steps will produce
approximations with roughly 4, 8, and 16 correct digits (doubling each
time), and so it rapidly converges to full machine precision.

For example, the approximations to \(\sqrt{2}\) generated by Newton's
method starting at \(x_0=0.5\) are:

\begin{Verbatim}[commandchars=\\\{\}]
\PYG{n}{k}\PYG{p}{,} \PYG{n}{x}\PYG{p}{,} \PYG{n}{f}\PYG{p}{(}\PYG{n}{x}\PYG{p}{)}\PYG{p}{:}   \PYG{l+m+mi}{1}    \PYG{l+m+mf}{0.500000000000000E+00}   \PYG{o}{\PYGZhy{}}\PYG{l+m+mf}{0.175000E+01}
\PYG{n}{k}\PYG{p}{,} \PYG{n}{x}\PYG{p}{,} \PYG{n}{f}\PYG{p}{(}\PYG{n}{x}\PYG{p}{)}\PYG{p}{:}   \PYG{l+m+mi}{2}    \PYG{l+m+mf}{0.225000000000000E+01}    \PYG{l+m+mf}{0.306250E+01}
\PYG{n}{k}\PYG{p}{,} \PYG{n}{x}\PYG{p}{,} \PYG{n}{f}\PYG{p}{(}\PYG{n}{x}\PYG{p}{)}\PYG{p}{:}   \PYG{l+m+mi}{3}    \PYG{l+m+mf}{0.156944444444444E+01}    \PYG{l+m+mf}{0.463156E+00}
\PYG{n}{k}\PYG{p}{,} \PYG{n}{x}\PYG{p}{,} \PYG{n}{f}\PYG{p}{(}\PYG{n}{x}\PYG{p}{)}\PYG{p}{:}   \PYG{l+m+mi}{4}    \PYG{l+m+mf}{0.142189036381514E+01}    \PYG{l+m+mf}{0.217722E\PYGZhy{}01}
\PYG{n}{k}\PYG{p}{,} \PYG{n}{x}\PYG{p}{,} \PYG{n}{f}\PYG{p}{(}\PYG{n}{x}\PYG{p}{)}\PYG{p}{:}   \PYG{l+m+mi}{5}    \PYG{l+m+mf}{0.141423428594007E+01}    \PYG{l+m+mf}{0.586155E\PYGZhy{}04}
\PYG{n}{k}\PYG{p}{,} \PYG{n}{x}\PYG{p}{,} \PYG{n}{f}\PYG{p}{(}\PYG{n}{x}\PYG{p}{)}\PYG{p}{:}   \PYG{l+m+mi}{6}    \PYG{l+m+mf}{0.141421356252493E+01}    \PYG{l+m+mf}{0.429460E\PYGZhy{}09}
\PYG{n}{k}\PYG{p}{,} \PYG{n}{x}\PYG{p}{,} \PYG{n}{f}\PYG{p}{(}\PYG{n}{x}\PYG{p}{)}\PYG{p}{:}   \PYG{l+m+mi}{7}    \PYG{l+m+mf}{0.141421356237310E+01}    \PYG{l+m+mf}{0.444089E\PYGZhy{}15}
\end{Verbatim}

The last value is correct to all digits.

There are many other methods that have been developed for finding zeros of
functions, and a number of software packages that are designed to be more
robust than Newton's method (less likely to fail to converge) while still
converging very rapidly.


\section{Timing code}
\label{timing:timing}\label{timing::doc}\label{timing:timing-code}

\subsection{Unix time command}
\label{timing:unix-time-command}\label{timing:timing-unix}
There is a built in command \titleref{time} that can be used to time other commands.
Just type \titleref{time command} at the prompt, e.g.:

\begin{Verbatim}[commandchars=\\\{\}]
\PYGZdl{} time ./a.out
\PYGZlt{}output from code\PYGZgt{}

real    0m5.279s
user    0m1.915s
sys     0m0.006s
\end{Verbatim}

This executes the command \titleref{./a.out} in this case (running a Fortran
executable) and then prints information
about the time required to execute, where, roughly speaking,
\emph{real} is the wall-clock time, \emph{user} is the time spent executing the
user's program, and \emph{sys} is the time spent on system tasks required by the
program.

There may be a big difference between the \emph{real} time and the sum of the
other two times if the computer is simulataneously executing many other
tasks and your program is only getting some of its attention.

Using \emph{time} does not allow you to examine how much CPU time different parts
of the code required, for example.


\subsection{Fortran timing utilities}
\label{timing:timing-fortran}\label{timing:fortran-timing-utilities}
There are two Fortran intrinsic functions that are very useful.

\titleref{system\_clock} tells the elapsed wall time between two successive calls, and
might be used as follows:

\begin{Verbatim}[commandchars=\\\{\}]
\PYG{n}{integer}\PYG{p}{(}\PYG{n}{kind}\PYG{o}{=}\PYG{l+m+mi}{8}\PYG{p}{)} \PYG{p}{:}\PYG{p}{:} \PYG{n}{tclock1}\PYG{p}{,} \PYG{n}{tclock2}\PYG{p}{,} \PYG{n}{clock\PYGZus{}rate}
\PYG{n}{real}\PYG{p}{(}\PYG{n}{kind}\PYG{o}{=}\PYG{l+m+mi}{8}\PYG{p}{)} \PYG{p}{:}\PYG{p}{:} \PYG{n}{elapsed\PYGZus{}time}

\PYG{n}{call} \PYG{n}{system\PYGZus{}clock}\PYG{p}{(}\PYG{n}{tclock1}\PYG{p}{)}

\PYG{o}{\PYGZlt{}}\PYG{n}{code} \PYG{n}{to} \PYG{n}{be} \PYG{n}{timed}\PYG{o}{\PYGZgt{}}

\PYG{n}{call} \PYG{n}{system\PYGZus{}clock}\PYG{p}{(}\PYG{n}{tclock2}\PYG{p}{,} \PYG{n}{clock\PYGZus{}rate}\PYG{p}{)}
\PYG{n}{elapsed\PYGZus{}time} \PYG{o}{=} \PYG{n+nb}{float}\PYG{p}{(}\PYG{n}{tclock2} \PYG{o}{\PYGZhy{}} \PYG{n}{tclock1}\PYG{p}{)} \PYG{o}{/} \PYG{n+nb}{float}\PYG{p}{(}\PYG{n}{clock\PYGZus{}rate}\PYG{p}{)}
\end{Verbatim}

\titleref{cpu\_time} tells the CPU time used between two successive calls:

\begin{Verbatim}[commandchars=\\\{\}]
\PYG{n}{real}\PYG{p}{(}\PYG{n}{kind}\PYG{o}{=}\PYG{l+m+mi}{8}\PYG{p}{)} \PYG{p}{:}\PYG{p}{:} \PYG{n}{t1}\PYG{p}{,} \PYG{n}{t2}\PYG{p}{,} \PYG{n}{elapsed\PYGZus{}cpu\PYGZus{}time}

\PYG{n}{call} \PYG{n}{cpu\PYGZus{}time}\PYG{p}{(}\PYG{n}{t1}\PYG{p}{)}

\PYG{o}{\PYGZlt{}}\PYG{n}{code} \PYG{n}{to} \PYG{n}{be} \PYG{n}{timed}\PYG{o}{\PYGZgt{}}

\PYG{n}{call} \PYG{n}{cpu\PYGZus{}time}\PYG{p}{(}\PYG{n}{t2}\PYG{p}{)}
\PYG{n}{elapsed\PYGZus{}cpu\PYGZus{}time} \PYG{o}{=} \PYG{n}{t2} \PYG{o}{\PYGZhy{}} \PYG{n}{t1}
\end{Verbatim}

Here is an example code that uses both, and tests matrix-matrix multiply.

\begin{Verbatim}[commandchars=\\\{\},numbers=left,firstnumber=1,stepnumber=1]
\PYG{c}{! \PYGZdl{}UWHPSC/codes/fortran/optimize/timings.f90}

\PYG{c}{! Illustrate timing utilities in Fortran.}
\PYG{c}{!  system\PYGZus{}clock can be used to compute elapsed time between}
\PYG{c}{!      two calls (wall time)}
\PYG{c}{!  cpu\PYGZus{}time can be used to compute CPU time used between two calls.}

\PYG{c}{! Try compiling with different levels of optimization, e.g. \PYGZhy{}O3}


\PYG{k}{program }\PYG{n}{timings}

    \PYG{k}{implicit }\PYG{k}{none}
\PYG{k}{    }\PYG{k+kt}{integer}\PYG{p}{,} \PYG{k}{parameter} \PYG{k+kd}{::} \PYG{n}{ntests} \PYG{o}{=} \PYG{l+m+mi}{20}
    \PYG{k+kt}{integer} \PYG{k+kd}{::} \PYG{n}{n} 
    \PYG{k+kt}{real}\PYG{p}{(}\PYG{n+nb}{kind}\PYG{o}{=}\PYG{l+m+mi}{8}\PYG{p}{)}\PYG{p}{,} \PYG{k}{allocatable}\PYG{p}{,} \PYG{k}{dimension}\PYG{p}{(}\PYG{p}{:}\PYG{p}{,}\PYG{p}{:}\PYG{p}{)} \PYG{k+kd}{::} \PYG{n}{a}\PYG{p}{,}\PYG{n}{b}\PYG{p}{,}\PYG{n}{c}
    \PYG{k+kt}{real}\PYG{p}{(}\PYG{n+nb}{kind}\PYG{o}{=}\PYG{l+m+mi}{8}\PYG{p}{)} \PYG{k+kd}{::} \PYG{n}{t1}\PYG{p}{,} \PYG{n}{t2}\PYG{p}{,} \PYG{n}{elapsed\PYGZus{}time}
    \PYG{k+kt}{integer}\PYG{p}{(}\PYG{n+nb}{kind}\PYG{o}{=}\PYG{l+m+mi}{8}\PYG{p}{)} \PYG{k+kd}{::} \PYG{n}{tclock1}\PYG{p}{,} \PYG{n}{tclock2}\PYG{p}{,} \PYG{n}{clock\PYGZus{}rate}
    \PYG{k+kt}{integer} \PYG{k+kd}{::} \PYG{n}{i}\PYG{p}{,}\PYG{n}{j}\PYG{p}{,}\PYG{n}{k}\PYG{p}{,}\PYG{n}{itest}

    \PYG{k}{call }\PYG{n+nb}{system\PYGZus{}clock}\PYG{p}{(}\PYG{n}{tclock1}\PYG{p}{)}

    \PYG{k}{print} \PYG{o}{*}\PYG{p}{,} \PYG{l+s+s2}{\PYGZdq{}Will multiply n by n matrices, input n: \PYGZdq{}}
    \PYG{k}{read} \PYG{o}{*}\PYG{p}{,} \PYG{n}{n}

    \PYG{k}{allocate}\PYG{p}{(}\PYG{n}{a}\PYG{p}{(}\PYG{n}{n}\PYG{p}{,}\PYG{n}{n}\PYG{p}{)}\PYG{p}{,} \PYG{n}{b}\PYG{p}{(}\PYG{n}{n}\PYG{p}{,}\PYG{n}{n}\PYG{p}{)}\PYG{p}{,} \PYG{n}{c}\PYG{p}{(}\PYG{n}{n}\PYG{p}{,}\PYG{n}{n}\PYG{p}{)}\PYG{p}{)}

    \PYG{c}{! fill a and b with 1\PYGZsq{}s just for demo purposes:}
    \PYG{n}{a} \PYG{o}{=} \PYG{l+m+mf}{1.}\PYG{n}{d0}
    \PYG{n}{b} \PYG{o}{=} \PYG{l+m+mf}{1.}\PYG{n}{d0}

    \PYG{k}{call }\PYG{n+nb}{cpu\PYGZus{}time}\PYG{p}{(}\PYG{n}{t1}\PYG{p}{)}   \PYG{c}{! start cpu timer}
    \PYG{k}{do }\PYG{n}{itest}\PYG{o}{=}\PYG{l+m+mi}{1}\PYG{p}{,}\PYG{n}{ntests}
        \PYG{k}{do }\PYG{n}{j} \PYG{o}{=} \PYG{l+m+mi}{1}\PYG{p}{,}\PYG{n}{n}
            \PYG{k}{do }\PYG{n}{i} \PYG{o}{=} \PYG{l+m+mi}{1}\PYG{p}{,}\PYG{n}{n}
                \PYG{n}{c}\PYG{p}{(}\PYG{n}{i}\PYG{p}{,}\PYG{n}{j}\PYG{p}{)} \PYG{o}{=} \PYG{l+m+mf}{0.}\PYG{n}{d0}
                \PYG{k}{do }\PYG{n}{k}\PYG{o}{=}\PYG{l+m+mi}{1}\PYG{p}{,}\PYG{n}{n}
                    \PYG{n}{c}\PYG{p}{(}\PYG{n}{i}\PYG{p}{,}\PYG{n}{j}\PYG{p}{)} \PYG{o}{=} \PYG{n}{c}\PYG{p}{(}\PYG{n}{i}\PYG{p}{,}\PYG{n}{j}\PYG{p}{)} \PYG{o}{+} \PYG{n}{a}\PYG{p}{(}\PYG{n}{i}\PYG{p}{,}\PYG{n}{k}\PYG{p}{)}\PYG{o}{*}\PYG{n}{b}\PYG{p}{(}\PYG{n}{k}\PYG{p}{,}\PYG{n}{j}\PYG{p}{)}
                    \PYG{n}{enddo}
                \PYG{n}{enddo}
            \PYG{n}{enddo}
        \PYG{n}{enddo}

    \PYG{k}{call }\PYG{n+nb}{cpu\PYGZus{}time}\PYG{p}{(}\PYG{n}{t2}\PYG{p}{)}   \PYG{c}{! end cpu timer}
    \PYG{k}{print }\PYG{l+m+mi}{10}\PYG{p}{,} \PYG{n}{ntests}\PYG{p}{,} \PYG{n}{t2}\PYG{o}{\PYGZhy{}}\PYG{n}{t1}
 \PYG{l+m+mi}{10} \PYG{k}{format}\PYG{p}{(}\PYG{l+s+s2}{\PYGZdq{}Performed \PYGZdq{}}\PYG{p}{,}\PYG{n}{i4}\PYG{p}{,} \PYG{l+s+s2}{\PYGZdq{} matrix multiplies: CPU time = \PYGZdq{}}\PYG{p}{,}\PYG{n}{f12}\PYG{p}{.}\PYG{l+m+mi}{8}\PYG{p}{,} \PYG{l+s+s2}{\PYGZdq{} seconds\PYGZdq{}}\PYG{p}{)}

    
    \PYG{k}{call }\PYG{n+nb}{system\PYGZus{}clock}\PYG{p}{(}\PYG{n}{tclock2}\PYG{p}{,} \PYG{n}{clock\PYGZus{}rate}\PYG{p}{)}
    \PYG{n}{elapsed\PYGZus{}time} \PYG{o}{=} \PYG{n+nb}{float}\PYG{p}{(}\PYG{n}{tclock2} \PYG{o}{\PYGZhy{}} \PYG{n}{tclock1}\PYG{p}{)} \PYG{o}{/} \PYG{n+nb}{float}\PYG{p}{(}\PYG{n}{clock\PYGZus{}rate}\PYG{p}{)}
    \PYG{k}{print }\PYG{l+m+mi}{11}\PYG{p}{,} \PYG{n}{elapsed\PYGZus{}time}
 \PYG{l+m+mi}{11} \PYG{k}{format}\PYG{p}{(}\PYG{l+s+s2}{\PYGZdq{}Elapsed time = \PYGZdq{}}\PYG{p}{,}\PYG{n}{f12}\PYG{p}{.}\PYG{l+m+mi}{8}\PYG{p}{,} \PYG{l+s+s2}{\PYGZdq{} seconds\PYGZdq{}}\PYG{p}{)}

\PYG{k}{end }\PYG{k}{program }\PYG{n}{timings}
\end{Verbatim}

Note that the matrix-matrix product is computed 20 times over to give a
better estimate of the timings.

You might want to experiment with this code and various sizes of the
matrices and optimization levels.  Here are a few sample results on a
MacBook Pro.

First, with no optimization and \(200\times 200\) matrices:

\begin{Verbatim}[commandchars=\\\{\}]
\PYGZdl{} gfortran timings.f90
\PYGZdl{} ./a.out
 Will multiply n by n matrices, input n:
200
Performed   20 matrix multiplies: CPU time =   0.81523600 seconds
Elapsed time =   5.94083357 seconds
\end{Verbatim}

Note that the elapsed time include the time required to type in the response
to the prompt for \titleref{n}, as well as the time required to allocate and
initialize the matrices, whereas the CPU time is just for the matrix
multiplication loops.

Trying a larger \(400 \times 400\) case gives:

\begin{Verbatim}[commandchars=\\\{\}]
\PYGZdl{} ./a.out
 Will multiply n by n matrices, input n:
400
Performed   20 matrix multiplies: CPU time =   7.33721500 seconds
Elapsed time =   9.87114525 seconds
\end{Verbatim}

Since computing the product of \(n \times n\) matrices takes
\(O(n^3)\) flops,
we expect this to take about 8 times as much CPU time as the previous test.
This is roughly what we observe.

Doubling the size again gives requires much more that 8 times as long
however:

\begin{Verbatim}[commandchars=\\\{\}]
\PYGZdl{} ./a.out
 Will multiply n by n matrices, input n:
800
Performed   20 matrix multiplies: CPU time =  90.49682200 seconds
Elapsed time =  93.98917389 seconds
\end{Verbatim}

Note that 3 matrices that are \(400\times 400\) require 3.84 MB of memory,
whereas three \(800 \times 800\) matrices require 15.6 MB.  Since the MacBook
used for this experiment
has only 6 MB of L3 cache, the data no longer fit in cache.

\textbf{Compiling with optimization}

If we recompile with the -O3 optimization flag, the last two experiments
give these results:

\begin{Verbatim}[commandchars=\\\{\}]
\PYGZdl{} gfortran \PYGZhy{}O3 timings.f90
\PYGZdl{} ./a.out
 Will multiply n by n matrices, input n:
400
Performed   20 matrix multiplies: CPU time =   1.39002200 seconds
Elapsed time =   3.58041191 seconds
\end{Verbatim}

and

\begin{Verbatim}[commandchars=\\\{\}]
\PYGZdl{} ./a.out
 Will multiply n by n matrices, input n:
800
Performed   20 matrix multiplies: CPU time =  66.39167200 seconds
Elapsed time =  68.29921722 seconds
\end{Verbatim}

Both have sped up relative to the un-optimized code, the first much more
dramatically.


\subsection{Timing OpenMP code}
\label{timing:timing-openmp-code}
The code in \titleref{\$UWHPSC/codes/openmp/timings.f90} shows an analogous code for
matrix multiplication using OpenMP.

The code has been slightly modified so that \titleref{system\_clock} is only timing
the inner loops in order to illustrate that \titleref{cpu\_time} now computes the sum
of the CPU time of all threads.

Here's one sample result:

\begin{Verbatim}[commandchars=\\\{\}]
\PYGZdl{} gfortran \PYGZhy{}fopenmp \PYGZhy{}O3 timings.f90
\PYGZdl{} ./a.out
 Using OpenMP, how many threads?
4
 Will multiply n by n matrices, input n:
400
Performed   20 matrix multiplies: CPU time =   1.99064000 seconds
Elapsed time =   0.58711302 seconds
\end{Verbatim}

Note that the CPU time reported is nearly 2 seconds but the elapsed time is
only 0.58 seconds in this case when 4 threads are being used.

The total CPU time is slightly more than the previous code that did not use
OpenMP, but the wall time is considerably less.

For \(800\times 800\) matrices there is a similar speedup:

\begin{Verbatim}[commandchars=\\\{\}]
\PYGZdl{} ./a.out
 Using OpenMP, how many threads?
4
 Will multiply n by n matrices, input n:
800
Performed   20 matrix multiplies: CPU time =  79.70573500 seconds
Elapsed time =  21.37633133 seconds
\end{Verbatim}

Here is the code:

\begin{Verbatim}[commandchars=\\\{\},numbers=left,firstnumber=1,stepnumber=1]
\PYG{c}{! \PYGZdl{}UWHPSC/codes/fortran/optimize/timings1.f90}

\PYG{c}{! Illustrate timing utilities in Fortran.}
\PYG{c}{!  system\PYGZus{}clock can be used to compute elapsed time between}
\PYG{c}{!      two calls (wall time)}
\PYG{c}{!  cpu\PYGZus{}time can be used to compute CPU time used between two calls.}

\PYG{c}{! Try compiling with different levels of optimization, e.g. \PYGZhy{}O3}


\PYG{k}{program }\PYG{n}{timings1}

    \PYG{k}{use }\PYG{n}{omp\PYGZus{}lib}

    \PYG{k}{implicit }\PYG{k}{none}
\PYG{k}{    }\PYG{k+kt}{integer}\PYG{p}{,} \PYG{k}{parameter} \PYG{k+kd}{::} \PYG{n}{ntests} \PYG{o}{=} \PYG{l+m+mi}{20}
    \PYG{k+kt}{integer} \PYG{k+kd}{::} \PYG{n}{n}\PYG{p}{,} \PYG{n}{nthreads}
    \PYG{k+kt}{real}\PYG{p}{(}\PYG{n+nb}{kind}\PYG{o}{=}\PYG{l+m+mi}{8}\PYG{p}{)}\PYG{p}{,} \PYG{k}{allocatable}\PYG{p}{,} \PYG{k}{dimension}\PYG{p}{(}\PYG{p}{:}\PYG{p}{,}\PYG{p}{:}\PYG{p}{)} \PYG{k+kd}{::} \PYG{n}{a}\PYG{p}{,}\PYG{n}{b}\PYG{p}{,}\PYG{n}{c}
    \PYG{k+kt}{real}\PYG{p}{(}\PYG{n+nb}{kind}\PYG{o}{=}\PYG{l+m+mi}{8}\PYG{p}{)} \PYG{k+kd}{::} \PYG{n}{t1}\PYG{p}{,} \PYG{n}{t2}\PYG{p}{,} \PYG{n}{elapsed\PYGZus{}time}
    \PYG{k+kt}{integer}\PYG{p}{(}\PYG{n+nb}{kind}\PYG{o}{=}\PYG{l+m+mi}{8}\PYG{p}{)} \PYG{k+kd}{::} \PYG{n}{tclock1}\PYG{p}{,} \PYG{n}{tclock2}\PYG{p}{,} \PYG{n}{clock\PYGZus{}rate}
    \PYG{k+kt}{integer} \PYG{k+kd}{::} \PYG{n}{i}\PYG{p}{,}\PYG{n}{j}\PYG{p}{,}\PYG{n}{k}\PYG{p}{,}\PYG{n}{itest}

    \PYG{c}{! Specify number of threads to use:}
    \PYG{c}{!\PYGZdl{} print *, \PYGZdq{}Using OpenMP, how many threads? \PYGZdq{}}
    \PYG{c}{!\PYGZdl{} read *, nthreads }
    \PYG{c}{!\PYGZdl{} call omp\PYGZus{}set\PYGZus{}num\PYGZus{}threads(nthreads)}

    \PYG{k}{print} \PYG{o}{*}\PYG{p}{,} \PYG{l+s+s2}{\PYGZdq{}Will multiply n by n matrices, input n: \PYGZdq{}}
    \PYG{k}{read} \PYG{o}{*}\PYG{p}{,} \PYG{n}{n}

    \PYG{k}{allocate}\PYG{p}{(}\PYG{n}{a}\PYG{p}{(}\PYG{n}{n}\PYG{p}{,}\PYG{n}{n}\PYG{p}{)}\PYG{p}{,} \PYG{n}{b}\PYG{p}{(}\PYG{n}{n}\PYG{p}{,}\PYG{n}{n}\PYG{p}{)}\PYG{p}{,} \PYG{n}{c}\PYG{p}{(}\PYG{n}{n}\PYG{p}{,}\PYG{n}{n}\PYG{p}{)}\PYG{p}{)}

    \PYG{c}{! fill a and b with 1\PYGZsq{}s just for demo purposes:}
    \PYG{n}{a} \PYG{o}{=} \PYG{l+m+mf}{1.}\PYG{n}{d0}
    \PYG{n}{b} \PYG{o}{=} \PYG{l+m+mf}{1.}\PYG{n}{d0}

    \PYG{k}{call }\PYG{n+nb}{system\PYGZus{}clock}\PYG{p}{(}\PYG{n}{tclock1}\PYG{p}{)}  \PYG{c}{! start wall timer}

    \PYG{k}{call }\PYG{n+nb}{cpu\PYGZus{}time}\PYG{p}{(}\PYG{n}{t1}\PYG{p}{)}   \PYG{c}{! start cpu timer}
    \PYG{k}{do }\PYG{n}{itest}\PYG{o}{=}\PYG{l+m+mi}{1}\PYG{p}{,}\PYG{n}{ntests}
        \PYG{c}{!\PYGZdl{}omp parallel do private(i,k)}
        \PYG{k}{do }\PYG{n}{j} \PYG{o}{=} \PYG{l+m+mi}{1}\PYG{p}{,}\PYG{n}{n}
            \PYG{k}{do }\PYG{n}{i} \PYG{o}{=} \PYG{l+m+mi}{1}\PYG{p}{,}\PYG{n}{n}
                \PYG{n}{c}\PYG{p}{(}\PYG{n}{i}\PYG{p}{,}\PYG{n}{j}\PYG{p}{)} \PYG{o}{=} \PYG{l+m+mf}{0.}\PYG{n}{d0}
                \PYG{k}{do }\PYG{n}{k}\PYG{o}{=}\PYG{l+m+mi}{1}\PYG{p}{,}\PYG{n}{n}
                    \PYG{n}{c}\PYG{p}{(}\PYG{n}{i}\PYG{p}{,}\PYG{n}{j}\PYG{p}{)} \PYG{o}{=} \PYG{n}{c}\PYG{p}{(}\PYG{n}{i}\PYG{p}{,}\PYG{n}{j}\PYG{p}{)} \PYG{o}{+} \PYG{n}{a}\PYG{p}{(}\PYG{n}{i}\PYG{p}{,}\PYG{n}{k}\PYG{p}{)}\PYG{o}{*}\PYG{n}{b}\PYG{p}{(}\PYG{n}{k}\PYG{p}{,}\PYG{n}{j}\PYG{p}{)}
                    \PYG{n}{enddo}
                \PYG{n}{enddo}
            \PYG{n}{enddo}
        \PYG{n}{enddo}

    \PYG{k}{call }\PYG{n+nb}{cpu\PYGZus{}time}\PYG{p}{(}\PYG{n}{t2}\PYG{p}{)}   \PYG{c}{! end cpu timer}
    \PYG{k}{print }\PYG{l+m+mi}{10}\PYG{p}{,} \PYG{n}{ntests}\PYG{p}{,} \PYG{n}{t2}\PYG{o}{\PYGZhy{}}\PYG{n}{t1}
 \PYG{l+m+mi}{10} \PYG{k}{format}\PYG{p}{(}\PYG{l+s+s2}{\PYGZdq{}Performed \PYGZdq{}}\PYG{p}{,}\PYG{n}{i4}\PYG{p}{,} \PYG{l+s+s2}{\PYGZdq{} matrix multiplies: CPU time = \PYGZdq{}}\PYG{p}{,}\PYG{n}{f12}\PYG{p}{.}\PYG{l+m+mi}{8}\PYG{p}{,} \PYG{l+s+s2}{\PYGZdq{} seconds\PYGZdq{}}\PYG{p}{)}

    
    \PYG{k}{call }\PYG{n+nb}{system\PYGZus{}clock}\PYG{p}{(}\PYG{n}{tclock2}\PYG{p}{,} \PYG{n}{clock\PYGZus{}rate}\PYG{p}{)}
    \PYG{n}{elapsed\PYGZus{}time} \PYG{o}{=} \PYG{n+nb}{float}\PYG{p}{(}\PYG{n}{tclock2} \PYG{o}{\PYGZhy{}} \PYG{n}{tclock1}\PYG{p}{)} \PYG{o}{/} \PYG{n+nb}{float}\PYG{p}{(}\PYG{n}{clock\PYGZus{}rate}\PYG{p}{)}
    \PYG{k}{print }\PYG{l+m+mi}{11}\PYG{p}{,} \PYG{n}{elapsed\PYGZus{}time}
 \PYG{l+m+mi}{11} \PYG{k}{format}\PYG{p}{(}\PYG{l+s+s2}{\PYGZdq{}Elapsed time = \PYGZdq{}}\PYG{p}{,}\PYG{n}{f12}\PYG{p}{.}\PYG{l+m+mi}{8}\PYG{p}{,} \PYG{l+s+s2}{\PYGZdq{} seconds\PYGZdq{}}\PYG{p}{)}

\PYG{k}{end }\PYG{k}{program }\PYG{n}{timings1}
\end{Verbatim}


\section{Linear Algebra software}
\label{linalg:linear-algebra-software}\label{linalg::doc}\label{linalg:linalg}

\subsection{The BLAS}
\label{linalg:blas}\label{linalg:the-blas}
The Basic Linear Algebra Subroutines are extensively used in LAPACK, in
other linear algebra packages, and elsewhere.

There are three levels of BLAS:
\begin{itemize}
\item {} 
Level 1: Scalar and vector operations

\item {} 
Level 2: Matrix-vector operations

\item {} 
Level 3: Matrix-matrix operations

\end{itemize}

For general information about BLAS see
\url{http://www.netlib.org/blas/faq.html}.

Optimized versions of the BLAS are available for many computer
architectures.  See
\begin{itemize}
\item {} 
\href{http://xianyi.github.io/OpenBLAS/}{OpenBLAS}

\item {} 
\href{http://www.tacc.utexas.edu/tacc-projects/gotoblas2}{GotoBLAS}

\end{itemize}

See also:
\begin{itemize}
\item {} 
\href{http://math-atlas.sourceforge.net/}{Automatically Tuned Linear Algebra Software (ATLAS)}

\item {} 
\href{http://www.netlib.org/blacs/}{BLACS} Basic Linear Algebra Communication
Subprograms with message passing.

\item {} 
\href{http://www.ce.uniroma2.it/psblas/}{PSBLAS} -- Parallel sparse BLAS.

\end{itemize}


\subsection{LAPACK}
\label{linalg:lapack}\label{linalg:linalg-lapack}\begin{itemize}
\item {} 
\href{http://www.netlib.org/lapack/}{LAPACK}

\item {} 
\href{http://www.netlib.org/lapack/lug/}{LAPACK User's Guide}

\item {} 
\url{http://en.wikipedia.org/wiki/LAPACK}

\item {} 
\href{http://www.netlib.org/scalapack/}{ScaLAPACK} for parallel distributed memory
machines

\end{itemize}

To install BLAS and LAPACK to work with gfortran, see:
\begin{itemize}
\item {} 
\url{http://gcc.gnu.org/wiki/GfortranBuild}

\end{itemize}

On some linux systems, including the VM for the class, you can install both
BLAS and LAPACK via:

\begin{Verbatim}[commandchars=\\\{\}]
\PYGZdl{} sudo apt\PYGZhy{}get install liblapack\PYGZhy{}dev
\end{Verbatim}


\subsection{Direct methods for sparse systems}
\label{linalg:direct-methods-for-sparse-systems}\label{linalg:linalg-spdirect}
Although iterative methods are often used for sparse systems, there are also
excellent software packages for direct methods (such as Gaussian
elimination):
\begin{itemize}
\item {} 
\href{http://www.cise.ufl.edu/research/sparse/umfpack/}{UMFPACK}

\item {} 
\href{http://crd-legacy.lbl.gov/~xiaoye/SuperLU/}{SuperLU}

\item {} 
\href{http://graal.ens-lyon.fr/MUMPS/}{MUMPS}

\item {} 
\href{http://www.pardiso-project.org/}{Pardiso}

\end{itemize}


\subsection{Other references}
\label{linalg:other-references}\begin{itemize}
\item {} 
\DUrole{xref,std,std-ref}{lapack\_examples} for some examples.

\item {} 
\href{http://www.netlib.org/utk/people/JackDongarra/la-sw.html}{Recent list of freely available linear algebra software}

\end{itemize}


\section{Random number generators}
\label{random:random-number-generators}\label{random:random}\label{random::doc}
For some description and documentation of pseudo-random number generators,
see:
\begin{itemize}
\item {} 
\href{https://en.wikipedia.org/wiki/Pseudorandom\_number\_generator}{Wikipedia}

\item {} 
\href{http://gcc.gnu.org/onlinedocs/gfortran/RANDOM\_005fNUMBER.html}{Fortran RANDOM\_NUMBER subroutine}

\item {} 
\href{http://gcc.gnu.org/onlinedocs/gfortran/RANDOM\_005fSEED.html\#RANDOM\_005fSEED}{Fortran RANDOM\_SEED subroutine}

\item {} 
\href{http://docs.python.org/2/library/random.html}{Python random}

\item {} 
\href{http://docs.scipy.org/doc/numpy/reference/routines.random.html}{More Python options}

\item {} 
\href{http://mira.math.udel.edu/ParallelKMC/doku.php?id=software}{OpenCL PRNG}

\end{itemize}


\chapter{Applications}
\label{index:applications}\label{index:toc-apps}

\section{Numerical methods for the Poisson problem}
\label{poisson:numerical-methods-for-the-poisson-problem}\label{poisson::doc}\label{poisson:poisson}
The steady state diffusion equation gives rise to a \emph{Poisson problem}

\(u_{xx} + u_{yy} = -f(x,y)\)

where \(f(x,y)\) is the source term.  In the simplest case
\(f(x,y) = 0\) this reduces to \emph{Laplace's equation}.
This must be augmented with boundary conditions around the edge of some
two-dimensional region.  \emph{Dirichlet boundary conditions} consist of
specifying the solution \(u(x,y)\) at all points around the boundary.
\emph{Neumann boundary conditions} consist of specifying the normal derivative
(i.e. the direction derivative of the solution in the direction orthogonal
to the boundary) and are used in physical situations where the if the flux of
heat or the diffused quantity is known along the boundary rather than the
value of the solution itself (for example an \emph{insulated boundary} has no
flux and the normal derivative is zero).  We will only study Dirichlet
problems, where \(u\) itself is known at boundary points.  We will also
concentrate on problems in a rectangular domain \(a_x < x < b_x\) and
\(a_y < y < b_y\), in which case it is natural to discretize
on a \emph{Cartesian grid} aligned with the axes.

The Poisson problem can be discretized on a two-dimensional Cartesian grid
with equal grid
spacing \(h\) in the \(x\) and \(y\) directions as

\(U_{i-1,j} + U_{i+1,j} + U_{i,j-1} + U_{i,j+1} - 4u_{ij} = -h^2
f(x_i,y_j)\).

This gives a coupled system of equations with \(n_x n_y\) unknowns,
where it is assumed that \(h(n_x+1) = b_x - a_x\) and
\(h(n_y+1) = b_y - a_y\).  The linear system has a very sparse
coefficient matrix since each of the \(n_x n_y\) rows has at most 5
nonzero entries.

If the boundary data varies smoothly around the boundary then it can be
shown that solving this linear system gives an approximate solution
of the partial differential equation that is \({\cal O}(h^2)\) accurate
at each point.  There are many books that contain much
more about the development and analysis of such finite difference methods.


\subsection{Iterative methods for the Poisson problem}
\label{poisson:poisson-iter}\label{poisson:iterative-methods-for-the-poisson-problem}
Simple iterative methods such as Jacobi, Gauss-Siedel, and Successive
Over-Relaxation (SOR) are discussed in the lectures and used as examples for
implementations in OpenMP and MPI.  For three implementation of Jacobi in
one space dimension, see
\begin{itemize}
\item {} 
{\hyperref[jacobi1d_omp1:jacobi1d\string-omp1]{\crossref{\DUrole{std,std-ref}{Jacobi iteration using OpenMP with parallel do constructs}}}}

\item {} 
{\hyperref[jacobi1d_omp2:jacobi1d\string-omp2]{\crossref{\DUrole{std,std-ref}{Jacobi iteration using OpenMP with coarse-grain parallel block}}}}

\item {} 
{\hyperref[jacobi1d_mpi:jacobi1d\string-mpi]{\crossref{\DUrole{std,std-ref}{Jacobi iteration using MPI}}}}

\end{itemize}

A sample implementation of Jacobi in two space dimensions can be found
in \titleref{\$UWHPSC/lectures/lecture1}.


\subsection{Monte Carlo methods for the steady state diffusion equation}
\label{poisson:poisson-mc}\label{poisson:monte-carlo-methods-for-the-steady-state-diffusion-equation}
Solving the linear system described above would give an approximate solution
to the Poisson problem at each point on the grid.  Suppose we only want to
approximate the solution at a single point \((x_0,y_0)\) for some reason.
Is there a way
to estimate this without solving the system for all values \(U_{ij}\)?
Not easily from the linear system, but there are other approaches that might
be used.

We will consider a Monte Carlo approach in which a large number of
\emph{random walks} starting from the point of interest are used to estimate the
solution.  See {\hyperref[random:random]{\crossref{\DUrole{std,std-ref}{Random number generators}}}} for a discussion of random number generators
and Monte Carlo methods more generally.

We will assume there is no source term, \(f(x,y)=0\) so that we are
solving Laplace's equation.  The random walk solution is more complicated if
there is a source term.

A random walk starting at some point \((x_0,y_0)\) wanders randomly in
the domain until it hits the boundary at some point.  We do this many times
over and keep track of the boundary value given for \(u(x,y)\) at the
point where each walk hits the boundary.  It can be shown that if we
do this for
a large number of walks and average the results, this converges to the
desired solution value \(u(x_0,y_0)\).  Note that we expect more walks
to hit the boundary at parts of the boundary near \((x_0,y_0)\) than
at points further away, so the boundary conditions at such points will have
more influence on the solution.  This is intuitively what we expect for a
steady state solution of a diffusion or heat conduction problem.

To implement this numerically we will consider the simplification
of a \emph{lattice random walk}, in which we put down a grid on the domain as in
the finite difference discretization and allow the random walk to only go in
one of 4 directions in each time step, from a point on the grid to one of
its four neighbors.  For isotropic diffusion as we are considering,
we can define a random walk by choosing 1 of the four
neighbors with equal probability in each step.

The code \titleref{\$UWHPSC/codes/project/laplace\_mc.py} illustrates this.
Run this code with
\titleref{plot\_walk = True} to see plots of a few random walks on a coarse grid, or with
\titleref{plot\_walk = False} to report the solution after many random walks on a finer
grid.

With this lattice random walk we do not expect the approximate solution to
converge to the true solution of the PDE, as the number of trials increases.
Instead we expect it to converge to the solution of the linear system
determined by the finite difference method described above.
In other words if we choose \((x_0,y_0) = (x_i, y_j)\) for some grid
point \((i,j)\) then we expect the Monte Carlo solution to converge to
\(U_{ij}\) rather than to \(u(x_i,y_j)\).

\textbf{Why does this work?}  Here's one way to think about it.  Suppose doing this
random walk starting at \((x_i,y_j)\) converges to some value \(E_{ij}\),
the expected value of \(u\) at the boundary hit when starting a random walk at this
point.  If \((x_i,y_j)\) is one of the boundary points then
\(E_{ij} = U_{ij}\) since we immediately hit the boundary with zero
steps, so every random walk starting at this point returns \(u\) at this
point.  On the other hand, if \((x_i,y_j)\) is an interior point, then
after a single step of the random walk we will be at one of the four
neighbors.  Continuing our original random walk from this point is
equivalent to starting a new random walk at this point.  So for example
any random walk that first takes a step to the right from \((x_i,y_j)\)
to \((x_{i+1},y_j)\) has the same expected boundary value as obtained
from all random walks starting at \((x_{i+1},y_j)\), i.e. the value
\(E_{i+1,j}\).  But only 1/4 of the random walks starting at
\((x_i,y_j)\) go first to the right.  So the expected value over all
walks starting at \((x_i,y_j)\) is expected to be the average of the
expected value when starting at any of the 4 neighbors.  In other words,

\(E_{ij} = \frac 1 4 (E_{i-1,j} + E_{i+1,j} + E_{i,j-1} + E_{i,j+1})\)

But this means \(E_{ij}\) satisfies the same linear system of equations
as \(U_{ij}\) (and also the same boundary conditions),
and hence must be the same.


\section{Jacobi iteration using OpenMP with \titleref{parallel do} constructs}
\label{jacobi1d_omp1:jacobi-iteration-using-openmp-with-parallel-do-constructs}\label{jacobi1d_omp1:jacobi1d-omp1}\label{jacobi1d_omp1::doc}
The code below implements Jacobi iteration for solving the linear system
arising from the steady state heat equation
with a simple application of \titleref{parallel do} loops using OpenMP.

Compare to:
\begin{itemize}
\item {} 
{\hyperref[jacobi1d_omp2:jacobi1d\string-omp2]{\crossref{\DUrole{std,std-ref}{Jacobi iteration using OpenMP with coarse-grain parallel block}}}}

\item {} 
{\hyperref[jacobi1d_mpi:jacobi1d\string-mpi]{\crossref{\DUrole{std,std-ref}{Jacobi iteration using MPI}}}}

\end{itemize}

The code:

\begin{Verbatim}[commandchars=\\\{\},numbers=left,firstnumber=1,stepnumber=1]
\PYG{c}{! \PYGZdl{}UWHPSC/codes/openmp/jacobi1d\PYGZus{}omp1.f90}
\PYG{c}{!}
\PYG{c}{! Jacobi iteration illustrating fine grain parallelism with OpenMP.}
\PYG{c}{!}
\PYG{c}{! Several omp parallel do loops are used.  Each time threads will be}
\PYG{c}{! forked and the compiler will decide how to split up the loop.}

\PYG{k}{program }\PYG{n}{jacobi1d\PYGZus{}omp1}
    \PYG{k}{use }\PYG{n}{omp\PYGZus{}lib}
    \PYG{k}{implicit }\PYG{k}{none}
\PYG{k}{    }\PYG{k+kt}{integer} \PYG{k+kd}{::} \PYG{n}{n}\PYG{p}{,} \PYG{n}{nthreads}
    \PYG{k+kt}{real}\PYG{p}{(}\PYG{n+nb}{kind}\PYG{o}{=}\PYG{l+m+mi}{8}\PYG{p}{)}\PYG{p}{,} \PYG{k}{dimension}\PYG{p}{(}\PYG{p}{:}\PYG{p}{)}\PYG{p}{,} \PYG{k}{allocatable} \PYG{k+kd}{::} \PYG{n}{x}\PYG{p}{,}\PYG{n}{u}\PYG{p}{,}\PYG{n}{uold}\PYG{p}{,}\PYG{n}{f}
    \PYG{k+kt}{real}\PYG{p}{(}\PYG{n+nb}{kind}\PYG{o}{=}\PYG{l+m+mi}{8}\PYG{p}{)} \PYG{k+kd}{::} \PYG{n}{alpha}\PYG{p}{,} \PYG{n}{beta}\PYG{p}{,} \PYG{n}{dx}\PYG{p}{,} \PYG{n}{tol}\PYG{p}{,} \PYG{n}{dumax}
    \PYG{k+kt}{real}\PYG{p}{(}\PYG{n+nb}{kind}\PYG{o}{=}\PYG{l+m+mi}{8}\PYG{p}{)}\PYG{p}{,} \PYG{k}{intrinsic} \PYG{k+kd}{::} \PYG{n+nb}{exp}
\PYG{n+nb}{    }\PYG{k+kt}{real}\PYG{p}{(}\PYG{n+nb}{kind}\PYG{o}{=}\PYG{l+m+mi}{8}\PYG{p}{)} \PYG{k+kd}{::} \PYG{n}{t1}\PYG{p}{,}\PYG{n}{t2}
    \PYG{k+kt}{integer} \PYG{k+kd}{::} \PYG{n}{i}\PYG{p}{,}\PYG{n}{iter}\PYG{p}{,}\PYG{n}{maxiter} 

    \PYG{c}{! Specify number of threads to use:}
    \PYG{n}{nthreads} \PYG{o}{=} \PYG{l+m+mi}{2}
    \PYG{c}{!\PYGZdl{} call omp\PYGZus{}set\PYGZus{}num\PYGZus{}threads(nthreads)}
    \PYG{c}{!\PYGZdl{} print \PYGZdq{}(\PYGZsq{}Using OpenMP with \PYGZsq{},i3,\PYGZsq{} threads\PYGZsq{})\PYGZdq{}, nthreads}

    \PYG{k}{print} \PYG{o}{*}\PYG{p}{,} \PYG{l+s+s2}{\PYGZdq{}Input n ... \PYGZdq{}}
    \PYG{k}{read} \PYG{o}{*}\PYG{p}{,} \PYG{n}{n}

    \PYG{c}{! allocate storage for boundary points too:}
    \PYG{k}{allocate}\PYG{p}{(}\PYG{n}{x}\PYG{p}{(}\PYG{l+m+mi}{0}\PYG{p}{:}\PYG{n}{n}\PYG{o}{+}\PYG{l+m+mi}{1}\PYG{p}{)}\PYG{p}{,} \PYG{n}{u}\PYG{p}{(}\PYG{l+m+mi}{0}\PYG{p}{:}\PYG{n}{n}\PYG{o}{+}\PYG{l+m+mi}{1}\PYG{p}{)}\PYG{p}{,} \PYG{n}{uold}\PYG{p}{(}\PYG{l+m+mi}{0}\PYG{p}{:}\PYG{n}{n}\PYG{o}{+}\PYG{l+m+mi}{1}\PYG{p}{)}\PYG{p}{,} \PYG{n}{f}\PYG{p}{(}\PYG{l+m+mi}{0}\PYG{p}{:}\PYG{n}{n}\PYG{o}{+}\PYG{l+m+mi}{1}\PYG{p}{)}\PYG{p}{)}

    \PYG{k}{open}\PYG{p}{(}\PYG{n}{unit}\PYG{o}{=}\PYG{l+m+mi}{20}\PYG{p}{,} \PYG{k}{file}\PYG{o}{=}\PYG{l+s+s2}{\PYGZdq{}heatsoln.txt\PYGZdq{}}\PYG{p}{,} \PYG{n}{status}\PYG{o}{=}\PYG{l+s+s2}{\PYGZdq{}unknown\PYGZdq{}}\PYG{p}{)}

    \PYG{k}{call }\PYG{n+nb}{cpu\PYGZus{}time}\PYG{p}{(}\PYG{n}{t1}\PYG{p}{)}

    \PYG{c}{! grid spacing:}
    \PYG{n}{dx} \PYG{o}{=} \PYG{l+m+mf}{1.}\PYG{n}{d0} \PYG{o}{/} \PYG{p}{(}\PYG{n}{n}\PYG{o}{+}\PYG{l+m+mf}{1.}\PYG{n}{d0}\PYG{p}{)}

    \PYG{c}{! boundary conditions:}
    \PYG{n}{alpha} \PYG{o}{=} \PYG{l+m+mi}{2}\PYG{l+m+mf}{0.}\PYG{n}{d0}
    \PYG{n}{beta} \PYG{o}{=} \PYG{l+m+mi}{6}\PYG{l+m+mf}{0.}\PYG{n}{d0}

    \PYG{c}{!\PYGZdl{}omp parallel do}
    \PYG{k}{do }\PYG{n}{i}\PYG{o}{=}\PYG{l+m+mi}{0}\PYG{p}{,}\PYG{n}{n}\PYG{o}{+}\PYG{l+m+mi}{1}
        \PYG{c}{! grid points:}
        \PYG{n}{x}\PYG{p}{(}\PYG{n}{i}\PYG{p}{)} \PYG{o}{=} \PYG{n}{i}\PYG{o}{*}\PYG{n}{dx}
        \PYG{c}{! source term:}
        \PYG{n}{f}\PYG{p}{(}\PYG{n}{i}\PYG{p}{)} \PYG{o}{=} \PYG{l+m+mi}{10}\PYG{l+m+mf}{0.}\PYG{o}{*}\PYG{n+nb}{exp}\PYG{p}{(}\PYG{n}{x}\PYG{p}{(}\PYG{n}{i}\PYG{p}{)}\PYG{p}{)}
        \PYG{c}{! initial guess:}
        \PYG{n}{u}\PYG{p}{(}\PYG{n}{i}\PYG{p}{)} \PYG{o}{=} \PYG{n}{alpha} \PYG{o}{+} \PYG{n}{x}\PYG{p}{(}\PYG{n}{i}\PYG{p}{)}\PYG{o}{*}\PYG{p}{(}\PYG{n}{beta}\PYG{o}{\PYGZhy{}}\PYG{n}{alpha}\PYG{p}{)}
        \PYG{n}{enddo}

    \PYG{c}{! tolerance and max number of iterations:}
    \PYG{n}{tol} \PYG{o}{=} \PYG{l+m+mf}{0.1} \PYG{o}{*} \PYG{n}{dx}\PYG{o}{**}\PYG{l+m+mi}{2}
    \PYG{k}{print} \PYG{o}{*}\PYG{p}{,} \PYG{l+s+s2}{\PYGZdq{}Convergence tolerance: tol = \PYGZdq{}}\PYG{p}{,}\PYG{n}{tol}
    \PYG{n}{maxiter} \PYG{o}{=} \PYG{l+m+mi}{100000}
    \PYG{k}{print} \PYG{o}{*}\PYG{p}{,} \PYG{l+s+s2}{\PYGZdq{}Maximum number of iterations: maxiter = \PYGZdq{}}\PYG{p}{,}\PYG{n}{maxiter}

    \PYG{c}{! Jacobi iteratation:}

    \PYG{n}{uold} \PYG{o}{=} \PYG{n}{u}  \PYG{c}{! starting values before updating}

    \PYG{k}{do }\PYG{n}{iter}\PYG{o}{=}\PYG{l+m+mi}{1}\PYG{p}{,}\PYG{n}{maxiter}
        \PYG{n}{dumax} \PYG{o}{=} \PYG{l+m+mf}{0.}\PYG{n}{d0}
        \PYG{c}{!\PYGZdl{}omp parallel do reduction(max : dumax)}
        \PYG{k}{do }\PYG{n}{i}\PYG{o}{=}\PYG{l+m+mi}{1}\PYG{p}{,}\PYG{n}{n}
            \PYG{n}{u}\PYG{p}{(}\PYG{n}{i}\PYG{p}{)} \PYG{o}{=} \PYG{l+m+mf}{0.5}\PYG{n}{d0}\PYG{o}{*}\PYG{p}{(}\PYG{n}{uold}\PYG{p}{(}\PYG{n}{i}\PYG{o}{\PYGZhy{}}\PYG{l+m+mi}{1}\PYG{p}{)} \PYG{o}{+} \PYG{n}{uold}\PYG{p}{(}\PYG{n}{i}\PYG{o}{+}\PYG{l+m+mi}{1}\PYG{p}{)} \PYG{o}{+} \PYG{n}{dx}\PYG{o}{**}\PYG{l+m+mi}{2}\PYG{o}{*}\PYG{n}{f}\PYG{p}{(}\PYG{n}{i}\PYG{p}{)}\PYG{p}{)}
            \PYG{n}{dumax} \PYG{o}{=} \PYG{n+nb}{max}\PYG{p}{(}\PYG{n}{dumax}\PYG{p}{,} \PYG{n+nb}{abs}\PYG{p}{(}\PYG{n}{u}\PYG{p}{(}\PYG{n}{i}\PYG{p}{)}\PYG{o}{\PYGZhy{}}\PYG{n}{uold}\PYG{p}{(}\PYG{n}{i}\PYG{p}{)}\PYG{p}{)}\PYG{p}{)}
            \PYG{n}{enddo}
        \PYG{k}{if} \PYG{p}{(}\PYG{n+nb}{mod}\PYG{p}{(}\PYG{n}{iter}\PYG{p}{,}\PYG{l+m+mi}{10000}\PYG{p}{)}\PYG{o}{==}\PYG{l+m+mi}{0}\PYG{p}{)} \PYG{k}{then}
\PYG{k}{            }\PYG{k}{print} \PYG{o}{*}\PYG{p}{,} \PYG{n}{iter}\PYG{p}{,} \PYG{n}{dumax}
            \PYG{n}{endif}
        \PYG{c}{! check for convergence:}
        \PYG{k}{if} \PYG{p}{(}\PYG{n}{dumax} \PYG{p}{.}\PYG{n}{lt}\PYG{p}{.} \PYG{n}{tol}\PYG{p}{)} \PYG{k}{exit}

        \PYG{c}{!\PYGZdl{}omp parallel do }
        \PYG{k}{do }\PYG{n}{i}\PYG{o}{=}\PYG{l+m+mi}{1}\PYG{p}{,}\PYG{n}{n}
            \PYG{n}{uold}\PYG{p}{(}\PYG{n}{i}\PYG{p}{)} \PYG{o}{=} \PYG{n}{u}\PYG{p}{(}\PYG{n}{i}\PYG{p}{)}   \PYG{c}{! for next iteration}
            \PYG{n}{enddo}
        \PYG{n}{enddo}

        \PYG{k}{call }\PYG{n+nb}{cpu\PYGZus{}time}\PYG{p}{(}\PYG{n}{t2}\PYG{p}{)}
        \PYG{k}{print} \PYG{l+s+s1}{\PYGZsq{}(\PYGZdq{}CPU time = \PYGZdq{},f12.8, \PYGZdq{} seconds\PYGZdq{})\PYGZsq{}}\PYG{p}{,} \PYG{n}{t2}\PYG{o}{\PYGZhy{}}\PYG{n}{t1}

        \PYG{k}{print} \PYG{o}{*}\PYG{p}{,} \PYG{l+s+s2}{\PYGZdq{}Total number of iterations: \PYGZdq{}}\PYG{p}{,}\PYG{n}{iter}

    \PYG{k}{write}\PYG{p}{(}\PYG{l+m+mi}{20}\PYG{p}{,}\PYG{o}{*}\PYG{p}{)} \PYG{l+s+s2}{\PYGZdq{}          x                  u\PYGZdq{}}
    \PYG{k}{do }\PYG{n}{i}\PYG{o}{=}\PYG{l+m+mi}{0}\PYG{p}{,}\PYG{n}{n}\PYG{o}{+}\PYG{l+m+mi}{1}
        \PYG{k}{write}\PYG{p}{(}\PYG{l+m+mi}{20}\PYG{p}{,}\PYG{l+s+s1}{\PYGZsq{}(2e20.10)\PYGZsq{}}\PYG{p}{)}\PYG{p}{,} \PYG{n}{x}\PYG{p}{(}\PYG{n}{i}\PYG{p}{)}\PYG{p}{,} \PYG{n}{u}\PYG{p}{(}\PYG{n}{i}\PYG{p}{)}
        \PYG{n}{enddo}

    \PYG{k}{print} \PYG{o}{*}\PYG{p}{,} \PYG{l+s+s2}{\PYGZdq{}Solution is in heatsoln.txt\PYGZdq{}}


    \PYG{k}{close}\PYG{p}{(}\PYG{l+m+mi}{20}\PYG{p}{)}

\PYG{k}{end }\PYG{k}{program }\PYG{n}{jacobi1d\PYGZus{}omp1}
\end{Verbatim}


\section{Jacobi iteration using OpenMP with coarse-grain \titleref{parallel} block}
\label{jacobi1d_omp2:jacobi1d-omp2}\label{jacobi1d_omp2::doc}\label{jacobi1d_omp2:jacobi-iteration-using-openmp-with-coarse-grain-parallel-block}
The code below implements Jacobi iteration for solving the linear system
arising from the steady state heat equation
with a single \titleref{parallel} block.  Work is split up manually between threads.

Compare to:
\begin{itemize}
\item {} 
{\hyperref[jacobi1d_omp1:jacobi1d\string-omp1]{\crossref{\DUrole{std,std-ref}{Jacobi iteration using OpenMP with parallel do constructs}}}}

\item {} 
{\hyperref[jacobi1d_mpi:jacobi1d\string-mpi]{\crossref{\DUrole{std,std-ref}{Jacobi iteration using MPI}}}}

\end{itemize}

The code:

\begin{Verbatim}[commandchars=\\\{\},numbers=left,firstnumber=1,stepnumber=1]
\PYG{c}{! \PYGZdl{}UWHPSC/codes/openmp/jacobi1d\PYGZus{}omp2.f90}
\PYG{c}{!}
\PYG{c}{! Domain decomposition version of Jacobi iteration illustrating}
\PYG{c}{! coarse grain parallelism with OpenMP.}
\PYG{c}{!}
\PYG{c}{! The grid points are split up into nthreads disjoint sets and each thread}
\PYG{c}{! is assigned one set that it updates for all iterations.}

\PYG{k}{program }\PYG{n}{jacobi1d\PYGZus{}omp2}
    \PYG{k}{use }\PYG{n}{omp\PYGZus{}lib}
    \PYG{k}{implicit }\PYG{k}{none}
\PYG{k}{    }\PYG{k+kt}{real}\PYG{p}{(}\PYG{n+nb}{kind}\PYG{o}{=}\PYG{l+m+mi}{8}\PYG{p}{)}\PYG{p}{,} \PYG{k}{dimension}\PYG{p}{(}\PYG{p}{:}\PYG{p}{)}\PYG{p}{,} \PYG{k}{allocatable} \PYG{k+kd}{::} \PYG{n}{x}\PYG{p}{,}\PYG{n}{u}\PYG{p}{,}\PYG{n}{uold}\PYG{p}{,}\PYG{n}{f}
    \PYG{k+kt}{real}\PYG{p}{(}\PYG{n+nb}{kind}\PYG{o}{=}\PYG{l+m+mi}{8}\PYG{p}{)} \PYG{k+kd}{::} \PYG{n}{alpha}\PYG{p}{,} \PYG{n}{beta}\PYG{p}{,} \PYG{n}{dx}\PYG{p}{,} \PYG{n}{tol}\PYG{p}{,} \PYG{n}{dumax}\PYG{p}{,} \PYG{n}{dumax\PYGZus{}thread}
    \PYG{k+kt}{real}\PYG{p}{(}\PYG{n+nb}{kind}\PYG{o}{=}\PYG{l+m+mi}{8}\PYG{p}{)}\PYG{p}{,} \PYG{k}{intrinsic} \PYG{k+kd}{::} \PYG{n+nb}{exp}
\PYG{n+nb}{    }\PYG{k+kt}{real}\PYG{p}{(}\PYG{n+nb}{kind}\PYG{o}{=}\PYG{l+m+mi}{8}\PYG{p}{)} \PYG{k+kd}{::} \PYG{n}{t1}\PYG{p}{,}\PYG{n}{t2}
    \PYG{k+kt}{integer} \PYG{k+kd}{::} \PYG{n}{n}\PYG{p}{,} \PYG{n}{nthreads}\PYG{p}{,} \PYG{n}{points\PYGZus{}per\PYGZus{}thread}\PYG{p}{,}\PYG{n}{thread\PYGZus{}num}
    \PYG{k+kt}{integer} \PYG{k+kd}{::} \PYG{n}{i}\PYG{p}{,}\PYG{n}{iter}\PYG{p}{,}\PYG{n}{maxiter}\PYG{p}{,}\PYG{n}{istart}\PYG{p}{,}\PYG{n}{iend}\PYG{p}{,}\PYG{n}{nprint}

    \PYG{c}{! Specify number of threads to use:}
    \PYG{n}{nthreads} \PYG{o}{=} \PYG{l+m+mi}{1}       \PYG{c}{! need this value in serial mode}
    \PYG{c}{!\PYGZdl{} nthreads = 2    }
    \PYG{c}{!\PYGZdl{} call omp\PYGZus{}set\PYGZus{}num\PYGZus{}threads(nthreads)}
    \PYG{c}{!\PYGZdl{} print \PYGZdq{}(\PYGZsq{}Using OpenMP with \PYGZsq{},i3,\PYGZsq{} threads\PYGZsq{})\PYGZdq{}, nthreads}

    \PYG{n}{nprint} \PYG{o}{=} \PYG{l+m+mi}{10000} \PYG{c}{! print dumax every nprint iterations}

    \PYG{k}{print} \PYG{o}{*}\PYG{p}{,} \PYG{l+s+s2}{\PYGZdq{}Input n ... \PYGZdq{}}
    \PYG{k}{read} \PYG{o}{*}\PYG{p}{,} \PYG{n}{n}

    \PYG{c}{! allocate storage for boundary points too:}
    \PYG{k}{allocate}\PYG{p}{(}\PYG{n}{x}\PYG{p}{(}\PYG{l+m+mi}{0}\PYG{p}{:}\PYG{n}{n}\PYG{o}{+}\PYG{l+m+mi}{1}\PYG{p}{)}\PYG{p}{,} \PYG{n}{u}\PYG{p}{(}\PYG{l+m+mi}{0}\PYG{p}{:}\PYG{n}{n}\PYG{o}{+}\PYG{l+m+mi}{1}\PYG{p}{)}\PYG{p}{,} \PYG{n}{uold}\PYG{p}{(}\PYG{l+m+mi}{0}\PYG{p}{:}\PYG{n}{n}\PYG{o}{+}\PYG{l+m+mi}{1}\PYG{p}{)}\PYG{p}{,} \PYG{n}{f}\PYG{p}{(}\PYG{l+m+mi}{0}\PYG{p}{:}\PYG{n}{n}\PYG{o}{+}\PYG{l+m+mi}{1}\PYG{p}{)}\PYG{p}{)}

    \PYG{k}{open}\PYG{p}{(}\PYG{n}{unit}\PYG{o}{=}\PYG{l+m+mi}{20}\PYG{p}{,} \PYG{k}{file}\PYG{o}{=}\PYG{l+s+s2}{\PYGZdq{}heatsoln.txt\PYGZdq{}}\PYG{p}{,} \PYG{n}{status}\PYG{o}{=}\PYG{l+s+s2}{\PYGZdq{}unknown\PYGZdq{}}\PYG{p}{)}

    \PYG{k}{call }\PYG{n+nb}{cpu\PYGZus{}time}\PYG{p}{(}\PYG{n}{t1}\PYG{p}{)}

    \PYG{c}{! grid spacing:}
    \PYG{n}{dx} \PYG{o}{=} \PYG{l+m+mf}{1.}\PYG{n}{d0} \PYG{o}{/} \PYG{p}{(}\PYG{n}{n}\PYG{o}{+}\PYG{l+m+mf}{1.}\PYG{n}{d0}\PYG{p}{)}

    \PYG{c}{! boundary conditions:}
    \PYG{n}{alpha} \PYG{o}{=} \PYG{l+m+mi}{2}\PYG{l+m+mf}{0.}\PYG{n}{d0}
    \PYG{n}{beta} \PYG{o}{=} \PYG{l+m+mi}{6}\PYG{l+m+mf}{0.}\PYG{n}{d0}

    \PYG{c}{! tolerance and max number of iterations:}
    \PYG{n}{tol} \PYG{o}{=} \PYG{l+m+mf}{0.1} \PYG{o}{*} \PYG{n}{dx}\PYG{o}{**}\PYG{l+m+mi}{2}
    \PYG{k}{print} \PYG{o}{*}\PYG{p}{,} \PYG{l+s+s2}{\PYGZdq{}Convergence tolerance: tol = \PYGZdq{}}\PYG{p}{,}\PYG{n}{tol}
    \PYG{n}{maxiter} \PYG{o}{=} \PYG{l+m+mi}{100000}
    \PYG{k}{print} \PYG{o}{*}\PYG{p}{,} \PYG{l+s+s2}{\PYGZdq{}Maximum number of iterations: maxiter = \PYGZdq{}}\PYG{p}{,}\PYG{n}{maxiter}

    \PYG{c}{! Determine how many points to handle with each thread.}
    \PYG{c}{! Note that dividing two integers and assigning to an integer will}
    \PYG{c}{! round down if the result is not an integer.  }
    \PYG{c}{! This, together with the min(...) in the definition of iend below,}
    \PYG{c}{! insures that all points will get distributed to some thread.}
    \PYG{n}{points\PYGZus{}per\PYGZus{}thread} \PYG{o}{=} \PYG{p}{(}\PYG{n}{n} \PYG{o}{+} \PYG{n}{nthreads} \PYG{o}{\PYGZhy{}} \PYG{l+m+mi}{1}\PYG{p}{)} \PYG{o}{/} \PYG{n}{nthreads}
    \PYG{k}{print} \PYG{o}{*}\PYG{p}{,} \PYG{l+s+s2}{\PYGZdq{}points\PYGZus{}per\PYGZus{}thread = \PYGZdq{}}\PYG{p}{,}\PYG{n}{points\PYGZus{}per\PYGZus{}thread}


    \PYG{c}{! Start of the parallel block... }
    \PYG{c}{! \PYGZhy{}\PYGZhy{}\PYGZhy{}\PYGZhy{}\PYGZhy{}\PYGZhy{}\PYGZhy{}\PYGZhy{}\PYGZhy{}\PYGZhy{}\PYGZhy{}\PYGZhy{}\PYGZhy{}\PYGZhy{}\PYGZhy{}\PYGZhy{}\PYGZhy{}\PYGZhy{}\PYGZhy{}\PYGZhy{}\PYGZhy{}\PYGZhy{}\PYGZhy{}\PYGZhy{}\PYGZhy{}\PYGZhy{}\PYGZhy{}\PYGZhy{}\PYGZhy{}\PYGZhy{}}

    \PYG{c}{! This is the only time threads are forked in this program:}

    \PYG{c}{!\PYGZdl{}omp parallel private(thread\PYGZus{}num, iter, istart, iend, i, dumax\PYGZus{}thread) }

    \PYG{n}{thread\PYGZus{}num} \PYG{o}{=} \PYG{l+m+mi}{0}     \PYG{c}{! needed in serial mode}
    \PYG{c}{!\PYGZdl{} thread\PYGZus{}num = omp\PYGZus{}get\PYGZus{}thread\PYGZus{}num()    ! unique for each thread}

    \PYG{c}{! Determine start and end index for the set of points to be }
    \PYG{c}{! handled by this thread:}
    \PYG{n}{istart} \PYG{o}{=} \PYG{n}{thread\PYGZus{}num} \PYG{o}{*} \PYG{n}{points\PYGZus{}per\PYGZus{}thread} \PYG{o}{+} \PYG{l+m+mi}{1}
    \PYG{n}{iend} \PYG{o}{=} \PYG{n+nb}{min}\PYG{p}{(}\PYG{p}{(}\PYG{n}{thread\PYGZus{}num}\PYG{o}{+}\PYG{l+m+mi}{1}\PYG{p}{)} \PYG{o}{*} \PYG{n}{points\PYGZus{}per\PYGZus{}thread}\PYG{p}{,} \PYG{n}{n}\PYG{p}{)}

    \PYG{c}{!\PYGZdl{}omp critical}
    \PYG{k}{print} \PYG{l+s+s1}{\PYGZsq{}(\PYGZdq{}Thread \PYGZdq{},i2,\PYGZdq{} will take i = \PYGZdq{},i6,\PYGZdq{} through i = \PYGZdq{},i6)\PYGZsq{}}\PYG{p}{,} \PYG{p}{\PYGZam{}}
          \PYG{n}{thread\PYGZus{}num}\PYG{p}{,} \PYG{n}{istart}\PYG{p}{,} \PYG{n}{iend}
    \PYG{c}{!\PYGZdl{}omp end critical}

    \PYG{c}{! Initialize:}
    \PYG{c}{! \PYGZhy{}\PYGZhy{}\PYGZhy{}\PYGZhy{}\PYGZhy{}\PYGZhy{}\PYGZhy{}\PYGZhy{}\PYGZhy{}\PYGZhy{}\PYGZhy{}}

    \PYG{c}{! each thread sets part of these arrays:}
    \PYG{k}{do }\PYG{n}{i}\PYG{o}{=}\PYG{n}{istart}\PYG{p}{,} \PYG{n}{iend}
        \PYG{c}{! grid points:}
        \PYG{n}{x}\PYG{p}{(}\PYG{n}{i}\PYG{p}{)} \PYG{o}{=} \PYG{n}{i}\PYG{o}{*}\PYG{n}{dx}
        \PYG{c}{! source term:}
        \PYG{n}{f}\PYG{p}{(}\PYG{n}{i}\PYG{p}{)} \PYG{o}{=} \PYG{l+m+mi}{10}\PYG{l+m+mf}{0.}\PYG{o}{*}\PYG{n+nb}{exp}\PYG{p}{(}\PYG{n}{x}\PYG{p}{(}\PYG{n}{i}\PYG{p}{)}\PYG{p}{)}
        \PYG{c}{! initial guess:}
        \PYG{n}{u}\PYG{p}{(}\PYG{n}{i}\PYG{p}{)} \PYG{o}{=} \PYG{n}{alpha} \PYG{o}{+} \PYG{n}{x}\PYG{p}{(}\PYG{n}{i}\PYG{p}{)}\PYG{o}{*}\PYG{p}{(}\PYG{n}{beta}\PYG{o}{\PYGZhy{}}\PYG{n}{alpha}\PYG{p}{)}
        \PYG{n}{enddo}
    
    \PYG{c}{! boundary conditions need to be added:}
    \PYG{n}{u}\PYG{p}{(}\PYG{l+m+mi}{0}\PYG{p}{)} \PYG{o}{=} \PYG{n}{alpha}
    \PYG{n}{u}\PYG{p}{(}\PYG{n}{n}\PYG{o}{+}\PYG{l+m+mi}{1}\PYG{p}{)} \PYG{o}{=} \PYG{n}{beta}

    \PYG{n}{uold} \PYG{o}{=} \PYG{n}{u}   \PYG{c}{! initialize, including boundary values}


    \PYG{c}{! Jacobi iteratation:}
    \PYG{c}{! \PYGZhy{}\PYGZhy{}\PYGZhy{}\PYGZhy{}\PYGZhy{}\PYGZhy{}\PYGZhy{}\PYGZhy{}\PYGZhy{}\PYGZhy{}\PYGZhy{}\PYGZhy{}\PYGZhy{}\PYGZhy{}\PYGZhy{}\PYGZhy{}\PYGZhy{}\PYGZhy{}\PYGZhy{}}


    \PYG{k}{do }\PYG{n}{iter}\PYG{o}{=}\PYG{l+m+mi}{1}\PYG{p}{,}\PYG{n}{maxiter}

        \PYG{c}{! initialize uold to u (note each thread does part!)}
        \PYG{n}{uold}\PYG{p}{(}\PYG{n}{istart}\PYG{p}{:}\PYG{n}{iend}\PYG{p}{)} \PYG{o}{=} \PYG{n}{u}\PYG{p}{(}\PYG{n}{istart}\PYG{p}{:}\PYG{n}{iend}\PYG{p}{)} 

        \PYG{c}{!\PYGZdl{}omp single}
        \PYG{n}{dumax} \PYG{o}{=} \PYG{l+m+mf}{0.}\PYG{n}{d0}     \PYG{c}{! global max initialized by one thread}
        \PYG{c}{!\PYGZdl{}omp end single}

        \PYG{c}{! Make sure all of uold is initialized before iterating}
        \PYG{c}{!\PYGZdl{}omp barrier}
        \PYG{c}{! Make sure uold is consitent in memory:}
        \PYG{c}{!\PYGZdl{}omp flush}

        \PYG{n}{dumax\PYGZus{}thread} \PYG{o}{=} \PYG{l+m+mf}{0.}\PYG{n}{d0}   \PYG{c}{! max seen by this thread}
        \PYG{k}{do }\PYG{n}{i}\PYG{o}{=}\PYG{n}{istart}\PYG{p}{,}\PYG{n}{iend}
            \PYG{n}{u}\PYG{p}{(}\PYG{n}{i}\PYG{p}{)} \PYG{o}{=} \PYG{l+m+mf}{0.5}\PYG{n}{d0}\PYG{o}{*}\PYG{p}{(}\PYG{n}{uold}\PYG{p}{(}\PYG{n}{i}\PYG{o}{\PYGZhy{}}\PYG{l+m+mi}{1}\PYG{p}{)} \PYG{o}{+} \PYG{n}{uold}\PYG{p}{(}\PYG{n}{i}\PYG{o}{+}\PYG{l+m+mi}{1}\PYG{p}{)} \PYG{o}{+} \PYG{n}{dx}\PYG{o}{**}\PYG{l+m+mi}{2}\PYG{o}{*}\PYG{n}{f}\PYG{p}{(}\PYG{n}{i}\PYG{p}{)}\PYG{p}{)}
            \PYG{n}{dumax\PYGZus{}thread} \PYG{o}{=} \PYG{n+nb}{max}\PYG{p}{(}\PYG{n}{dumax\PYGZus{}thread}\PYG{p}{,} \PYG{n+nb}{abs}\PYG{p}{(}\PYG{n}{u}\PYG{p}{(}\PYG{n}{i}\PYG{p}{)}\PYG{o}{\PYGZhy{}}\PYG{n}{uold}\PYG{p}{(}\PYG{n}{i}\PYG{p}{)}\PYG{p}{)}\PYG{p}{)}
            \PYG{n}{enddo}

        \PYG{c}{!\PYGZdl{}omp critical}
        \PYG{c}{! update global dumax using value from this thread:}
        \PYG{n}{dumax} \PYG{o}{=} \PYG{n+nb}{max}\PYG{p}{(}\PYG{n}{dumax}\PYG{p}{,} \PYG{n}{dumax\PYGZus{}thread}\PYG{p}{)}
        \PYG{c}{!\PYGZdl{}omp end critical}

        \PYG{c}{! make sure all threads are done with dumax:}
        \PYG{c}{!\PYGZdl{}omp barrier}

        \PYG{c}{!\PYGZdl{}omp single}
        \PYG{c}{! only one thread will do this print statement:}
        \PYG{k}{if} \PYG{p}{(}\PYG{n+nb}{mod}\PYG{p}{(}\PYG{n}{iter}\PYG{p}{,}\PYG{n}{nprint}\PYG{p}{)}\PYG{o}{==}\PYG{l+m+mi}{0}\PYG{p}{)} \PYG{k}{then}
\PYG{k}{            }\PYG{k}{print} \PYG{l+s+s1}{\PYGZsq{}(\PYGZdq{}After \PYGZdq{},i8,\PYGZdq{} iterations, dumax = \PYGZdq{},d16.6,/)\PYGZsq{}}\PYG{p}{,} \PYG{n}{iter}\PYG{p}{,} \PYG{n}{dumax}
            \PYG{n}{endif}
        \PYG{c}{!\PYGZdl{}omp end single}

        \PYG{c}{! check for convergence:}
        \PYG{c}{! note that all threads have same value of dumax }
        \PYG{c}{! at this point, so they will all exit on the same iteration.}
        
        \PYG{k}{if} \PYG{p}{(}\PYG{n}{dumax} \PYG{p}{.}\PYG{n}{lt}\PYG{p}{.} \PYG{n}{tol}\PYG{p}{)} \PYG{k}{exit}

        \PYG{c}{! need to synchronize here so no thread resets dumax = 0}
        \PYG{c}{! at start of next iteration before all have done the test above.}
        \PYG{c}{!\PYGZdl{}omp barrier}

        \PYG{n}{enddo}

    \PYG{k}{print} \PYG{l+s+s1}{\PYGZsq{}(\PYGZdq{}Thread number \PYGZdq{},i2,\PYGZdq{} finished after \PYGZdq{},i9, \PYGZam{}}
\PYG{l+s+s1}{            \PYGZdq{} iterations, dumax = \PYGZdq{}, e16.6)\PYGZsq{}}\PYG{p}{,} \PYG{p}{\PYGZam{}}
          \PYG{n}{thread\PYGZus{}num}\PYG{p}{,}\PYG{n}{iter}\PYG{p}{,}\PYG{n}{dumax}

    \PYG{c}{!\PYGZdl{}omp end parallel}

    \PYG{k}{call }\PYG{n+nb}{cpu\PYGZus{}time}\PYG{p}{(}\PYG{n}{t2}\PYG{p}{)}
    \PYG{k}{print} \PYG{l+s+s1}{\PYGZsq{}(\PYGZdq{}CPU time = \PYGZdq{},f12.8, \PYGZdq{} seconds\PYGZdq{})\PYGZsq{}}\PYG{p}{,} \PYG{n}{t2}\PYG{o}{\PYGZhy{}}\PYG{n}{t1}


    \PYG{c}{! Write solution to heatsoln.txt:}
    \PYG{k}{write}\PYG{p}{(}\PYG{l+m+mi}{20}\PYG{p}{,}\PYG{o}{*}\PYG{p}{)} \PYG{l+s+s2}{\PYGZdq{}          x                  u\PYGZdq{}}
    \PYG{k}{do }\PYG{n}{i}\PYG{o}{=}\PYG{l+m+mi}{0}\PYG{p}{,}\PYG{n}{n}\PYG{o}{+}\PYG{l+m+mi}{1}
        \PYG{k}{write}\PYG{p}{(}\PYG{l+m+mi}{20}\PYG{p}{,}\PYG{l+s+s1}{\PYGZsq{}(2e20.10)\PYGZsq{}}\PYG{p}{)}\PYG{p}{,} \PYG{n}{x}\PYG{p}{(}\PYG{n}{i}\PYG{p}{)}\PYG{p}{,} \PYG{n}{u}\PYG{p}{(}\PYG{n}{i}\PYG{p}{)}
        \PYG{n}{enddo}

    \PYG{k}{print} \PYG{o}{*}\PYG{p}{,} \PYG{l+s+s2}{\PYGZdq{}Solution is in heatsoln.txt\PYGZdq{}}

    \PYG{k}{close}\PYG{p}{(}\PYG{l+m+mi}{20}\PYG{p}{)}

\PYG{k}{end }\PYG{k}{program }\PYG{n}{jacobi1d\PYGZus{}omp2}
\end{Verbatim}


\section{Jacobi iteration using MPI}
\label{jacobi1d_mpi::doc}\label{jacobi1d_mpi:jacobi-iteration-using-mpi}\label{jacobi1d_mpi:jacobi1d-mpi}
The code below implements Jacobi iteration for solving the linear system
arising from the steady state heat equation
using MPI.  Note that in this code each process, or task, has only a portion
of the arrays and must exchange boundary data using message passing.

Compare to:
\begin{itemize}
\item {} 
{\hyperref[jacobi1d_omp1:jacobi1d\string-omp1]{\crossref{\DUrole{std,std-ref}{Jacobi iteration using OpenMP with parallel do constructs}}}}

\item {} 
{\hyperref[jacobi1d_omp2:jacobi1d\string-omp2]{\crossref{\DUrole{std,std-ref}{Jacobi iteration using OpenMP with coarse-grain parallel block}}}}

\end{itemize}

The code:

\begin{Verbatim}[commandchars=\\\{\},numbers=left,firstnumber=1,stepnumber=1]
\PYG{c}{! \PYGZdl{}UWHSPC/codes/mpi/jacobi1d\PYGZus{}mpi.f90}
\PYG{c}{!}
\PYG{c}{! Domain decomposition version of Jacobi iteration illustrating}
\PYG{c}{! coarse grain parallelism with MPI.}
\PYG{c}{!}
\PYG{c}{! The one\PYGZhy{}dimensional Poisson problem is solved, u\PYGZsq{}\PYGZsq{}(x) = \PYGZhy{}f(x)}
\PYG{c}{! with u(0) = alpha and u(1) = beta.}
\PYG{c}{!}
\PYG{c}{! The grid points are split up into ntasks disjoint sets and each task}
\PYG{c}{! is assigned one set that it updates for all iterations.  The tasks }
\PYG{c}{! correspond to processes.}
\PYG{c}{!}
\PYG{c}{! The task (or process) number is called \PYGZdq{}me\PYGZdq{} in this code for brevity}
\PYG{c}{! rather than proc\PYGZus{}num.}
\PYG{c}{!}
\PYG{c}{! Note that each task allocates only as much storage as needed for its }
\PYG{c}{! portion of the arrays.}
\PYG{c}{!}
\PYG{c}{! Each iteration, boundary values at the edge of each grid must be}
\PYG{c}{! exchanged with the neighbors.}


\PYG{k}{program }\PYG{n}{jacobi1d\PYGZus{}mpi}
    \PYG{k}{use }\PYG{n}{mpi}

    \PYG{k}{implicit }\PYG{k}{none}

\PYG{k}{    }\PYG{k+kt}{integer}\PYG{p}{,} \PYG{k}{parameter} \PYG{k+kd}{::} \PYG{n}{maxiter} \PYG{o}{=} \PYG{l+m+mi}{100000}\PYG{p}{,} \PYG{n}{nprint} \PYG{o}{=} \PYG{l+m+mi}{5000}
    \PYG{k+kt}{real} \PYG{p}{(}\PYG{n+nb}{kind}\PYG{o}{=}\PYG{l+m+mi}{8}\PYG{p}{)}\PYG{p}{,} \PYG{k}{parameter} \PYG{k+kd}{::} \PYG{n}{alpha} \PYG{o}{=} \PYG{l+m+mi}{2}\PYG{l+m+mf}{0.}\PYG{n}{d0}\PYG{p}{,} \PYG{n}{beta} \PYG{o}{=} \PYG{l+m+mi}{6}\PYG{l+m+mf}{0.}\PYG{n}{d0}

    \PYG{k+kt}{integer} \PYG{k+kd}{::} \PYG{n}{i}\PYG{p}{,} \PYG{n}{iter}\PYG{p}{,} \PYG{n}{istart}\PYG{p}{,} \PYG{n}{iend}\PYG{p}{,} \PYG{n}{points\PYGZus{}per\PYGZus{}task}\PYG{p}{,} \PYG{n}{itask}\PYG{p}{,} \PYG{n}{n}
    \PYG{k+kt}{integer} \PYG{k+kd}{::} \PYG{n}{ierr}\PYG{p}{,} \PYG{n}{ntasks}\PYG{p}{,} \PYG{n}{me}\PYG{p}{,} \PYG{n}{req1}\PYG{p}{,} \PYG{n}{req2}
    \PYG{k+kt}{integer}\PYG{p}{,} \PYG{k}{dimension}\PYG{p}{(}\PYG{n}{MPI\PYGZus{}STATUS\PYGZus{}SIZE}\PYG{p}{)} \PYG{k+kd}{::} \PYG{n}{mpistatus}
    \PYG{k+kt}{real} \PYG{p}{(}\PYG{n+nb}{kind} \PYG{o}{=} \PYG{l+m+mi}{8}\PYG{p}{)}\PYG{p}{,} \PYG{k}{dimension}\PYG{p}{(}\PYG{p}{:}\PYG{p}{)}\PYG{p}{,} \PYG{k}{allocatable} \PYG{k+kd}{::} \PYG{n}{f}\PYG{p}{,} \PYG{n}{u}\PYG{p}{,} \PYG{n}{uold}
    \PYG{k+kt}{real} \PYG{p}{(}\PYG{n+nb}{kind} \PYG{o}{=} \PYG{l+m+mi}{8}\PYG{p}{)} \PYG{k+kd}{::} \PYG{n}{x}\PYG{p}{,} \PYG{n}{dumax\PYGZus{}task}\PYG{p}{,} \PYG{n}{dumax\PYGZus{}global}\PYG{p}{,} \PYG{n}{dx}\PYG{p}{,} \PYG{n}{tol}

    \PYG{c}{! Initialize MPI; get total number of tasks and ID of this task}
    \PYG{k}{call }\PYG{n}{mpi\PYGZus{}init}\PYG{p}{(}\PYG{n}{ierr}\PYG{p}{)}
    \PYG{k}{call }\PYG{n}{mpi\PYGZus{}comm\PYGZus{}size}\PYG{p}{(}\PYG{n}{MPI\PYGZus{}COMM\PYGZus{}WORLD}\PYG{p}{,} \PYG{n}{ntasks}\PYG{p}{,} \PYG{n}{ierr}\PYG{p}{)}
    \PYG{k}{call }\PYG{n}{mpi\PYGZus{}comm\PYGZus{}rank}\PYG{p}{(}\PYG{n}{MPI\PYGZus{}COMM\PYGZus{}WORLD}\PYG{p}{,} \PYG{n}{me}\PYG{p}{,} \PYG{n}{ierr}\PYG{p}{)}

    \PYG{c}{! Ask the user for the number of points}
    \PYG{k}{if} \PYG{p}{(}\PYG{n}{me} \PYG{o}{==} \PYG{l+m+mi}{0}\PYG{p}{)} \PYG{k}{then}
\PYG{k}{        }\PYG{k}{print} \PYG{o}{*}\PYG{p}{,} \PYG{l+s+s2}{\PYGZdq{}Input n ... \PYGZdq{}}
        \PYG{k}{read} \PYG{o}{*}\PYG{p}{,} \PYG{n}{n}
    \PYG{k}{end }\PYG{k}{if}
    \PYG{c}{! Broadcast to all tasks; everybody gets the value of n from task 0}
    \PYG{k}{call }\PYG{n}{mpi\PYGZus{}bcast}\PYG{p}{(}\PYG{n}{n}\PYG{p}{,} \PYG{l+m+mi}{1}\PYG{p}{,} \PYG{n}{MPI\PYGZus{}INTEGER}\PYG{p}{,} \PYG{l+m+mi}{0}\PYG{p}{,} \PYG{n}{MPI\PYGZus{}COMM\PYGZus{}WORLD}\PYG{p}{,} \PYG{n}{ierr}\PYG{p}{)}

    \PYG{n}{dx} \PYG{o}{=} \PYG{l+m+mf}{1.}\PYG{n}{d0}\PYG{o}{/}\PYG{k+kt}{real}\PYG{p}{(}\PYG{n}{n}\PYG{o}{+}\PYG{l+m+mi}{1}\PYG{p}{,} \PYG{n+nb}{kind}\PYG{o}{=}\PYG{l+m+mi}{8}\PYG{p}{)}
    \PYG{n}{tol} \PYG{o}{=} \PYG{l+m+mf}{0.1}\PYG{n}{d0}\PYG{o}{*}\PYG{n}{dx}\PYG{o}{**}\PYG{l+m+mi}{2}

    \PYG{c}{! Determine how many points to handle with each task}
    \PYG{n}{points\PYGZus{}per\PYGZus{}task} \PYG{o}{=} \PYG{p}{(}\PYG{n}{n} \PYG{o}{+} \PYG{n}{ntasks} \PYG{o}{\PYGZhy{}} \PYG{l+m+mi}{1}\PYG{p}{)}\PYG{o}{/}\PYG{n}{ntasks}
    \PYG{k}{if} \PYG{p}{(}\PYG{n}{me} \PYG{o}{==} \PYG{l+m+mi}{0}\PYG{p}{)} \PYG{k}{then}   \PYG{c}{! Only one task should print to avoid clutter}
        \PYG{k}{print} \PYG{o}{*}\PYG{p}{,} \PYG{l+s+s2}{\PYGZdq{}points\PYGZus{}per\PYGZus{}task = \PYGZdq{}}\PYG{p}{,} \PYG{n}{points\PYGZus{}per\PYGZus{}task}
    \PYG{k}{end }\PYG{k}{if}

    \PYG{c}{! Determine start and end index for this task\PYGZsq{}s points}
    \PYG{n}{istart} \PYG{o}{=} \PYG{n}{me} \PYG{o}{*} \PYG{n}{points\PYGZus{}per\PYGZus{}task} \PYG{o}{+} \PYG{l+m+mi}{1}
    \PYG{n}{iend} \PYG{o}{=} \PYG{n+nb}{min}\PYG{p}{(}\PYG{p}{(}\PYG{n}{me} \PYG{o}{+} \PYG{l+m+mi}{1}\PYG{p}{)}\PYG{o}{*}\PYG{n}{points\PYGZus{}per\PYGZus{}task}\PYG{p}{,} \PYG{n}{n}\PYG{p}{)}

    \PYG{c}{! Diagnostic: tell the user which points will be handled by which task}
    \PYG{k}{print} \PYG{l+s+s1}{\PYGZsq{}(\PYGZdq{}Task \PYGZdq{},i2,\PYGZdq{} will take i = \PYGZdq{},i6,\PYGZdq{} through i = \PYGZdq{},i6)\PYGZsq{}}\PYG{p}{,} \PYG{p}{\PYGZam{}}
        \PYG{n}{me}\PYG{p}{,} \PYG{n}{istart}\PYG{p}{,} \PYG{n}{iend}


    \PYG{c}{! Initialize:}
    \PYG{c}{! \PYGZhy{}\PYGZhy{}\PYGZhy{}\PYGZhy{}\PYGZhy{}\PYGZhy{}\PYGZhy{}\PYGZhy{}\PYGZhy{}\PYGZhy{}\PYGZhy{}}

    \PYG{c}{! This makes the indices run from istart\PYGZhy{}1 to iend+1}
    \PYG{c}{! This is more or less cosmetic, but makes things easier to think about}
    \PYG{k}{allocate}\PYG{p}{(}\PYG{n}{f}\PYG{p}{(}\PYG{n}{istart}\PYG{o}{\PYGZhy{}}\PYG{l+m+mi}{1}\PYG{p}{:}\PYG{n}{iend}\PYG{o}{+}\PYG{l+m+mi}{1}\PYG{p}{)}\PYG{p}{,} \PYG{n}{u}\PYG{p}{(}\PYG{n}{istart}\PYG{o}{\PYGZhy{}}\PYG{l+m+mi}{1}\PYG{p}{:}\PYG{n}{iend}\PYG{o}{+}\PYG{l+m+mi}{1}\PYG{p}{)}\PYG{p}{,} \PYG{n}{uold}\PYG{p}{(}\PYG{n}{istart}\PYG{o}{\PYGZhy{}}\PYG{l+m+mi}{1}\PYG{p}{:}\PYG{n}{iend}\PYG{o}{+}\PYG{l+m+mi}{1}\PYG{p}{)}\PYG{p}{)}

    \PYG{c}{! Each task sets its own, independent array}
    \PYG{k}{do }\PYG{n}{i} \PYG{o}{=} \PYG{n}{istart}\PYG{p}{,} \PYG{n}{iend}
        \PYG{c}{! Each task is a single thread with all its variables private}
        \PYG{c}{! so re\PYGZhy{}using the scalar variable x from one loop iteration to}
        \PYG{c}{! the next does not produce a race condition.}
        \PYG{n}{x} \PYG{o}{=} \PYG{n}{dx}\PYG{o}{*}\PYG{k+kt}{real}\PYG{p}{(}\PYG{n}{i}\PYG{p}{,} \PYG{n+nb}{kind}\PYG{o}{=}\PYG{l+m+mi}{8}\PYG{p}{)}
        \PYG{n}{f}\PYG{p}{(}\PYG{n}{i}\PYG{p}{)} \PYG{o}{=} \PYG{l+m+mi}{10}\PYG{l+m+mf}{0.}\PYG{n}{d0}\PYG{o}{*}\PYG{n+nb}{exp}\PYG{p}{(}\PYG{n}{x}\PYG{p}{)}               \PYG{c}{! Source term}
        \PYG{n}{u}\PYG{p}{(}\PYG{n}{i}\PYG{p}{)} \PYG{o}{=} \PYG{n}{alpha} \PYG{o}{+} \PYG{n}{x}\PYG{o}{*}\PYG{p}{(}\PYG{n}{beta} \PYG{o}{\PYGZhy{}} \PYG{n}{alpha}\PYG{p}{)}    \PYG{c}{! Initial guess}
    \PYG{k}{end }\PYG{k}{do}
    
    \PYG{c}{! Set boundary conditions if this task is keeping track of a boundary}
    \PYG{c}{! point}
    \PYG{k}{if} \PYG{p}{(}\PYG{n}{me} \PYG{o}{==} \PYG{l+m+mi}{0}\PYG{p}{)}        \PYG{n}{u}\PYG{p}{(}\PYG{n}{istart}\PYG{o}{\PYGZhy{}}\PYG{l+m+mi}{1}\PYG{p}{)} \PYG{o}{=} \PYG{n}{alpha}
    \PYG{k}{if} \PYG{p}{(}\PYG{n}{me} \PYG{o}{==} \PYG{n}{ntasks}\PYG{o}{\PYGZhy{}}\PYG{l+m+mi}{1}\PYG{p}{)} \PYG{n}{u}\PYG{p}{(}\PYG{n}{iend}\PYG{o}{+}\PYG{l+m+mi}{1}\PYG{p}{)}   \PYG{o}{=} \PYG{n}{beta}


    \PYG{c}{! Jacobi iteratation:}
    \PYG{c}{! \PYGZhy{}\PYGZhy{}\PYGZhy{}\PYGZhy{}\PYGZhy{}\PYGZhy{}\PYGZhy{}\PYGZhy{}\PYGZhy{}\PYGZhy{}\PYGZhy{}\PYGZhy{}\PYGZhy{}\PYGZhy{}\PYGZhy{}\PYGZhy{}\PYGZhy{}\PYGZhy{}\PYGZhy{}}

    \PYG{k}{do }\PYG{n}{iter} \PYG{o}{=} \PYG{l+m+mi}{1}\PYG{p}{,} \PYG{n}{maxiter}
        \PYG{n}{uold} \PYG{o}{=} \PYG{n}{u}

        \PYG{c}{! Send endpoint values to tasks handling neighboring sections}
        \PYG{c}{! of the array.  Note that non\PYGZhy{}blocking sends are used; note}
        \PYG{c}{! also that this sends from uold, so the buffer we\PYGZsq{}re sending}
        \PYG{c}{! from won\PYGZsq{}t be modified while it\PYGZsq{}s being sent.}
        \PYG{c}{!}
        \PYG{c}{! tag=1 is used for messages sent to the left}
        \PYG{c}{! tag=2 is used for messages sent to the right}

        \PYG{k}{if} \PYG{p}{(}\PYG{n}{me} \PYG{o}{\PYGZgt{}} \PYG{l+m+mi}{0}\PYG{p}{)} \PYG{k}{then}
            \PYG{c}{! Send left endpoint value to process to the \PYGZdq{}left\PYGZdq{}}
            \PYG{k}{call }\PYG{n}{mpi\PYGZus{}isend}\PYG{p}{(}\PYG{n}{uold}\PYG{p}{(}\PYG{n}{istart}\PYG{p}{)}\PYG{p}{,} \PYG{l+m+mi}{1}\PYG{p}{,} \PYG{n}{MPI\PYGZus{}DOUBLE\PYGZus{}PRECISION}\PYG{p}{,} \PYG{n}{me} \PYG{o}{\PYGZhy{}} \PYG{l+m+mi}{1}\PYG{p}{,} \PYG{p}{\PYGZam{}}
                \PYG{l+m+mi}{1}\PYG{p}{,} \PYG{n}{MPI\PYGZus{}COMM\PYGZus{}WORLD}\PYG{p}{,} \PYG{n}{req1}\PYG{p}{,} \PYG{n}{ierr}\PYG{p}{)}
        \PYG{k}{end }\PYG{k}{if}
\PYG{k}{        }\PYG{k}{if} \PYG{p}{(}\PYG{n}{me} \PYG{o}{\PYGZlt{}} \PYG{n}{ntasks}\PYG{o}{\PYGZhy{}}\PYG{l+m+mi}{1}\PYG{p}{)} \PYG{k}{then}
            \PYG{c}{! Send right endpoint value to process on the \PYGZdq{}right\PYGZdq{}}
            \PYG{k}{call }\PYG{n}{mpi\PYGZus{}isend}\PYG{p}{(}\PYG{n}{uold}\PYG{p}{(}\PYG{n}{iend}\PYG{p}{)}\PYG{p}{,} \PYG{l+m+mi}{1}\PYG{p}{,} \PYG{n}{MPI\PYGZus{}DOUBLE\PYGZus{}PRECISION}\PYG{p}{,} \PYG{n}{me} \PYG{o}{+} \PYG{l+m+mi}{1}\PYG{p}{,} \PYG{p}{\PYGZam{}}
                \PYG{l+m+mi}{2}\PYG{p}{,} \PYG{n}{MPI\PYGZus{}COMM\PYGZus{}WORLD}\PYG{p}{,} \PYG{n}{req2}\PYG{p}{,} \PYG{n}{ierr}\PYG{p}{)}
        \PYG{k}{end }\PYG{k}{if}

        \PYG{c}{! Accept incoming endpoint values from other tasks.  Note that}
        \PYG{c}{! these are blocking receives, because we can\PYGZsq{}t run the next step}
        \PYG{c}{! of the Jacobi iteration until we\PYGZsq{}ve received all the}
        \PYG{c}{! incoming data.}

        \PYG{k}{if} \PYG{p}{(}\PYG{n}{me} \PYG{o}{\PYGZlt{}} \PYG{n}{ntasks}\PYG{o}{\PYGZhy{}}\PYG{l+m+mi}{1}\PYG{p}{)} \PYG{k}{then}
            \PYG{c}{! Receive right endpoint value}
            \PYG{k}{call }\PYG{n}{mpi\PYGZus{}recv}\PYG{p}{(}\PYG{n}{uold}\PYG{p}{(}\PYG{n}{iend}\PYG{o}{+}\PYG{l+m+mi}{1}\PYG{p}{)}\PYG{p}{,} \PYG{l+m+mi}{1}\PYG{p}{,} \PYG{n}{MPI\PYGZus{}DOUBLE\PYGZus{}PRECISION}\PYG{p}{,} \PYG{n}{me} \PYG{o}{+} \PYG{l+m+mi}{1}\PYG{p}{,} \PYG{p}{\PYGZam{}}
                \PYG{l+m+mi}{1}\PYG{p}{,} \PYG{n}{MPI\PYGZus{}COMM\PYGZus{}WORLD}\PYG{p}{,} \PYG{n}{mpistatus}\PYG{p}{,} \PYG{n}{ierr}\PYG{p}{)}
        \PYG{k}{end }\PYG{k}{if}
\PYG{k}{        }\PYG{k}{if} \PYG{p}{(}\PYG{n}{me} \PYG{o}{\PYGZgt{}} \PYG{l+m+mi}{0}\PYG{p}{)} \PYG{k}{then}
            \PYG{c}{! Receive left endpoint value}
            \PYG{k}{call }\PYG{n}{mpi\PYGZus{}recv}\PYG{p}{(}\PYG{n}{uold}\PYG{p}{(}\PYG{n}{istart}\PYG{o}{\PYGZhy{}}\PYG{l+m+mi}{1}\PYG{p}{)}\PYG{p}{,} \PYG{l+m+mi}{1}\PYG{p}{,} \PYG{n}{MPI\PYGZus{}DOUBLE\PYGZus{}PRECISION}\PYG{p}{,} \PYG{n}{me} \PYG{o}{\PYGZhy{}} \PYG{l+m+mi}{1}\PYG{p}{,} \PYG{p}{\PYGZam{}}
                \PYG{l+m+mi}{2}\PYG{p}{,} \PYG{n}{MPI\PYGZus{}COMM\PYGZus{}WORLD}\PYG{p}{,} \PYG{n}{mpistatus}\PYG{p}{,} \PYG{n}{ierr}\PYG{p}{)}
        \PYG{k}{end }\PYG{k}{if}

\PYG{k}{        }\PYG{n}{dumax\PYGZus{}task} \PYG{o}{=} \PYG{l+m+mf}{0.}\PYG{n}{d0}   \PYG{c}{! Max seen by this task}

        \PYG{c}{! Apply Jacobi iteration on this task\PYGZsq{}s section of the array}
        \PYG{k}{do }\PYG{n}{i} \PYG{o}{=} \PYG{n}{istart}\PYG{p}{,} \PYG{n}{iend}
            \PYG{n}{u}\PYG{p}{(}\PYG{n}{i}\PYG{p}{)} \PYG{o}{=} \PYG{l+m+mf}{0.5}\PYG{n}{d0}\PYG{o}{*}\PYG{p}{(}\PYG{n}{uold}\PYG{p}{(}\PYG{n}{i}\PYG{o}{\PYGZhy{}}\PYG{l+m+mi}{1}\PYG{p}{)} \PYG{o}{+} \PYG{n}{uold}\PYG{p}{(}\PYG{n}{i}\PYG{o}{+}\PYG{l+m+mi}{1}\PYG{p}{)} \PYG{o}{+} \PYG{n}{dx}\PYG{o}{**}\PYG{l+m+mi}{2}\PYG{o}{*}\PYG{n}{f}\PYG{p}{(}\PYG{n}{i}\PYG{p}{)}\PYG{p}{)}
            \PYG{n}{dumax\PYGZus{}task} \PYG{o}{=} \PYG{n+nb}{max}\PYG{p}{(}\PYG{n}{dumax\PYGZus{}task}\PYG{p}{,} \PYG{n+nb}{abs}\PYG{p}{(}\PYG{n}{u}\PYG{p}{(}\PYG{n}{i}\PYG{p}{)} \PYG{o}{\PYGZhy{}} \PYG{n}{uold}\PYG{p}{(}\PYG{n}{i}\PYG{p}{)}\PYG{p}{)}\PYG{p}{)}
        \PYG{k}{end }\PYG{k}{do}

        \PYG{c}{! Take global maximum of dumax values}
        \PYG{k}{call }\PYG{n}{mpi\PYGZus{}allreduce}\PYG{p}{(}\PYG{n}{dumax\PYGZus{}task}\PYG{p}{,} \PYG{n}{dumax\PYGZus{}global}\PYG{p}{,} \PYG{l+m+mi}{1}\PYG{p}{,} \PYG{n}{MPI\PYGZus{}DOUBLE\PYGZus{}PRECISION}\PYG{p}{,} \PYG{p}{\PYGZam{}}
            \PYG{n}{MPI\PYGZus{}MAX}\PYG{p}{,} \PYG{n}{MPI\PYGZus{}COMM\PYGZus{}WORLD}\PYG{p}{,} \PYG{n}{ierr}\PYG{p}{)}
        \PYG{c}{! Note that this MPI\PYGZus{}ALLREDUCE call acts as an implicit barrier,}
        \PYG{c}{! since no process can return from it until all processes}
        \PYG{c}{! have called it.  Because of this, after this call we know}
        \PYG{c}{! that all the send and receive operations initiated at the}
        \PYG{c}{! top of the loop have finished \PYGZhy{}\PYGZhy{} all the MPI\PYGZus{}RECV calls have}
        \PYG{c}{! finished in order for each process to get here, and if the}
        \PYG{c}{! MPI\PYGZus{}RECV calls have finished, the corresponding MPI\PYGZus{}ISEND}
        \PYG{c}{! calls have also finished.  Thus we can safely modify uold}
        \PYG{c}{! again.}

        \PYG{c}{! Also periodically report progress to the user}
        \PYG{k}{if} \PYG{p}{(}\PYG{n}{me} \PYG{o}{==} \PYG{l+m+mi}{0}\PYG{p}{)} \PYG{k}{then}
\PYG{k}{            }\PYG{k}{if} \PYG{p}{(}\PYG{n+nb}{mod}\PYG{p}{(}\PYG{n}{iter}\PYG{p}{,} \PYG{n}{nprint}\PYG{p}{)}\PYG{o}{==}\PYG{l+m+mi}{0}\PYG{p}{)} \PYG{k}{then}
\PYG{k}{                }\PYG{k}{print} \PYG{l+s+s1}{\PYGZsq{}(\PYGZdq{}After \PYGZdq{},i8,\PYGZdq{} iterations, dumax = \PYGZdq{},d16.6,/)\PYGZsq{}}\PYG{p}{,} \PYG{p}{\PYGZam{}}
                    \PYG{n}{iter}\PYG{p}{,} \PYG{n}{dumax\PYGZus{}global}
            \PYG{k}{end }\PYG{k}{if}
\PYG{k}{        }\PYG{k}{end }\PYG{k}{if}

        \PYG{c}{! All tasks now have dumax\PYGZus{}global, and can check for convergence}
        \PYG{k}{if} \PYG{p}{(}\PYG{n}{dumax\PYGZus{}global} \PYG{o}{\PYGZlt{}} \PYG{n}{tol}\PYG{p}{)} \PYG{k}{exit}
\PYG{k}{    }\PYG{k}{end }\PYG{k}{do}

\PYG{k}{    }\PYG{k}{print} \PYG{l+s+s1}{\PYGZsq{}(\PYGZdq{}Task number \PYGZdq{},i2,\PYGZdq{} finished after \PYGZdq{},i9,\PYGZdq{} iterations, dumax = \PYGZdq{},\PYGZam{}}
\PYG{l+s+s1}{            e16.6)\PYGZsq{}}\PYG{p}{,} \PYG{n}{me}\PYG{p}{,} \PYG{n}{iter}\PYG{p}{,} \PYG{n}{dumax\PYGZus{}global}


    \PYG{c}{! Output result:}
    \PYG{c}{! \PYGZhy{}\PYGZhy{}\PYGZhy{}\PYGZhy{}\PYGZhy{}\PYGZhy{}\PYGZhy{}\PYGZhy{}\PYGZhy{}\PYGZhy{}\PYGZhy{}\PYGZhy{}\PYGZhy{}\PYGZhy{}}

    \PYG{c}{! Note: this only works if all processes share a file system}
    \PYG{c}{! and can all open and write to the same file!}

    \PYG{c}{! Synchronize to keep the next part from being non\PYGZhy{}deterministic}
    \PYG{k}{call }\PYG{n}{mpi\PYGZus{}barrier}\PYG{p}{(}\PYG{n}{MPI\PYGZus{}COMM\PYGZus{}WORLD}\PYG{p}{,} \PYG{n}{ierr}\PYG{p}{)}

    \PYG{c}{! Have each task output to a file in sequence, using messages to}
    \PYG{c}{! coordinate}

    \PYG{k}{if} \PYG{p}{(}\PYG{n}{me} \PYG{o}{==} \PYG{l+m+mi}{0}\PYG{p}{)} \PYG{k}{then}    \PYG{c}{! Task 0 goes first}
        \PYG{c}{! Open file for writing, replacing any previous version:}
        \PYG{k}{open}\PYG{p}{(}\PYG{n}{unit}\PYG{o}{=}\PYG{l+m+mi}{20}\PYG{p}{,} \PYG{k}{file}\PYG{o}{=}\PYG{l+s+s2}{\PYGZdq{}heatsoln.txt\PYGZdq{}}\PYG{p}{,} \PYG{n}{status}\PYG{o}{=}\PYG{l+s+s2}{\PYGZdq{}replace\PYGZdq{}}\PYG{p}{)}
        \PYG{k}{write}\PYG{p}{(}\PYG{l+m+mi}{20}\PYG{p}{,}\PYG{o}{*}\PYG{p}{)} \PYG{l+s+s2}{\PYGZdq{}          x                  u\PYGZdq{}}
        \PYG{k}{write}\PYG{p}{(}\PYG{l+m+mi}{20}\PYG{p}{,} \PYG{l+s+s1}{\PYGZsq{}(2e20.10)\PYGZsq{}}\PYG{p}{)} \PYG{l+m+mf}{0.}\PYG{n}{d0}\PYG{p}{,} \PYG{n}{u}\PYG{p}{(}\PYG{l+m+mi}{0}\PYG{p}{)}    \PYG{c}{! Boundary value at left end}

        \PYG{k}{do }\PYG{n}{i} \PYG{o}{=} \PYG{n}{istart}\PYG{p}{,} \PYG{n}{iend}
            \PYG{k}{write}\PYG{p}{(}\PYG{l+m+mi}{20}\PYG{p}{,} \PYG{l+s+s1}{\PYGZsq{}(2e20.10)\PYGZsq{}}\PYG{p}{)} \PYG{n}{i}\PYG{o}{*}\PYG{n}{dx}\PYG{p}{,} \PYG{n}{u}\PYG{p}{(}\PYG{n}{i}\PYG{p}{)}
        \PYG{k}{end }\PYG{k}{do}

\PYG{k}{        }\PYG{k}{close}\PYG{p}{(}\PYG{n}{unit}\PYG{o}{=}\PYG{l+m+mi}{20}\PYG{p}{)}
        \PYG{c}{! Closing the file should guarantee that all the ouput }
        \PYG{c}{! will be written to disk.}
        \PYG{c}{! If the file isn\PYGZsq{}t closed before the next process starts writing,}
        \PYG{c}{! output may be jumbled or missing.}

        \PYG{c}{! Send go\PYGZhy{}ahead message to next task}
        \PYG{c}{! Only the fact that the message was sent is important, not its contents}
        \PYG{c}{! so we send the special address MPI\PYGZus{}BOTTOM and length 0.}
        \PYG{c}{! tag=4 is used for this message.}

        \PYG{k}{if} \PYG{p}{(}\PYG{n}{ntasks} \PYG{o}{\PYGZgt{}} \PYG{l+m+mi}{1}\PYG{p}{)} \PYG{k}{then}
\PYG{k}{            }\PYG{k}{call }\PYG{n}{mpi\PYGZus{}send}\PYG{p}{(}\PYG{n}{MPI\PYGZus{}BOTTOM}\PYG{p}{,} \PYG{l+m+mi}{0}\PYG{p}{,} \PYG{n}{MPI\PYGZus{}INTEGER}\PYG{p}{,} \PYG{l+m+mi}{1}\PYG{p}{,} \PYG{l+m+mi}{4}\PYG{p}{,} \PYG{p}{\PYGZam{}}
                          \PYG{n}{MPI\PYGZus{}COMM\PYGZus{}WORLD}\PYG{p}{,} \PYG{n}{ierr}\PYG{p}{)}
            \PYG{n}{endif}

    \PYG{k}{else}
        \PYG{c}{! Wait for go\PYGZhy{}ahead message from previous task}
        \PYG{k}{call }\PYG{n}{mpi\PYGZus{}recv}\PYG{p}{(}\PYG{n}{MPI\PYGZus{}BOTTOM}\PYG{p}{,} \PYG{l+m+mi}{0}\PYG{p}{,} \PYG{n}{MPI\PYGZus{}INTEGER}\PYG{p}{,} \PYG{n}{me} \PYG{o}{\PYGZhy{}} \PYG{l+m+mi}{1}\PYG{p}{,} \PYG{l+m+mi}{4}\PYG{p}{,} \PYG{p}{\PYGZam{}}
                          \PYG{n}{MPI\PYGZus{}COMM\PYGZus{}WORLD}\PYG{p}{,} \PYG{n}{mpistatus}\PYG{p}{,} \PYG{n}{ierr}\PYG{p}{)}
        \PYG{c}{! Open file for appending; do not destroy previous contents}
        \PYG{k}{open}\PYG{p}{(}\PYG{n}{unit}\PYG{o}{=}\PYG{l+m+mi}{20}\PYG{p}{,} \PYG{k}{file}\PYG{o}{=}\PYG{l+s+s2}{\PYGZdq{}heatsoln.txt\PYGZdq{}}\PYG{p}{,} \PYG{n}{status}\PYG{o}{=}\PYG{l+s+s2}{\PYGZdq{}old\PYGZdq{}}\PYG{p}{,} \PYG{n+nb}{access}\PYG{o}{=}\PYG{l+s+s2}{\PYGZdq{}append\PYGZdq{}}\PYG{p}{)}
        \PYG{k}{do }\PYG{n}{i} \PYG{o}{=} \PYG{n}{istart}\PYG{p}{,} \PYG{n}{iend}
            \PYG{k}{write}\PYG{p}{(}\PYG{l+m+mi}{20}\PYG{p}{,} \PYG{l+s+s1}{\PYGZsq{}(2e20.10)\PYGZsq{}}\PYG{p}{)} \PYG{n}{i}\PYG{o}{*}\PYG{n}{dx}\PYG{p}{,} \PYG{n}{u}\PYG{p}{(}\PYG{n}{i}\PYG{p}{)}
        \PYG{k}{end }\PYG{k}{do}

        \PYG{c}{! Boundary value at right end:}
        \PYG{k}{if} \PYG{p}{(}\PYG{n}{me} \PYG{o}{==} \PYG{n}{ntasks} \PYG{o}{\PYGZhy{}} \PYG{l+m+mi}{1}\PYG{p}{)} \PYG{k}{write}\PYG{p}{(}\PYG{l+m+mi}{20}\PYG{p}{,} \PYG{l+s+s1}{\PYGZsq{}(2e20.10)\PYGZsq{}}\PYG{p}{)} \PYG{l+m+mf}{1.}\PYG{n}{d0}\PYG{p}{,} \PYG{n}{u}\PYG{p}{(}\PYG{n}{iend}\PYG{o}{+}\PYG{l+m+mi}{1}\PYG{p}{)}    

        \PYG{c}{! Flush all pending writes to disk}
        \PYG{k}{close}\PYG{p}{(}\PYG{n}{unit}\PYG{o}{=}\PYG{l+m+mi}{20}\PYG{p}{)}

        \PYG{k}{if} \PYG{p}{(}\PYG{n}{me} \PYG{o}{\PYGZlt{}} \PYG{n}{ntasks} \PYG{o}{\PYGZhy{}} \PYG{l+m+mi}{1}\PYG{p}{)} \PYG{k}{then}
            \PYG{c}{! Send go\PYGZhy{}ahead message to next task}
            \PYG{k}{call }\PYG{n}{mpi\PYGZus{}send}\PYG{p}{(}\PYG{n}{MPI\PYGZus{}BOTTOM}\PYG{p}{,} \PYG{l+m+mi}{0}\PYG{p}{,} \PYG{n}{MPI\PYGZus{}INTEGER}\PYG{p}{,} \PYG{n}{me} \PYG{o}{+} \PYG{l+m+mi}{1}\PYG{p}{,} \PYG{l+m+mi}{4}\PYG{p}{,} \PYG{p}{\PYGZam{}}
                          \PYG{n}{MPI\PYGZus{}COMM\PYGZus{}WORLD}\PYG{p}{,} \PYG{n}{ierr}\PYG{p}{)}
        \PYG{k}{end }\PYG{k}{if}
\PYG{k}{    }\PYG{k}{end }\PYG{k}{if}

    \PYG{c}{! Notify the user when all tasks have finished writing}
    \PYG{k}{if} \PYG{p}{(}\PYG{n}{me} \PYG{o}{==} \PYG{n}{ntasks} \PYG{o}{\PYGZhy{}} \PYG{l+m+mi}{1}\PYG{p}{)} \PYG{k}{print} \PYG{o}{*}\PYG{p}{,} \PYG{l+s+s2}{\PYGZdq{}Solution is in heatsoln.txt\PYGZdq{}}

    \PYG{c}{! Close out MPI}
    \PYG{k}{call }\PYG{n}{mpi\PYGZus{}finalize}\PYG{p}{(}\PYG{n}{ierr}\PYG{p}{)}

\PYG{k}{end }\PYG{k}{program }\PYG{n}{jacobi1d\PYGZus{}mpi}
\end{Verbatim}


\chapter{References}
\label{index:toc-biblio}\label{index:references}

\section{Bibliography and further reading}
\label{biblio:bibliography-and-further-reading}\label{biblio::doc}\label{biblio:biblio}
Many other pages in these notes have links not listed below.  These are some
references that are partiuclarly useful or are cited often elsewhere.


\subsection{Books}
\label{biblio:biblio-books}\label{biblio:books}

\subsection{Other courses with useful slides or webpages}
\label{biblio:other-courses}\label{biblio:other-courses-with-useful-slides-or-webpages}

\subsection{Other Links}
\label{biblio:other-links}\label{biblio:biblio-links}

\subsection{Software}
\label{biblio:software}

\strong{See also:}


{\hyperref[software_installation:software\string-installation]{\crossref{\DUrole{std,std-ref}{Downloading and installing software for this class}}}} for links to software download pages.




\subsection{Virtual machine references}
\label{biblio:biblio-vm}\label{biblio:virtual-machine-references}

\subsection{Sphinx references}
\label{biblio:sphinx-references}\label{biblio:biblio-sphinx}

\subsection{Python:}
\label{biblio:biblio-python}\label{biblio:id5}

\subsection{Numerical Python references}
\label{biblio:biblio-numpy}\label{biblio:numerical-python-references}

\subsection{Unix, bash references}
\label{biblio:biblio-unix}\label{biblio:unix-bash-references}

\subsection{Version control systems references}
\label{biblio:biblio-vcs}\label{biblio:version-control-systems-references}

\subsection{Git references}
\label{biblio:git-references}\label{biblio:biblio-git}

\subsection{Mercurial references}
\label{biblio:mercurial-references}

\subsection{Reproducibility references}
\label{biblio:reproducibility-references}\label{biblio:biblio-repro}

\subsection{Fortran references}
\label{biblio:biblio-fortran}\label{biblio:fortran-references}
Many tutorials and references are available online.  Search for
``fortran 90 tutorial'' or ``fortran 95 tutorial'' to find many others.


\subsection{Makefile references}
\label{biblio:makefile-references}\label{biblio:biblio-make}

\subsection{Computer architecture references}
\label{biblio:computer-architecture-references}\label{biblio:biblio-computer-arch}

\subsection{Floating point arithmetic}
\label{biblio:floating-point-arithmetic}

\subsection{Languages and compilers}
\label{biblio:languages-and-compilers}

\subsection{OpenMP references}
\label{biblio:biblio-openmp}\label{biblio:openmp-references}

\subsection{MPI references}
\label{biblio:biblio-mpi}\label{biblio:mpi-references}
See also {[}Gropp-Lusk-Skjellum-MPI{]} and  {[}Snir-Dongarra-etal-MPI{]}

See {\hyperref[mpi:mpi]{\crossref{\DUrole{std,std-ref}{MPI}}}} for more references.


\subsection{Exa-scale computing}
\label{biblio:exa-scale-computing}

\bigskip\hrule{}\bigskip


\textbf{More will be added, check back later}

\begin{thebibliography}{wikipedia-revision-control-software}
\bibitem[Lin-Snyder]{Lin-Snyder}{\phantomsection\label{biblio:lin-snyder} 
C. Lin and L. Snyder, \emph{Principles of Parallel Programming},
2008.
}
\bibitem[Scott-Clark-Bagheri]{Scott-Clark-Bagheri}{\phantomsection\label{biblio:scott-clark-bagheri} 
L. R. Scott, T. Clark, B. Bagheri,
\emph{Scientific Parallel Computing}, Princeton University Press, 2005.
}
\bibitem[McCormack-scientific-fortran]{McCormack-scientific-fortran}{\phantomsection\label{biblio:mccormack-scientific-fortran} 
D. McCormack, \emph{Scientific Software Development
in Fortran}, Lulu Press, ...
\href{http://www.lulu.com/product/download/scientific-software-development-in-fortran/6208176}{ebook}  ...
\href{http://www.lulu.com/product/paperback/scientific-software-development-in-fortran/6208175}{paperback}
}
\bibitem[Rauber-Ruenger]{Rauber-Ruenger}{\phantomsection\label{biblio:rauber-ruenger} 
T. Rauber and G. Ruenger,
\emph{Parallel Programming For Multicore and Cluster Systems},
Springer, 2010 ... \href{http://www.springer.com/computer/swe/book/978-3-642-04817-3}{book}
... \href{http://www.springerlink.com/content/978-3-642-04817-3\#section=675480\&page=1\&locus=0}{ebook}
}
\bibitem[Chandra-etal-openmp]{Chandra-etal-openmp}{\phantomsection\label{biblio:chandra-etal-openmp} 
R. Chandra, L. Dagum, et. al., \emph{Parallel Programming
in OpenMP}, Academic Press, 2001.
}
\bibitem[Gropp-Lusk-Skjellum-MPI]{Gropp-Lusk-Skjellum-MPI}{\phantomsection\label{biblio:gropp-lusk-skjellum-mpi} 
W. Gropp, E. Lusk, A. Skjellum, \emph{Using MPI},
Second Edition, MIT Press, 1999.
\href{http://books.google.com/books?id=xpBZ0RyRb-oC\&printsec=frontcover\&dq=Gropp+lusk+skjellum+mpi\&source=bl\&ots=u9fzi2MK9Z\&sig=GvK20XXrv9xMgYSyJ3JXKt45fhY\&hl=en\&ei=zpOlS-6HIIOeswPPwu3YAw\&sa=X\&oi=book\_result\&ct=result\&resnum=1\&ved=0CAYQ6AEwAA\#v=onepage\&q=\&f=false}{Google books}
}
\bibitem[Snir-Dongarra-etal-MPI]{Snir-Dongarra-etal-MPI}{\phantomsection\label{biblio:snir-dongarra-etal-mpi} 
M. Snir, J. Dongarra, J. S. Kowalik, S.
Huss-Lederman, S. W. Otto, D. W. Walker,
\emph{MPI: The Complete Reference (2-volume set)} , MIT Press, 2000.
}
\bibitem[Dive-into-Python]{Dive-into-Python}{\phantomsection\label{biblio:dive-into-python} 
M. Pilgram, \emph{Dive Into Python},
\url{http://www.diveintopython.org/}.
}
\bibitem[Python]{Python}{\phantomsection\label{biblio:python} 
G. van Rossum, \emph{An Introduction to Python},
\url{http://www.network-theory.co.uk/docs/pytut/index.html}
}
\bibitem[Langtangen-scripting]{Langtangen-scripting}{\phantomsection\label{biblio:langtangen-scripting} 
H. P. Langangen, \emph{Python Scripting for
Computational Science}, 3rd edition, Springer, 2008.
\href{http://folk.uio.no/hpl/scripting/}{book and scripts} ...
\href{http://heim.ifi.uio.no/~hpl/scripting/all-nosplit/}{lots of slides}
}
\bibitem[Langtangen-Primer]{Langtangen-Primer}{\phantomsection\label{biblio:langtangen-primer} 
H. P. Langtangen, \emph{A Primer on Scientific Programming
with Python}, Springer 2009  \href{http://folk.uio.no/hpl/scripting/book\_comparison.html}{What's the difference from the previous
one?}
}
\bibitem[Goedecker-Hoisie-optimization]{Goedecker-Hoisie-optimization}{\phantomsection\label{biblio:goedecker-hoisie-optimization} 
S. Goedecker and A. Hoisie,
\emph{Performance Optimization of Numerically intensive Codes}, SIAM 2001.
}
\bibitem[Matloff-Salzman-debugging]{Matloff-Salzman-debugging}{\phantomsection\label{biblio:matloff-salzman-debugging} 
N. Matloff and P. J. Salzman, \emph{The Art
of Debugging with GDB, DDD, and Eclipse}, no starch press, San Francisco,
2008.
}
\bibitem[Overton-IEEE]{Overton-IEEE}{\phantomsection\label{biblio:overton-ieee} 
M. Overton, \emph{Numerical Computing with IEEE Floating Point
Arithmetic}, SIAM, 2001.
}
\bibitem[Eijkhout]{Eijkhout}{\phantomsection\label{biblio:eijkhout} 
V. Eijkhout, \emph{Introduction to High-Performance Scientific Computing},
\href{http://tacc-web.austin.utexas.edu/veijkhout/public\_html/istc/istc.html}{{[}link to book and slides{]}}
}
\bibitem[software-carpentry]{software-carpentry}{\phantomsection\label{biblio:software-carpentry} 
Greg Wilson, \url{http://software-carpentry.org/}
See {\hyperref[software_carpentry:software\string-carpentry]{\crossref{\DUrole{std,std-ref}{Software Carpentry}}}} for links to some useful sections.
}
\bibitem[Reynolds-class]{Reynolds-class}{\phantomsection\label{biblio:reynolds-class} 
Dan Reynolds, SMU \url{http://dreynolds.math.smu.edu/Courses/Math6370\_Spring11/}.
}
\bibitem[Snyder-UW-CSE524]{Snyder-UW-CSE524}{\phantomsection\label{biblio:snyder-uw-cse524} 
Larry Snyder, \href{http://www.cs.washington.edu/education/courses/524/08wi/}{UW CSE 524, Parallel Algorithms}
}
\bibitem[Gropp-UIUC]{Gropp-UIUC}{\phantomsection\label{biblio:gropp-uiuc} 
William Gropp \href{http://www.cs.uiuc.edu/homes/wgropp/cs598/index.htm}{UIUC Topics in HPC}
}
\bibitem[Yelick-UCB]{Yelick-UCB}{\phantomsection\label{biblio:yelick-ucb} 
Kathy Yelick, \href{http://www.cs.berkeley.edu/~yelick/cs267/}{Berkeley course on parallel computing}
}
\bibitem[Demmel-UCB]{Demmel-UCB}{\phantomsection\label{biblio:demmel-ucb} 
Jim Demmel,
\href{http://www.cs.berkeley.edu/~demmel/cs267\_Spr12/}{Berkeley course on parallel computing}
}
\bibitem[Kloekner-Berger-NYU]{Kloekner-Berger-NYU}{\phantomsection\label{biblio:kloekner-berger-nyu} 
Andreas Kloeckner and Marsha Berger,
\href{http://wiki.tiker.net/Teaching/HPCFall2012/}{NYU course}
}
\bibitem[Berger-Bindel-NYU]{Berger-Bindel-NYU}{\phantomsection\label{biblio:berger-bindel-nyu} 
Marsha Berger and David Bindel,
\href{http://www.cs.nyu.edu/courses/fall08/G22.2945-001/index.html}{NYU course}
}
\bibitem[LLNL-HPC]{LLNL-HPC}{\phantomsection\label{biblio:llnl-hpc} 
\href{https://computing.llnl.gov/?set=training\&page=index}{Livermore HPC tutorials}
}
\bibitem[NERSC-tutorials]{NERSC-tutorials}{\phantomsection\label{biblio:nersc-tutorials} 
\href{http://www.nersc.gov/nusers/help/tutorials/}{NERSC tutorials}
}
\bibitem[HPC-University]{HPC-University}{\phantomsection\label{biblio:hpc-university} 
\url{http://www.hpcuniv.org/}
}
\bibitem[CosmicProject]{CosmicProject}{\phantomsection\label{biblio:cosmicproject} 
\href{http://cosmicproject.org/links.html}{links to open source Python software}
}
\bibitem[VirtualBox]{VirtualBox}{\phantomsection\label{biblio:virtualbox} 
\url{http://www.virtualbox.org/}
}
\bibitem[VirtualBox-documentation]{VirtualBox-documentation}{\phantomsection\label{biblio:virtualbox-documentation} 
\url{http://www.virtualbox.org/wiki/Documentation}
}
\bibitem[sphinx]{sphinx}{\phantomsection\label{biblio:sphinx} 
\url{http://sphinx-doc.org}
}
\bibitem[sphinx-documentation]{sphinx-documentation}{\phantomsection\label{biblio:sphinx-documentation} 
\url{http://sphinx-doc.org/contents.html}
}
\bibitem[sphinx-rst]{sphinx-rst}{\phantomsection\label{biblio:sphinx-rst} 
\url{http://sphinx-doc.org/rest.html}
}
\bibitem[rst-documentation]{rst-documentation}{\phantomsection\label{biblio:rst-documentation} 
\url{http://docutils.sourceforge.net/rst.html}
}
\bibitem[sphinx-cheatsheet]{sphinx-cheatsheet}{\phantomsection\label{biblio:sphinx-cheatsheet} 
\url{http://matplotlib.sourceforge.net/sampledoc/cheatsheet.html}
}
\bibitem[sphinx-examples]{sphinx-examples}{\phantomsection\label{biblio:sphinx-examples} 
\url{http://sphinx-doc.org/examples.html}
}
\bibitem[sphinx-sampledoc]{sphinx-sampledoc}{\phantomsection\label{biblio:sphinx-sampledoc} 
\url{http://matplotlib.sourceforge.net/sampledoc/index.html}
}
\bibitem[Python-2.5-tutorial]{Python-2.5-tutorial}{\phantomsection\label{biblio:python-2-5-tutorial} 
\url{http://www.python.org/doc/2.5.2/tut/tut.html}
}
\bibitem[Python-2.7-tutorial]{Python-2.7-tutorial}{\phantomsection\label{biblio:python-2-7-tutorial} 
\url{http://docs.python.org/tutorial/}
}
\bibitem[Python-documentation]{Python-documentation}{\phantomsection\label{biblio:python-documentation} 
\url{http://docs.python.org/2/contents.html}
}
\bibitem[Python-3.0-tutorial]{Python-3.0-tutorial}{\phantomsection\label{biblio:python-3-0-tutorial} 
\url{http://docs.python.org/3.0/tutorial/}
(we are \emph{not} using Python 3.0 in this class!)
}
\bibitem[IPython-documentation]{IPython-documentation}{\phantomsection\label{biblio:ipython-documentation} 
\url{http://ipython.org/documentation.html}
(With lots of links to other documentation and tutorials)
}
\bibitem[IPython-notebook]{IPython-notebook}{\phantomsection\label{biblio:ipython-notebook} 
\url{http://ipython.org/ipython-doc/dev/interactive/htmlnotebook.html}
}
\bibitem[Python-pdb]{Python-pdb}{\phantomsection\label{biblio:python-pdb} 
\href{http://docs.python.org/2/library/pdb.html}{Python debugger documentation}
}
\bibitem[IPython-book]{IPython-book}{\phantomsection\label{biblio:ipython-book} 
Cyrille Rossant,
\emph{Learning IPython for Interactive Computing and Data Visualization},
Packt Publishing, 2013. \url{http://ipython.rossant.net/}.
}
\bibitem[NumPy-tutorial]{NumPy-tutorial}{\phantomsection\label{biblio:numpy-tutorial} 
\url{http://www.scipy.org/Tentative\_NumPy\_Tutorial}
}
\bibitem[NumPy-reference]{NumPy-reference}{\phantomsection\label{biblio:numpy-reference} 
\url{http://docs.scipy.org/doc/numpy/reference/}
}
\bibitem[NumPy-SciPy-docs]{NumPy-SciPy-docs}{\phantomsection\label{biblio:numpy-scipy-docs} 
\url{http://docs.scipy.org/doc/}
}
\bibitem[NumPy-for-Matlab-Users]{NumPy-for-Matlab-Users}{\phantomsection\label{biblio:numpy-for-matlab-users} 
\url{http://www.scipy.org/NumPy\_for\_Matlab\_Users}
}
\bibitem[NumPy-pros-cons]{NumPy-pros-cons}{\phantomsection\label{biblio:numpy-pros-cons} 
\url{http://www.scipy.org/NumPyProConPage}
}
\bibitem[Numerical-Python-links]{Numerical-Python-links}{\phantomsection\label{biblio:numerical-python-links} 
\url{http://wiki.python.org/moin/NumericAndScientific}
}
\bibitem[Reynolds-unix]{Reynolds-unix}{\phantomsection\label{biblio:reynolds-unix} 
\href{http://dreynolds.math.smu.edu/Courses/Math6370\_Spring11/unix.html}{Dan Reynolds unix page} has good links.
}
\bibitem[Wikipedia-unix-utilities]{Wikipedia-unix-utilities}{\phantomsection\label{biblio:wikipedia-unix-utilities} 
\url{http://en.wikipedia.org/wiki/List\_of\_Unix\_utilities}
}
\bibitem[Bash-Beginners-Guide]{Bash-Beginners-Guide}{\phantomsection\label{biblio:bash-beginners-guide} 
\url{http://tldp.org/LDP/Bash-Beginners-Guide/html/index.html}
}
\bibitem[gnu-bash]{gnu-bash}{\phantomsection\label{biblio:gnu-bash} 
\url{http://www.gnu.org/software/bash/bash.html}
}
\bibitem[Wikipedia-bash]{Wikipedia-bash}{\phantomsection\label{biblio:wikipedia-bash} 
\url{http://en.wikipedia.org/wiki/Bash(Unix\_shell)}
}
\bibitem[wikipedia-tar]{wikipedia-tar}{\phantomsection\label{biblio:wikipedia-tar} 
\url{http://en.wikipedia.org/wiki/Tar\_\%28file\_format\%29}
(tar files)
}
\bibitem[wikipedia-revision-control]{wikipedia-revision-control}{\phantomsection\label{biblio:wikipedia-revision-control} 
\url{http://en.wikipedia.org/wiki/Revision\_control}
}
\bibitem[wikipedia-revision-control-software]{wikipedia-revision-control-software}{\phantomsection\label{biblio:wikipedia-revision-control-software} 
\url{http://en.wikipedia.org/wiki/List\_of\_revision\_control\_software}
}
\bibitem[git-try]{git-try}{\phantomsection\label{biblio:git-try} 
\href{http://try.github.com/}{Online interactive tutorial}
}
\bibitem[git-tutorials]{git-tutorials}{\phantomsection\label{biblio:git-tutorials} 
List of 10 tutorials \url{http://sixrevisions.com/resources/git-tutorials-beginners/}
}
\bibitem[gitref]{gitref}{\phantomsection\label{biblio:gitref} 
\url{http://gitref.org/index.html}
}
\bibitem[git-book]{git-book}{\phantomsection\label{biblio:git-book} 
Git Book \url{http://git-scm.com/book/en/Getting-Started-Git-Basics}
}
\bibitem[github-help]{github-help}{\phantomsection\label{biblio:github-help} 
Github help page: \url{http://help.github.com/}
}
\bibitem[git-parable]{git-parable}{\phantomsection\label{biblio:git-parable} 
\url{http://tom.preston-werner.com/2009/05/19/the-git-parable.html}
}
\bibitem[hgbook]{hgbook}{\phantomsection\label{biblio:hgbook} 
\url{http://hgbook.red-bean.com/}
}
\bibitem[hg-faq]{hg-faq}{\phantomsection\label{biblio:hg-faq} 
\url{http://mercurial.selenic.com/wiki/FAQ}
}
\bibitem[sci-code-manifesto]{sci-code-manifesto}{\phantomsection\label{biblio:sci-code-manifesto} 
\href{http://sciencecodemanifesto.org/}{Science Code Manifesto}
}
\bibitem[icerm-workshop]{icerm-workshop}{\phantomsection\label{biblio:icerm-workshop} 
\href{http://icerm.brown.edu/tw12-5-rcem-wiki.php}{Links from a recent workshop on the topic}
}
\bibitem[winter-school]{winter-school}{\phantomsection\label{biblio:winter-school} 
\href{http://www.sintef.no/Projectweb/eVITA/Winter-Schools/2013/}{A recent Winter School on the topic in Geilo, Norway:}
}
\bibitem[Reynolds-fortran]{Reynolds-fortran}{\phantomsection\label{biblio:reynolds-fortran} 
Dan Reynolds fortran page \url{http://dreynolds.math.smu.edu/Courses/Math6370\_Spring11/fortran.html}
}
\bibitem[Shene-fortran]{Shene-fortran}{\phantomsection\label{biblio:shene-fortran} 
C.-K. Shene's Fortran 90 tutorial \url{http://www.cs.mtu.edu/~shene/COURSES/cs201/NOTES/fortran.html}
}
\bibitem[Dodson-fortran]{Dodson-fortran}{\phantomsection\label{biblio:dodson-fortran} 
Zane Dodson's Fortran 90 tutorial \url{http://www.cisl.ucar.edu/tcg/consweb/Fortran90/F90Tutorial/tutorial.html}
}
\bibitem[fortran-tutorials]{fortran-tutorials}{\phantomsection\label{biblio:fortran-tutorials} 
Links to a few other tutorials \url{http://gcc.gnu.org/wiki/Fortran\%2095\%20tutorials\%20available\%20online}
}
\bibitem[advanced-fortran]{advanced-fortran}{\phantomsection\label{biblio:advanced-fortran} 
Kaiser, Advanced Fortran 90 \url{http://www.sdsc.edu/~tkaiser/f90.html}
}
\bibitem[carpentry-make]{carpentry-make}{\phantomsection\label{biblio:carpentry-make} 
\url{http://software-carpentry.org/4\_0/make/}
}
\bibitem[gnu-make]{gnu-make}{\phantomsection\label{biblio:gnu-make} 
\url{http://www.gnu.org/software/make/manual/make.html}
}
\bibitem[make-tutorial]{make-tutorial}{\phantomsection\label{biblio:make-tutorial} 
\url{http://mrbook.org/tutorials/make/}
}
\bibitem[Wikipedia-make]{Wikipedia-make}{\phantomsection\label{biblio:wikipedia-make} 
\url{http://en.wikipedia.org/wiki/Make\_\%28software\%29}
}
\bibitem[wikipedia-computer-architecture]{wikipedia-computer-architecture}{\phantomsection\label{biblio:wikipedia-computer-architecture} 
\url{http://en.wikipedia.org/wiki/Computer\_architecture}
}
\bibitem[wikipedia-memory-hierachy]{wikipedia-memory-hierachy}{\phantomsection\label{biblio:wikipedia-memory-hierachy} 
\url{http://en.wikipedia.org/wiki/Memory\_hierarchy}
}
\bibitem[wikipedia-moores-law]{wikipedia-moores-law}{\phantomsection\label{biblio:wikipedia-moores-law} 
\url{http://en.wikipedia.org/wiki/Moore\%27s\_law}.
}
\bibitem[Arnold-disasters]{Arnold-disasters}{\phantomsection\label{biblio:arnold-disasters} 
Doug Arnold's descriptions of some disasters due to
bad numerical computing,
\url{http://www.ima.umn.edu/~arnold/disasters/}
}
\bibitem[wikipedia-machine-code]{wikipedia-machine-code}{\phantomsection\label{biblio:wikipedia-machine-code} 
\url{http://en.wikipedia.org/wiki/Machine\_code}
}
\bibitem[wikipedia-assembly]{wikipedia-assembly}{\phantomsection\label{biblio:wikipedia-assembly} 
\url{http://en.wikipedia.org/wiki/Assembly\_language}
}
\bibitem[wikipedia-compilers]{wikipedia-compilers}{\phantomsection\label{biblio:wikipedia-compilers} 
\url{http://en.wikipedia.org/wiki/Compilers}
}
\bibitem[Chandra-etal-openmp]{Chandra-etal-openmp}{\phantomsection\label{biblio:id6} 
R. Chandra, L. Dagum, et. al., \emph{Parallel Programming
in OpenMP}, Academic Press, 2001.
}
\bibitem[Chapman-Jost]{Chapman-Jost}{\phantomsection\label{biblio:chapman-jost} 
B. Chapman, G. Jost, R. van der Pas, \emph{Using OpenMP:
Portable Shared Memory Parallel Programming}, MIT Press, 2007
}
\bibitem[openmp-RR]{openmp-RR}{\phantomsection\label{biblio:openmp-rr} 
Section 6.3 and beyond of \phantomsection\label{biblio:id7}{\hyperref[biblio:rauber\string-ruenger]{\crossref{{[}Rauber-Ruenger{]}}}}
}
\bibitem[openmp.org]{openmp.org}{\phantomsection\label{biblio:openmp-org} 
\url{http://openmp.org/wp/resources} contains pointers to
many books and tutorials.
}
\bibitem[openmp-specs]{openmp-specs}{\phantomsection\label{biblio:openmp-specs} 
\url{http://openmp.org/wp/openmp-specifications/}
has the latest official specifications
}
\bibitem[openmp-llnl]{openmp-llnl}{\phantomsection\label{biblio:openmp-llnl} 
\url{https://computing.llnl.gov/tutorials/openMP/} Livermore
tutorials
}
\bibitem[openmp-gfortran]{openmp-gfortran}{\phantomsection\label{biblio:openmp-gfortran} 
\url{http://gcc.gnu.org/onlinedocs/gfortran/OpenMP.html}
}
\bibitem[openmp-gfortran2]{openmp-gfortran2}{\phantomsection\label{biblio:openmp-gfortran2} 
\url{http://sites.google.com/site/gfortransite/}
}
\bibitem[openmp-refcard]{openmp-refcard}{\phantomsection\label{biblio:openmp-refcard} 
\href{http://openmp.org/mp-documents/OpenMP3.0-FortranCard.pdf}{OpenMP in Fortran Reference card}
}
\bibitem[openmp-RR]{openmp-RR}{\phantomsection\label{biblio:id8} 
Chapter 5 of \phantomsection\label{biblio:id9}{\hyperref[biblio:rauber\string-ruenger]{\crossref{{[}Rauber-Ruenger{]}}}}
}
\bibitem[exascale-doe]{exascale-doe}{\phantomsection\label{biblio:exascale-doe} 
\href{http://www.er.doe.gov/ascr/ProgramDocuments/Docs/TownHall.pdf}{Modeling and Simulation at the Exascale for Energy and the
Environment, DOE Town Hall Meetings Report}
}
\bibitem[exascale-sc08]{exascale-sc08}{\phantomsection\label{biblio:exascale-sc08} 
\url{http://www.lbl.gov/CS/html/SC08ExascalePowerWorkshop/index.html}
}
\end{thebibliography}



\renewcommand{\indexname}{Index}
\printindex
\end{document}
